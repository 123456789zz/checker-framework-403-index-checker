\htmlhr
\chapter{Integration with external tools\label{external-tools}}

This chapter discusses how to run a checker from the command line, from a
build system, or from an IDE\@.  You can skip to the appropriate section:

% Keep this list up to date with the sections of this chapter and with TWO
% copies of the list in file introduction.tex .
\begin{itemize}
\item javac (Section~\ref{javac-installation})
\item Ant (Section~\ref{ant-task})
\item Maven (Section~\ref{maven})
\item Gradle (Section~\ref{gradle})
\item IntelliJ IDEA (Section~\ref{intellij})
\item Eclipse (Section~\ref{eclipse})
\item tIDE (Section~\ref{tide})
\end{itemize}

If your build system or IDE is not listed above, you should customize how
it runs the javac command on your behalf.  See your build system or IDE
documentation to learn how to
customize it, adapting the instructions for javac in Section~\ref{javac-installation}.
If you make another tool support running a checker, please
inform us via the
\href{https://groups.google.com/forum/#!forum/checker-framework-discuss}{mailing
  list} or
\href{https://github.com/typetools/checker-framework/issues}{issue tracker} so
we can add it to this manual.

This chapter also discusses type inference tools (see
Section~\ref{type-inference-tools}).

All examples in this chapter are in the public domain, with no copyright nor
licensing restrictions.


\section{Javac compiler\label{javac-installation}}

To perform pluggable type-checking, run the \<javac> compiler with the
Checker Framework on the classpath.
There are three ways to achieve this.  You can use any
one of them.  However, if you are using the Windows command shell, you must
use the last one.
% Is the last one required for Cygwin, as well as for the Windows command shell?


\begin{itemize}
  \item
    Option 1:
    Add directory
    \code{.../checker-framework-1.9.11/checker/bin} to your path, \emph{before} any other
    directory that contains a \<javac> executable.

    If you are
    using the bash shell, a way to do this is to add the following to your
    \verb|~/.profile| (or alternately \verb|~/.bash_profle| or \verb|~/.bashrc|) file:
\begin{Verbatim}
  export CHECKERFRAMEWORK=${HOME}/checker-framework-1.9.11
  export PATH=${CHECKERFRAMEWORK}/checker/bin:${PATH}
\end{Verbatim}
    then log out and back in to ensure that the environment variable
    setting takes effect.

    Now, whenever you run \code{javac}, you will use the ``Checker
    Framework compiler''.  It is exactly the same as the OpenJDK compiler,
    with two small differences:  it includes the Checker Framework jar file
    on its classpath, and it recognizes type annotations in comments (see
    Section~\ref{annotations-in-comments}).
   
  \item
    \begin{sloppypar}
    Option 2:
    Whenever this document tells you to run \code{javac}, you
    can instead run \code{\$CHECKERFRAMEWORK/checker/bin/javac}.
    \end{sloppypar}

    You can simplify this by introducing an alias.  Then,
    whenever this document tells you to run \code{javac}, instead use that
    alias.  Here is the syntax for your
    \verb|~/.bashrc| file:
% No Windows example because this doesn't work under Windows.
\begin{Verbatim}
  export CHECKERFRAMEWORK=${HOME}/checker-framework-1.9.11
  alias javacheck='$CHECKERFRAMEWORK/checker/bin/javac'
\end{Verbatim}

    If you are using a Java 7 JVM, then add command-line arguments to so
    indicate:

\begin{Verbatim}
  export CHECKERFRAMEWORK=${HOME}/checker-framework-1.9.11
  alias javacheck='$CHECKERFRAMEWORK/checker/bin/javac -source 7 -target 7'
\end{Verbatim}

   If you do not add the \<-source 7 -target 7> command-line arguments, you
   may get the following error when running a class that was compiled by
   javacheck:
\begin{Verbatim}
  UnsupportedClassVersionError: ... : Unsupported major.minor version 52.0
\end{Verbatim}

   \item
   Option 3:
   Whenever this document tells you to run \code{javac}, instead
   run checker.jar via \<java> (not \<javac>) as in:

\begin{Verbatim}
  java -jar $CHECKERFRAMEWORK/checker/dist/checker.jar ...
\end{Verbatim}

    You can simplify the above command by introducing an alias.  Then,
    whenever this document tells you to run \code{javac}, instead use that
    alias.  For example:

\begin{Verbatim}
  # Unix
  export CHECKERFRAMEWORK=${HOME}/checker-framework-1.9.11
  alias javacheck='java -jar $CHECKERFRAMEWORK/checker/dist/checker.jar'

  # Windows
  set CHECKERFRAMEWORK = C:\Program Files\checker-framework-1.9.11\
  doskey javacheck=java -jar %CHECKERFRAMEWORK%\checker\dist\checker.jar $*
\end{Verbatim}

   \noindent
   and add \<-source 7 -target 7> if you use a Java 7 JVM.

   (Explanation for advanced users:  More generally, anywhere that you would use \<javac.jar>, you can substitute
   \<\$CHECKERFRAMEWORK/checker/dist/checker.jar>;
   the result is to use the Checker
   Framework compiler instead of the regular \<javac>.)

\end{itemize}


To ensure that you are using the Checker Framework compiler, run
\<javac -version> (possibly using the
full pathname to \<javac> or the alias, if you did not add the Checker
Framework \<javac> to your path).
The output should be:

\begin{Verbatim}
  javac 1.8.0-jsr308-1.9.11
\end{Verbatim}




%% Does this work?  Text elsewhere in the manual imples that it does not.
% \item
% \begin{sloppypar}
%   In order to use the updated compiler when you type \code{javac}, add the
%   directory \<C:\ttbs{}Program Files\ttbs{}checker-framework\ttbs{}checkers\ttbs{}binary> to the
%   beginning of your path variable.  Also set a \code{CHECKERFRAMEWORK} variable.
% \end{sloppypar}
%
% % Instructions stolen from http://www.webreference.com/js/tips/020429.html
%
% To set an environment variable, you have two options:  make the change
% temporarily or permanently.
% \begin{itemize}
% \item
% To make the change \textbf{temporarily}, type at the command shell prompt:
%
% \begin{alltt}
% path = \emph{newdir};%PATH%
% \end{alltt}
%
% For example:
%
% \begin{Verbatim}
% set CHECKERFRAMEWORK = C:\Program Files\checker-framework
% path = %CHECKERFRAMEWORK%\checker\bin;%PATH%
% \end{Verbatim}
%
% This is a temporary change that endures until the window is closed, and you
% must re-do it every time you start a new command shell.
%
% \item
% To make the change \textbf{permanently},
% Right-click the \<My Computer> icon and
% select \<Properties>. Select the \<Advanced> tab and click the
% \<Environment Variables> button. You can set the variable as a ``System
% Variable'' (visible to all users) or as a ``User Variable'' (visible to
% just this user).  Both work; the instructions below show how to set as a
% ``System Variable''.
% In the \<System Variables> pane, select
% \<Path> from the list and click \<Edit>. In the \<Edit System Variable>
% dialog box, move the cursor to the beginning of the string in the
% \<Variable Value> field and type the full directory name (not using the
% \verb|%CHECKERFRAMEWORK%| environment variable) followed by a
% semicolon (\<;>).
%
% % This is for the benefit of the Ant task.
% Similarly, set the \code{CHECKERFRAMEWORK} variable.
%
% This is a permanent change that only needs to be done once ever.
% \end{itemize}



\section{Ant task\label{ant-task}}

If you use the \href{http://ant.apache.org/}{Ant} build tool to compile
your software, then you can add an Ant task that runs a checker.  We assume
that your Ant file already contains a compilation target that uses the
\code{javac} task.

\begin{enumerate}
\item
Set the \code{jsr308javac} property:

%BEGIN LATEX
\begin{smaller}
%END LATEX
\begin{Verbatim}
  <property environment="env"/>

  <property name="checkerframework" value="${env.CHECKERFRAMEWORK}" />

  <!-- On Mac/Linux, use the javac shell script; on Windows, use javac.bat -->
  <condition property="cfJavac" value="javac.bat" else="javac">
      <os family="windows" />
  </condition>

  <presetdef name="jsr308.javac">
    <javac fork="yes" executable="${checkerframework}/checker/bin/${cfJavac}" >
      <!-- JSR-308-related compiler arguments -->
      <compilerarg value="-version"/>
      <compilerarg value="-implicit:class"/>
    </javac>
  </presetdef>
\end{Verbatim}
%BEGIN LATEX
\end{smaller}
%END LATEX

\item \textbf{Duplicate} the compilation target, then \textbf{modify} it slightly as
indicated in this example:

%BEGIN LATEX
\begin{smaller}
%END LATEX
\begin{Verbatim}
  <target name="check-nullness"
          description="Check for null pointer dereferences"
          depends="clean,...">
    <!-- use jsr308.javac instead of javac -->
    <jsr308.javac ... >
      <compilerarg line="-processor org.checkerframework.checker.nullness.NullnessChecker"/>
      <!-- optional, to not check uses of library methods:
        <compilerarg value="-AskipUses=^(java\.awt\.|javax\.swing\.)"/>
      -->
      <compilerarg line="-Xmaxerrs 10000"/>
      ...
    </jsr308.javac>
  </target>
\end{Verbatim}
%BEGIN LATEX
\end{smaller}
%END LATEX

Fill in each ellipsis (\ldots) from the original compilation target.

In the example, the target is named \code{check-nullness}, but you can
name it whatever you like.
\end{enumerate}

\subsection{Explanation\label{ant-task-explanation}}

This section explains each part of the Ant task.

\begin{enumerate}
\item Definition of \code{jsr308.javac}:

The \code{fork} field of the \code{javac} task
ensures that an external javac program is called.  Otherwise, Ant will run
javac via a Java method call, and there is no guarantee that it will get
the Checker Framework compiler that is distributed with the Checker Framework.

The \code{-version} compiler argument is just for debugging; you may omit
it.

The \code{-implicit:class} compiler argument causes annotation processing
to be performed on implicitly compiled files.  (An implicitly compiled file
is one that was not specified on the command line, but for which the source
code is newer than the \code{.class} file.)  This is the default, but
supplying the argument explicitly suppresses a compiler warning.

%% -Awarns was removed above without removing it here.
% The \code{-Awarns} compiler argument is optional, and causes the checker to
% treat errors as warnings so that compilation does not fail even if
% pluggable type-checking fails; see Section~\ref{checker-options}.

\item The \code{check-nullness} target:

The target assumes the existence of a \code{clean} target that removes all
\code{.class} files.  That is necessary because Ant's \code{javac} target
doesn't re-compile \code{.java} files for which a \code{.class} file
already exists.

The \code{-processor ...} compiler argument indicates which checker to
run.  You can supply additional arguments to the checker as well.

\end{enumerate}


\section{Maven\label{maven}}

If you use the \href{http://maven.apache.org/}{Maven} tool,
then you can enable Checker Framework checkers by following the
instructions below.

These instructions use the artifacts from
\href{http://search.maven.org/#search\%7Cga\%7C1\%7Corg.checkerframework}{Maven Central}.

See the demonstration \code{examples/MavenExample/} for an example of the use of
these instructions in a Maven project.  This example can be used to verify that
Maven is correctly downloading and executing the Checker Framework on your machine.

Please note that the \<-AoutputArgsToFile> command-line option
(see Section~\ref{debugging-options-output-args}) and shorthands for built-in checkers
(see Section~\ref{shorthand-for-checkers}) are not available when
following these instructions.  Both these features are available only when a checker is
launched via \<checker.jar> such as when \code{\$CHECKERFRAMEWORK/checker/bin/javac}
is run.  These instructions bypass \<checker.jar> and cause the compiler to run a
checker as an annotation processor directly.

\begin{enumerate}

\item Declare a dependency on the Checker Framework artifacts.  Find the
  existing \code{<dependencies>} section and add the following new
  \code{<dependency>} items:

\begin{alltt}
    <dependencies>
        ... existing <dependency> items ...

        <!-- annotations from the Checker Framework: nullness, interning, locking, ... -->
        <dependency>
            <groupId>org.checkerframework</groupId>
            <artifactId>checker-qual</artifactId>
            <version>\ReleaseVersion{}</version>
        </dependency>
        <dependency>
            <groupId>org.checkerframework</groupId>
            <artifactId>checker</artifactId>
            <version>\ReleaseVersion{}</version>
        </dependency>
        <!-- The type annotations compiler - uncomment if using Java 7 -->
        <!-- <dependency>
            <groupId>org.checkerframework</groupId>
            <artifactId>compiler</artifactId>
            <version>\ReleaseVersion{}</version>
        </dependency> -->
        <!-- The annotated JDK to use (change to jdk7 if using Java 7) -->
        <dependency>
            <groupId>org.checkerframework</groupId>
            <artifactId>jdk8</artifactId>
            <version>\ReleaseVersion{}</version>
        </dependency>
    </dependencies>
\end{alltt}

\item Use Maven properties to hold the locations of the
  annotated JDK and, if using Java 7, the type annotations compiler.  Both were declared as Maven dependencies above.
To set the value of these properties automatically, use the Maven Dependency plugin.

First, create the properties in the \code{properties} section of the POM:

\begin{alltt}
<properties>
    <!-- These properties will be set by the Maven Dependency plugin -->
    <!-- Change to jdk7 if using Java 7 -->
    <annotatedJdk>$\{org.checkerframework:jdk8:jar\}</annotatedJdk>
    <!-- The type annotations compiler is required if using Java 7. -->
    <!-- Uncomment the following line if using Java 7. -->
    <!-- <typeAnnotationsJavac>$\{org.checkerframework:compiler:jar\}</typeAnnotationsJavac> -->
</properties>
\end{alltt}

Change the reference to the \code{maven-dependency-plugin} within the \code{<plugins>}
section, or add it if it is not present.

\begin{alltt}
    <plugin>
        <!-- This plugin will set properties values using dependency information -->
        <groupId>org.apache.maven.plugins</groupId>
        <artifactId>maven-dependency-plugin</artifactId>
        <version>2.3</version>
        <executions>
            <execution>
                <goals>
                    <goal>properties</goal>
                </goals>
            </execution>
        </executions>
    </plugin>
\end{alltt}


\item Direct the Maven compiler plugin to use the desired checkers.
Change the reference to the \code{maven-compiler-plugin} within the \code{<plugins>}
section, or add it if it is not present.

For example, to use the \code{org.checkerframework.checker.nullness.NullnessChecker}:

\begin{mysmall}
\begin{alltt}
    <plugin>
        <artifactId>maven-compiler-plugin</artifactId>
        <version>3.3</version>
        <configuration>
            <!-- Change source and target to 1.7 if using Java 7 -->
            <source>1.8</source>
            <target>1.8</target>
            <fork>true</fork>
            <annotationProcessors>
                <!-- Add all the checkers you want to enable here -->
                <annotationProcessor>org.checkerframework.checker.nullness.NullnessChecker</annotationProcessor>
            </annotationProcessors>
            <compilerArgs>
                <!-- location of the annotated JDK, which comes from a Maven dependency -->
                <arg>-Xbootclasspath/p:$\{annotatedJdk\}</arg>
                <!-- Uncomment the following line if using Java 7. -->
                <!-- <arg>-J-Xbootclasspath/p:$\{typeAnnotationsJavac\}</arg> -->
            </compilerArgs>
        </configuration>
    </plugin>
\end{alltt}
\end{mysmall}

Now, building with Maven should run the checkers during compilation.

Notice that using this approach, no external setup is necessary,
so your Maven build should be reproducible on any server.

If you want to allow Maven to compile your code without running the checkers, you may
want to move the declarations above to within a Maven profile, so that the checkers
would only run if the profile was enabled.

\end{enumerate}


\section{Gradle\label{gradle}}

If you use \href{http://gradle.org/}{Gradle},
then you can run Checker Framework checkers by following the
instructions below.

These instructions use the artifacts from
\href{http://search.maven.org/#search\%7Cga\%7C1\%7Corg.checkerframework}{Maven Central}.

See the directory \code{examples/GradleExamples/} for examples of running a
checker in a Gradle project.  You can use these examples to verify that
Gradle is correctly downloading and executing the Checker Framework.

Please note that the \<-AoutputArgsToFile> command-line option
(see Section~\ref{debugging-options-output-args}) and shorthands for built-in checkers
(see Section~\ref{shorthand-for-checkers}) are not available when
following these instructions.  Both these features are available only when a checker is
launched via \<checker.jar> such as when \code{\$CHECKERFRAMEWORK/checker/bin/javac}
is run.  The Gradle instructions in this section
bypass \<checker.jar> and cause the compiler to run a
checker as an annotation processor directly.

These instructions can be used with Java 7 or Java 8.  If using Java 7, then
type annotations should be placed in comments as explained in
Section~\ref{annotations-in-comments}.

\begin{enumerate}

\item Add a property to indicate which version of Java to which the source and
class files are targeted.  This property can be set based on the current version
of Java as shown below or hard code.

\begin{Verbatim}
ext.targetJavaVersion = JavaVersion.current().isJava7() ? JavaVersion.VERSION_1_7 : JavaVersion.VERSION_1_8
\end{Verbatim}

\item Define the Maven Central repostiory:

\begin{Verbatim}
repositories {
  ... extisting repositories...
  mavenCentral()
}
\end{Verbatim}

\item Add dependency configurations for the annotated JDK, \<checker.jar>, and the Type Annotations compiler:

\begin{Verbatim}
configurations {
    ... extisting configurations ...
    if (targetJavaVersion.isJava7()) {
        checkerFrameworkJavac {
            description = 'a customization of the Open JDK javac compiler with additional support for type annotations'
        }
    }
    checkerFrameworkAnnotatedJDK {
       description = 'a copy of JDK classes with Checker Framework type qualifers inserted'
    }
    checkerFramework {
       description = 'The Checker Framework: custom pluggable types for Java'
    }
}
\end{Verbatim}

\item Using the Maven coordinates, declare the Checker Framework dependencies:

\begin{Verbatim}
dependencies {
    ... extisting dependencies...
    ext.checkerFrameworkVersion = '1.9.11'
    ext.jdkVersion = JavaVersion.current().isJava7() ? 'jdk7' : 'jdk8'
    checkerFrameworkAnnotatedJDK "org.checkerframework:${jdkVersion}:${checkerFrameworkVersion}"

    if (targetJavaVersion.isJava7()) {
        checkerFrameworkJavac "org.checkerframework:compiler:${checkerFrameworkVersion}"
    }
    checkerFramework "org.checkerframework:checker:${checkerFrameworkVersion}"
    compile "org.checkerframework:checker-qual:${checkerFrameworkVersion}"
}
\end{Verbatim}

\item Direct all tasks of type JavaCompile to use the desired checkers:

\begin{Verbatim}
allprojects {
    tasks.withType(JavaCompile).all { JavaCompile compile ->
        compile.options.compilerArgs = [
                '-processor', 'org.checkerframework.checker.nullness.NullnessChecker',
                '-processorpath', "${configurations.checkerFramework.asPath}",
                // uncomment to turn Checker Framework errors into warnings
                //'-Awarns',
                "-Xbootclasspath/p:${configurations.checkerFrameworkAnnotatedJDK.asPath}"
        ]
        if (targetJavaVersion.isJava7()) {
            compile.options.compilerArgs += ['-source', '7', '-target', '7']
            options.fork = true
            options.forkOptions.jvmArgs += ["-Xbootclasspath/p:${configurations.checkerFrameworkJavac.asPath}"]
        }
    }
}
\end{Verbatim}
\end{enumerate}

\section{IntelliJ IDEA\label{intellij}}

IntelliJ IDEA (Maia release)
\href{http://blogs.jetbrains.com/idea/2009/07/type-annotations-jsr-308-support/}{supports}
the Type Annotations (JSR-308) syntax.
See \url{http://blogs.jetbrains.com/idea/2009/07/type-annotations-jsr-308-support/}.

\section{Eclipse\label{eclipse}}

% Eclipse supports type annotations.
% Eclipse does not directly support running the Checker Framework,
% nor is Eclipse necessary for running the Checker Framework.

There are two ways to run a checker from within the Eclipse IDE:  via Ant
or using an Eclipse plugin.  These two methods are described below.

No matter what method you choose, we suggest that
all Checker Framework annotations be written in the comments
if you are using a version of Eclipse that
does not support Java 8.  This will avoid many
text highlighting errors with versions of Eclipse that don't support Java 8
and type annotations.

Even in a version of Eclipse that supports Java 8's type annotations, you
still need to run the Checker Framework via Ant or via the plug-in, rather
than by supplying the \<-processor> command-line option to the ejc
compiler.  The reason is that the Checker Framework is built upon javac,
and ejc represents the Java program differently.  (If both javac and ejc
implemented JSR 198~\cite{JSR198}, then it would be possible to build
a type-checking plug-in that works with both compilers.)


\subsection{Using an Ant task\label{eclipse-ant}}

Add an Ant target as described in Section~\ref{ant-task}.  You can
run the Ant target by executing the following steps
(instructions copied from {\codesize\url{http://help.eclipse.org/luna/index.jsp?topic=%2Forg.eclipse.platform.doc.user%2FgettingStarted%2Fqs-84_run_ant.htm}}):

\begin{enumerate}

\item
  Select \code{build.xml} in one of the navigation views and choose
  {\bf Run As $>$ Ant Build...} from its context menu.

\item
  A launch configuration dialog is opened on a launch configuration
  for this Ant buildfile.

\item
  In the {\bf Targets} tab, select the new ant task (e.g., check-interning).

\item
  Click {\bf Run}.

\item
  The Ant buildfile is run, and the output is sent to the Console view.

\end{enumerate}


\label{eclipse-plug-in}         % previous label for this section
\subsection{Eclipse plugin for the Checker Framework\label{eclipse-plugin}}

The Checker framework Eclipse Plugin enables the use of the Checker
Framework within the Eclipse IDE\@.
Its website (\myurl{http://types.cs.washington.edu/checker-framework/eclipse/}).
The website contains instructions for installing and using the plugin.
% The plugin has been substantially improved through a Google Summer of Code 2010 project
% and supports all checkers that are distributed with the Checker Framework.


\subsection{Troubleshooting Eclipse}

Eclipse issues an ``Unhandled Token in @SuppressWarnings'' warning if you
write a \<@SuppressWarnings> annotation containing a string that does not
know about.  Unfortunately, Eclipse
\href{https://bugs.eclipse.org/bugs/show_bug.cgi?id=122475}{hard-codes}
this list and there is not a way for the Eclipse plug-in to extend it.

To eliminate the warnings, you have two options:

\begin{enumerate}
\item
Write \<@SuppressWarnings> annotations related to the Checker Framework in
comments (see Section~\ref{annotations-in-comments}).
\item
  Disable all ``Unhandled Token in @SuppressWarnings'' warnings in Eclipse.
  Look under the menu headings Java $\rightarrow$ Compiler $\rightarrow$ Errors/Warnings $\rightarrow$ Annotations $\rightarrow$ Unhandled Token in '@SuppressWarnings', and set it to ignore.
\end{enumerate}


\section{tIDE\label{tide}}

\begin{sloppypar}
tIDE, an open-source Java IDE, supports the Checker Framework.  See its
documentation at \myurl{http://tide.olympe.in/}.
\end{sloppypar}


\section{Type inference tools\label{type-inference-tools}}

\subsection{Varieties of type inference\label{type-inference-varieties}}

There are two different tasks that are commonly called ``type inference''.

\begin{enumerate}
\item
  Type inference during type-checking (Section~\ref{type-refinement}):
  During type-checking, if certain variables have no type qualifier, the
  type-checker determines whether there is some type qualifier that would
  permit the program to type-check.  If so, the type-checker uses that type
  qualifier, but never tells the programmer what it was.  Each time the
  type-checker runs, it re-infers the type qualifier for that variable.  If
  no type qualifier exists that permits the program to type-check, the
  type-checker issues a type warning.

  This variety of type inference is built into the Checker Framework.  Every
  checker can take advantage of it at no extra effort.  However, it only
  works within a method, not across method boundaries.

  Advantages of this variety of type inference include:
  \begin{itemize}
  \item
    If the type qualifier is obvious to the programmer, then omitting it
    can reduce annotation clutter in the program.
  \item
    The type inference can take advantage of only the code currently being
    compiled, rather than having to be correct for all possible calls.
    Additionally, if the code changes, then there is no old annotation to
    update.
  \end{itemize}


\item
  Type inference to annotate a program (Section~\ref{type-inference-to-annotate}):
  As a separate step before type-checking, a type inference tool takes the
  program as input, and outputs a set of type qualifiers that would
  type-check.  These qualifiers are inserted into the source code or the
  class file.  They can be viewed and adjusted by the programmer, and can
  be used by tools such as the type-checker.

  This variety of type inference must be provided by a separate tool.  It
  is not built into the Checker Framework.

  Advantages of this variety of type inference include:
  \begin{itemize}
  \item
    The program contains documentation in the form of type qualifiers,
    which can aid programmer understanding.
  \item
    Error messages may be more comprehensible.  With type inference
    during type-checking, error messages can be obscure, because the
    compiler has already inferred (possibly incorrect) types for a number
    of variables.
  \item
    A minor advantage is speed:  type-checking can be modular, which can be
    faster than re-doing type inference every time the
    program is type-checked.
  \end{itemize}

\end{enumerate}

Advantages of both varieties of inference include:
\begin{itemize}
\item
  Less work for the programmer.
\item
  The tool chooses the most general type, whereas a programmer might
  accidentally write a more specific, less generally-useful annotation.
\end{itemize}


Each variety of type inference has its place.  When using the Checker
Framework, type inference during type-checking is performed only
\emph{within} a method (Section~\ref{type-refinement}).  Every method
signature (arguments and return values) and field must have already been explicitly annotated,
either by the programmer or by a separate type-checking tool
(Section~\ref{type-inference-to-annotate}).
This approach enables modular checking (one class or method at a time) and
gives documentation benefits.
The programmer still has to
put in some effort, but much less than without inference:  typically, a
programmer does not have to write any qualifiers
inside the body of a method.


\subsection{Type inference to annotate a program\label{type-inference-to-annotate}}

This section lists tools that take a program and output a set of
annotations for it.

Section~\ref{nullness-inference} lists several tools that infer
annotations for the Nullness Checker.

Section~\ref{javari-inference} lists a tool that infers
annotations for the Javari Checker, which detects mutation errors.

\href{https://github.com/reprogrammer/cascade/}{Cascade}~\cite{VakilianPEJ2014}
is an Eclipse plugin that implements interactive type qualifier inference.
Cascade is interactive rather than fully-automated:  it makes it easier for
a developer to insert annotations.
Cascade starts with an unannotated program and runs a type-checker.  For each
warning it suggests multiple fixes, the developer chooses a fix, and
Cascade applies it.  Cascade works with any checker built on the Checker
Framework.
You can find installation instructions and a video tutorial at \url{https://github.com/reprogrammer/cascade}.


% LocalWords:  jsr plugin Warski xml buildfile tIDE java Awarns pom lifecycle
% LocalWords:  IntelliJ Maia newdir classpath Unconfuse nullness Gradle
% LocalWords:  compilerArgs Xbootclasspath JSR308 jsr308 jsr308javac mvn
%  LocalWords:  plugins proc procOnly DirectoryScanner setIncludes groupId
%  LocalWords:  setExcludes checkerFrameworkVersion javacParams javaParams
%%  LocalWords:  artifactId quals failOnError ejc CHECKERFRAMEWORK
%%  LocalWords:  javacheck checkerframework MavenExample org arg
%%  LocalWords:  AoutputArgsToFile qual jdk7 jdk8 annotatedJdk Unhandled
%%  LocalWords:  annotationProcessors annotationProcessor
%%  LocalWords:  typeAnnotationsJavac
