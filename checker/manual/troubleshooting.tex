\htmlhr
\chapter{Troubleshooting and getting help\label{troubleshooting}}

\begin{sloppypar}
The manual might already answer your question, so first please look for
your answer in the manual,
including this chapter and the FAQ (Chapter~\ref{faq}).
If not, you can use the mailing list,
\code{checker-framework-discuss@googlegroups.com}, to ask other users for
help.  For archives and to subscribe, see \url{https://groups.google.com/forum/#!forum/checker-framework-discuss}.
To report bugs, please see Section~\ref{reporting-bugs}.
If you want to help out, you can give feedback (including on the
documentation), choose a bug and fix it, or select a
project from the ideas list at
\url{https://github.com/typetools/checker-framework/wiki/Ideas}.
\end{sloppypar}


\section{Common problems and solutions\label{common-problems}}

\begin{itemize}
\item
To verify that you are using the compiler you think you are, you can add
\code{-version} to the command line.  For instance, instead of running
\code{javac -g MyFile.java}, you can run \code{javac \underline{-version} -g
  MyFile.java}.  Then, javac will print out its version number in addition
to doing its normal processing.

\end{itemize}



\subsection{Unable to run the checker, or checker crashes\label{common-problems-running}}

If you are unable to run the checker, or if the checker or the compiler
terminates with an error, then the problem may be a problem with your environment.
(If the checker or the compiler crashes, that is a bug in the Checker
Framework; please report it.  See Section~\ref{reporting-bugs}.)
This section describes some possible problems and solutions.

\begin{itemize}
\item
If you get the error

%BEGIN LATEX
\begin{smaller}
%END LATEX
\begin{Verbatim}
com.sun.tools.javac.code.Symbol$CompletionFailure: class file for com.sun.source.tree.Tree not found
\end{Verbatim}
%BEGIN LATEX
\end{smaller}
%END LATEX

\noindent
then you are using the source installation and file \code{tools.jar} is not
on your classpath.  See the installation instructions
(Section~\ref{installation}).

\item
If you get an error such as

\begin{Verbatim}
package org.checkerframework.checker.nullness.qual does not exist
\end{Verbatim}

  \noindent
\begin{sloppypar}
  despite no apparent use of \code{import org.checkerframework.checker.nullness.qual.*;} in
  the source code, then perhaps
  \code{jsr308\_imports} is set as a Java system property, a shell
  environment variable, or a command-line option.
  You should solve this by unsetting the variable/option, which it is deprecated.
\end{sloppypar}

If the error is

\begin{Verbatim}
package org.checkerframework.checker.nullness.qual does not exist
\end{Verbatim}

\begin{sloppypar}
\noindent
(note the extra apostrophe!), then you have probably misused quoting when
supplying the (deprecated) \code{jsr308\_imports} environment variable.
\end{sloppypar}

\item
If you get an error like one of the following,

%BEGIN LATEX
\begin{smaller}
%END LATEX
\begin{Verbatim}
...\build.xml:59: Error running ${env.CHECKERFRAMEWORK}\checker\bin\javac.bat compiler
\end{Verbatim}

\begin{Verbatim}
.../bin/javac: Command not found
\end{Verbatim}
%BEGIN LATEX
\end{smaller}
%END LATEX

\noindent
then the problem may be that you have not set the \code{CHECKERFRAMEWORK} environment
variable, as described in Section~\ref{javac-installation}.  Or, maybe
you made it a user variable instead of a system variable.

\item
If you get one of these errors:

\begin{alltt}
The hierarchy of the type \emph{ClassName} is inconsistent

The type com.sun.source.util.AbstractTypeProcessor cannot be resolved.
  It is indirectly referenced from required .class files
\end{alltt}

\begin{sloppypar}
\noindent
then you are likely \textbf{not} using the Checker Framework compiler.  Use
either \code{\$CHECKERFRAMEWORK/checker/bin/javac} one of the alternatives
described in Section~\ref{installation}.
\end{sloppypar}


\item
If you get the error

\begin{Verbatim}
  java.lang.ArrayStoreException: sun.reflect.annotation.TypeNotPresentExceptionProxy
\end{Verbatim}

\noindent
If you get an error such as

\begin{Verbatim}
  java.lang.NoClassDefFoundError: java/util/Objects
\end{Verbatim}

\noindent
then you are trying to run the compiler using a JDK 6 or earlier JVM\@.
Install and use a Java 7 or 8 JDK, at least for running the Checker
Framework.

\noindent
% I'm not 100% sure of the following explanation and solution.
then an annotation is not present at run time that was present at compile
time.  For example, maybe when you compiled the code, the \<@Nullable>
annotation was available, but it was not available at run time.
You can use JDK 8 at run time, or compile
with a Java 6 or 7 compiler that will ignore the annotations in comments.

\item
A ``class file for \ldots\ not found'' error, especially for an inner class
in the JDK, is probably due to a JDK version mismatch.

In general, Java issues a ``class file for \ldots\ not found'' error when
your classpath contains code that was compiled
with some library, but your classpath does not contain that library itself.

For example, suppose that when you run the compiler, you are using JDK 8,
but some library on your classpath was compiled against JDK 6 or 7, and the
compiled library refers to a class that only appears in JDK 6 or 7.  (If only
one version of Java existed, or the Checker Framework didn't try to support
multiple different versions of Java, this would not be a problem.)

Examples of classes that were in JDK 7 but were removed in JDK 8 include:

\begin{Verbatim}
  class file for java.util.TimeZone$DisplayNames not found
\end{Verbatim}

Examples of classes that were in JDK 6 but were removed in JDK 7 include:

\begin{Verbatim}
  class file for java.io.File$LazyInitialization not found
  class file for java.util.Hashtable$EmptyIterator not found
  java.lang.NoClassDefFoundError: java/util/Hashtable$EmptyEnumerator
\end{Verbatim}

Examples of classes that were not in JDK 7 but were introduced in JDK 8 include:

\begin{Verbatim}
  The type java.lang.Class$ReflectionData cannot be resolved
\end{Verbatim}

Examples of classes that were not in JDK 6 but were introduced in JDK 7 include:

\begin{Verbatim}
  class file for java.util.Vector$Itr not found
\end{Verbatim}

There are even classes that were introduced within a single JDK release.
Classes that appear in JDK 7 release 71 but not in JDK 7 release 45 include:

\begin{Verbatim}
  class file for java.lang.Class$ReflectionData not found
\end{Verbatim}

You may be able to solve the problem by running

\begin{Verbatim}
  cd checker
  ant jdk.jar bindist
\end{Verbatim}

\noindent
to re-generate files \code{checker/jdk/jdk\{7,8\}.jar} and \code{checker/bin/jdk\{7,8\}.jar}.

That usually works, but if not, then you should recompile the Checker
Framework from source rather than using the pre-compiled distribution.


\item
A \<NoSuchFieldError> such as this:

\begin{Verbatim}
java.lang.NoSuchFieldError: NATIVE_HEADER_OUTPUT
\end{Verbatim}

\noindent
Field \<NATIVE\_HEADER\_OUTPUT> was added in JDK 8.
The error message suggests that
you're not executing with the right bootclasspath: some classes were
compiled with the JDK 8 version and expect the field, but you're
executing the compiler on a JDK without the field.

One possibility is that you are not running the Checker Framework compiler
--- use \<javac -version> to check this, then use the right one.  (Maybe
the Checker Framework javac is at the end rather than the beginning of your
path.)

If you are using Ant, then one possibility
is that the javac compiler is using the same JDK as Ant is using.  You can
correct this by being sure to use \<fork="yes"> (see
Section~\ref{ant-task}) and/or setting the \<build.compiler> property to
\<extJavac>.

If you are building from source, you might need to rebuild the Annotation
File Utilities before recompiling or using the Checker Framework.


\item
If you get an error that contains lines like these:

\begin{Verbatim}
Caused by: java.util.zip.ZipException: error in opening zip file
    at java.util.zip.ZipFile.open(Native Method)
    at java.util.zip.ZipFile.<init>(ZipFile.java:131)
\end{Verbatim}

\noindent
then one possibility is that you have installed the Checker Framework in a
directory that contains special characters that Java's ZipFile
implementation cannot handle.  For instance, if the directory name contains
``\<+>'', then Java 1.6 throws a ZipException, and Java 1.7 throws a
FileNotFoundException and prints out the directory name with ``\<+>''
replaced by blanks.

\item
If you get an error

\begin{Verbatim}
error: scoping construct for static nested type cannot be annotated
\end{Verbatim}

\noindent
\begin{sloppypar}
then you have probably written something like \<@Nullable java.util.List>.
The correct syntax is \<java.util.@Nullable List>.  But, it's usually
better to add \<import java.util.List> to your source file, so that you can
just write \<@Nullable List>.  Likewise, you must write \<Outer.@Nullable
StaticNestedClass> rather than \<@Nullable Outer.StaticNestedClass>.
\end{sloppypar}

Java 8 requires that a type qualifier be written directly on the type that
it qualifies, rather than on a scoping mechanism that assists in resolving
the name.  Examples of scoping mechanisms are package names and outer
classes of static nested classes.

The reason for the Java 8 syntax is to avoid syntactic irregularity.  When
writing a member nested class (also known as an inner class), it is
possible to write annotations on both the outer and the inner class:  \<@A1
Outer. @A2 Inner>.  Therefore, when writing a static nested class, the
annotations should go on the same place:  \<Outer. @A3 StaticNested> (rather
than \<@ConfusingAnnotation Outer. Nested> where
\<@ConfusingAnnotation> applies to \<Outer> if \<Nested> is a member class
and applies to \<Nested> if \<Nested> is a static class).  It's not legal
to write an annotation on the outer class of a static nested class, because
neither annotations nor instantiations of the outer class affect the static
nested class.

Similar arguments apply when annotating \<package.Outer.Nested>.

\end{itemize}


\subsection{Unexpected type-checking results\label{common-problems-typechecking}}

This section describes possible problems that can lead the type-checker to
give unexpected results.


\begin{itemize}
\item
  If the Checker Framework is unable to verify a property that you know is
  true, then it is helpful to formulate an argument about why the property
  is true.  Recall that the Checker Framework does modular verification,
  one procedure at a time; it observes the specifications, but not the
  implementations, of other methods.

  If any aspects of your argument are not expressed as annotations, then
  you may need to write more annotations.  If any aspects of your argument
  are not expressible as annotations, then you may need to extend the
  type-checker.

\item
If a checker seems to be ignoring the annotation on a method, then it is
possible that the checker is reading the method's signature from its
\code{.class} file, but the \code{.class} file was not created by the JSR
308 compiler.  You can check whether the annotations actually appear in the
\code{.class} file by using the \code{javap} tool.

If the annotations do not appear in the \code{.class} file, here are two
ways to solve the problem:
\begin{itemize}
\item
  Re-compile the method's class with the Checker Framework compiler.  This will
  ensure that the type annotations are written to the class file, even if
  no type-checking happens during that execution.
\item
  Pass the method's file explicitly on the command line when type-checking,
  so that the compiler reads its source code instead of its \code{.class}
  file.
\end{itemize}

\item
If a checker issues a warning about a property that it accepted (or that
was checked) on a previous line, then probably there was a side-effecting
method call in between that could invalidate the property.  For example, in
this code:

\begin{Verbatim}
if (currentOutgoing != null && !message.isCompleted()) {
    currentOutgoing.continueBuffering(message);
}
\end{Verbatim}

\noindent
the Nullness Checker will issue a warning on the second line:
\begin{Verbatim}
warning: [dereference.of.nullable] dereference of possibly-null reference currentOutgoing
    currentOutgoing.continueBuffering(message);
    ^
\end{Verbatim}

If \<currentOutgoing> is a field rather than a local variable, and
\<isCompleted()> is not a pure method, then a null pointer
dereference can occur at the given location, because \<isCompleted()> might set
the field \<currentOutgoing> to \<null>.

If you want to communicate that
isCompleted() does not set the field \<currentOutgoing> to \<null>, you can use
\<\refqualclass{dataflow/qual}{Pure}>,
\<\refqualclass{dataflow/qual}{SideEffectFree}>,
or \<\refqualclass{checker/nullness/qual}{EnsuresNonNull}> on the
declaration of \<isCompleted()>; see Sections~\ref{type-refinement-purity}
and~\ref{nullness-method-annotations}.


\item
If a checker issues a type-checking error for a call that the library's
documentation states is correct, then maybe that library method has not yet
been annotated, so default annotations are being used.

To solve the problem, add the missing annotations to the library (see
Chapter~\ref{annotating-libraries}).  Depending on the checker, the
annotations might be expressed in the form of stub files (which appear
together with the checker's source code, such as in file
\code{checker/src/org/checkerframework/checker/interning/jdk.astub} for the
Interning Checker) or in the form of annotated libraries (which appear
under \code{checker/jdk/}, such as at \code{checker/jdk/nullness/src/} for
the Nullness Checker.

\item
If the compiler reports that it cannot find a method from the JDK or
another external library, then maybe the stub/skeleton file for that class
is incomplete.

To solve the problem, add the missing annotations to the library, as
described in the previous item.

The error might take one of these forms:

\begin{Verbatim}
method sleep in class Thread cannot be applied to given types
cannot find symbol: constructor StringBuffer(StringBuffer)
\end{Verbatim}

\item
If you get an error related to a bounded type parameter and a literal such
as \<null>, the problem may be missing defaulting.  Here is an example:

\begin{Verbatim}
mypackage/MyClass.java:2044: warning: incompatible types in assignment.
      T retval = null;
                 ^
  found   : null
  required: T extends @MyQualifier Object
\end{Verbatim}

\noindent
A value that can be assigned to a variable of type \<T extends @MyQualifier
Object> only if that value is of the bottom type, since the bottom type is
the only one that is a subtype of every subtype of \<T extends @MyQualifier
Object>.  The value \<null> satisfies this for the Java type system, and it
must be made to satisfy it for the pluggable type system as well.  The
typical way to address this is to write the meta-annotation
\<@ImplicitFor(trees={Tree.Kind.NULL\_LITERAL})> on the definition of the
bottom type qualifier.

\item
An error such as

\begin{Verbatim}
MyFile.java:123: error: incompatible types in argument.
                        myModel.addElement("Scanning directories...");
                                           ^
  found   : String
  required: ? extends Object
\end{Verbatim}

\noindent
may stem from use of raw types.  (``\<String>'' might be a different type
and might have type annotations.)  If your declaration was

\begin{Verbatim}
  DefaultListModel myModel;
\end{Verbatim}

\noindent
then it should be
\begin{Verbatim}
  DefaultListModel<String> myModel;
\end{Verbatim}

Running the regular Java compiler with the \<-Xlint:unchecked> command-line
option will help you to find and fix problems such as raw types.


\item
The error

\begin{Verbatim}
error: annotation type not applicable to this kind of declaration
    ... List<@NonNull String> ...
\end{Verbatim}

\noindent
indicates that you are using a definition of \<@NonNull> that is a
declaration annotation, which cannot be used in that syntactic location.
For example, many legacy annotations such as those listed in
Figure~\ref{fig-nullness-refactoring} are declaration annotations.  You can
fix the problem by instead using a definition of \<@NonNull> that is a type
annotation, such as the Checker Framework's annotations; often this only
requires changing an \<import> statement.
Alternately, if you wish to continue using the legacy annotations in
declaration locations, see Section~\ref{declaration-annotations-for-java7}.


\item
This compile-time error

\begin{Verbatim}
  unknown enum constant java.lang.annotation.ElementType.TYPE_USE
\end{Verbatim}

\noindent
indicates that you are compiling using a Java 6 or 7 JDK, but your code
references an enum constant that is only defined in the Java 8 JDK.  The
problem might be that your code uses a library that references the enum
constant.  In particular, the type annotations shipped with the Checker
Framework reference
\sunjavadoc{java/lang/annotation/ElementType.html\#TYPE\_USE}{ElementType.TYPE\_USE}.  You can use the Checker
Framework, but still compile and run your code in a Java 6 or 7 JVM, by
following the instructions in Section~\ref{backward-compatibility}.

If you ignore the error and run your code in a Java 6 or 7 JVM, then you will get a run-time error:

\begin{Verbatim}
  java.lang.ArrayStoreException: sun.reflect.annotation.EnumConstantNotPresentExceptionProxy
\end{Verbatim}


\item
If Eclipse gives the warning

\begin{Verbatim}
The annotation @NonNull is disallowed for this location
\end{Verbatim}

\noindent
then you have the wrong version of the \<org.eclipse.jdt.annotation>
classes.  Eclipse includes two incompatible versions of these annotations.
You want the one with a name like
\<org.eclipse.jdt.annotation\_2.0.0.....jar>, which you can find in the
\<plugins> subdirectory under the Eclipse installation directory.
Add this .jar file to your build path.


\end{itemize}


\subsection{Unable to build the checker, or to run programs\label{common-problems-running-java}}

An error like this

\begin{Verbatim}
Unsupported major.minor version 52.0
\end{Verbatim}

means that you have compiled some files into the Java 8 format (version
52.0), but you are trying to run them with Java 7 or earlier.
Likewise, ``Unsupported major.minor version 51.0''
means that you have compiled some files into the Java 7 format (version
51.0), but you are trying to run them with Java 6 or earlier.
Here are ways to solve the problem:

\begin{itemize}
\item
Use a newer JVM (run \<java -version> to
determine the version you are using)
\item
Use the Checker Framework to type-check your code, then afterword produce a
classfile that targets an earlier JVM by supplying arguments such as
\<javac -source 7 -target 7 ...>.
\end{itemize}


\subsection{Classfile version warning\label{common-problems-classfile-version}}

The following warning is innocuous and you can ignore it, or you can
suppress it using the \<-Xlint:-classfile> command-line argument to javac:

%BEGIN LATEX
\begin{smaller}
%END LATEX
\begin{Verbatim}
warning: [classfile] RuntimeVisibleTypeAnnotations attribute introduced in version 52.0 class files is ignored in version 51.0 class files
\end{Verbatim}
%BEGIN LATEX
\end{smaller}
%END LATEX

This warning results when you compile a library using the Checker Framework
compiler, then use a normal Java compiler to compile client code that uses
the library.
The Checker Framework compiler puts Java 8 type annotations even in Java 7
classfiles, for the benefit of modular typechecking.  The Checker Framework
compiler reads these annotations in Java 7, and other compilers ignore them.


\section{How to report problems (bug reporting)\label{reporting-bugs}}

If you have a problem with any checker, or with the Checker Framework,
please file a bug at
\url{https://github.com/typetools/checker-framework/issues}.
(First, check whether there is an existing bug report for that issue.)

Alternately (especially if your communication is not a bug report), you can
send mail to checker-framework-dev@googlegroups.com.
We welcome suggestions, annotated libraries, bug fixes, new
features, new checker plugins, and other improvements.

Please ensure that your bug report is clear and that it is complete.
Otherwise, we may be unable to understand it or to reproduce it, either of
which would prevent us from fixing the bug.  Your bug report will be most
helpful if you:

\begin{itemize}
\item
  Add \code{-version -verbose -AprintErrorStack -AprintAllQualifiers} to the javac options.  This causes the compiler to output
  debugging information, including its version number.
\item
  Indicate exactly what you did.  Don't skip any steps, and don't merely
  describe your actions in words.  Show the exact commands by attaching a
  file or using cut-and-paste from your command shell (a screenshot is not
  as useful).
\item
  Include all files that are necessary to reproduce the problem.  This
  includes every file that is used by any of the commands you reported, and
  possibly other files as well.  Please attach the files, rather than
  pasting their contents into the body of your bug report or email message,
  because some mailers mangle formatting of pasted text.
\item
  Indicate exactly what the result was by attaching a file or using
  cut-and-paste from your command shell (don't merely describe it in
  words).
\item
  Indicate what you expected the result to be, since a bug is a difference
  between desired and actual outcomes.  Also, please indicate \textbf{why}
  you expected that result --- explaining your reasoning can help you
  understand how your reasoning is different than the checker's and which
  one is wrong.  Remember that the checker reasons modularly and
  intraprocedurally:  it examines one method at a time, using only the
  method signatures of other methods.
\item
  Indicate what you have already done to try to understand the problem.
  Did you do any additional experiments?  What parts of the manual did
  read, and what else did you search for in the manual?  This information
  will prevent you being given redundant suggestions.
\end{itemize}

A particularly useful format for a test case is as a new file, or a diff to
an existing file, for the existing Checker Framework test suite.  For
instance, for the Nullness
Checker, see directory \<checker-framework/checker/tests/nullness/>.
But, please report your bug even if you do not report it in this format.


\section{Building from source\label{build-source}}

The Checker Framework release (Section~\ref{installation}) contains
everything that most users need, both to use the distributed checkers and
to write your own checkers.  This section describes how to compile its
binaries from source.  You will be using the latest development version of
the Checker Framework, rather than an official release.

% Doing
% so permits you to examine and modify the implementation of the distributed
% checkers and of the checker framework.  It may also help you to debug
% problems more effectively.


\subsection{Obtain the source\label{building-obtain-source}}

Obtain the latest source code from the version control repository:

\begin{Verbatim}
export JSR308=$HOME/jsr308
mkdir -p $JSR308
cd $JSR308
hg clone https://bitbucket.org/typetools/jsr308-langtools jsr308-langtools
git clone https://github.com/typetools/checker-framework.git checker-framework
git clone https://github.com/typetools/annotation-tools.git annotation-tools
\end{Verbatim}
% $ to unconfuse Emacs LaTeX mode


\subsection{Build the Type Annotations compiler\label{building-compiler}}

The Checker Framework compiler is built upon a compiler called the Type
Annotations compiler.  The Type Annotations compiler is a variant of the
OpenJDK javac that supports annotations in comments.  The Checker Framework
compiler is a small wrapper around the Type Annotations compiler, which
adds annotated JDKs and the Checker Framework jars to the classpath.


\begin{enumerate}
\item
% Why is this necessary?  What goes wrong if it is not set?  Can I avoid
% the need to set it?  It's used for:
%  * the location of tools.jar, below.
%  * the default location of RTJAR, in checker/jdk/Makefile.
Set the \<JAVA\_HOME> environment variable to the location of your JDK
7 or 8 installation (not the JRE installation, and not JDK 6 or earlier).
This needs to be an Oracle JDK.
(The \<JAVA\_HOME> environment
variable might already be set, because it is needed for Ant to work.)

In the bash shell, the following command \emph{sometimes} works (it might
not because \<java> might be the version in the JDK or in the JRE):
% Can someone give a simpler command?
%BEGIN LATEX
\begin{smaller}
%END LATEX
\begin{Verbatim}
  export JAVA_HOME=${JAVA_HOME:-$(dirname $(dirname $(dirname $(readlink -f $(/usr/bin/which java)))))}
\end{Verbatim}
%BEGIN LATEX
\end{smaller}
%END LATEX

\item
Compile the Type Annotations tools:

\begin{Verbatim}
  cd $JSR308/jsr308-langtools/make
  ant clean-and-build-all-tools
\end{Verbatim}
% $ to unconfuse Emacs LaTeX mode

\item
 Add the \<jsr308-langtools/dist/bin> directory to the front of your PATH environment variable.
  Example command:

\begin{Verbatim}
  export PATH=$JSR308/jsr308-langtools/dist/bin:${PATH}
\end{Verbatim}

\end{enumerate}

% Java 8 extends the Java language to permit annotations to appear on types,
% as in \code{List<@NonNull String>} (see Section~\ref{writing-annotations}).
% This change will be part of the Java 8 language.  We recommend that you
% write annotations in comments, as in \code{List</*@NonNull*/ String>} (see
% Section~\ref{annotations-in-comments}).  The Checker Framework compiler still reads
% such annotations, but this syntax permits you to use a compiler other than
% the Checker Framework compiler.  For example, you can compile your code with a Java 5
% compiler, and you can use a checker as an external tool in an IDE.



\subsection{Build the Annotation File Utilities\label{afu-building}}

This is simply done by:

\begin{Verbatim}
  cd $JSR308/annotation-tools
  ant
\end{Verbatim}

You do not need to add the Annotation File Utilities to the path, as
the Checker Framework build finds it using relative paths.


\subsection{Build the Checker Framework\label{building}}

% Building (compiling) the checkers and framework from source creates the
% \code{checker.jar} file.  A pre-compiled \code{checker.jar} is included
% in the distribution, so building it is optional.  It is mostly useful for
% people who are developing compiler plug-ins (type-checkers).  If you only
% want to \emph{use} the compiler and existing plug-ins, it is sufficient to
% use the pre-compiled version.

\begin{enumerate}
% \item
% Edit \code{checker/build.properties} file so that the
% \code{compiler.lib} property specifies the location of the Checker
% Framework
% \code{javac.jar} library.  (If you also installed the JSR 308 compiler from
% source, and you made the \code{checker-framework} and \code{jsr308-langtools} directories
% siblings, then you don't need to edit \code{checker/build.properties}.)

\item
Run \code{ant} to build the Checker Framework:

\begin{Verbatim}
  cd $JSR308/checker-framework/checker
  ant
\end{Verbatim}
% $ to unconfuse Emacs LaTeX mode

\item
Once it is built, you may wish to put the Checker Framework's \<javac>
even earlier in your \<PATH>:

\begin{Verbatim}
export PATH=$JSR308/checker-framework/checker/bin:$JSR308/jsr308-langtools/dist/bin:${PATH}
\end{Verbatim}

The Checker Framework's \code{javac} ensures that all required
libraries are on your classpath and boot classpath, but is otherwise
identical to the Type Annotations compiler.

Putting the Checker Framework's \<javac> earlier in your \<PATH> will
ensure that the Checker Framework's version is used.

\item
If you are developing a checker within the Checker Framework, there is
a developer version of \<javac> in the \<bin-devel> directory. This
version will use compiled classes from \<dataflow/build>,
\<javacutil/build>, \<stubparser/build>, \<framework/build>,
and \<checker/build> in the \<checker-framework> directory instead of
the compiled jar files, and by default will print stack traces for all
errors.

To use it, set your \<PATH> to use \<javac> in the \<bin-devel> directory:

\begin{Verbatim}
export PATH=$JSR308/checker-framework/checker/bin-devel:$JSR308/jsr308-langtools/dist/bin:${PATH}
\end{Verbatim}

The developer version of \<javac> allows you to not have to rebuild
the jar files after every code change, in turn allowing you to test
your changes faster. Source files can be compiled using command \<ant
build> in the \<checker> directory, or can be automatically compiled by
an IDE such as Eclipse.

\item Test that everything works:

  \begin{itemize}

  \item Run \code{ant all-tests} in the \code{checker} directory:
\begin{Verbatim}
  cd $JSR308/checker-framework/checker
  ant all-tests
\end{Verbatim}
% $ to unconfuse Emacs LaTeX mode

  \item Run the Nullness Checker examples (see
    Section~\refwithpage{nullness-example}).

  \end{itemize}

\end{enumerate}

\subsection{Build the Checker Framework Manual (this document)\label{building-manual}}

\begin{enumerate}
\item
To build the manual you will need
{\hevea} (\myurl{http://hevea.inria.fr/}) installed, and either
\<rsvg-convert> or \<convert>.  If you use the Ubuntu operating system, do
\begin{Verbatim}
sudo apt-get install hevea librsvg2-bin
\end{Verbatim}

\item
Run \code{make} in the \code{checker/manual} directory to build both the PDF and HTML versions of the manual.
\end{enumerate}


\section{Publications\label{publications}\label{learning-more}}

Here are two technical papers about the Checker Framework itself:

\begin{itemize}
\item
``Practical pluggable types for Java''~\cite{PapiACPE2008}
(ISSTA 2008, \myurl{http://homes.cs.washington.edu/~mernst/pubs/pluggable-checkers-issta2008.pdf})
describes the design and implementation of the Checker Framework.
The paper also describes case
studies in which the Nullness, Interning, Javari, and IGJ Checkers found
previously-unknown errors in real software.
The case studies also yielded new insights about type systems.

\item
``Building and using pluggable
type-checkers''~\cite{DietlDEMS2011}
(ICSE 2011, \myurl{http://homes.cs.washington.edu/~mernst/pubs/pluggable-checkers-icse2011.pdf})
discusses further experience with the Checker Framework, increasing the
number of lines of verified code to 3 million.  The case studies are of the
Fake Enum, Signature String, Interning, and Nullness Checkers.
The paper also evaluates the ease
of pluggable type-checking with the Checker Framework:  type-checkers
were easy to write, easy for novices to use, and effective in finding
errors.
\end{itemize}

Here are some papers about type systems that were implemented and evaluated
using the Checker Framework:

\begin{description}
\item[Nullness (Chapter~\ref{nullness-checker})]
See the two papers about the Checker Framework, described above.

\item[Rawness initialization (Section~\ref{initialization-rawness-checker})]
``Inference of field initialization'' (ICSE 2011, \myurl{http://homes.cs.washington.edu/~mernst/pubs/initialization-icse2011-abstract.html})
describes inference for the Rawness Initialization Checker.

\item[Interning (Chapter~\ref{interning-checker})]
See the two papers about the Checker Framework, described above.

\item[Fake enumerations (Chapter~\ref{fenum-checker})]
See the ICSE 2011 paper about the Checker Framework, described above.

\item[Regular expressions (Chapter~\ref{regex-checker})]
``A type system for regular expressions''~\cite{SpishakDE2012} (FTfJP 2012, \myurl{http://homes.cs.washington.edu/~mernst/pubs/regex-types-ftfjp2012-abstract.html})
            describes the Regex Checker.

\item[Format Strings (Chapter~\ref{formatter-checker})]
``A type system for format strings''~\cite{WeitzKSE2014} (ISSTA 2014, \myurl{http://homes.cs.washington.edu/~mernst/pubs/format-string-issta2014-abstract.html})
            describes the Format String Checker.

\item[Signature strings (Chapter~\ref{signature-checker})]
See the ICSE 2011 paper about the Checker Framework, described above.

\item[GUI Effects (Chapter~\ref{guieffect-checker})]
``JavaUI: Effects for controlling UI object access''~\cite{GordonDEG2013} (ECOOP 2013, \myurl{http://homes.cs.washington.edu/~mernst/pubs/gui-thread-ecoop2013-abstract.html})
            describes the GUI Effect Checker.

\item
``Verification games: Making verification fun'' (FTfJP 2012, \myurl{http://homes.cs.washington.edu/~mernst/pubs/verigames-ftfjp2012-abstract.html})
            describes a general inference approach that, at the time, had only been implemented for the Nullness Checker (Section~\ref{nullness-checker}).

\item[IGJ and OIGJ immutability (Chapter~\ref{igj-checker})]
``Object and reference immutability using Java generics''~\cite{ZibinPAAKE2007} (ESEC/FSE 2007, \myurl{http://homes.cs.washington.edu/~mernst/pubs/immutability-generics-fse2007-abstract.html})
and
``Ownership and immutability in generic Java''~\cite{ZibinPLAE2010} (OOPSLA 2010, \myurl{http://homes.cs.washington.edu/~mernst/pubs/ownership-immutability-oopsla2010-abstract.html})
            describe the IGJ and OIGJ immutability type systems.
For further case studies, also see the ISSTA 2008 paper about the Checker
Framework, described above.

\item[Javari immutability (Chapter~\ref{javari-checker})]
``Javari: Adding reference immutability to Java''~\cite{TschantzE2005} (OOPSLA 2005, \myurl{http://homes.cs.washington.edu/~mernst/pubs/ref-immutability-oopsla2005-abstract.html})
            describes the Javari type system.
For inference, see
``Inference of reference immutability''~\cite{QuinonezTE2008} (ECOOP 2008, \myurl{http://homes.cs.washington.edu/~mernst/pubs/infer-refimmutability-ecoop2008-abstract.html})
and
``Parameter reference immutability: Formal definition, inference tool, and comparison''~\cite{ArtziQKE2009} (J.ASE 2009, \myurl{http://homes.cs.washington.edu/~mernst/pubs/mutability-jase2009-abstract.html}).
For further case studies, also see the ISSTA 2008 paper about the Checker
Framework, described above.

\item[Thread locality (Section~\ref{loci-thread-locality-checker})]
``Loci: Simple thread-locality for Java''~\cite{WrigstadPMZV2009} (ECOOP 2009,
\myurl{http://janvitek.org/pubs/ecoop09.pdf})

\item[\href{https://ece.uwaterloo.ca/~wdietl/ownership/}{Generic Universe
    Types} (Section~\ref{gut-checker})]
``Tunable static inference for Generic Universe Types'' (ECOOP 2011, \myurl{http://homes.cs.washington.edu/~mernst/pubs/tunable-typeinf-ecoop2011-abstract.html})
            describes inference for the Generic Universe Types type system.

Another implementation of Universe Types and \href{http://www.cs.rpi.edu/~huangw5/cf-inference/}{ownership types} is  described in
``Inference and checking of object ownership''~\cite{HuangDME2012} (ECOOP 2012, \myurl{http://homes.cs.washington.edu/~mernst/pubs/infer-ownership-ecoop2012-abstract.html}).

\item[Approximate data (Section~\ref{enerj-checker})]
``EnerJ: Approximate Data Types for Safe and General Low-Power Computation''~\cite{SampsonDFGCG2011} (PLDI 2011, \myurl{http://adriansampson.net/media/papers/enerj-pldi2011.pdf})

\item[Information flow and tainting (Section~\ref{sparta-checker})]
``Collaborative Verification of Information Flow
for a High-Assurance App Store''~\cite{ErnstJMDPRKBBHVW2014} (CCS 2014, \myurl{http://homes.cs.washington.edu/~mernst/pubs/infoflow-ccs2014.pdf}) describes the SPARTA information flow type system.

\item[ReIm immutability]
% TODO:  (Section~\ref{reim-checker})
``ReIm \& ReImInfer: Checking and inference of reference immutability and method purity''~\cite{HuangMDE2012} (OOPSLA 2012, \myurl{http://homes.cs.washington.edu/~mernst/pubs/infer-refimmutability-oopsla2012-abstract.html})
            describes the ReIm immutability type system.

\end{description}

In addition to these papers that discuss use the Checker Framework
directly, other academic papers use the Checker Framework in their
implementation or evaluation.
Most educational use of the Checker
Framework is never published, and most commercial use of the Checker
Framework is never discussed publicly.

(If you know of a paper or other use that is not listed here, please inform
the Checker Framework developers so we can add it.)


\section{Comparison to other tools\label{other-tools}}

A pluggable type-checker, such as those created by the Checker Framework,
aims to help you prevent or detect all errors of a given variety.  An
alternate approach is to use a bug detector such as
\href{http://findbugs.sourceforge.net/}{FindBugs},
\href{http://jlint.sourceforge.net/}{Jlint}, or
\href{http://pmd.sourceforge.net/}{PMD}.

A pluggable type-checker
differs from a bug detector in several ways:
\begin{itemize}
\item
  A type-checker aims to find \emph{all} errors.  Thus, it can verify the
  \emph{absence} of errors:  if the type-checker says there are no null
  pointer errors in your code, then there are none.  (This guarantee only
  holds for the code it checks, of course; see
  Section~\ref{checker-guarantees}.)

  A bug detector aims to find \emph{some} of the most obvious errors.  Even
  if it reports no errors, then there may still be errors in your code.

  Both types of tools may issue false alarms,
  also known as false positive warnings;
  see Section~\ref{suppressing-warnings}.

\item
  A type-checker requires you to annotate your code with type qualifiers,
  or to run an inference tool that does so for you.  A bug detector may not
  require annotations.  This means that it may be easier to get started
  running a bug detector.

\item
  A type-checker may use a more sophisticated and complete analysis.
  A bug detector typically does a more lightweight analysis, coupled with
  heuristics to suppress false positives.

  As one example, a type-checker can take advantage of annotations on
  generic type parameters, such as \code{List<@NonNull String>}, permitting
  it to be much more precise for code that uses generics.

\end{itemize}

A case study~\cite[\S6]{PapiACPE2008} compared the Checker Framework's nullness
checker with those of FindBugs, Jlint, and PMD\@.  The case study was on a
well-tested program in daily use.  The Checker Framework tool found 8
nullness errors (that is, null pointer dereferences).  None of the other
tools found any errors.

Also see the
\href{http://types.cs.washington.edu/jsr308/}{JSR 308}~\cite{JSR308-2008-09-12}
documentation for a detailed discussion of related work.



\section{Credits, changelog, and license\label{credits}}

The key developers of the Checker Framework are Mahmood Ali, Telmo Correa,
Werner M. Dietl, Michael D. Ernst, and Matthew M. Papi.
Many other developers have also contributed, for example by writing
the checkers that are distributed with the Checker Framework.
Many, many users to list have provided valuable feedback, for which we are
grateful.

%% Not so accurate, since Mahmood is really an author of the nullness and
%% interned checkers too, and since then they have been essentially
%% rewritten from scratch.
% The Checker Framework was implemented by
% The Nullness Checker was implemented by Matthew M. Papi.
% The Interning Checker was implemented by Matthew M. Papi.
% The Javari Checker was implemented by Telmo Correa.
% The IGJ Checker was implemented by Mahmood Ali.
% The Subtyping Checker was implemented by Matthew M. Papi.
% The Fake Enum Checker was written by Werner M. Dietl.
% ... many others ...

Differences from previous versions of the checkers and framework can be found
in the \code{changelog.txt} file.  This file is included in the
Checker Framework distribution and is also available on the web at
\myurl{http://types.cs.washington.edu/checker-framework/current/changelog.txt}.

Two different licenses apply to different parts of the Checker Framework.
\begin{itemize}
\item
The Checker Framework itself is licensed under the GNU General Public License
(GPL), version 2.  The GPL is the same license that OpenJDK is licensed
under.  That means that type-checking your code using the Checker Framework
is no more dangerous (from an intellectual property point of view) than
compiling your code using javac.
\item
The more permissive MIT License applies
to code that you might want to include in your own
program, such as the annotations.
\end{itemize}
\noindent
For details, see file \<LICENSE.txt>.



% LocalWords:  jsr unsetting plugins langtools zipfile cp plugin Nullness txt
% LocalWords:  nullness classpath NonNull MyObject javac uref changelog MyEnum
% LocalWords:  subtyping containsKey proc classfiles SourceChecker javap jdk
% LocalWords:  MyFile buildfiles ClassName JRE java jsr308 bootclasspath
%  LocalWords:  extJavac ZipFile AprintErrorStack AprintAllQualifiers Jlint
%  LocalWords:  Telmo Correa Papi NoSuchFieldError ZipException Xlint A1
%  LocalWords:  FileNotFoundException MyQualifier ImplicitFor A2 A3 JDKs
%  LocalWords:  StaticNestedClass StaticNested ConfusingAnnotation
%%  LocalWords:  CHECKERFRAMEWORK currentOutgoing isCompleted
%%  LocalWords:  EnsuresNonNull
