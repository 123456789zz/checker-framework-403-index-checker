\htmlhr
\chapter{Value Checker\label{value-checker}}

The Value Checker attempts to track the values of variables in a
program at any given point in the execution. This checker is intended
primarily as a tool for other checkers to use; the Reflection Checker,
for example, makes use of the Value Checker extensively for attempting
to determine the target of reflective calls, see
Section~\ref{reflection-checker}.

By itself, the Value Checker does not provide much useful
functionality unless a programmer wants to attempt to enforce specific
constraints on the values of a variable, such as indicating a certain
int should always be either 0 or 1.

To run the Value Checker, supply the -processor
checkers.value.ValueChecker command-line option to javac.

\section{Value Annotations}

The Value Checker uses annotations to track values for all Java
primitives, their wrappers (e.g. int, and Integer), Strings, and the
length field of arrays. The types covered correspond to the names of
the annotations as shown in Figure~\ref{fig-value-hierarchy}

\begin{figure}
\includeimage{value-hierarchy}{9cm}
\caption{The overall annotation hierarchy of the Value
annotations. Dashed lines indicate partial subtyping relationships
based on the contents of the value arrays (explained further in
Figure~\ref{fig-value-subtyping}. Qualifiers in gray are used
internally by the type system but should never be written by a
programmer.}
\label{fig-value-hierarchy}
\end{figure}

Each annotation has an array of values representing all the values the
annotated type could be. This array is limited to 10 entries for
performance reasons. Therefore, if a variable, due to program logic,
could be more than 10 different values at any given point the Value
Checker gives up and it is annotated as @UnknownVal instead.

In addition to the hierarchy shown above, an annotation is also
considered a subtype of another annotation if its value array is a
subset of the value array of the other annotation (see
Figure~\ref{fig-value-subtype}. This allows the possible values of
variables to continue to be tracked even when code branches cannot be
predicted at compile-time as in Figure~\ref{fig-value-multivalue}.

\begin{figure}
\includeimage{value-subtype}{2.5cm}
\caption{Subtyping relationship between value annotations of the same type.}
\label{fig-value-subtype}
\end{figure}

\begin{figure}
\begin{Verbatim}
public void foo(boolean b) {
   int i = 1;     // i is @IntVal({1})
   if (b) {  
      i = 2;      // i is now @IntVal({2})
   }        
   // i is now @IntVal({1,2})
   
   i = i + 1;     // i is now @IntVal({2,3})
}
\end{Verbatim}
\caption{An example of the Value Checker accumulating possible values
    of a variable.}
\label{fig-value-multivalue}
\end{figure}

The third subtyping relationship among the value annotations is the
hierarchy of numeric value types. If two annotations satisfy the
subset relationship described above and one of them is numerically a
less precise type than the other it will be automatically ``casted''
up to the type of the other. The full extent of this subtyping is
shown in Figure~\ref{fig-value-subtyping}

\begin{figure}
\includeimage{value-subtyping}{9cm}
\caption{Subtyping of different numeric value annotations.}
\label{fig-value-subtyping}
\end{figure}

\subsection{@Analyzable}
In addition to the value annotations on variables discussed above,
the Value Checker also uses a method annotation,
@Analyzable. As part of the Value Checker operation the checker
attempts to resolve operators (e.g. !, +, ==) and method calls to
track the values returned. By default only the methods used by String
and the wrapper classes are annotated with @Analyzable. However, a
programmer can additionally annotate their own methods with
@Analyzable if appropriate.

Annotating a method with @Analyzable will allow the value of the
return of that method to be tracked via the Value Checker, so long as
all inputs to the function have known value annotations as well.

Since an @Analyzable method is resolved by performing a reflective
call at compile-time, it should only be added to a method (and only
add useful functionality) if the method meets the following
conditions:

\begin{enumerate}

\item The method must be \code{@\refclass{dataflow/quals}{Pure}} (side-effect free and deterministic)

\item The method must only call other \code{@Analyzable} methods

\item The method must be available on the Classpath (since it will be reflectively called at compile-time)

\item The method's parameters are only Strings and/or Java primitives
(the types covered by the Value Checker)

\item The method returns a String or Java primitive (the types covered
by the Value Checker)

\end{enumerate}

\section{Warnings}
The Value Checker may issue warnings, mostly based on failing to
resolve methods marked as @Analyzable but which were not able to be
resolved when the checker was run. These will be displayed as warnings when
the checker is run but the only negative effect is that the return
value of the function will be @UnknownVal instead of being able to be
resolved to a specific value annotation.

