\htmlhr
\chapter{Lock Checker\label{lock-checker}}

The Lock Checker prevents certain kinds of concurrency errors.  If the Lock
checker issues no warnings for a given program, then the program holds the
appropriate lock every time that it accesses a variable.

Note:  This does \emph{not} mean that your program has \emph{no} concurrency
errors.  (You might have forgotten to annotate that a particular variable
should only be accessed when a lock is held.  You might release and
re-acquire the lock, when correctness requires you to hold it throughout a
computation.  And, there are other concurrency errors that cannot, or
should not, be solved with locks.)  However, ensuring that your
program obeys its locking discipline is an easy and effective way to
eliminate a common and important class of errors.


To run the Lock Checker, supply the \code{-processor
  checkers.lock.LockChecker} command-line option to javac.


\section{Lock annotations\label{lock-annotations}}

The Lock Checker uses two annotations.  One is a type qualifier, and the
other is a method annotation.

\begin{description}

\item[\code{@\refclass{checkers/lock/quals}{GuardedBy}}]
  indicates a type whose value may be accessed only when the given lock is
  held.
  See the \ahref{api/checkers/lock/quals/GuardedBy.html}{GuardedBy
    Javadoc} for an explanation of the argument and other details.  The lock
  acquisition and the value access may be arbitrarily far in the future;
  or, if the value is never accessed, the lock never need be held.
  Figure~\ref{fig-guardedby-hierarchy} gives the type hierarchy.

\begin{figure}
\includeimage{guardedby}{2.5cm}
\caption{Type hierarchy for the \code{@GuardedBy} annotation of the lock
  type system.  The \<@GuardedByTop> annotation is for internal use by the
  type checker; a programmer cannot write it.}
\label{fig-guardedby-hierarchy}
\end{figure}


\item[\code{@\refclass{checkers/lock/quals}{Holding}}]
  is a method annotation (not a type qualifier).  It indicates that when
  the method is called, the given lock must be held by the caller.
  In other words, the given lock is already held at the time the method is
  called.

\end{description}

\subsection{Examples}

The most common use of \code{@GuardedBy} is to annotate a field declaration
type.  However, other uses of \code{@GuardedBy} are possible.

\paragraph{Return values}

A return value may be annotated with \code{@GuardedBy}:

\begin{Verbatim}
  @GuardedBy("MyClass.myLock") Object myMethod() { ... }

  // reassignments without holding the lock are OK.
  @GuardedBy("MyClass.myLock") Object x = myMethod();
  @GuardedBy("MyClass.myLock") Object y = x;
  Object z = x;  // ILLEGAL (assuming no lock inference),
                 // because z can be freely accessed.
  x.toString() // ILLEGAL because the lock is not held
  synchronized(MyClass.myLock) {
    y.toString();  // OK: the lock is held
  }
\end{Verbatim}

\paragraph{Formal parameters}

A parameter may be annotated with \code{@GuardedBy}, which indicates that
the method body must acquire the lock before accessing the parameter.  A
client may pass a non-\code{@GuardedBy} reference as an argument, since it
is legal to access such a reference after the lock is acquired.

\begin{Verbatim}
  void helper1(@GuardedBy("MyClass.myLock") Object a) {
    a.toString(); // ILLEGAL: the lock is not held
    synchronized(MyClass.myLock) {
      a.toString();  // OK: the lock is held
    }
  }
  @Holding("MyClass.myLock")
  void helper2(@GuardedBy("MyClass.myLock") Object b) {
    b.toString(); // OK: the lock is held
  }
  void helper3(Object c) {
    helper1(c); // OK: passing a subtype in place of a the @GuardedBy supertype
    c.toString(); // OK: no lock constraints
  }
  void helper4(@GuardedBy("MyClass.myLock") Object d) {
    d.toString(); // ILLEGAL: the lock is not held
  }
  void myMethod2(@GuardedBy("MyClass.myLock") Object e) {
    helper1(e);  // OK to pass to another routine without holding the lock
    e.toString(); // ILLEGAL: the lock is not held
    synchronized (MyClass.myLock) {
      helper2(e);
      helper3(e);
      helper4(e); // OK, but helper4's body still does not type-check
    }
  }
\end{Verbatim}


    

\subsection{Discussion of \<@Holding>}

A programmer might choose to use the \code{@Holding} method annotation in
two different ways:  to specify a higher-level protocol, or to summarize
intended usage.  Both of these approaches are useful, and the Lock Checker
supports both.

\paragraph{Higher-level synchronization protocol}

  \code{@Holding} can specify a higher-level synchronization protocol that
  is not expressible as locks over Java objects.  By requiring locks to be
  held, you can create higher-level protocol primitives without giving up
  the benefits of the annotations and checking of them.

\paragraph{Method summary that simplifies reasoning}

  \code{@Holding} can be a method summary that simplifies reasoning.  In
  this case, the \code{@Holding} doesn't necessarily introduce a new
  correctness constraint; the program might be correct even if the lock
  were acquired later in the body of the method or in a method it calls, so
  long as the lock is acquired before accessing the data it protects.

  Rather, here \code{@Holding} expresses a fact about execution:  when
  execution reaches this point, the following locks are already held.  This
  fact enables people and tools to reason intra- rather than
  inter-procedurally.

  In Java, it is always legal to re-acquire a lock that is already held,
  and the re-acquisition always works.  Thus, whenever you write 

\begin{Verbatim}
  @Holding("myLock")
  void myMethod() {
    ...
  }
\end{Verbatim}

\noindent
it would be equivalent, from the point of view of which locks are held
during the body, to write

\begin{Verbatim}
  void myMethod() {
    synchronized (myLock) {   // no-op:  re-aquire a lock that is already held
      ...
    }
  }
\end{Verbatim}

The advantages of the \<@Holding> annotation include:
\begin{itemize}
\item
  The annotation documents the fact that the lock is intended to already be
  held.
\item
  The Lock Checker enforces that the lock is held when the method is
  called, rather than masking a programmer error by silently re-acquiring
  the lock.
\item
  The \<synchronized> statement can deadlock if, due to a programmer error,
  the lock is not already held.  The Lock Checker prevents this type of
  error.
\item
  The annotation has no run-time overhead.  Even if the lock re-acquisition
  succeeds, it still consumes time.
\end{itemize}


\section{Other lock annotations\label{other-lock-annotations}}

The Checker Framework's lock annotations are similar to annotations used
elsewhere.

If your code is already annotated with a different lock
annotation, you can reuse that effort.  The Checker Framework comes with
cleanroom re-implementations of annotations from other tools.  It treats
them exactly as if you had written the corresponding annotation from the
Lock Checker, as described in Figure~\ref{fig-lock-refactoring}.


% These lists should be kept in sync with LockAnnotatedTypeFactory.java .
\begin{figure}
\begin{center}
% The ~ around the text makes things look better in Hevea (and not terrible
% in LaTeX).
\begin{tabular}{ll}
\begin{tabular}{|l|}
\hline
 ~net.jcip.annotations.GuardedBy~ \\ \hline
\end{tabular}
&
$\Rightarrow$
~checkers.lock.quals.GuardedBy~
\end{tabular}
\end{center}
%BEGIN LATEX
\vspace{-1.5\baselineskip}
%END LATEX
\caption{Correspondence between other lock annotations and the
  Checker Framework's annotations.}
\label{fig-lock-refactoring}
\end{figure}

Alternately, the Checker Framework can process those other annotations (as
well as its own, if they also appear in your program).  The Checker
Framework has its own definition of the annotations on the left side of
Figure~\ref{fig-lock-refactoring}, so that they can be used as type
qualifiers.  The Checker Framework interprets them according to the right
side of Figure~\ref{fig-lock-refactoring}.


\subsection{Relationship to annotations in \emph{Java Concurrency in Practice}\label{jcip-annotations}}

The book \ahref{\url{http://jcip.net/}}{\emph{Java Concurrency in Practice}}~\cite{Goetz2006} defines a
\ahref{\url{http://jcip.net.s3-website-us-east-1.amazonaws.com/annotations/doc/net/jcip/annotations/GuardedBy.html}}{\code{@GuardedBy}} annotation that is the inspiration for ours.  The book's
\code{@GuardedBy} serves two related but distinct purposes:

\begin{itemize}
\item
  When applied to a field, it means that the given lock must be held when
  accessing the field.  The lock acquisition and the field access may be
  arbitrarily far in the future.
\item
  When applied to a method, it means that the given lock must be held by
  the caller at the time that the method is called --- in other words, at
  the time that execution passes the \code{@GuardedBy} annotation.
\end{itemize}

The Lock Checker renames the method annotation to
\code{@\refclass{checkers/lock/quals}{Holding}}, and it generalizes the 
\code{@\refclass{checkers/lock/quals}{GuardedBy}} annotation into a type qualifier
that can apply not just to a field but to an arbitrary type (including the
type of a parameter, return value, local variable, generic type parameter,
etc.).  This makes the annotations more expressive and also more amenable
to automated checking.  It also accommodates the distinct
meanings of the two annotations, and resolves ambiguity when \<@GuardedBy>
is written in a location that might apply to either the method or the
return type.

(The JCIP book gives some rationales for reusing the annotation name for
two purposes.  One rationale is
that there are fewer annotations to learn.  Another rationale is
that both variables and methods are ``members'' that can be ``accessed'';
variables can be accessed by reading or writing them (putfield, getfield),
and methods can be accessed by calling them (invokevirtual,
invokeinterface):  in both cases, \code{@GuardedBy} creates preconditions
for accessing so-annotated members.  This informal intuition is
inappropriate for a tool that requires precise semantics.)

% It would not work to retain the name \code{@GuardedBy} but put it on the
% receiver; an annotation on the receiver indicates what lock must be held
% when it is accessed in the future, not what must have already been held
% when the method was called.


\section{Possible extensions\label{lock-extensions}}

The Lock Checker validates some uses of locks, but not all.  It would be
possible to enrich it with additional annotations.  This would increase the
programmer annotation burden, but would provide additional guarantees.

Lock ordering:  Specify that one lock must be acquired before or after
another, or specify a global ordering for all locks.  This would prevent
deadlock.

Not-holding:  Specify that a method must not be called if any of the listed
locks are held.

These features are supported by 
\ahref{\url{http://clang.llvm.org/docs/LanguageExtensions.html#thread-safety-annotation-checking}}{Clang's thread-safety annotations}.


% LocalWords:  quals GuardedBy JCIP putfield getfield invokevirtual 5cm
% LocalWords:  invokeinterface threadsafety Clang's GuardedByTop cleanroom
