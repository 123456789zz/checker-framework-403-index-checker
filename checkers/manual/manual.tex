\documentclass[10pt]{article}
\usepackage{pslatex}
\usepackage{fullpage}
\usepackage{graphicx}
\usepackage{hevea}
\usepackage{listings}
\usepackage{url}
\usepackage{alltt}

\usepackage{relsize}
% \def\codesize{\smaller}
\def\codesize{\relax}           % for "pslatex"
%HEVEA \def\codesize{\relax}
\newcommand{\code}[1]{\ifmmode{\mbox{\codesize\ttfamily{#1}}}\else{\codesize\ttfamily #1}\fi}
% This can't handle a URL with an embedded "#" -- at least at UW CSE
\newcommand{\myurl}[1]{{\codesize\url{#1}}}
%HEVEA \def\myurl{\url}
\def\<#1>{\code{#1}}

\usepackage{fancyvrb}
%BEGIN LATEX
\RecustomVerbatimEnvironment{Verbatim}{Verbatim}{fontsize=\codesize}
%END LATEX

%HEVEA \footerfalse    % Disable hevea advertisement in footer

\newcommand{\htmlhr}{\relax}
%HEVEA \renewcommand{\htmlhr}{\@hr{}{}}

% Problem: Hevea writes this into manual.image.tex, and invokes LaTeX on
% it.  Sometimes running "make" leads to an error as a result, sometimes
% not.  I don't know the pattern of the failures, though running "make
% clean" and then "make" seems to work.  So maybe there's a problem with an
% auxiliary file.
%HEVEA \newcommand{\discretionary}[3]{\relax}

%HEVEA \newstyle{.lstframe}{margin:auto;margin-bottom:2em}

% Left and right curly braces in tt font
\newcommand{\ttlcb}{\texttt{\char "7B}}
\newcommand{\ttrcb}{\texttt{\char "7D}}
\newcommand{\ttbs}{\texttt{\char "5C}}


\title{The Checker Framework Manual}
\author{% MIT Program Analysis Group \\
\url{http://types.cs.washington.edu/checker-framework/}}
\newcommand{\ReleaseInfo}{0.9.8 (21 Aug 2009)}
\date{Version \ReleaseInfo{}}

\begin{document}

\maketitle

% %BEGIN LATEX
% \tableofcontents
% %END LATEX

%BEGIN LATEX
  %% Bring items closer together in list environments
  % Prevent infinite loops
  \let\Itemize =\itemize
  \let\Enumerate =\enumerate
  \let\Description =\description
  % Zero the vertical spacing parameters
  \def\Nospacing{\itemsep=0pt\topsep=0pt\partopsep=0pt\parskip=0pt\parsep=0pt}
  % Redefine the environments in terms of the original values
  \renewenvironment{itemize}{\Itemize\Nospacing}{\endlist}
  \renewenvironment{enumerate}{\Enumerate\Nospacing}{\endlist}
  \renewenvironment{description}{\Description\Nospacing}{\endlist}

  % Add line between figure and text
  \makeatletter
  \def\topfigrule{\kern3\p@ \hrule \kern -3.4\p@} % the \hrule is .4pt high
  \def\botfigrule{\kern-3\p@ \hrule \kern 2.6\p@} % the \hrule is .4pt high
  \def\dblfigrule{\kern3\p@ \hrule \kern -3.4\p@} % the \hrule is .4pt high
  \makeatother
%END LATEX


% Reference to Checker Framework Javadoc for a class (not a method, etc.).
% Arg 1: directory under checkers/, including internal "/", but no leading
% or trailing "/".
% Arg 2: class name.
% In the printed version, only the base class name appears.
% In the HTML version, it's a link to the Javadoc.
\newcommand{\refclass}[2]{\ahref{doc/checkers/#1/#2.html}{\<#2>}}
% Reference to Checker Framework Javadoc for a method or field.).
% Arg 1: directory under checkers/, including internal "/", but no leading
% or trailing "/".
% Arg 2: class name.
% Arg 3: method name.
% Arg 4: fully-qualified arguments.  Example: "(T)"
% In the printed version, only "class.method" appears.
% In the HTML version, it's a link to the Javadoc.
\newcommand{\refmethod}[4]{\ahref{doc/checkers/#1/#2.html\##3#4}{\<#2.#3>}}
% Reference to Sun Javadoc.
% Arg 1: .html reference, without the .../api/ prefix
% Arg 2: What will appear in the formatted manual.
% Problem:  the "?is-external=true" must appear before any "#".  But why is
% it necessary at all?
% \newcommand{\sunjavadoc}[2]{\ahref{http://java.sun.com/javase/6/docs/api/#1?is-external=true}{\<#2>}}
\newcommand{\sunjavadoc}[2]{\ahref{http://java.sun.com/javase/6/docs/api/#1}{\<#2>}}

\noindent
\textbf{For the impatient:}
Section~\ref{installation} describes how to \textbf{install and use} pluggable
type-checkers.

\newcommand{\sectionpageref}[1]{Section~\ref{#1}, page~\pageref{#1}}
%HEVEA \renewcommand{\sectionpageref}[1]{Section~\ref{#1}}

%BEGIN LATEX
\medskip
%END LATEX

\noindent
You can also jump directly to the documentation for a particular checker:
\begin{itemize}
\item Nullness checker: \sectionpageref{nullness-checker}
\item Interning checker: \sectionpageref{interning-checker}
\item IGJ (immutability) checker: \sectionpageref{igj-checker}
\item Javari (immutability) checker: \sectionpageref{javari-checker}
\item Lock checker: \sectionpageref{lock-checker}
\item Tainting checker: \sectionpageref{tainting-checker}
\item Basic checker: \sectionpageref{basic-checker}
\end{itemize}


%HEVEA \setcounter{tocdepth}{2}
\tableofcontents
\newpage

\htmlhr
\chapter{Introduction\label{introduction}}

The Checker Framework enhances Java's type system to make it more powerful
and useful.
This lets software developers detect and
prevent errors in their Java programs.

The Checker Framework comes with checkers for specific types of errors:

\begin{enumerate}

\item
  \ahrefloc{nullness-checker}{Nullness checker} for null pointer errors
  (see \chapterpageref{nullness-checker})
\item
  \ahrefloc{interning-checker}{Interning checker} for errors in equality
  testing and interning (see \chapterpageref{interning-checker})
\item
  \ahrefloc{lock-checker}{Lock checker} for concurrency and lock errors,
  inspired by the Java Concurrency in Practice (JCIP) annotations (see
  \chapterpageref{lock-checker})
\item
  \ahrefloc{fenum-checker}{Fake enum checker} to allow type-safe fake enum
  patterns (see \chapterpageref{fenum-checker})
\item
  \ahrefloc{tainting-checker}{Tainting checker} for trust and security errors
  (see \chapterpageref{tainting-checker})
\item
  \ahrefloc{regex-checker}{Regex checker} to prevent use of syntactically
  invalid regular expressions (see \chapterpageref{regex-checker})
\item
  \ahrefloc{propkey-checker}{Property file checker} to ensure that valid
  keys are used for property files and resource bundles (see
  \chapterpageref{propkey-checker}).
  Also includes a checker that code is properly internationalized.
  %% Not interesting enough to go at the beginning of the introduction.
  % and a checker for compiler message keys as used in the Checker Framework.
\item
  \ahrefloc{signature-checker}{Signature string checker} to ensure that the
  string representation of a type is properly used, for example in
  \<Class.forName> (see \chapterpageref{signature-checker}).
  Also includes a checker that code is properly internationalized.
  %% Not interesting enough to go at the beginning of the introduction.
  % and a checker for compiler message keys as used in the Checker Framework.
\item
  \ahrefloc{units-checker}{Units checker} to ensure operations are
  performed on correct units of measurement
  (see \chapterpageref{units-checker})
\item
  \ahrefloc{linear-checker}{Linear checker} to control aliasing and prevent
  re-use (see \chapterpageref{linear-checker})
\item
  \ahrefloc{igj-checker}{IGJ checker} for mutation errors (incorrect
  side effects), based on the IGJ type system (see
  \chapterpageref{igj-checker})
\item
  \ahrefloc{javari-checker}{Javari checker} for mutation errors
  (incorrect side effects), based on the Javari type system (see
  \chapterpageref{javari-checker})
\item
  \ahrefloc{basic-checker}{Basic checker} for customized checking without
  writing any code (see \chapterpageref{basic-checker})
\item
  \ahrefloc{typestate-checker}{Typestate checker} to ensure operations are
  performed on objects that are in the right state, such as only opened
  files being read (see \chapterpageref{typestate-checker})
\item
  \ahrefloc{third-party-checkers}{Third-party checkers} that are distributed
  separately from the Checker Framework
  (see \chapterpageref{third-party-checkers})

% Keep this list in sync with the list in manual.tex.
\end{enumerate}

\noindent
These checkers are easy to use and are invoked as arguments to \<javac>.


The Checker Framework also enables you to write new checkers of your
own; see Chapters~\ref{basic-checker} and~\ref{writing-a-checker}.


\section{How it works:  Pluggable types\label{pluggable-types}}

The Checker Framework supports adding
pluggable type systems to the Java language in a backward-compatible way.
Java's built-in typechecker finds and prevents many errors --- but it
doesn't find and prevent \emph{enough} errors.  The Checker Framework lets you
run an additional typechecker as a plug-in to the javac compiler.  Your
code stays completely backward-compatible:  your code compiles with any
Java compiler, it runs on any JVM, and your coworkers don't have to use the
enhanced type system if they don't want to.  You can check only part of
your program.  Type inference tools exist to help you annotate your
code.


A type system designer uses the Checker Framework to define type qualifiers
and their semantics, and a
compiler plug-in (a ``checker'') enforces the semantics.  Programmers can
write the type qualifiers in their programs and use the plug-in to detect
or prevent errors.  The Checker Framework is useful both to programmers who
wish to write error-free code, and to type system designers who wish to
evaluate and deploy their type systems.



% This manual is organized as follows.
% \begin{itemize}
% \item Chapter~\ref{introduction} overviews the Checker Framework and
%   describes how to \ahrefloc{installation}{install} it (Chapter~\ref{installation}).
% \item Chapter~\ref{using-a-checker} describes how to \ahrefloc{using-a-checker}{use a checker}.
% \item
%   The next chapters are user manuals for the \ahrefloc{nullness-checker}{Nullness}
%   (Chapter~\ref{nullness-checker}), \ahrefloc{interning-checker}{Interning}
%   (Chapter~\ref{interning-checker}), \ahrefloc{javari-checker}{Javari} (Chapter~\ref{javari-checker}),
%   \ahrefloc{igj-checker}{IGJ} (Chapter~\ref{igj-checker}), and \ahrefloc{basic-checker}{Basic}
%   (Chapter~\ref{basic-checker}) checkers.
% \item Chapter~\ref{annotating-libraries} describes an approach for \ahrefloc{annotating-libraries}{annotating external
% libraries}.
% \item Chapter~\ref{writing-a-checker} describes how to
%   \ahrefloc{writing-a-checker}{write a new checker} using the Checker Framework.
% \end{itemize}






This document uses the terms ``checker'', ``checker plugin'',
``type-checking compiler plugin'', and ``annotation processor'' as
synonyms.

\section{Installation\label{installation}}

This section describes how to install the Checker
Framework for use from the command line, Ant, and Maven.  If you wish to
use the Checker Framework from Eclipse, see the Checker Framework Eclipse
Plugin webpage: \ahrefurl{http://types.cs.washington.edu/checker-framework/eclipse/}.
The Checker Framework release contains everything that you need, both to
run checkers and to write your own checkers.  As an alternative, you can
build the latest development version from source
(Section~\refwithpage{build-source}).

% Not "\ahrefurl" because it looks bad in the printed manual.
\textbf{Requirement:}
You must have \textbf{JDK 6} or later installed.  You can get JDK 6 from
\ahref{http://www.oracle.com/technetwork/java/javase/downloads/index.html}{Oracle}
or elsewhere.  If you are using Apple Mac OS X, you can use
\ahref{http://developer.apple.com/search/index.php?q=java}{Apple's implementation},
\ahref{http://landonf.bikemonkey.org/static/soylatte/}{SoyLatte},
or the \ahref{http://openjdk.java.net/}{OpenJDK}.

The installation process is simple!
\begin{enumerate}
\item
  Download the Checker Framework distribution
  (\ahrefurl{http://types.cs.washington.edu/checker-framework/current/checkers.zip}).
\item 
  Unzip it to create a \code{checker-framework} directory.

%% Not needed any more.
%   Then, set the
%   \code{CHECKERS} environment variable to the
%   \code{checker-framework/checkers} directory.  For instance, if you use
%   the bash shell, then add this to your \code{.bashrc} file (don't forget
%   to replace the ``\code{...}''!):
% \begin{Verbatim}
%   export CHECKERS=.../checker-framework/checkers
% \end{Verbatim}
%   Also execute it on the command line, or log out and back in.

\item
  Optionally, update your execution path or create an alias.

  When doing pluggable type-checking, you need to use the ``Checker
  Framework compiler''.  The Checker Framework compiler in turn uses the
  ``Type Annotations compiler'', an advance version of the OpenJDK 8 javac
  compiler that understands type annotations.  The Type Annotations
  compiler is backward-compatible, so using it as your Java compiler, even
  when you are not doing pluggable type-checking, should have no negative
  consequences.

  You can use the Checker Framework compiler in three ways.  You can use any one of them.  However, if
  you are using the Windows command shell, you must use the last one.

  First, set a \code{CHECKERS} environment variable to the
  \code{.../checker-framework/checkers} directory.  Alternately, you can
  use an absolute path where the below instructions use the \code{CHECKERS}
  environment variable.

  % Is the last one required for Cygwin, as well as for the Windows command shell?

  \begin{itemize}
  \item
    Add directory
    \code{\$CHECKERS/binary} to your path, \emph{before} any other
    directory that contains a \<javac> executable.  Now, whenever
    you run \code{javac}, you will use the updated compiler.  If you are
    using the bash shell, a way to do this is to add the following to your
    \verb|~/.bashrc| file:
\begin{Verbatim}
  export PATH=${CHECKERS}/binary:${PATH}
\end{Verbatim}
  \item
    Whenever this document tells you to run \code{javac}, you
    can instead run \code{\$CHECKERS/binary/javac}.

    You can simplify this by introducing an alias.  Then,
    whenever this document tells you to run \code{javac}, instead use that
    alias.  Here is the syntax for your 
    \verb|~/.bashrc| file:
% No Windows example because this doesn't work under Windows.
\begin{Verbatim}
  alias javacheck='$CHECKERS/binary/javac'
\end{Verbatim}

 \item
   Whenever this document tells you to run \code{javac}, instead
   run checkers.jar via \<java> (not \<javac>) as in:

\begin{Verbatim}
  java -jar $CHECKERS/binary/checkers.jar ...
\end{Verbatim}

   More generally, anywhere that you use \<javac.jar>, you can substitute
   \<\$CHECKERS/binary/checkers.jar>; the result is to use the Checker
   Framework compiler instead of the regular \<javac>.

%     Note: Previous versions of these instructions required the use of
%     \$CHECKERS/binary/javac.jar and providing jsr308-all.jar on the bootclasspath.
%     This is no longer the case.  The provided javac script and checkers.jar
%     will automatically fix up the classpath and bootclasspath for use with
%     the Checker Framework and call the type-annotations compiler appropriately.
%     jsr308-all.jar no longer exists.  Finally, anywhere javac.jar might be used
%     you can substitute \$CHECKERS/binary/checkers.jar in order to use
%     the Checker Framework.

    You can simplify the above command by introducing an alias.  Then,
    whenever this document tells you to run \code{javac}, instead use that
    alias.  For example:

\begin{Verbatim}
  # Unix
  alias javacheck='java -jar $CHECKERS/binary/checkers.jar'

  # Windows
  doskey javacheck=java -jar %CHECKERS%\binary\checkers.jar $*
\end{Verbatim}
\end{itemize}

\end{enumerate}


To ensure that it was installed properly, run \<javac -version> (possibly using the
full pathname to \<javac> or the alias, if you did not add the Type Annotations \<javac> to your path).

The output should be:

\begin{Verbatim}
  javac 1.7.0-jsr308-1.6.4
\end{Verbatim}

That's all there is to it!  Now you are ready to start using the checkers.

Section~\ref{example-use} walks you through a simple example.  More detailed
instructions for using a checker appear in Chapter~\ref{using-a-checker}.
There is also a \ahreforurl{http://types.cs.washington.edu/checker-framework/tutorial/}{tutorial}
that walks you through how to use the Checker Framework in Eclipse or on
the command line.


\section{Example use:  detecting a null pointer bug\label{example-use}}

This section gives a very simple example of running the Checker Framework
from the command line.  There is also a \ahreforurl{http://types.cs.washington.edu/checker-framework/tutorial/}{tutorial}
that gives more extensive instructions for using the Checker Framework in
Eclipse or on the command line.


To run a checker on a source file, just run javac as usual, passing the
\<-processor> flag.  (You can also use an IDE or other build tool; see
Chapter~\ref{external-tools}.)

For instance, if you usually run the compiler like
this:

\begin{Verbatim}
  javac Foo.java Bar.java
\end{Verbatim}

\noindent
then you will instead use the command line:

\begin{alltt}
  javac -processor \textit{ProcessorName} Foo.java Bar.java
\end{alltt}

\noindent
but take note that the \code{javac} command must refer to the Type
Annotations compiler (see Section~\ref{installation}).


If you usually do your coding within an IDE, you will need to configure
the IDE.  This manual contains instructions for
Ant (Section~\ref{ant-task}),
Maven (Section~\ref{maven-plugin}),
IntelliJ IDEA (Section~\ref{intellij}),
Eclipse (Section~\ref{eclipse}), and
tIDE (Section~\ref{tide}).
Otherwise, see your IDE documentation for details.


\begin{enumerate}
\item
  Let's consider this very simple Java class.  One local variable is
  annotated as \<NonNull>, indicating that \<ref> must be a reference to a
  non-null object.  Save the file as \<GetStarted.java>.

\begin{Verbatim}
import checkers.nullness.quals.*;

public class GetStarted {
    void sample() {
        @NonNull Object ref = new Object();
    }
}
\end{Verbatim}

\item
  Run the nullness checker on the class.
  Either run this command:
\begin{Verbatim}
  javac -processor checkers.nullness.NullnessChecker GetStarted.java
\end{Verbatim}

\noindent
or compile from within your IDE, which you have customized to use the
Checker Framework compiler and to pass the extra arguments.

  The compilation should complete without any errors.

\item
  Let's introduce an error now.  Modify \<ref>'s assignment to:
\begin{Verbatim}
  @NonNull Object ref = null;
\end{Verbatim}

\item
  Run the nullness checker again, just as before.  This run should emit
  the following error:
\begin{Verbatim}
GetStarted.java:5: incompatible types.
found   : @Nullable <nulltype>
required: @NonNull Object
        @NonNull Object ref = null;
                              ^
1 error
\end{Verbatim}

\end{enumerate}

The type qualifiers (e.g., \<@NonNull>) are permitted anywhere
that would write a type, including generics and casts; see
Section~\ref{writing-annotations}.

\begin{alltt}
  \underline{@Interned} String intern() \ttlcb{} ... \ttrcb{}             // return value
  int compareTo(\underline{@NonNull} String other) \ttlcb{} ... \ttrcb{}  // parameter
  \underline{@NonNull} List<\underline{@Interned} String> messages;     // non-null list of interned Strings
\end{alltt}


\htmlhr
\chapter{Using a checker\label{using-a-checker}}

A pluggable type-checker enables you to detect certain bugs in your code,
or to prove that they are not present.  The verification happens at compile
time.


Finding bugs, or verifying their absence, with a checker plugin is a two-step process, whose steps are
described in Sections~\ref{writing-annotations} and \ref{running}.

\begin{enumerate}

\item The programmer writes annotations, such as \code{@\refclass{nullness/quals}{NonNull}} and
  \code{@\refclass{interning/quals}{Interned}}, that specify additional information about Java types.
  (Or, the programmer uses an inference tool to automatically insert
  annotations in his code:  see Sections~\ref{nullness-inference} and~\ref{javari-inference}.)
  It is possible to annotate only part of your code:  see
  Section~\ref{unannotated-code}.

\item The checker reports whether the program contains any erroneous code
  --- that is, code that is inconsistent with the annotations.

\end{enumerate}

This chapter is structured as follows:
\begin{itemize}
\item Section~\ref{writing-annotations}: How to write annotations
\item Section~\ref{running}:  How to run a checker
\item Section~\ref{checker-guarantees}: What the checker guarantees
\item Section~\ref{tips-about-writing-annotations}: Tips about writing annotations
\end{itemize}

Additional topics that apply to all checkers are covered later in the manual:
\begin{itemize}
\item Chapter~\ref{advanced-type-system-features}: Advanced type system features
\item Chapter~\ref{warnings-and-legacy}: Handling warnings and legacy code
\item Chapter~\ref{annotating-libraries}: Annotating libraries
\item Chapter~\ref{writing-a-checker}: How to create a new checker
\item Chapter~\ref{external-tools}: Integration with external tools
\end{itemize}


Finally, there is a 
\ahreforurl{http://types.cs.washington.edu/checker-framework/tutorial/}{tutorial}
that walks you through using the Checker Framework in Eclipse or on the
command line.

% The annotations have to be on your classpath even when you are not using
% the -processor, because of the existence of the import statement for
% the annotations.


\section{Writing annotations\label{writing-annotations}}

The syntax of type annotations in Java is specified by
\ahref{http://types.cs.washington.edu/jsr308/}{JSR 308}~\cite{JSR308-2008-09-12}.  Ordinary
Java permits annotations on declarations.  JSR 308 permits annotations
anywhere that you would write a type, including generics and casts.  You
can also write annotations to indicate type qualifiers for array levels and
receivers.  Here are a few examples:

\begin{alltt}
  \underline{@Interned} String intern() \ttlcb{} ... \ttrcb{}               // return value
  int compareTo(\underline{@NonNull} String other) \ttlcb{} ... \ttrcb{}    // parameter
  String toString(\underline{@ReadOnly} MyClass this) \ttlcb{} ... \ttrcb{} // receiver ("this" parameter)
  \underline{@NonNull} List<\underline{@Interned} String> messages;       // generics:  non-null list of interned Strings
  \underline{@Interned} String \underline{@NonNull} [] messages;          // arrays:  non-null array of interned Strings
  myDate = (\underline{@ReadOnly} Date) readonlyObject;       // cast
\end{alltt}

You can also write the annotations within comments, as in
\code{List</*@NonNull*/ String>}.  The Type Annotations compiler, which is
distributed with the Checker Framework, will still process
the annotations.
However, your code will remain compilable by people who are not using the
Type Annotations compiler.  For more details, see
Section~\ref{annotations-in-comments}.



\subsection{Distributing your annotated project\label{distributing}}

If your code contains annotations, then your code has a dependency on the
annotation declarations.  People who want to compile or run your code may
need declarations of the annotations on their classpath.

\begin{itemize}
\item
To perform pluggable type-checking, all of the Checker Framework (which
also contains the annotation declarations) is needed.
\item
To compile the code:
\begin{itemize}
\item
  If you wrote annotations in comments (see
  Section~\ref{annotations-in-comments}) and/or used implicit import
  statements (see Section~\ref{implicit-import-statements}), then the code
  can be compiled by any Java compiler, without needing declarations of the
  annotations.
\item
  Otherwise, compiling the code requires a declaration of the annotations.
  These appear in the full Checker Framework.  Additionally, the Checker
  Framework distribution \code{.zip} file contains a small jar file,
  \code{checkers-quals.jar}, that only contains the definitions of the
  distributed qualifiers, without any support for type-checking.
\end{itemize}
\item
To run the code:
\begin{itemize}
\item
  If you compiled the code without using the annotation declarations, then
  no annotation declarations are needed.
\item
  If you compiled the code using the annotation declarations, then users
  may need to have the annotation declarations on their classpath.
\end{itemize}
\end{itemize}

A simple rule of thumb is as follows.  When distributing your source code,
you may wish to include either the Checker Framework jar file or the
\code{checkers-quals.jar} file.  When distributing compiled binaries, you
may wish to compile them without using the annotations, or include the
contents of \code{checkers-quals.jar} in your distribution.


\section{Running a checker\label{running}}

To run a checker plugin, run the compiler \code{javac} as usual,
but pass the \code{-processor \emph{plugin\_class}} command-line
option.
(You can run a checker from within your favorite IDE or build system.  See
Chapter~\ref{external-tools} for details about
Ant (Section~\ref{ant-task}),
Maven (Section~\ref{maven-plugin}),
IntelliJ IDEA (Section~\ref{intellij}),
Eclipse (Section~\ref{eclipse}),
and
tIDE (Section~\ref{tide}), and about customizing other IDEs and build tools.)
Remember that you must be using the
Type Annotations version of \<javac>, which you already installed (see Section~\ref{installation}).

Two concrete examples (using the Nullness checker) are:

%BEGIN LATEX
\begin{smaller}
%END LATEX
\begin{Verbatim}
  javac -processor checkers.nullness.NullnessChecker MyFile.java
  java -jar $CHECKERS/binary/checkers.jar -processor checkers.nullness.NullnessChecker MyFile.java
\end{Verbatim}
%BEGIN LATEX
\end{smaller}
%END LATEX

\noindent
Note that the two invocations above are equivalent.  Each one invokes the
Checker Framework compiler, which in turn invokes the Type Annotations compiler
with an annotated JDK on its classpath.  For more information on annotated
JDKs, see Section~\ref{skeleton-using}.

The checker is run on only the Java files that javac compiles.
This includes all Java files specified on the command line (or
created by another annotation processor).  It may also include other of
your Java files (but not if a more recent \code{.class} file exists).
Even when the checker does not analyze a class (say, the class was
already compiled, or source code is not available), it does check
the \emph{uses} of those classes in the source code being compiled.

You can always compile the code without the \code{-processor}
command-line option, but in that case no checking of the type
annotations is performed.  The annotations are still written to the
resulting \<.class> files, however.



\subsection{Summary of command-line options\label{checker-options}}

You can pass command-line arguments to a checker via javac's standard \<-A>
option (``\<A>'' stands for ``annotation'').  All of the distributed
checkers support the following command-line options:

\begin{itemize}
\item \<-AskipUses> Suppress all errors and warnings at all uses of a
  given class; see Section~\ref{suppressing-warnings}
\item \<-AskipDefs> Suppress all errors and warnings within the definition of a
  given class; see Section~\ref{suppressing-warnings}
\item \<-Astubs> List of stub files or directories; see Section~\ref{stub-using}
\item \<-AstubWarnIfNotFound> Warn if a stub file entry could not be found; see Section~\ref{stub-using}
\item \<-AstubDebug> Output debug messages while parsing stub files; see Section~\ref{stub-using}
\item \<-Alint> Enable or disable optional checks; see Section~\ref{lint-options}
\item \<-Awarns> Treat checker errors as warnings.  If you use this, you
  may wish to also supply \code{-Xmaxwarns 10000}, because by default
  \<javac> prints at most 100 warnings.
\item \<-Afilenames>, \<-Anomsgtext>, <\-Adetailedmsgtext>, \<-Ashowchecks>, \<-AprintErrorStack>,
  \<-AprintAllQualifiers>, \<-Aignorejdkastub>
  Aids for testing or debugging a checker; see Section~\ref{debugging-options}
\item \<-AresourceStats> Display resource usage statistics at the end of compilation
\end{itemize}

\noindent
Some checkers support additional options, such as \<-Aquals> for the Basic
Checker to check; see Chapter~\ref{basic-checker}.


Here are some standard javac command-line options that you may find useful.
Many of them contain the word ``processor'', because in javac jargon, a
checker is a type of ``annotation processor''.

\begin{itemize}
\item \<-processor> Names the checker to be
  run; see Section~\ref{running}
\item \<-processorpath> Indicates where to search for the
  checker; should also contain any qualifiers used by the Basic
  Checker; see Section~\ref{basic-example}
\item \<-proc:>\{\<none>,\<only>\} Controls whether checking
  happens; \<-proc:none>
  means to skip checking; \<-proc:only> means to do only
  checking, without any subsequent compilation; see
  Section~\ref{checker-auto-discovery}
\item \<-Xbootclasspath/p:> Indicates where to find the annotated JDK classes;
  see Section~\ref{skeleton-using}
\item \<-implicit:class> Suppresses warnings about implicitly compiled files
  (not named on the command line); see Section~\ref{ant-task}
\item \<-XDTA:noannotationsincomments> and \<-XDTA:spacesincomments>
  to turn off parsing annotation comments and
  to turn on parsing annotation comments even when they
  contain spaces; applicable only to the Type Annotations compiler;
  see Section~\ref{annotations-in-comments}
\item \<-J> Supply an argument to the JVM that is running javac; example:
  \<-J-Djsr308\_imports=checkers.nullness.quals.*>; see Section~\ref{implicit-import-statements}
\end{itemize}


\subsection{Checker auto-discovery\label{checker-auto-discovery}}

``Auto-discovery'' makes the \code{javac} compiler always run a checker
plugin, even if you do not explicitly pass the \code{-processor}
command-line option.  This can make your command line shorter, and ensures
that your code is checked even if you forget the command-line option.

To enable auto-discovery, place a configuration file named
\code{META-INF/services/javax.annotation.processing.Processor}
in your classpath.  The file contains the names of the checker plugins to
be used, listed one per line.  For instance, to run the Nullness and the
Interning checkers automatically, the configuration file should contain:

%BEGIN LATEX
\begin{smaller}
%END LATEX
\begin{Verbatim}
  checkers.nullness.NullnessChecker
  checkers.interning.InterningChecker
\end{Verbatim}
%BEGIN LATEX
\end{smaller}
%END LATEX

You can disable this auto-discovery mechanism by passing the
\code{-proc:none} command-line option to \<javac>, which disables all
annotation processing including all pluggable type-checking.

%% Auto-discovering all the distributed checkers by default would be
%% problematic.  So, leave it up to the user to enable auto-discovery.
%%  1. We don't want to auto-discover both the Javari & IGJ type checkers,
%%     as then the user would see multiple, possibly contradictory, types
%%     of mutability diagnostics.
%%  2. The nullness and mutability checkers would issue lots of errors for
%%     unannotated code, and that would be irritating.



\section{What the checker guarantees\label{checker-guarantees}}

A checker can guarantee that a particular property holds throughout the
code.  For example, the Nullness checker (Chapter~\ref{nullness-checker})
guarantees that every expression whose type is a \code{@\refclass{nullness/quals}{NonNull}} type never
evaluates to null.  The Interning checker (Chapter~\ref{interning-checker})
guarantees that every expression whose type is an \code{@\refclass{interning/quals}{Interned}} type
evaluates to an interned value.  The checker makes its guarantee by
examining every part of your program and verifying that no part of the
program violates the guarantee.

There are some limitations to the guarantee.

\urldef{\jlsintersectiontypesurl}{\url}{http://docs.oracle.com/javase/specs/jls/se7/html/jls-4.html#jls-4.9}

\begin{itemize}

\item
  A compiler plugin can check only those parts of your program that you run
  it on.  If you compile some parts of your program without running the
  checker, then there is no guarantee that the entire program satisfies the
  property being checked.  Some examples of un-checked code are:

  \begin{itemize}
  \item
    Code compiled without the \code{-processor} switch, including any
    external library supplied as a \code{.class} file.
  \item
    Code compiled with the \code{-AskipUses} or \code{-AskipDefs}
    properties (see Section~\ref{suppressing-warnings}).
  \item
    Suppression of warnings, such as via the \code{@SuppressWarnings}
    annotation (see Section~\ref{suppressing-warnings}).
  \item
    Native methods (because the implementation is not Java code, it cannot
    be checked).
  \end{itemize}

  In each of these cases, any \emph{use} of the code is checked --- for
  example, a call to a native method must be compatible with any
  annotations on the native method's signature.
  However, the annotations on the un-checked code are trusted; there is no
  verification that the implementation of the native method satisfies the
  annotations.

\item
  Reflection can violate the Java type system, and
  the checkers are not sophisticated enough to reason about the possible
  effects of reflection.  Similarly, deserialization and cloning can
  create objects that could not result from normal constructor calls, and
  that therefore may violate the property being checked.

\item
  The Checker Framework assumes it is checking single-threaded code.
  Concurrent access to variables (that is, race conditions) could cause
  errors that the Checker Framework does not report.

\item
  The Checker Framework does not yet support annotations on intersection
  types (see
  \ahref{\jlsintersectiontypesurl}{JLS \S4.9}).  As a result, checkers cannot provide guarantees about
  intersection types.

\item
  Specific checkers may have other limitations; see their documentation for
  details.

\end{itemize}

A checker can be useful in finding bugs or in verifying part of a
program, even if the checker is unable to verify the correctness of an
entire program.

In order to avoid a flood of unhelpful warnings, many of the checkers avoid
issuing the same warning multiple times.  For example, in this code:

\begin{Verbatim}
  @Nullable Object x = ...;
  x.toString();                 // warning
  x.toString();                 // no warning
\end{Verbatim}

\noindent
In this case, the second call to \<toString> cannot possibly throw a null
pointer warning --- \<x> is non-null if control flows to the second
statement.
In other cases, a checker avoids issuing later warnings with the same cause
even when later code in a method might also fail.
This does not
affect the soundness guarantee, but a user may need to examine more
warnings after fixing the first ones identified.  (More often, at least in
our experience to date, a single fix corrects all the warnings.)

% It might be worthwhile to permit a user to see every warning -- though I
% would not advocate this setting for daily use.

If you find that a checker fails to issue a warning that it
should, then please report a bug (see Section~\ref{reporting-bugs}).


\section{Tips about writing annotations\label{tips-about-writing-annotations}}


\subsection{How to get started annotating legacy code\label{get-started-with-legacy-code}}

Annotating an entire existing program may seem like a daunting task.  But,
if you approach it systematically and do a little bit at a time, you will
find that it is manageable.

You should start with a property that matters to you, to achieve the best
benefits.  It is easiest to add annotations if you know the code or the
code contains documentation; you will find that you spend most of your time
understanding the code, and very little time actually writing annotations
or running the checker.

Don't get discouraged if you see many type-checker warnings at first.
Often, adding just a few missing annotations will eliminate many warnings,
and you'll be surprised how fast the process goes overall.

It is best to annotate one package at a time,
% Upcoming fix that applies different defaults to annotated and unannotated
% code will eliminate this reason.
and to annotate the entire package so that you don't forget any classes
(failing to annotate a class can lead to unexpected results).
Start as close to the leaves of the call tree as possible, such as with
libraries --- that is,
start with methods/classes/packages that have few dependences on other
code or, equivalently, start with code that a lot of your other code
depends on.  The reason for this is that it is
easiest to annotate a class if the code it calls has already been
annotated.

For each class, read its Javadoc.  For instance, if you are adding
annotations for the Nullness Checker (Section~\ref{nullness-checker}), then
you can search the documentation for ``null'' and then add \<@Nullable>
anywhere appropriate.  Do not annotate the method bodies yet ---
first, get the signatures and fields annotated.  The only reason to even
\emph{read} the method bodies yet is to determine signature annotations for
undocumented methods ---
for example, if the method returns null, you know its return type should be
annotated \<@Nullable>, and a parameter that is compared against \<null>
may need to be annotated \<@Nullable>.  If you are only annotating
signatures (say, for a library you do not maintain and do not wish to
check), you are now done.

If you wish to check the implementation, then after the signatures are
annotated, run the checker.  Then, add method body annotations (usually,
few are necessary), fix bugs in code, and add annotations to signatures
where necessary.  If signature annotations are necessary, then you may want
to fix the documentation that did not indicate the property; but this isn't
strictly necessary, since the annotations that you wrote provide that
documentation.

You may wonder about the effect of adding a given annotation --- how many
other annotations it will require, or whether it conflicts with other code.
Suppose you have added an annotation to a method parameter.  You could
manually examine all callees.  A better way can be to save the checker
output before adding the annotation, and to compare it to the checker
output after adding the annotation.  This helps you to focus on the
specific consequences of your change.

Also see Chapter~\ref{warnings-and-legacy}, which tells you what to do when
you are unable to eliminate checker warnings.



\subsection{Do not annotate local variables unless necessary\label{tips-local-inference}}

The checker infers annotations for local variables (see
Section~\ref{type-refinement}).  Usually, you only need to annotate fields
and method signatures.  After doing those, you can add annotations inside
method bodies if the checker is unable to infer the correct annotation, if
you need to suppress a warning (see Section~\ref{suppressing-warnings}),
etc.


\subsection{Annotations indicate normal behavior\label{annotate-normal-behavior}}

You should use annotations to indicate \emph{normal} behavior.  The
annotations indicate all the values that you \emph{want} to flow to
reference --- not every value that might possibly flow there if your
program has a bug.

Many methods are guaranteed to throw an exception if they are passed \code{null}
as an argument.  Examples include

\begin{Verbatim}
  java.lang.Double.valueOf(String)
  java.lang.String.contains(CharSequence)
  org.junit.Assert.assertNotNull(Object)
  com.google.common.base.Preconditions.checkNotNull(Object)
\end{Verbatim}

\code{@\refclass{nullness/quals}{Nullable}} (see Section~\ref{nullness-annotations})
might seem like a reasonable annotation for the parameter,
for two reasons.  First, \code{null} is a legal argument with a
well-defined semantics:  throw an exception.  Second, \code{@Nullable}
describes a possible program execution:  it might be possible for
\code{null} to flow there, if your program has a bug.

% (Checking for such a bug is the whole purpose of the \code{assertNotNull}
% and \code{checkNotNull} methods.)

However, it is never useful for a programmer to pass \code{null}.  It is
the programmer's intention that \code{null} never flows there.  If
\code{null} does flow there, the program will not continue normally.

Therefore, you should mark such parameters as
\code{@\refclass{nullness/quals}{NonNull}}, indicating
the intended use of the method.  When you use the \code{@NonNull}
annotation, the checker is able to issue compile-time warnings about
possible run-time exceptions, which is its purpose.  Marking the parameter
as \code{@Nullable} would suppress such warnings, which is undesirable.

% (The note at
% http://google-collections.googlecode.com/svn/trunk/javadoc/com/google/common/base/Preconditions.html
% argues that the parameter could be marked as @Nullable, since it is
% possible for null to flow there at run time.  However, since that is an
% erroneous case, the annotation would be counterproductive rather than
% useful.)


\subsection{Subclasses must respect superclass annotations\label{annotations-are-a-contract}}

An annotation indicates a guarantee that a client can depend upon.  A subclass
is not permitted to \emph{weaken} the contract; for example,
if a method accepts \code{null} as an argument, then every overriding
definition must also accept \code{null}.
A subclass is permitted to \emph{strengthen} the contract; for example,
if a method does \emph{not} accept \code{null} as an argument, then an
overriding definition is permitted to accept \code{null}.

As a bad example, consider an erroneous \code{@Nullable} annotation at
line 141 of \ahref{http://code.google.com/p/google-collections/source/browse/trunk/src/com/google/common/collect/Multiset.java}{\code{com/google/common/collect/Multiset.java}}, version r78:

\begin{Verbatim}
101  public interface Multiset<E> extends Collection<E> {
...
122    /**
123     * Adds a number of occurrences of an element to this multiset.
...
129     * @param element the element to add occurrences of; may be {@code null} only
130     *     if explicitly allowed by the implementation
...
137     * @throws NullPointerException if {@code element} is null and this
138     *     implementation does not permit null elements. Note that if {@code
139     *     occurrences} is zero, the implementation may opt to return normally.
140     */
141    int add(@Nullable E element, int occurrences);
\end{Verbatim}

There exist implementations of Multiset that permit \code{null} elements,
and implementations of Multiset that do not permit \code{null} elements.  A
client with a variable \code{Multiset ms} does not know which variety of
Multiset \code{ms} refers to.  However, the \code{@Nullable} annotation
promises that \code{ms.add(null, 1)} is permissible.  (Recall from
Section~\ref{annotate-normal-behavior} that annotations should indicate
normal behavior.)

If parameter \code{element} on line 141 were to be annotated, the correct
annotation would be \code{@NonNull}.  Suppose a client has a reference to
same Multiset \code{ms}.  The only way the client can be sure not to throw an exception is to pass
only non-\code{null} elements to \code{ms.add()}.  A particular class
that implements Multiset could declare \code{add} to take a
\code{@Nullable} parameter.  That still satisfies the original contract.
It strengthens the contract by promising even more:  a client with such a
reference can pass any non-\code{null} value to \code{add()}, and may also
pass \code{null}.

\textbf{However}, the best annotation for line 141 is no annotation at all.
The reason is that each implementation of the Multiset interface should
specify its own nullness properties when it specifies the type parameter
for Multiset.  For example, two clients could be written as

\begin{Verbatim}
  class MyNullPermittingMultiset implements Multiset<@Nullable Object> { ... }
  class MyNullProhibitingMultiset implements Multiset<@NonNull Object> { ... }
\end{Verbatim}

\noindent
or, more generally, as

\begin{Verbatim}
  class MyNullPermittingMultiset<E extends @Nullable Object> implements Multiset<E> { ... }
  class MyNullProhibitingMultiset<E extends @NonNull Object> implements Multiset<E> { ... }
\end{Verbatim}

Then, the specification is more informative, and the Checker Framework is
able to do more precise checking, than if line 141 has an annotation.

It is a pleasant feature of the Checker Framework that in many cases, no
annotations at all are needed on type parameters such as \code{E} in \<MultiSet>.


\subsection{Annotations on constructor invocations\label{annotations-on-constructor-invocations}}

%% I want to get rid of this syntax.
%% However, @Linear provides a compelling use case.

In the checkers distributed with the Checker Framework, an annotation on a
constructor invocation is equivalent to a cast on a constructor result.
That is, the following two expressions have identical semantics:  one is
just shorthand for the other.

\begin{Verbatim}
  new @ReadOnly Date()
  (@ReadOnly Date) new Date()
\end{Verbatim}

However, you should rarely have to use this.  The Checker Framework will
determine the qualifier on the result, based on the ``return value''
annotation on the constructor definition.  The ``return value'' annotation
appears before the constructor name, for example:

\begin{Verbatim}
  class MyClass {
    @ReadOnly MyClass() { ... }
  }
\end{Verbatim}

In general, you should only use an annotation on a constructor invocation
when you know that the cast is
guaranteed to succeed.  An example from the IGJ checker
(Chapter~\ref{igj-checker}) is \<new @Immutable MyClass()> or \<new
@Mutable MyClass()>, where you know that every other reference to the class
is annotated \<@ReadOnly>.


\subsection{When to use (and not use) type qualifiers\label{when-to-use-type-qualifiers}}

For some programming tasks, you can use either a Java subclass or a type
qualifier.  For instance, suppose that your code currently uses
\code{String} to represent an address.  You could create a new \code{Address}
class and refactor your code to use it, or you could create a
\code{@Address} annotation and apply it to some uses of \code{String} in
your code.  If both of these are truly possible, then it is probably more
foolproof to use the Java class.  We do not encourage you to use type
qualifiers as a poor substitute for classes.  However, sometimes type
qualifiers are a better choice.

Using a new class may make your code incompatible with existing libraries or
clients.  Brian Goetz expands on this issues in an article on the
pseudo-typedef antipattern~\cite{Goetz2006:typedef}.  Even if compatibility
is not a concern, a code change may introduce bugs, whereas adding
annotations does not change the run-time behavior.  It is possible to add
annotations to existing code, including code you do not maintain or cannot
change.  It is possible to annotate primitive types without converting them
to wrappers, which would make the code both uglier and slower.

Type qualifiers can be applied to any type, including final classes that
cannot be subclassed.

Type qualifiers permit you to remove operations, with a compile-time
guarantee.  An example is mutating methods that are forbidden by immutable
types (see Chapters~\ref{igj-checker} and~\ref{javari-checker}).  More
generally, type qualifiers permit creating a new supertype, not just a
subtype, of an existing Java type.

% This is the least important reason.
A final reason is efficiency.  Type qualifiers can be more
efficient, since there is no run-time representation such as a wrapper
or a separate class, nor introduction of dynamic dispatch for methods that
could otherwise be statically dispatched.


\subsection{What to do if a checker issues a warning about your code\label{handling-warnings}}

When you first run a type-checker on your code, it is likely to issue
warnings or errors.  For each warning, try to understand why the checker
issues it.  (For example, if you are using the
\ahrefloc{nullness-checker}{Nullness checker}
(\chapterpageref{nullness-checker}), try to understand why it cannot prove
that no null pointer exception ever occurs.)  The reason will sometimes be
an actually possible null dereference, sometimes be a weakness of the
annotations, and sometimes be a weakness of the checker.  You will need to
examine your code, and possibly write test cases, to understand the reason.

If there is an actual possible null dereference, then fix your code to
prevent that crash.

If there is a weakness in the annotations, then improve the annotations.
For example, continuing the Nullness Checker example, if a particular
variable is annotated as \code{@\refclass{nullness/quals}{Nullable}} but it
actually never contains \<null> at run time, then change the annotation to 
\code{@\refclass{nullness/quals}{NonNull}}.  The weakness might be in the
annotations in your code, or in the annotations in a library that your code
calls.  Another possible problem is that a library is unannotated (see
\chapterpageref{annotating-libraries}).

If there is a weakness in the checker, then your code is safe --- it never
suffers the specific run-time error --- but the checker cannot prove this
fact.  This is most often because the checker is not omniscient, and some
tricky coding paradigms are beyond its analysis capabilities; in this
case, you should suppress the warning (see
\chapterpageref{suppressing-warnings}).  In other cases, the problem is a
bug in the checker; in this case, please report the bug (see
\chapterpageref{reporting-bugs}).


% LocalWords:  NonNull zipfile processor classfiles annotationname javac htoc
% LocalWords:  SuppressWarnings un skipUses java plugins plugin TODO cp igj
% LocalWords:  nonnull javari langtools sourcepath classpath OpenJDK pre jsr
% LocalWords:  Djsr quals Alint javac's dotequals nullable supertype JLS Papi
% LocalWords:  deserialization Mahmood Telmo Correa changelog txt nullness ESC
% LocalWords:  Nullness Xspacesincomments unselect checkbox unsetting PolyNull
% LocalWords:  bashrc IDE xml buildfile PolymorphicQualifier enum API elts INF
% LocalWords:  typechecker proc discoverable Xlint util QualifierDefaults Foo
% LocalWords:  DefaultQualifier DefaultQualifiers SoyLatte GetStarted Formatter
% LocalWords:  Dcheckers Warski MyClass ProcessorName compareTo toString myDate
% LocalWords:  ReadOnly readonlyObject int XDTA spacesincomments newdir Awarns
% LocalWords:  subpackages bak tIDE Multiset NullPointerException AskipUses
% LocalWords:  html JCIP MultiSet Astubs Afilenames Anomsgtext Ashowchecks tex
% LocalWords:  Aquals processorpath regex RegEx Xmaxwarns Xbootclasspath com
% LocalWords:  IntelliJ assertNotNull checkNotNull Goetz antipattern subclassed
% LocalWords:  callees Xmx unconfuse fenum propkey forName jsr308
% LocalWords:  bootclasspath

\htmlhr
\chapter{Nullness checker\label{nullness-checker}}

If the Nullness checker issues no warnings for a given program, then
running that program will never throw a null pointer exception.  This
guarantee enables a programmer to prevent errors from occurring when a
program is run.  See Section~\ref{nullness-checks} for more details about
the guarantee and what is checked.


\section{Nullness annotations\label{nullness-annotations}}

The Nullness checker uses three separate type hierarchies:  one for nullness,
one for rawness (Section~\ref{raw-partially-initialized}),
and one for map keys (Section~\ref{map-keys})
The Nullness checker has three varieties of annotations:  nullness
qualifiers, nullness method annotations, rawness qualifiers, and map key
qualifiers.

\subsection{Nullness qualifiers\label{nullness-qualifiers}}

The nullness hierarchy contains these qualifiers:

\begin{description}

\item[\code{@\refclass{nullness/quals}{Nullable}}]
  indicates a type that includes the null value.  For example, the type \code{Boolean}
  is nullable:  a variable of type \code{Boolean} always has one of the
  values \code{TRUE}, \code{FALSE}, or \code{null}.

\item[\code{@\refclass{nullness/quals}{NonNull}}]
  indicates a type that does not include the null value.  The type
  \code{boolean} is non-null; a variable of type \code{boolean} always has
  one of the values \code{true} or \code{false}.  The type \code{@NonNull
    Boolean} is also non-null:  a variable of type \code{@NonNull Boolean}
  always has one of the values \code{TRUE} or \code{FALSE} --- never
  \code{null}.  Dereferencing an expression of non-null type can never cause
  a null pointer exception.

  The \<@NonNull> annotation is rarely written in a program, because it is
  the default (see Section~\ref{null-defaults}).

\item[\code{@\refclass{nullness/quals}{PolyNull}}]
  indicates qualifier polymorphism.  For a description of
  \<@\refclass{nullness/quals}{PolyNull}>, see
  Section~\ref{qualifier-polymorphism}.

\item[\code{@\refclass{nullness/quals}{LazyNonNull}}]
  indicates a reference that may be \code{null}, but if it ever becomes
  non-\code{null}, then it never becomes \code{null} again.  This is
  appropriate for lazily-initialized fields, among other uses.  When the
  variable is read, its type is treated as
  \code{@\refclass{nullness/quals}{Nullable}}, but when the variable is
  assigned, its type is treated as
  \code{@\refclass{nullness/quals}{NonNull}}.

  Because the Nullness checker works intraprocedurally (it analyzes one
  method at a time), when a \code{LazyNonNull} field is first read within a
  method, the field cannot be assumed to be non-null.  The benefit of
  LazyNonNull over Nullable is its different interaction with
  flow-sensitive type qualifier refinement (Section~\ref{type-refinement}).
  After a check of a LazyNonNull
  field, all subsequent accesses \emph{within that method} can be assumed
  to be NonNull, even after arbitrary external method calls that have
  access to the given field.

\end{description}

Figure~\ref{fig:nonnull-hierarchy} shows part of the type hierarchy for the
Nullness type system.

\begin{figure}
\includeimage{nullness-and-raw}{2.5cm}
\caption{Partial type hierarchy for the Nullness type system.
Java's \<Object> is expressed as \<@Nullable Object>.  Programmers can omit
most type qualifiers, because the default annotation
(Section~\ref{null-defaults}) is usually correct.  Also shown is the
type hierarchy for rawness (Section~\ref{raw-partially-initialized}), which
indicates whether
initialization has completed.  The two type hierarchies are independent but
inter-related, and the Nullness Checker verifies them both.}
\label{fig:nonnull-hierarchy}
\end{figure}


\subsection{Nullness method annotations\label{nullness-non-qualifiers}}

The Nullness checker supports several annotations that specify method
behavior.

\begin{description}

\item[\code{@\refclass{nullness/quals}{NonNullOnEntry}}]
  indicates a method precondition:  The annotated method expects the
  specified variables (typically field references) to be non-null when the
  method is invoked.

\item[\code{@\refclass{nullness/quals}{AssertNonNullAfter}}]
\item[\code{@\refclass{nullness/quals}{AssertNonNullIfTrue}}]
\item[\code{@\refclass{nullness/quals}{AssertNonNullIfFalse}}]
  indicates a method postcondition.  With \<@AssertNonNullAfter>, the given
  expressions are non-null after the method returns; this is useful for a
  method that initializes a field, for example.  With
  \<@AssertNonNullIfTrue> and \<@AssertNonNullIfFalse>, if the annotated
  method returns the given boolean value (true or false), then the given
  expressions are non-null.  See Section~\ref{conditional-nullness} and the
  Javadoc for examples of their use.

\item[\code{@\refclass{nullness/quals}{Pure}}]
  indicates that the method has no (visible) side effects.  Furthermore,
  if the method is called multiple times with the same
  arguments, then it returns the same result.  This property cannot be
  assumed in general.  For example, suppose that the return value of method
  \code{m} is nullable.  Then this code will pass the type-checker:

\begin{Verbatim}
        if (m(arg) != null) {
          m(arg).toString();
        }
\end{Verbatim}

\noindent
only if method \code{m} is annotated as \code{@Pure}.

\item[\code{@\refclass{nullness/quals}{AssertParametersNonNull}}]
  % Indicates a method precondition:  The annotated method expects all of
  % its parameters to be non-null.
  is used for suppressing warnings, in very rare cases.  See the Javadoc for
  details.

\end{description}


\subsection{Rawness qualifiers\label{rawness-qualifiers}}

The Nullness Checker supports rawness annotations that indicate whether
an object is fully initialized --- that is, whether its fields have all
been assigned.

\begin{description}
\item[\code{@\refclass{nullness/quals}{Raw}}]
\item[\code{@\refclass{nullness/quals}{NonRaw}}]
\item[\code{@\refclass{nullness/quals}{PolyRaw}}]
\end{description}

Use of these annotations can help you to type-check more
code.  Figure~\ref{fig:nonnull-hierarchy} shows its type hierarchy.  For
details, see Section~\ref{raw-partially-initialized}.


\subsection{Map key qualifiers\label{map-key-qualifiers}}

The Nullness Checker supports a map key annotation, \code{@\refclass{nullness/quals}{KeyFor}} that indicates whether
a value is a key for a given map --- that is, whether
\code{map.containsKey(value)} would evaluate to \code{true}.

\begin{description}
\item[\code{@\refclass{nullness/quals}{KeyFor}}]
\end{description}

Use of this annotation can help you to type-check more code.  For details,
see Section~\ref{map-keys}.


\section{Writing nullness annotations\label{writing-nullness-annotations}}

\subsection{Implicit qualifiers\label{nullness-implicit-qualifiers}}

As described in Section~\ref{effective-qualifier}, the Nullness checker
adds implicit qualifiers, reducing the number of annotations that must
appear in your code.
For example, enum types are implicitly non-null, so you never need to write
\<@NonNull MyEnumType>.

For a complete description of all implicit nullness qualifiers, see the
Javadoc for \refclass{nullness}{NullnessAnnotatedTypeFactory}.



\subsection{Default annotation\label{null-defaults}}

Unannotated references are treated as if they had a default annotation,
using the NNEL (non-null except locals) rule described below.
A user may choose a different rule for defaults using the
\code{@\refclass{quals}{DefaultQualifier}} annotation; see
Section~\ref{defaults}.

%BEGIN LATEX
\begin{sloppy}
%END LATEX
Here are three possible default rules you may wish to use.  Other rules are
possible but are not as useful.
\begin{itemize}
\item
  \code{@\refclass{nullness/quals}{Nullable}}:  Unannotated types are regarded as possibly-null, or
  nullable.  This default is backward-compatible with Java, which permits
  any reference to be null.  You can activate this default by writing
  a \code{@DefaultQualifier("Nullable")} annotation on a
  % package/
  class or method
  % /variable
  declaration.
\item
  \code{@\refclass{nullness/quals}{NonNull}}:  Unannotated types are treated as non-null.
  % This may leads to fewer annotations written in your code.
  You can activate this
  default via the
  \code{@DefaultQualifier("NonNull")} annotation.
\item
  Non-null except locals (NNEL):  Unannotated types are treated as
  \code{@\refclass{nullness/quals}{NonNull}}, \emph{except} that the
  unannotated raw type of a local variable is treated as
  \code{@\refclass{nullness/quals}{Nullable}}.  (Any generic arguments to a
  local variable still default to
  \code{@\refclass{nullness/quals}{NonNull}}.)  This is the standard
  behavior.  You can explicitly activate this default via the
  \code{@DefaultQualifier(value="NonNull",
    locations=\discretionary{}{}{}\{DefaultLocation\discretionary{}{}{}.ALL\_EXCEPT\_LOCALS\})}
  annotation.

  The NNEL default leads to the smallest number of explicit annotations in
  your code~\cite{PapiACPE2008}.  It is what we recommend.  If you do not
  explicitly specify a different default, then NNEL is the default.
\end{itemize}
%BEGIN LATEX
\end{sloppy}
%END LATEX

\subsection{Conditional nullness\label{conditional-nullness}}

The Nullness Checker supports a form of conditional nullness types, via the
\code{@\refclass{nullness/quals}{AssertNonNullIfTrue}} and \code{@\refclass{nullness/quals}{AssertNonNullIfFalse}} method annotations.
The annotation on a method declares that some expressions are non-null, if
the method returns true.

Consider \sunjavadoc{java/io/File.html}{java.io.File}.
Method
\sunjavadoc{java/io/File.html#listFiles()}{File.listFiles()} may
return null, but is specified to return a non-null value if
\sunjavadoc{java/io/File.html#isDirectory()}{File.isDirectory()} is
true.  The same holds for method
\sunjavadoc{java/io/File.html#listFiles()}{File.list()}.
You can declare this relationship in the following way:

\begin{Verbatim}
  class File {

    @AssertNonNullIfTrue({"list()", "listFiles()"})
    public boolean isDirectory() { ... }

    public File @Nullable [] listFiles();
  }
\end{Verbatim}

A client that checks that a \code{File} reference is indeed that of a directory,
can then de-reference \code{File.isDirectory} safely without any nullness check.

\begin{Verbatim}
  static void analyze(File file) {
    if (file.isDirectory()) {
      for (File child : file.listFiles()) {  // no possible null dereference
        analyze(child);
      }
    } else {
        ... analyze file
    }
  }
\end{Verbatim}


\subsection{Inference of \code{@NonNull} and \code{@Nullable} annotations\label{nullness-inference}}

It can be tedious to write annotations in your code.  Tools exist that
can automatically infer annotations and insert them in your source code.
(This is different than type qualifier refinement for local variables
(Section~\ref{type-refinement}), which infers a more specific type for
local variables and uses them during type-checking but does not insert them
in your source code.  Type qualifier refinement is always enabled, no
matter how annotations on signatures got inserted in your source code.)

Your choice of tool depends on what default annotation (see
Section~\ref{null-defaults}) your code uses.  You only need one of these tools.

\begin{itemize}

\item
  Inference of \code{@\refclass{nullness/quals}{Nullable}}:
  %
  If your code uses the standard NNEL (non-null-except-locals) default or
  the \refclass{nullness/quals}{NonNull} default, then use the
  \ahref{http://groups.csail.mit.edu/pag/daikon/download/doc/daikon.html#AnnotateNullable}{AnnotateNullable}
  tool of the \ahref{http://pag.csail.mit.edu/daikon/}{Daikon invariant
    detector}.

\item
  Inference of \code{@\refclass{nullness/quals}{NonNull}}:
  %
  If your code uses the Nullable default, use one of these tools:
\begin{itemize}
\item
  \ahref{http://julia.scienze.univr.it:8080/julia/}{Julia analyzer},
\item
  \ahref{http://nit.gforge.inria.fr}{Nit: Nullability Inference Tool},
\item
  \ahref{http://jastadd.org/jastadd-tutorial-examples/non-null-types-for-java}{Non-null
    checker and inferencer} of the \ahref{http://jastadd.org/}{JastAdd
    Extensible Compiler}.
\end{itemize}

\end{itemize}



\section{What the Nullness checker checks\label{nullness-checks}}

The checker issues a warning in three cases:

\begin{enumerate}

\item
  When an expression of non-\code{@\refclass{nullness/quals}{NonNull}} type
  is dereferenced, because it might cause a null pointer exception.
  Dereferences occur not only when a field is accessed, but when an array
  is indexed, an exception is thrown, a lock is taken in a synchronized
  block, and more.  For a complete description of all checks performed by
  the Nullness checker, see the Javadoc for
  \refclass{nullness}{NullnessVisitor}.

\item
  When an expression of \code{@\refclass{nullness/quals}{NonNull}} type
  might become null, because it
  is a misuse of the type:  the null value could flow to a dereference that
  the checker does not warn about.

\item
  \label{lint-nulltest}
  When a null check is performed against a value that is guaranteed to be
  non-null, as in \code{("m" == null)}, because this might indicate a
  programmer error or misunderstanding, and is unnecessary.  This check is
  performed only if the \code{nulltest} lint option is enabled via the
  \code{-Alint=nulltest} command-line option.  The lint option is disabled
  by default because sometimes such checks are part of ordinary defensive
  programming.  See Section~\ref{suppressing-warnings} for more details
  about the \code{-Alint} command-line option.

\end{enumerate}

This example illustrates the programming errors that the checker detects:

\begin{Verbatim}
           Object   obj;  // might be null
  @NonNull Object nnobj;  // never null
  ...
  obj.toString()         // checker warning:  dereference might cause null pointer exception
  nnobj = obj;           // checker warning:  nnobj may become null
  if (nnobj == null)     // checker warning:  redundant test
\end{Verbatim}

Parameter passing and return values are checked analogously to assignments.

The Nullness Checker also checks the correctness, and correct use, of
rawness annotations for checking initialization.  See
Section~\ref{raw-partially-initialized}.


\section{Suppressing nullness warnings\label{suppressing-warnings-nullness}}

The Checker Framework supplies several ways to suppress warnings, most
notably the \<@SuppressWarnings("nullness")> annotation (see
Section~\ref{suppressing-warnings}).  An example use is

\begin{Verbatim}
    // might return null
    @Nullable Object getObject() { ... }

    void myMethod() {
      // The programmer knows that this partucular call never returns null.
      @SuppressWarnings("nullness")
      @NonNull Object o2 = getObject();
\end{Verbatim}


The Nullness Checker supports an additional warning suppression key,
\<nullness:collection-typeargs>.
Use of \<@SuppressWarnings("nullness:generic.argument")> causes the Nullness
Checker to suppress warnings related to misuse of generic type
arguments.  One use for this key is when a class is declared to take only
\<@NonNull> type arguments, but you want to instantiate the class with a
\<@Nullable> type argument, as in \code{List<@Nullable Object>}.  For a more
complete explanation of this example, see
Section~\refwithpage{faq-list-map-nonnull-typeargs}.

The Nullness Checker also permits you to use assertions or method calls to
suppress warnings; see below.

% TODO: check whether the SuppressWarnings keys are correct.


\subsection{Suppressing warnings with assertions and method calls\label{suppressing-warnings-with-assertions}}

Occasionally, it is inconvenient or
verbose to use the \<@SuppressWarnings> annotation.  For example, Java does
not permit annotations such as \<@SuppressWarnings> to appear on statements.

For situations when the \<@SuppressWarnings> annotation is inconvenient,
the Nullness Checker provides three additional ways to suppress warnings:
via an \<assert> statement, the \<castNonNull> method, and the
\<@AssertParametersNonNull> annotation.  These are
appropriate when the Nullness Checker issues a warning, but the programmer
knows for sure that the warning is a false positive, because the value
cannot ever be null at run time.

  % "@SuppressWarnings(nullness)" might be a better string.
  % That enables a programmer to search the source code to find all instances.
\newcommand{\nullnessSuppressionString}{nullness}

\begin{enumerate}
\item
  Use an assertion.  If the string ``\<\nullnessSuppressionString>''
  appears in the message body, then the Nullness Checker treats the
  assertion as suppressing a warning and assumes that the assertion always
  succeeds.  For example, the checker assumes that no null pointer
  exception can occur in code such as
\begin{Verbatim}
  assert x != null : "@SuppressWarnings(nullness)";
  ... x.f ...
\end{Verbatim}

  If the string ``\<\nullnessSuppressionString>'' does not appear in the
  assertion message, then the Nullness Checker treats the assertion as being
  used for defensive programming, and it warns if the method might throw a
  nullness-related exception.

  A downside of putting the string in the assertion message is that if the
  assertion ever fails, then a user might see the string and be confused.
  But the string should only be used if the programmer has reasoned that
  the assertion can never fail.

% (Another way of stating the Nullness checker's use of assertions is as an
% additional caveat to the guarantees provided by a checker
% (Section~\ref{checker-guarantees}).  The Nullness checker prevents null
% pointer errors in your code under the assumption that assertions are
% enabled, and it does not guarantee that all of your assertions succeed.)

\item
  Use the \refmethod{nullness}{NullnessUtils}{castNonNull}{(T)} method.

The Nullness
 Checker considers both the return value, and also the argument, to
 be non-null after the method call.  Therefore, the
 \<castNonNull> method can be used either as a cast expression or
 as a statement.  The Nullness Checker issues no warnings in any of
the following code:

\begin{Verbatim}
  // one way to use as a cast:
  @NonNull String s = castNonNull(possiblyNull1);

  // another way to use as a cast:
  castNonNull(possiblyNull2).toString();

  // one way to use as a statement:
  castNonNull(possiblyNull3);
  possiblyNull3.toString();`
\end{Verbatim}

  The method also throws \<AssertionError> if Java assertions are enabled and
  the argument is \<null>.  However, it is not intended for general defensive
  programming; see Section~\ref{defensive-programming}.

  A potential disadvantage of using the \<castNonNull> method is that your
  code becomes dependent on the Checker Framework at run time as well as at
  compile time.  You can avoid this by copying the implementation of
  \<castNonNull> into your own code, and possibly renaming it if you do not
  like the name.  Be sure to retain the documentation that indicates that
  your copy is intended for use only to suppress warnings and not for
  defensive programming.  See Section~\ref{defensive-programming} for an
  explanation of the distinction.

\item
  Use the \code{@\refclass{nullness/quals}{AssertParametersNonNull}}
  annotation.  It is used on \<castNonNull>, and may be used on other
  methods with the same semantics; it should probably never be used in any
  other situation.

\end{enumerate}


\subsection{Suppressing warnings on nullness-checking routines and defensive programming\label{defensive-programming}}

%% Work this in
% As explained in Section~\ref{annotate-normal-behavior}, annotations should
% indicate normal behavior that will not cause an exception.
%
% TODO: discuss how to write your own, and why the default doesn't have
% assert or checking methods suppress warnings.


One way to suppress warnings in the Nullness Checker is to use
method \code{castNonNull}.
(Section~\ref{suppressing-warnings-with-assertions} gives other techniques.)

This section explains why the Nullness Checker introduces a new method
rather than re-using the \<assert> statement (as in
\<assert x != null>) or an existing method such as:

\begin{Verbatim}
  org.junit.Assert.assertNotNull(Object)
  com.google.common.base.Preconditions.checkNotNull(Object)
\end{Verbatim}

In each case, the assertion or method indicates an application invariant --- a
fact that should always be true.  There are two distinct reasons a
programmer may have written the invariant, depending on whether the
programmer is 100\% sure that the application invariant holds.

\begin{enumerate}
\item
  A programmer might write it as \textbf{defensive programming}.  This causes
  the program to throw an exception, which is useful for debugging because
  it gives an earlier run-time indication of the error.
  A programmer would use an assertion in this way if the programmer is not
  100\% sure that the application invariant holds.

  % , or even to document what the program
  % is intended to do.

\item
  A programmer might write it to \textbf{suppress} false positive
  \textbf{warning messages} from a checker.  A programmer would use an
  assertion this way if the programmer is 100\% sure that the application
  invariant holds, and the reference can never be null at run time.

\end{enumerate}

With assertions and existing methods like JUnit's \<assertNotNull>, there
is no way of knowing the programmer's intent in using the method.
Different programmers or codebases may use them in different ways.
Guessing wrong would make the Nullness Checker less useful, because it
would either miss real errors or issue warnings where there is no real
error.  Also, different checking tools issue different false warnings that
need to be suppressed, so warning suppression needs to be customized for
each tool rather than inferred from general-purpose code.


As an example of using assertions for defensive programming, some style
guides suggest using assertions or method calls to indicate nullness.  A
programmer might write

\begin{Verbatim}
    String s = ...
    assert s != null;    // or:  assertNotNull(s);   or: checkNotNull(s);
    ... Double.valueOf(s) ...
\end{Verbatim}

A programming error might cause \<s> to be null, in which case the code
would throw an exception at run time.
If the assertion caused the Nullness Checker to assume that \<s> is not
\<null>, then the Nullness Checker would issue no warning for this code.
That would be undesirable, because the whole purpose of the Nullness
Checker is to give a compile-time warning about possible run-time
exceptions.  Furthermore, if the programmer uses assertions for defensive
programming systematically throughout the codebase, then many useful
Nullness Checker warnings would be suppressed.


Because it is important to distinguish between the two uses of assertions
(defensive programming vs.~suppressing warnings), the Checker Framework
introduces the \refmethod{nullness}{NullnessUtils}{castNonNull}{(T)} method.
Unlike existing assertions and
methods, \<castNonNull> is intended only to suppress false warnings that are
issued by the Nullness Checker, not for defensive programming.

If you know that a particular codebase uses
% the \<assert> statement or
a nullness-checking method not for defensive programming but to indicate
facts that are guaranteed to be true (that is, these assertions will never
fail at run time), then you can cause the Nullness Checker to suppress
warnings related to them, just as it does for \<castNonNull>.
Annotate its definition just as
\refmethod{nullness}{NullnessUtils}{castNonNull}{(T)} is annotated (see the
source code for the Checker Framework).
% TODO:
% For an assert statement, XXXXX.
Also, be sure to document the intention in the method's Javadoc, so that
programmers do not
accidentally misuse it for defensive programming.


If you are annotating a codebase that already contains precondition checks,
such as:

\begin{Verbatim}
  public String get(String key, String def) {
    checkNotNull(key, "key"); //NOI18N
    ...
  }
\end{Verbatim}

\noindent
then you should mark the appropriate parameter as \<@NonNull> (which is the
default).  This will prevent the checker from issuing a warning about the
\<checkNotNull> call.


\section{\code{@Raw} annotation for partially-initialized objects\label{raw-partially-initialized}}

The rawness hierarchy indicates whether an object is fully initialized ---
that is, whether its fields have all been assigned.  This is mostly
relevant within the constructor, or for references to \code{this} that
escape the constructor.  Most readers can skip this section on first
reading; you can return to it once you have mastered the rest of the
nullness checker.

The rawness hierarchy is independent of the
nullness hierarchy, and is shown in Figure~\ref{fig:nonnull-hierarchy}.
The rawness hierarchy contains these qualifiers:

\begin{description}

\item[\code{@\refclass{nullness/quals}{Raw}}]
  indicates a type that contains a partially-initialized object.  In a
  partially-initialized object, fields that are annotated as
  \code{@\refclass{nullness/quals}{NonNull}} may be null because the field
  has not yet been assigned.  Within the constructor,
  \code{this} has \code{@\refclass{nullness/quals}{Raw}} type until all
  the fields have been assigned.

% Cut this?
\item[\code{@\refclass{nullness/quals}{NonRaw}}]
  indicates a type that contains a fully-initialized object.  \code{NonRaw}
  is the default, so there is little need for a programmer to write this
  explicitly.

\item[\code{@\refclass{nullness/quals}{PolyRaw}}]
  indicates qualifier polymorphism over rawness (see
  Section~\ref{qualifier-polymorphism}).

\end{description}

Suppose a class contains a field ``\code{@\refclass{nullness/quals}{NonNull}
  Date d;}''.  Java executes the class's constructor by first setting \<d> to
\code{null}.  The constructor sets field \<d> to its final value, either
directly or by calling helper methods.  Before the constructor sets field
\<d>, its initial value \code{null} violates its type \code{@NonNull Date}.
In general, code can depend on field \<d> not being null, but not in a
partially-initialized object.  A partially-initialized object (\code{this}
in a constructor) may be passed to a helper method or stored in a variable;
if so, the method receiver, or the field, would have to be annotated as
\<@Raw>.

% However, if the constructor makes
% a method call (passing \code{this} as a parameter or the receiver), then
% the called method could observe the object in an illegal state.

The \code{@\refclass{nullness/quals}{Raw}} type annotation represents a
partially-initialized object.  If a reference has
\code{@Raw} type, then all of its \code{@NonNull} fields are treated as
\code{@\refclass{nullness/quals}{LazyNonNull}}:  when read, they are
treated as being \code{@\refclass{nullness/quals}{Nullable}}, but when
written, they are treated as being
\code{@\refclass{nullness/quals}{NonNull}}.


The rawness hierarchy is orthogonal to the nullness hierarchy.  It
is legal for a reference to be \<@NonNull @Raw>, \<@Nullable @Raw>,
\<@NonNull @NonRaw>, or \<@Nullable @NonRaw>.  The nullness hierarchy tells
you about the reference itself:  might the reference be null?  The rawness
hierarchy tells you about the \<@NonNull> fields in the referred-to object:
might those fields be temporarily null in contravention of their
declaration?


% Does our implementation handle static fields soundly?


You can suppress warnings related to partially-initialized objects with
\<@SuppressWarnings("rawness")>.  (Do not confuse this with the unrelated
\<@SuppressWarnings("rawtypes")> annotation for non-instantiated generic types!)


\paragraph{How an object becomes non-raw}

Within the constructor,
\code{this} starts out with \code{@\refclass{nullness/quals}{Raw}} type.
As soon as all of the \code{@\refclass{nullness/quals}{NonNull}} fields
have been initialized, then \code{this} is treated as non-raw.

The Nullness checker issues an error if the constructor fails to initialize
any non-null field.  This ensures that the object is in a legal (non-raw)
state by the time that the constructor exits.
\urldef{\jlsdefiniteassignmenturl}{\url}{http://java.sun.com/docs/books/jls/third_edition/html/defAssign.html}
\urldef{\jlsfinalvariablesurl}{\url}{http://java.sun.com/docs/books/jls/third_edition/html/typesValues.html#4.12.4}
This is different than Java's test for definite assignment (see
\ahref{\jlsdefiniteassignmenturl}{JLS ch.16}),
% , which requires that local
% variables (and blank \code{final} fields) must be assigned.  Java does not
% require that non-\code{final} fields be assigned, since
which does not apply to fields (except blank final ones, defined in
\ahref{\jlsdefiniteassignmenturl}{JLS \S 4.12.4}) because fields
have a default value of null.


% and can only be passed to methods when the corresponding parameter is
% annotated with \code{@\refclass{nullness/quals}{Raw}}.  Similar
% restrictions apply to assigning \code{this} to a field.

\paragraph{Invoking the superclass constructor; rawness of the superclass reference}

Suppose that class B extends class A.  Within the B
constructor, until the A superclass constructor is called, \code{this} has
type \code{@Raw B} and also \code{@Raw A}.  After the
superclass constructor has been exited, then \code{this} has type
\code{@Raw B} and also \code{@NonRaw A}.
By the time that the constructor exits, \<this> has type \<@NonRaw B> and
also \<@NonRaw A>.

When you write \<@Raw>, the annotation applies only to the given class, not
to any superclass.  For instance, the checker interprets \<@Raw B> as
``\code{@Raw B} and also \code{{\bfseries @NonRaw} A}'', rather than
``\code{@Raw B} and also \code{@Raw A}'', which would be less useful.  The
only exception is when a method overriding relationship forces the
superclass to also be raw.  For example:

\begin{Verbatim}
  class A extends Object {
    // receiver is "@NonRaw A"
    void nonRawAReceiver() { }
    // annotation forces receiver to be "@Raw A"; also is "@NonRaw Object"
    void rawAReciever() @Raw { }
  }

  class B extends A {
    // annotation forces receiver to be "@Raw B", method overriding forces "@Raw A"
    void rawAReceiver() @Raw {
      super.nonRawAReceiver();  // illegal!  rawness of A does not match
    }
    // annotation forces receiver to be "@Raw B"; also is "@NonRaw A"
    void rawBReceiver() @Raw {
      super.nonRawAReceiver();  // OK
    }
  }
\end{Verbatim}




% \urldef{\jlsconstructorbodyurl}{\url}{http://java.sun.com/docs/books/jls/third_edition/html/classes.html#8.8.7}
% (Recall that the superclass constructor is called on the first line, or is
% inserted automatically by the compiler before the first line, see
% \ahref{\jlsconstructorbodyurl}{JLS \S8.8.7}.)



% Should we change the terminology?
\paragraph{A note about the terminology ``raw''}

The name ``raw'' comes from a research paper that proposed this
approach~\cite{FahndrichL2003}.
A better name might have been ``not yet initialized'' or ``partially
initialized'', but the term ``raw'' is now well-known.
The \code{@\refclass{nullness/quals}{Raw}}
annotation has nothing to do with the raw types of Java Generics.


\section{Map key annotations\label{map-keys}}

Java's
\sunjavadoc{java/util/Map.html#get(java.lang.Object)}{\code{Map.get}}
method always has the possibility to return null, if the key is not in the
map.  Thus, to guarantee that the value returned from \code{Map.get} is
non-null, it is necessary that the map contains only non-null values,
\emph{and} the key is in the map.
The \code{@\refclass{nullness/quals}{KeyFor}} annotation states the latter
property.

If a type is annotated as \code{@KeyFor("m")}, then any value v with that type
is a key in Map m.  Another way of saying this is that the expression
\code{m.containsKey(v)} evaluates to true.

You usually do not have to write \code{@KeyFor} explicitly, because the
checker infers it based on usage patterns, such as calls to
\code{containsKey} or iteration over a map's
\sunjavadoc{java/util/Map.html#keySet()}{key set}.

One usage pattern where you \emph{do} have to write \<@KeyFor> is for a
user-managed collection that is a subset of the key set:

\begin{Verbatim}
Map<String, Object> m;
Set<@KeyOf("m") String> matchingKeys; // keys that match some criterion
for (String k : matchingKeys) {    // checker infers type of k to be: @KeyOf("m") String
  ... m.get(k) ...  // known to be non-null
}
\end{Verbatim}

As with any annotation, use of the \<@KeyFor> annotation may force you to
slightly refactor your code.  For example, this would be illegal:

\begin{Verbatim}
  Map<K,V> m;
  Collection<@KeyFor("m") K> coll;
  coll.add(x);
  ...            // at this point, the @KeyFor annotation is violated
  m.put(x, ...);
\end{Verbatim}

but this would be OK:

\begin{Verbatim}
  @KeysFor("m") Collection coll;
  m.put(x, ...);
  coll.add(x);
\end{Verbatim}


\section{Examples\label{nullness-example}}

\subsection{Tiny examples\label{nullness-tiny-examples}}

To try the Nullness checker on a source file that uses the \code{@\refclass{nullness/quals}{NonNull}} qualifier,
use the following command (where \code{javac} is the JSR 308 compiler that
is distributed with the Checker Framework):

\begin{Verbatim}
  javac -processor checkers.nullness.NullnessChecker examples/NullnessExample.java
\end{Verbatim}

\noindent
Compilation will complete without warnings.

To see the checker warn about incorrect usage of annotations (and therefore the
possibility of a null pointer exception at run time), use the following command:

\begin{Verbatim}
  javac -processor checkers.nullness.NullnessChecker examples/NullnessExampleWithWarnings.java
\end{Verbatim}

\noindent
The compiler will issue three warnings regarding violation of the semantics of
\code{@\refclass{nullness/quals}{NonNull}}.
% in the \code{NonNullExampleWithWarnings.java} file.


\subsection{Annotated library\label{nullness-annotated-library}}

Some libraries that or annotated with nullness qualifiers are:

\begin{itemize}
\item
The Nullness checker itself.

\item
The
\ahref{http://code.google.com/p/plume-lib/}{Plume-lib library}.
Run the command \code{make check-nullness}.


\item
The
\ahref{http://groups.csail.mit.edu/pag/daikon/}{Daikon invariant detector}.
Run the command \code{make check-nullness}.

% \item
% The annotation scene library.
% To run the Nullness checker on the annotation scene library,
% % TODO: how does one do this?
% first download the scene library suite (which includes build
% dependencies for the scene library as well as its source code) and extract it
% into your checkers installation. The checker can then be run on the annotation
% scene library with Apache Ant using the following commands:
%
% \begin{Verbatim}
%   cd checkers
%   ant -f scene-lib-test.xml
% \end{Verbatim}
%
% % \noindent
% % where \code{checkers} is the location of the Checker Framework installation.
%
% You can view the annotated source code, which contains \code{@\refclass{nullness/quals}{NonNull}} annotations, in
% the
% %BEGIN LATEX
% \begin{smaller}
% %END LATEX
% \code{checkers/scene-lib-test/src/annotations/}
% %BEGIN LATEX
% \end{smaller}
% %END LATEX
% directory.

\end{itemize}


\section{Other tools for nullness checking\label{nullness-related-work}}

\newcommand{\linktoNonNull}{\code{\refclass{nullness/quals}{NonNull}}}
\newcommand{\linktoNullable}{\code{\refclass{nullness/quals}{Nullable}}}

The Checker Framework's nullness annotation is similar to annotations used
in IntelliJ IDEA, FindBugs, JML, the JSR 305 proposal, and others.  Also
see Section~\ref{other-tools} for a comparison to other tools.

You might prefer to use the Checker Framework because it has a more
powerful analysis that can warn you about more null pointer errors in your
code.

If you have already annotated your code with a different nullness
annotation, you can reuse that effort by converting them to the Checker
Framework's nullness annotations.  Perform the refactoring described in
Figure~\ref{fig:findbugs-refactoring}.


\begin{figure}
\begin{center}
% The ~ around the text makes things look better in Hevea (and not terrible
% in LaTeX).
\begin{tabular}{ll}
\begin{tabular}{|l|}
\hline
 ~edu.umd.cs.findbugs.annotations.NonNull~ \\ \hline
 ~javax.annotation.Nonnull~ \\ \hline
 ~org.jetbrains.annotations.NotNull~ \\ \hline
\end{tabular}
&
$\Rightarrow$
~checkers.nullness.quals.NonNull~
\\
\
\\
\begin{tabular}{|l|l|}
\hline
 ~edu.umd.cs.findbugs.annotations.Nullable~ \\ \hline
 ~edu.umd.cs.findbugs.annotations.CheckForNull~ \\ \hline
 ~edu.umd.cs.findbugs.annotations.UnknownNullness~ \\ \hline
 ~javax.annotation.Nullable~ \\ \hline
 ~javax.annotation.CheckForNull~ \\ \hline
 ~org.jetbrains.annotations.Nullable~ \\ \hline
\end{tabular}
&
$\Rightarrow$
~checkers.nullness.quals.Nullable~
\end{tabular}
\end{center}
%BEGIN LATEX
\vspace{-1.5\baselineskip}
%END LATEX
\caption{Refactoring for converting nullness annotations from FindBugs, the
  JSR~305 proposal, and IntelliJ to the Checker Framework.}
\label{fig:findbugs-refactoring}
\end{figure}

%% Removed, because it's tedious and should be obvious to a decent programmer.
% Your IDE may be able to do that for you.  Alternately, do the following:
% \begin{enumerate}
% \item
%   replace \<@Nonnull> by \<@NonNull> (note capitalization difference)
% \item
%   replace \<@CheckForNull> by \<@Nullable>
% \item
%   replace \<@UnknownNullness> by \<@Nullable>
% \item
%   convert each single-type import statement (without a ``\<*>'' character)
%    according to the table above.
% \item
%   convert each on-demand import statements, such as ``\<import
%    edu.umd.cs.findbugs.annotations.*;>''.
% \begin{itemize}
%    \item
%   One approach is to change it into a set of single-type imports,
%       then convert the relevant ones.
%    \item
%   Another approach is to change it according to the table above, then
%       try to compile and re-introduce the single-type imports as necessary.
% \end{itemize}
%    These approaches let you continue to use other annotations in the
%    \<edu.umd.cs.findbugs.annotations> package, even though you are not using
%    its nullness annotations.
% \end{enumerate}


Alternately, the Checker Framework can process those other annotations (as
well as its own, if they also appear in your program).  The Checker
Framework has its own definition of the annotations on the left side of
Figure~\ref{fig:findbugs-refactoring}, so that they can be used as type
qualifiers.  The Checker Framework interprets them according to the right
side of Figure~\ref{fig:findbugs-refactoring}.

The Checker Framework may issue more or fewer errors than another tool.
This is expected, since each tool uses a different analysis.  Remember that
the Checker Framework aims at soundness:  it aims to never miss a possible
null dereference, while at the same time limiting false reports.

Because some of the names are the same (\<NonNull>, \<Nullable>), it is
unpleasant to use nullness annotations from multiple different packages in
the same codebase.  You can import at most one of the annotations with
conflicting names; the other(s) must be written out fully rather than
imported.  Also, note FindBugs's non-standard meaning for
\<@Nullable> (Section~\ref{findbugs-nullable}).


\subsection{Which tool is right for you?\label{choosing-nullness-tool}}

Different tools are appropriate in different circumstances.  Here is a
brief comparison with FindBugs, but similar points apply to other tools.

Checker Framework has a more powerful nullness analysis; FindBugs misses
some real
errors.  However, FindBugs does not require you to annotate your code as
thoroughly as the Checker Framework does.  Depending on the importance of
your code, you may wish to do no nullness checking; the cursory checking of
FindBugs; or the thorough checking of the Checker Framework.  You might
even want to ensure that both tools run, for example if your coworkers or
some other organization are still using FindBugs.  If you know that you
will eventually want to use the Checker Framework, there is no point using
FindBugs first; it is easier to go straight to using the Checker Framework.

FindBugs can find other errors in addition to nullness errors; here
we focus on its nullness checks.  Even if you use FindBugs for its other
features, you may want to use the Checker Framework for analyses that can
be expressed as pluggable type-checking, such as detecting nullness errors.

Regardless of whether you wish to use the FindBugs nullness analysis, you
may continue running all of the other FindBugs analyses at the same time as
the Checker Framework; there are no interactions among them.

If FindBugs (or any other tool) discovers a nullness error that the Checker
Framework does not, please report it to us (see
Section~\ref{reporting-bugs}) so that we can enhance the Checker Framework.



\subsection{Incompatibility note about FindBugs \tt{@Nullable}\label{findbugs-nullable}}

FindBugs has a non-standard definition of \<@Nullable>.  FindBugs's treatment is not
documented in its own
\ahref{http://findbugs.sourceforge.net/api/edu/umd/cs/findbugs/annotations/Nullable.html}{Javadoc};
it is different from the definition of \<@Nullable> in every other tool for
nullness analysis; it means tho same thing as \<@NonNull> when applied to a
formal parameter; and it inevitably surprises programmers.  Thus, FindBugs's
\<@Nullable> is detrimental rather than useful as documentation.
In practice, your best bet is to not rely on FindBugs for nullness analysis,
even if you find FindBugs useful for other purposes.

You can skip the rest of this section unless you wish to learn more details.

FindBugs suppresses all warnings at uses of a \<@Nullable> variable.
(You have to use \<@CheckForNull> to
indicate a nullable variable that FindBugs should check.)  For example:

\begin{Verbatim}
     // declare getObject() to possibly return null
     @Nullable Object getObject() { ... }

     void myMethod() {
       @Nullable Object o = getObject();
       // FindBugs issues no warning about calling toString on a possibly-null reference!
       o.toString();
     }
\end{Verbatim}

\noindent
The Checker Framework does not emulate this non-standard behavior of
FindBugs, even if the code uses FindBugs annotations.

FindBugs takes the approach of annotating a declaration, and thus
suppressing checking at \emph{all} client uses, even the places that you
want to check.
It is better to suppress warnings at only the specific client uses
where the value is known to be non-null; the Checker Framework supports
this, if you write \<@SuppressWarnings> at the client uses.
The Checker Framework also supports suppressing checking at all client uses,
by writing a \<@SuppressWarnings> annotation at the declaration site.

In general, the Checker Framework will issue more warnings than FindBugs,
and some of them may be about real bugs in your program.
See Section~\ref{suppressing-warnings-nullness} for information about
suppressing nullness warnings.

(FindBugs made a poor choice of names.  The choice of names should make a
clear distinction between annotations that specify whether a reference is
null, and annotations that suppress false warnings.  The choice of names
should also have been consistent for other tools, and intuitively clear to
programmers.  The FindBugs choices make the FindBugs annotations less
helpful to people, and much less useful for other tools.  The FindBugs
analysis is also very imprecise.  For type-related analyses, it is best to
stay away from the FindBugs nullness annotations, and use a more capable
tool like the Checker Framework.)



% As background, here is an explanation of the (sometimes surprising)
% semantics of the FindBugs nullness annotations.
%
%  * edu.umd.cs.findbugs.annotations.NonNull     javax.annotation.Nonnull
%    These mean the obvious thing:   the reference is never null.
%
%  * edu.umd.cs.findbugs.annotations.Nullable     javax.annotation.Nullable
%    This means that the value may be null, but that *all warnings should be
%    suppressed* regarding its use.  In other words, the value is really
%    nullable, but clients should treat it as non-null.  For example:
%
%      // declare getObject() to possibly return null
%      @Nullable Object getObject() { ... }
%
%      // FindBugs issues no warning about calling toString on a possibly-null reference
%      getObject().toString();
%
%    In the Checker Framework, this corresponds to declaring the method
%    return value as @Nullable, then suppressing warnings at client uses
%    that are known to be non-null.  An easy way to suppress the warning
%    is by adding an assert statement about the return value.
%
%    (Alternately, you could declare the method return value as @NonNull, and
%    suppress warnings within the method definition where it returns null,
%    but this approach is not recommended because the @NonNull annotation on
%    the return value would be misleading, and warnings should be suppressed
%    at particular sites where they are known to be unnecessary, not at all
%    call sites whatsoever.)
%
%  * edu.umd.cs.findbugs.annotations.CheckForNull      javax.annotation.CheckForNull
%    This means that the value may be null.  To avoid a NullPointerException,
%    every client should check nullness before dereferencing the value.
%    In the Checker Framework, this corresponds to @Nullable.



% LocalWords:  NonNull plugin quals un NonNullExampleWithWarnings java ahndrich
% LocalWords:  NotNull IntelliJ FindBugs Nullable TODO Alint nullable NNEL JSR
% LocalWords:  DefaultLocation Nullness PolyNull nullness AnnotateNullable JLS
% LocalWords:  Daikon JastAdd javac DefaultQualifier boolean MyEnumType NonRaw
% LocalWords:  NullnessAnnotatedTypeFactory NullnessVisitor LazyNonNull PolyRaw
% LocalWords:  inferencer Nonnull CheckForNull UnknownNullness rawtypes de ch
% LocalWords:  castNonNull NullnessUtils assertNotNull codebases checkNotNull
% LocalWords:  Nullability typeargs nulltest AssertNonNullIfTrue listFiles faq
% LocalWords:  isDirectory AssertionError intraprocedurally SuppressWarnings
% LocalWords:  FindBugs's getObject NonNullOnEntry AssertNonNullIfFalse
% LocalWords:  AssertParametersNonNull nonnull

\htmlhr
\chapter{Interning Checker\label{interning-checker}}

\urldef{\jlsboxingurl}{\url}{http://docs.oracle.com/javase/specs/jls/se7/html/jls-5.html#jls-5.1.7}

If the Interning Checker issues no errors for a given program, then all
reference equality tests (i.e., all uses of ``\code{==}'') are proper;
that is,
\code{==} is not misused where \code{equals()} should have been used instead.

Interning is a design pattern in which the same object is used whenever two
different objects would be considered equal.  Interning is also known as
canonicalization or hash-consing, and it is related to the flyweight design
pattern.
Interning has two benefits:  it can save memory, and it can speed up testing for
equality by permitting use of \code{==}.

The Interning Checker prevents two types of errors in your code.  First, 
\code{==} should be used
only on interned values; using \code{==} on
non-interned values can result in subtle bugs.  For example:

\begin{Verbatim}
  Integer x = new Integer(22);
  Integer y = new Integer(22);
  System.out.println(x == y);  // prints false!
\end{Verbatim}

\noindent
The Interning Checker helps programmers to prevent such bugs.
Second, 
the Interning Checker also helps to prevent performance problems that result
from failure to use interning.
(See Section~\ref{checker-guarantees} for caveats to the checker's guarantees.)

Interning is such an important design pattern that Java builds it in for
these types: \<String>, \<Boolean>, \<Byte>, \<Character>, \<Integer>,
\<Short>.  Every string literal in the program is guaranteed to be interned
(\ahref{\url{http://docs.oracle.com/javase/specs/jls/se7/html/jls-3.html#jls-3.10.5}}{JLS
  \S3.10.5}), and the
\sunjavadoc{java/lang/String.html#intern()}{String.intern()} method
performs interning for strings that are computed at run time.
The \<valueOf> methods in wrapper classes always (\<Boolean>, \<Byte>) or
sometimes (\<Character>, \<Integer>, \<Short>) return an interned result
(\ahref{\jlsboxingurl}{JLS \S5.1.7}).
Users can also write their own interning methods for other types.

It is a proper optimization to use \code{==}, rather than \code{equals()},
whenever the comparison is guaranteed to produce the same result --- that
is, whenever the comparison is never provided with two different objects
for which \code{equals()} would return true.  Here are three reasons that
this property could hold:

\begin{enumerate}
\item
  Interning.  A factory method ensures that, globally, no two different
  interned objects are \code{equals()} to one another.  (In some cases
  other, non-interned objects of the class might be \code{equals()} to one
  another; in other cases, every object of the class is interned.)
  Interned objects should always be immutable.
\item
  Global control flow.  The program's control flow is such that the
  constructor for class $C$ is called a limited number of times, and with
  specific values that ensure the results are not \code{equals()} to one
  another.  Objects of class $C$ can always be compared with \code{==}.
  Such objects may be mutable or immutable.
\item
  Local control flow.  Even though not all objects of the given type may be
  compared with \code{==}, the specific objects that can reach a given
  comparison may be.  For example, suppose that an array contains no
  duplicates.  Then testing to find the index of a given element that is
  known to be in the array can use \code{==}.
\end{enumerate}

To eliminate Interning Checker errors, you will need to annotate the
declarations of any expression used as an argument to \code{==}.
Thus, the Interning Checker
could also have been called the Reference Equality Checker.  In the
future, the checker will include annotations that target the non-interning
cases above, but for now you need to use \<@Interned>, \<@UsesObjectEquals>
(which handles a surprising number of cases), and/or
\<@SuppressWarnings>.

To run the Interning Checker, supply the \code{-processor
  checkers.interning.InterningChecker} command-line option to javac.  For
examples, see Section~\ref{interning-example}.


\section{Interning annotations\label{interning-annotations}}

These qualifiers are part of the Interning type system:

\begin{description}

\item[\code{@\refclass{checkers/interning/quals}{Interned}}]
  indicates a type that includes only interned values (no non-interned
  values).

\item[\<@\refclass{checkers/interning/quals}{PolyInterned}>]
  indicates qualifier polymorphism.  For a description of
  \<@\refclass{checkers/interning/quals}{PolyInterned}>, see
  Section~\ref{qualifier-polymorphism}.

\item[\<@\refclass{checkers/interning/quals}{UsesObjectEquals}>]
  is a class (not type) annotation that indicates that this class's
  \<equals> method is the same as that of \<Object>.  In other words,
  neither this class nor any of its superclasses overrides the \<equals>
  method.  Since \<Object.equals> uses reference equality, this means that
  for such a class, \<==> and \<equals> are equivalent, and so the
  Interning Checker does not issue errors or warnings for either one.

\end{description}


\section{Annotating your code with \code{@Interned}\label{annotating-with-interned}}

\begin{figure}
\includeimage{interning}{2.5cm}
\caption{Type hierarchy for the Interning type system.}
\label{fig-interning-hierarchy}
\end{figure}

In order to perform checking, you must annotate your code with the \code{@\refclass{checkers/interning/quals}{Interned}}
type annotation, which indicates a type for the canonical representation of an
object:

\begin{Verbatim}
            String s1 = ...;  // type is (uninterned) "String"
  @Interned String s2 = ...;  // Java type is "String", but checker treats it as "Interned String"
\end{Verbatim}

The type system enforced by the checker plugin ensures that only interned
values can be assigned to \code{s2}.

To specify that \emph{all} objects of a given type are interned, annotate the
class declaration:

\begin{Verbatim}
  public @Interned class MyInternedClass { ... }
\end{Verbatim}

This is equivalent to annotating every use of \code{MyInternedClass}, in a
declaration or elsewhere.  For example, \code{enum} classes are implicitly
so annotated.


\subsection{Implicit qualifiers\label{interning-implicit-qualifiers}}

As described in Section~\ref{effective-qualifier}, the Interning Checker
adds implicit qualifiers, reducing the number of annotations that must
appear in your code.
For example, String literals and the null literal are always considered interned, and
object creation expressions (using \code{new}) are never considered
\code{@\refclass{checkers/interning/quals}{Interned}} unless they are annotated as such, as in

%BEGIN LATEX
\begin{smaller}
%END LATEX
\begin{Verbatim}
@Interned Double internedDoubleZero = new @Interned Double(0); // canonical representation for Double zero
\end{Verbatim}
%BEGIN LATEX
\end{smaller}
%END LATEX

For a complete description of all implicit interning qualifiers, see the
Javadoc for \refclass{checkers/interning}{InterningAnnotatedTypeFactory}.


\section{What the Interning Checker checks\label{interning-checks}}

Objects of an \code{@\refclass{checkers/interning/quals}{Interned}} type may be safely compared using the ``\code{==}''
operator.

The checker issues an error in two cases:

\begin{enumerate}

\item
  When a reference (in)equality operator (``\code{==}'' or ``\code{!=}'')
  has an operand of non-\code{@\refclass{checkers/interning/quals}{Interned}} type.

\item
  When a non-\code{@\refclass{checkers/interning/quals}{Interned}} type is used where an \code{@\refclass{checkers/interning/quals}{Interned}} type
  is expected.

\end{enumerate}

This example shows both sorts of problems:

\begin{Verbatim}
            Object  obj;
  @Interned Object iobj;
  ...
  if (obj == iobj) { ... }  // error: reference equality test is unsafe
  iobj = obj;               // error: iobj's referent may no longer be interned
\end{Verbatim}

\label{lint-dotequals}

The checker also issues a warning when \code{.equals} is used where
\code{==} could be safely used.  You can disable this behavior via the
javac \code{-Alint} command-line option, like so: \code{-Alint=-dotequals}.

For a complete description of all checks performed by
  the checker, see the Javadoc for
  \refclass{checkers/interning}{InterningVisitor}.

\label{checking-class}
You can also restrict which types the checker should examine and type-check,
using the \code{-Acheckclass} option.  For example, to find only the
interning errors related to uses of \code{String}, you can pass
\code{-Acheckclass=java.lang.String}.  The Interning Checker always checks all
subclasses and superclasses of the given class.


\subsection{Limitations of the Interning Checker\label{interning-limitations}}

% There is no point to linking to the Javadoc for the valueOf methods,
% which don't discuss interning.

The Interning Checker conservatively assumes that the \<Character>, \<Integer>,
and \<Short> \<valueOf> methods return a non-interned value.  In fact, these
methods sometimes return an interned value and sometimes a non-interned
value, depending on the run-time argument (\ahref{\jlsboxingurl}{JLS
\S5.1.7}).  If you know that the run-time argument to \<valueOf> implies that
the result is interned, then you will need to suppress an error.  (An
alternative would be to enhance the Interning Checker to estimate the upper
and lower bounds on char, int, and short values so that it can more
precisely determine whether the result of a given \<valueOf> call is
interned.)



\section{Examples\label{interning-example}}

To try the Interning Checker on a source file that uses the \code{@\refclass{checkers/interning/quals}{Interned}} qualifier,
use the following command (where \code{javac} is the JSR 308 compiler that
is distributed with the Checker Framework):

\begin{Verbatim}
  javac -processor checkers.interning.InterningChecker examples/InterningExample.java
\end{Verbatim}

\noindent
Compilation will complete without errors or warnings.

To see the checker warn about incorrect usage of annotations, use the following
command:

\begin{Verbatim}
  javac -processor checkers.interning.InterningChecker examples/InterningExampleWithWarnings.java
\end{Verbatim}

\noindent
The compiler will issue an error regarding violation of the semantics of
\code{@\refclass{checkers/interning/quals}{Interned}}.
% in the \code{InterningExampleWithWarnings.java} file.


The Daikon invariant detector
(\myurl{http://groups.csail.mit.edu/pag/daikon/}) is also annotated with
\code{@\refclass{checkers/interning/quals}{Interned}}.  From directory \code{java},
run \code{make check-interning}.



\section{Other interning annotations\label{other-interning-annotations}}

The Checker Framework's interning annotations are similar to annotations used
elsewhere.

If your code is already annotated with a different interning
annotation, you can reuse that effort.  The Checker Framework comes with
cleanroom re-implementations of annotations from other tools.  It treats
them exactly as if you had written the corresponding annotation from the
Interning Checker, as described in Figure~\ref{fig-interning-refactoring}.


% These lists should be kept in sync with InterningAnnotatedTypeFactory.java .
\begin{figure}
\begin{center}
% The ~ around the text makes things look better in Hevea (and not terrible
% in LaTeX).
\begin{tabular}{ll}
\begin{tabular}{|l|}
\hline
 ~com.sun.istack.Interned~ \\ \hline
\end{tabular}
&
$\Rightarrow$
~checkers.interning.quals.Interned~
\end{tabular}
\end{center}
%BEGIN LATEX
\vspace{-1.5\baselineskip}
%END LATEX
\caption{Correspondence between other interning annotations and the
  Checker Framework's annotations.}
\label{fig-interning-refactoring}
\end{figure}

Alternately, the Checker Framework can process those other annotations (as
well as its own, if they also appear in your program).  The Checker
Framework has its own definition of the annotations on the left side of
Figure~\ref{fig-interning-refactoring}, so that they can be used as type
qualifiers.  The Checker Framework interprets them according to the right
side of Figure~\ref{fig-interning-refactoring}.



% LocalWords:  plugin MyInternedClass enum InterningExampleWithWarnings java
% LocalWords:  PolyInterned Alint dotequals quals InterningAnnotatedTypeFactory
% LocalWords:  javac InterningVisitor JLS Acheckclass UsesObjectEquals 5cm
%  LocalWords:  consing valueOf superclasses s2 cleanroom

\htmlhr
\chapter{IGJ immutability checker\label{igj-checker}}
% Use until that checker's manual chapter is reinstated.
\label{oigj-checker}

\textbf{Note:} The IGJ type-checker has some known bugs and limitations.
Nonetheless, it may still be useful to you.

IGJ is a Java language extension that helps programmers to avoid mutation errors
(unintended side effects).
If the IGJ Checker issues no warnings for a given program, then that program
will never change objects that should not be changed.  This guarantee
enables a programmer to detect and prevent mutation-related errors.
(See Section~\ref{checker-guarantees} for caveats to the guarantee.)

To run the IGJ Checker, supply the \code{-processor org.checkerframework.checker.igj.IGJChecker}
command-line option to javac.  For examples, see Section~\ref{igj-example}.


\section{IGJ and Mutability\label{igj-and-mutability}}

IGJ~\cite{ZibinPAAKE2007} permits a
programmer to express that a particular object should never be modified via any
reference (object immutability), or that a reference should never be used to
modify its referent (reference immutability). Once a programmer has expressed
these facts, an automatic checker analyzes the code to either locate mutability
bugs or to guarantee that the code contains no such bugs.

\begin{figure}
\includeimage{igj}{5cm}
\caption{Type hierarchy for three of IGJ's type qualifiers.}
\label{fig-igj-hierarchy}
\end{figure}

To learn more details of the IGJ language and type system, please see the
ESEC/FSE 2007 paper ``\href{http://homes.cs.washington.edu/~mernst/pubs/immutability-generics-fse2007-abstract.html}{Object and reference immutability using Java
generics}''~\cite{ZibinPAAKE2007}.
The IGJ Checker supports Annotation IGJ (Section~\ref{annotation-igj-dialect}),
which is a slightly different dialect
of IGJ than that described in the ESEC/FSE paper.


\section{IGJ Annotations\label{igj-annotations}}

Each object is either immutable (it can never be modified) or mutable (it
can be modified).  The following qualifiers are part of the IGJ type system.

\begin{description}

\item[\refqualclass{checker/igj/qual}{Immutable}]
  An immutable reference always refers to an immutable object.  Neither the
  reference, nor any aliasing reference, may modify the object.

\item[\refqualclass{checker/igj/qual}{Mutable}]
  A mutable reference refers to a mutable object.  The reference, or some
  aliasing mutable reference, may modify the object.

\item[\refqualclass{checker/igj/qual}{ReadOnly}]
  A readonly reference cannot be used to modify its referent.  The referent
  may be an immutable or a mutable object.  In other words, it is possible
  for the referent to change via an aliasing mutable reference, even though
  the referent cannot be changed via the readonly reference.

\item[\refqualclass{checker/igj/qual}{Assignable}]
  The annotated field may be re-assigned regardless of the
  immutability of the enclosing class or object instance.

\item[\refqualclass{checker/igj/qual}{AssignsFields}]
  is similar to \<@Mutable>, but permits only limited mutation ---
  assignment of fields --- and is intended for use by constructor helper
  methods.  \<@AssignsFields> is assumed to be true of the result of a
  constructor, so it does not need to be written there.

\item[\refqualclass{checker/igj/qual}{I}]
  simulates mutability overloading or the template behavior of generics.
  It can be applied to classes, methods, and parameters.  See
  Section~\ref{igj-templating}.

\end{description}

For additional details, see~\cite{ZibinPAAKE2007}.


\section{What the IGJ Checker checks\label{igj-checks}}

The IGJ Checker issues an error whenever mutation happens through a
readonly reference, when fields of a readonly reference which are not
explicitly marked with \refqualclass{checker/igj/qual}{Assignable} are
reassigned, or when a readonly reference is assigned to a mutable
variable.  The checker also emits a warning when casts increase the
mutability access of a reference.

% There is no visitor to reference!
% For a complete description of all checks performed by
% the checker, see the Javadoc for \refclass{checker/igj}{IGJVisitor}.


\section{Implicit and default qualifiers\label{igj-implicit-qualifiers}}

As described in Section~\ref{effective-qualifier}, the IGJ Checker
adds implicit qualifiers, reducing the number of annotations that must
appear in your code.
% For example, ...

For a complete description of all implicit IGJ qualifiers, see the
Javadoc for \refclass{checker/igj}{IGJAnnotatedTypeFactory}.

The default annotation (for types that are unannotated and not given an
implicit qualifier) is as follows:
\begin{itemize}
\item
  \code{@Mutable} for almost all references.  This is backward-compatible
  with Java, since Java permits any reference to be mutated.
\item
  \code{@Readonly} for local variables.  This qualifier may be refined by
  flow-sensitive local type refinement (see Section~\ref{type-refinement}).
\item
  \code{@Readonly} for type parameter and wildcard bounds.  For example,

\begin{Verbatim}
  interface List<T extends Object> { ... }
\end{Verbatim}

\noindent
is defaulted to

\begin{Verbatim}
  interface List<T extends @Readonly Object> { ... }
\end{Verbatim}

This default is not backward-compatible --- that is, you may have to
explicitly add \code{@Mutable} annotations to some type parameter bounds in
order to make unannotated Java code type-check under IGJ\@.  However, this
reduces the number of annotations you must write overall (since most
variables of generic type are in fact not modified), and permits more
client code to type-check (otherwise a client could not write
\code{List<@Readonly Date>}).

\end{itemize}



\section{Annotation IGJ Dialect\label{annotation-igj-dialect}}

The IGJ Checker supports the Annotation IGJ dialect of IGJ\@.  The syntax of
Annotation IGJ is based on type annotations.

The syntax of the original IGJ
dialect~\cite{ZibinPAAKE2007} was based on Java 5's generics and annotation mechanisms. The original
IGJ dialect was not backward-compatible with Java (either syntactically or
semantically). The dialect of IGJ checked by the IGJ Checker corrects these
problems.

The differences between the Annotation IGJ dialect and the original IGJ dialect
are as follows.

\subsection{Semantic Changes}

\begin{itemize}

\item
  Annotation IGJ does not permit covariant changes in generic type
  arguments, for backward compatibility with Java.  In ordinary Java, types
  with different generic type arguments, such as \code{Vector<Integer>} and
  \code{Vector<Number>}, have no subtype relationship, even if the
  arguments (\code{Integer} and \code{Number}) do. The original IGJ dialect
  changed the Java subtyping rules to permit safely varying a type argument
  covariantly in certain circumstances. For example,

\begin{Verbatim}
  Vector<Mutable, Integer>  <:  Vector<ReadOnly, Integer>
                            <:  Vector<ReadOnly, Number>
                            <:  Vector<ReadOnly, Object>
\end{Verbatim}

is valid in IGJ, but in Annotation IGJ, only

\begin{Verbatim}
  @Mutable Vector<Integer>  <:  @ReadOnly Vector<Integer>
\end{Verbatim}

holds and the other two subtype relations do not hold

\begin{Verbatim}
  @ReadOnly Vector<Integer> </:  @ReadOnly Vector<Number>
                            </:  @ReadOnly Vector<Object>
\end{Verbatim}


\item
  Annotation IGJ supports array immutability. The original IGJ dialect did
  not permit the (im)mutability of array elements to be specified, because
  the generics syntax used by the original IGJ dialect cannot be applied to
  array elements.

\end{itemize}

\subsection{Syntax Changes}

\begin{itemize}

\item  Immutability is specified through
  \href{http://types.cs.washington.edu/jsr308/}{type annotations}~\cite{JSR308-2008-09-12} (Section~\ref{igj-annotations}),
not through a combination of generics and annotations.  Use of type
annotations makes Annotation IGJ backward compatible with Java syntax.

\item Templating over Immutability: The annotation \refqualclass{checker/igj/qual}{I}\<(>\emph{\<id>}\<)> is used to template
over immutability.  See Section~\ref{igj-templating}.

\end{itemize}


\subsection{Templating Over Immutability: \code{@I}\label{igj-templating}}

\refqualclass{checker/igj/qual}{I} is a template annotation over IGJ Immutability annotations. It acts
similarly to type variables in Java's generic types, and the name
\refqualclass{checker/igj/qual}{I} mimics the standard \code{<I>} type variable name used in code
written in the original IGJ dialect.  The annotation value string is used
to distinguish between multiple instances of \refqualclass{checker/igj/qual}{I} --- in the
generics-based original dialect, these would be expressed as two type
variables \code{<I>} and \code{<J>}.

\paragraph{Usage on classes\label{igj-usage-on-classes}}

A class declaration annotated with \refqualclass{checker/igj/qual}{I} can then be
used with any IGJ Immutability annotation.  The actual immutability that
\refqualclass{checker/igj/qual}{I} is resolved to dictates the immutability type for all the non-static
appearances of \refqualclass{checker/igj/qual}{I} with the same value as the class declaration.

  Example:
\begin{Verbatim}
    @I
    public class FileDescriptor {
       private @Immutable Date creationData;
       private @I Date lastModData;

       public @I Date getLastModDate(@ReadOnly FileDescriptor this) { }
    }

    ...
    void useFileDescriptor() {
       @Mutable FileDescriptor file =
                         new @Mutable FileDescriptor(...);
       ...
       @Mutable Data date = file.getLastModDate();

    }
\end{Verbatim}

In the last example, \refqualclass{checker/igj/qual}{I} was resolved to \refqualclass{checker/igj/qual}{Mutable} for the instance file.

\paragraph{Usage on methods\label{igj-usage-on-methods}}

For example, it could be used for method parameters, return values, and the
actual IGJ immutability value would be resolved based on the method invocation.

For example, the below method \code{getMidpoint} returns a \code{Point} with the same
immutability type as the passed parameters if \code{p1} and \code{p2} match
in immutability, otherwise \refqualclass{checker/igj/qual}{I} is resolved to \refqualclass{checker/igj/qual}{ReadOnly}:

\begin{Verbatim}
  static @I Point getMidpoint(@I Point p1, @I Point p2) { ... }
\end{Verbatim}

The \refqualclass{checker/igj/qual}{I} annotation value distinguishes between \refqualclass{checker/igj/qual}{I}
declarations.  So, the below method \code{findUnion} returns a collection of the same
immutability type as the \emph{first} collection parameter:

\begin{Verbatim}
  static <E> @I("First") Collection<E> findUnion(@I("First") Collection<E> col1,
                                                 @I("Second") Collection<E> col2) { ... }
\end{Verbatim}


\section{Iterators and their abstract state\label{igj-library-annotations}}

This section explains why the receiver of \<Iterator.next()> is annotated
as \<@ReadOnly>.

An iterator conceptually has two pieces of state:
\begin{enumerate}
\item
  the underlying collection
\item
  an index into that collection (indicating the next object to be returned)
\end{enumerate}

We choose to exclude the index from the abstract state of the iterator.
That is, a change to the index does not count as a mutation of the
iterator itself.

Changes to the underlying collection are more important and interesting,
and unintentional changes are much more likely to lead to important
errors.  Therefore, this choice about the iterator's abstract state
appears to be more useful than other choices.  For example, if the
iterator's abstract state included both the underlying collection and
the index, then there would be no way to express, or check, that
\<Iterator.next> does not change the underlying collection.


\section{Examples\label{igj-example}}

To try the IGJ Checker on a source file that uses the IGJ qualifier, use
the following command (where \code{javac} is the Checker Framework compiler that
is distributed with the Checker Framework).

\begin{Verbatim}
  javac -processor org.checkerframework.checker.igj.IGJChecker examples/IGJExample.java
\end{Verbatim}

The IGJ Checker itself is also annotated with IGJ annotations.


% LocalWords:  plugin ReadOnly AssignsFields im templating getMidpoint cp TODO
% LocalWords:  findUnion igj IGJ's quals ESEC readonly covariant 5cm 5's
% LocalWords:  NullnessAnnotatedTypeFactory IGJAnnotatedTypeFactory p1 p2

\htmlhr
\chapter{Javari immutability checker\label{javari-checker}}

Javari~\cite{TschantzE2005,QuinonezTE2008} is a Java language extension that helps programmers to avoid mutation
errors that result from unintended side effects.
If the Javari checker issues no warnings for a given program, then that
program will never change objects that should not be changed.  This
guarantee enables a programmer to detect and prevent mutation-related
errors.  (See Section~\ref{checker-guarantees} for caveats to the guarantee.)
The Javari webpage (\myurl{http://types.cs.washington.edu/javari/}) contains
papers that explain the Javari language and type system.
By contrast to those papers, the Javari checker uses an annotation-based
dialect of the Javari language.

The Javarifier tool infers Javari types for an existing program; see
Section~\ref{javari-inference}.

Also consider the IGJ checker (Chapter~\ref{igj-checker}).  The IGJ type
system is more expressive than that of Javari, and the IGJ checker is a bit
more robust.  However, IGJ lacks a type inference tool such as Javarifier.

To run the Javari Checker, supply the \code{-processor
  checkers.javari.JavariChecker} command-line option to javac.  For
examples, see Section~\ref{javari-examples}.



\begin{figure}
\includeimage{javari}{2.5cm}
\caption{Type hierarchy for Javari's ReadOnly type qualifier.}
\label{fig:javari-hierarchy}
\end{figure}


\section{Javari annotations\label{javary-annotations}}

The following six annotations make up the Javari type system.

\begin{description}

\item[\code{@\refclass{javari/quals}{ReadOnly}}]
  indicates a type that provides only read-only access.  A reference of
  this type may not be used to modify its referent, but aliasing references
  to that object might change it.

\item[\code{@\refclass{javari/quals}{Mutable}}]
  indicates a mutable type.
  
\item[\code{@\refclass{javari/quals}{Assignable}}]
  is a field annotation, not a type qualifier.  It indicates that the given
  field may always be assigned, no matter what the type of the reference
  used to access the field.
  
\item[\code{@\refclass{javari/quals}{QReadOnly}}]
  corresponds to Javari's ``\code{?\ readonly}'' for wildcard types.  An
  example of its use is \code{List<@QReadOnly Date>}.  It allows only the
  operations which are allowed for both readonly and mutable types.

\item[\code{@\refclass{javari/quals}{PolyRead}}]
  (previously named \code{@RoMaybe}) specifies polymorphism over
  mutability; it simulates mutability overloading.  It can be applied to
  methods and parameters.  See Section~\ref{qualifier-polymorphism} and the
  \code{@\refclass{javari/quals}{PolyRead}} Javadoc for more details.

\item[\code{@\refclass{javari/quals}{ThisMutable}}]
  means that the mutability of the field is the same as that of the
  reference that contains it.  \code{@ThisMutable} is the default on
  fields, and does not make sense to write elsewhere.  Therefore,
  \code{@ThisMutable} should never appear in a program.

\end{description}


\section{Writing Javari annotations\label{writing-javari-annotations}}


\subsection{Implicit qualifiers\label{javari-implicit-qualifiers}}

As described in Section~\ref{effective-qualifier}, the Javari checker
adds implicit qualifiers, reducing the number of annotations that must
appear in your code.
% For example, ...

For a complete description of all implicit Javari qualifiers, see the
Javadoc for \refclass{javari}{JavariAnnotatedTypeFactory}.


\subsection{Inference of Javari annotations\label{javari-inference}}

It can be tedious to write annotations in your code.  The Javarifier tool
(\myurl{http://types.cs.washington.edu/javari/javarifier/}) infers 
Javari types for an existing program.  It 
automatically inserts Javari annotations in your Java program or
in \code{.class} files.

This has two benefits:  it relieves the programmer of the tedium of writing
annotations (though the programmer can always refine the inferred
annotations), and it annotates libraries, permitting checking of programs
that use those libraries.



\section{What the Javari checker checks\label{javari-checks}}

The checker issues an error whenever mutation happens through a readonly
reference, when fields of a readonly reference which are not explicitly
marked with \code{@\refclass{javari/quals}{Assignable}} are reassigned, or
when a readonly expression is assigned to a mutable variable.  The checker
also emits a warning when casts increase the mutability access of a
reference.

% There is no Javadoc as of 2/2009.
% For a complete description of all checks performed by
% the Nullness checker, see the Javadoc for
% \refclass{javari}{JavariVisitor}.


\section{Iterators and their abstract state\label{javari-library-annotations}}

For an explanation of why the receiver of \<Iterator.next()> is annotated
as \<@ReadOnly>, see Section~\ref{igj-library-annotations}.


\section{Examples\label{javari-examples}}

To try the Javari checker on a source file that uses the Javari
qualifier, use the following command (where \code{javac} is the JSR 308
compiler  that
is distributed with the Checker Framework).  Alternately, you may
specify just one of the test files.

\begin{Verbatim}
  javac -processor checkers.javari.JavariChecker tests/javari/*.java
\end{Verbatim}

\noindent
The compiler should issue the errors and warnings (if any) specified in the
\code{.out} files with same name.

To run the test suite for the Javari checker, use \code{ant javari-tests}.

The Javari checker itself is also annotated with Javari annotations.


% LocalWords:  PolyRead javari cp plugin ReadOnly QReadOnly romaybe Javarifier
% LocalWords:  readonly wildcard Javadoc javac RoMaybe quals IGJ ThisMutable
% LocalWords:  JavariAnnotatedTypeFactory

\htmlhr
\chapter{Lock Checker\label{lock-checker}}

The Lock Checker prevents certain kinds of concurrency errors.  If the Lock
checker issues no warnings for a given program, then the program holds the
appropriate lock every time that it accesses a variable.

Note:  This does \emph{not} mean that your program has \emph{no} concurrency
errors.  (You might have forgotten to annotate that a particular variable
should only be accessed when a lock is held.  You might release and
re-acquire the lock, when correctness requires you to hold it throughout a
computation.  And, there are other concurrency errors that cannot, or
should not, be solved with locks.)  However, ensuring that your
program obeys its locking discipline is an easy and effective way to
eliminate a common and important class of errors.


To run the Lock Checker, supply the \code{-processor
  checkers.lock.LockChecker} command-line option to javac.


\section{Lock annotations\label{lock-annotations}}

The Lock Checker uses two annotations.  One is a type qualifier, and the
other is a method annotation.

\begin{description}

\item[\code{@\refclass{lock/quals}{GuardedBy}}]
  indicates a type whose value may be accessed only when the given lock is
  held.
  See the \ahref{doc/checkers/lock/quals/GuardedBy.html}{GuardedBy
    Javadoc} for an explanation of the argument and other details.  The lock
  acquisition and the value access may be arbitrarily far in the future;
  or, if the value is never accessed, the lock never need be held.
  Figure~\ref{fig:guardedby-hierarchy} gives the type hierarchy.

\begin{figure}
\includeimage{guardedby}{2.5cm}
\caption{Type hierarchy for the \code{@GuardedBy} annotation of the lock
  type system.  The \<@GuardedByTop> annotation is for internal use by the
  type checker; a programmer cannot write it.}
\label{fig:guardedby-hierarchy}
\end{figure}


\item[\code{@\refclass{lock/quals}{Holding}}]
  is a method annotation (not a type qualifier).  It indicates that when
  the method is called, the given lock must be held by the caller.
  In other words, the given lock is already held at the time the method is
  called.

\end{description}

\subsection{Examples}

The most common use of \code{@GuardedBy} is to annotate a field declaration
type.  However, other uses of \code{@GuardedBy} are possible.

\paragraph{Return values}

A return value may be annotated with \code{@GuardedBy}:

\begin{Verbatim}
  @GuardedBy("MyClass.myLock") Object myMethod() { ... }

  // reassignments without holding the lock are OK.
  @GuardedBy("MyClass.myLock") Object x = myMethod();
  @GuardedBy("MyClass.myLock") Object y = x;
  Object z = x;  // ILLEGAL (assuming no lock inference),
                 // because z can be freely accessed.
  x.toString() // ILLEGAL because the lock is not held
  synchronized(MyClass.myLock) {
    y.toString();  // OK: the lock is held
  }
\end{Verbatim}

\paragraph{Formal parameters}

A parameter may be annotated with \code{@GuardedBy}, which indicates that
the method body must acquire the lock before accessing the parameter.  A
client may pass a non-\code{@GuardedBy} reference as an argument, since it
is legal to access such a reference after the lock is acquired.

\begin{Verbatim}
  void helper1(@GuardedBy("MyClass.myLock") Object a) {
    a.toString(); // ILLEGAL: the lock is not held
    synchronized(MyClass.myLock) {
      a.toString();  // OK: the lock is held
    }
  }
  @Holding("MyClass.myLock")
  void helper2(@GuardedBy("MyClass.myLock") Object b) {
    b.toString(); // OK: the lock is held
  }
  void helper3(Object c) {
    helper1(c); // OK: passing a subtype in place of a the @GuardedBy supertype
    c.toString(); // OK: no lock constraints
  }
  void helper4(@GuardedBy("MyClass.myLock") Object d) {
    d.toString(); // ILLEGAL: the lock is not held
  }
  void myMethod2(@GuardedBy("MyClass.myLock") Object e) {
    helper1(e);  // OK to pass to another routine without holding the lock
    e.toString(); // ILLEGAL: the lock is not held
    synchronized (MyClass.myLock) {
      helper2(e);
      helper3(e);
      helper4(e); // OK, but helper4's body still does not type-check
    }
  }
\end{Verbatim}


    

\subsection{Discussion of \<@Holding>}

A programmer might choose to use the \code{@Holding} method annotation in
two different ways:  to specify a higher-level protocol, or to summarize
intended usage.  Both of these approaches are useful, and the Lock Checker
supports both.

\paragraph{Higher-level synchronization protocol}

  \code{@Holding} can specify a higher-level synchronization protocol that
  is not expressible as locks over Java objects.  By requiring locks to be
  held, you can create higher-level protocol primitives without giving up
  the benefits of the annotations and checking of them.

\paragraph{Method summary that simplifies reasoning}

  \code{@Holding} can be a method summary that simplifies reasoning.  In
  this case, the \code{@Holding} doesn't necessarily introduce a new
  correctness constraint; the program might be correct even if the lock
  were acquired later in the body of the method or in a method it calls, so
  long as the lock is acquired before accessing the data it protects.

  Rather, here \code{@Holding} expresses a fact about execution:  when
  execution reaches this point, the following locks are already held.  This
  fact enables people and tools to reason intra- rather than
  inter-procedurally.

  In Java, it is always legal to re-acquire a lock that is already held,
  and the re-acquisition always works.  Thus, whenever you write 

\begin{Verbatim}
  @Holding("myLock")
  void myMethod() {
    ...
  }
\end{Verbatim}

\noindent
it would be equivalent, from the point of view of which locks are held
during the body, to write

\begin{Verbatim}
  void myMethod() {
    synchronized (myLock) {   // no-op:  re-aquire a lock that is already held
      ...
    }
  }
\end{Verbatim}

The advantages of the \<@Holding> annotation include:
\begin{itemize}
\item
  The annotation documents the fact that the lock is intended to already be
  held.
\item
  The Lock Checker enforces that the lock is held when the method is
  called, rather than masking a programmer error by silently re-acquiring
  the lock.
\item
  The \<synchronized> statement can deadlock if, due to a programmer error,
  the lock is not already held.  The Lock Checker prevents this type of
  error.
\item
  The annotation has no run-time overhead.  Even if the lock re-acquisition
  succeeds, it still consumes time.
\end{itemize}


\section{Other lock annotations\label{other-lock-annotations}}

The Checker Framework's lock annotations are similar to annotations used
elsewhere.

If your code is already annotated with a different lock
annotation, you can reuse that effort.  The Checker Framework comes with
cleanroom re-implementations of annotations from other tools.  It treats
them exactly as if you had written the corresponding annotation from the
Lock Checker, as described in Figure~\ref{fig:lock-refactoring}.


% These lists should be kept in sync with LockAnnotatedTypeFactory.java .
\begin{figure}
\begin{center}
% The ~ around the text makes things look better in Hevea (and not terrible
% in LaTeX).
\begin{tabular}{ll}
\begin{tabular}{|l|}
\hline
 ~net.jcip.annotations.GuardedBy~ \\ \hline
\end{tabular}
&
$\Rightarrow$
~checkers.lock.quals.GuardedBy~
\end{tabular}
\end{center}
%BEGIN LATEX
\vspace{-1.5\baselineskip}
%END LATEX
\caption{Correspondence between other lock annotations and the
  Checker Framework's annotations.}
\label{fig:lock-refactoring}
\end{figure}

Alternately, the Checker Framework can process those other annotations (as
well as its own, if they also appear in your program).  The Checker
Framework has its own definition of the annotations on the left side of
Figure~\ref{fig:lock-refactoring}, so that they can be used as type
qualifiers.  The Checker Framework interprets them according to the right
side of Figure~\ref{fig:lock-refactoring}.


\subsection{Relationship to annotations in \emph{Java Concurrency in Practice}\label{jcip-annotations}}

The book \ahref{http://jcip.net/}{\emph{Java Concurrency in Practice}}~\cite{Goetz2006} defines a
\ahref{http://jcip.net.s3-website-us-east-1.amazonaws.com/annotations/doc/net/jcip/annotations/GuardedBy.html}{\code{@GuardedBy}} annotation that is the inspiration for ours.  The book's
\code{@GuardedBy} serves two related but distinct purposes:

\begin{itemize}
\item
  When applied to a field, it means that the given lock must be held when
  accessing the field.  The lock acquisition and the field access may be
  arbitrarily far in the future.
\item
  When applied to a method, it means that the given lock must be held by
  the caller at the time that the method is called --- in other words, at
  the time that execution passes the \code{@GuardedBy} annotation.
\end{itemize}

The Lock Checker renames the method annotation to
\code{@\refclass{lock/quals}{Holding}}, and it generalizes the 
\code{@\refclass{lock/quals}{GuardedBy}} annotation into a type qualifier
that can apply not just to a field but to an arbitrary type (including the
type of a parameter, return value, local variable, generic type parameter,
etc.).  This makes the annotations more expressive and also more amenable
to automated checking.  It also accommodates the distinct
meanings of the two annotations, and resolves ambiguity when \<@GuardedBy>
is written in a location that might apply to either the method or the
return type.

(The JCIP book gives some rationales for reusing the annotation name for
two purposes.  One rationale is
that there are fewer annotations to learn.  Another rationale is
that both variables and methods are ``members'' that can be ``accessed'';
variables can be accessed by reading or writing them (putfield, getfield),
and methods can be accessed by calling them (invokevirtual,
invokeinterface):  in both cases, \code{@GuardedBy} creates preconditions
for accessing so-annotated members.  This informal intuition is
inappropriate for a tool that requires precise semantics.)

% It would not work to retain the name \code{@GuardedBy} but put it on the
% receiver; an annotation on the receiver indicates what lock must be held
% when it is accessed in the future, not what must have already been held
% when the method was called.


\section{Possible extensions\label{lock-extensions}}

The Lock Checker validates some uses of locks, but not all.  It would be
possible to enrich it with additional annotations.  This would increase the
programmer annotation burden, but would provide additional guarantees.

Lock ordering:  Specify that one lock must be acquired before or after
another, or specify a global ordering for all locks.  This would prevent
deadlock.

Not-holding:  Specify that a method must not be called if any of the listed
locks are held.

These features are supported by 
\ahref{http://clang.llvm.org/docs/LanguageExtensions.html#thread-safety-annotation-checking}{Clang's thread-safety annotations}.


% LocalWords:  quals GuardedBy JCIP putfield getfield invokevirtual
% LocalWords:  invokeinterface threadsafety Clang's

\htmlhr
\chapter{Tainting Checker\label{tainting-checker}}

The Tainting Checker prevents certain kinds of trust errors.
A \emph{tainted}, or untrusted, value is one that comes from an arbitrary,
possibly malicious source, such as user input or unvalidated data.
In certain parts of your application, using a tainted value can compromise
the application's integrity, causing it to crash, corrupt data, leak
private data, etc.

% Ought to have many more examples

For example, a user-supplied pointer, handle, or map key should be
validated before being dereferenced.
As another example, a user-supplied string should not be concatenated into a
SQL query, lest the program be subject to a 
\ahref{\url{http://en.wikipedia.org/wiki/Sql_injection}}{SQL injection} attack.
A location in your program where malicious data could do damage is
called a \emph{sensitive sink}.

A program must ``sanitize'' or ``untaint'' an untrusted value before using
it at a sensitive sink.  There are two general ways to untaint a value:
by checking
that it is innocuous/legal (e.g., it contains no characters that can be
interpreted as SQL commands when pasted into a string context), or by
transforming the value to be legal (e.g., quoting all the characters that
can be interpreted as SQL commands).  A correct program must use one of
these two techniques so that tainted values never flow to a sensitive sink.
The Tainting Checker ensures that your program does so.

If the Tainting Checker issues no warning for a given program, then no
tainted value ever flows to a sensitive sink.  However, your program is not
necessarily free from all trust errors.  As a simple example, you might
have forgotten to annotate a sensitive sink as requiring an untainted type,
or you might have forgotten to annotate untrusted data as having a tainted
type.

To run the Tainting Checker, supply the \code{-processor
  checkers.tainting.TaintingChecker} command-line option to javac.
%TODO: For examples, see Section~\ref{tainting-examples}.


\section{Tainting annotations\label{tainting-annotations}}

% TODO: add both qualifiers explicitly, and then describe their relationship.

The Tainting type system uses the following annotations:
\begin{itemize}
\item
  \code{@\refclass{checkers/tainting/quals}{Untainted}} indicates
  a type that includes only untainted, trusted values.
\item
  \code{@\refclass{checkers/tainting/quals}{Tainted}} indicates
  a type that may include only tainted, untrusted values.
  \code{@Tainted} is a supertype of \code{@Untainted}.
\item
  \code{@\refclass{checkers/tainting/quals}{PolyTainted}} is a qualifier that is
  polymorphic over tainting (see Section~\ref{qualifier-polymorphism}).
\end{itemize}


\section{Tips on writing \code{@Untainted} annotations\label{writing-untainted}}

Most programs are designed with a boundary that surrounds sensitive
computations, separating them from untrusted values.  Outside this
boundary, the program may manipulate malicious values, but no malicious
values ever pass the boundary to be operated upon by sensitive
computations.

In some programs, the area outside the boundary is very small:  values are
sanitized as soon as they are received from an external source.  In other
programs, the area inside the boundary is very small:  values are sanitized
only immediately before being used at a sensitive sink.  Either approach
can work, so long as every possibly-tainted value is sanitized before it
reaches a sensitive sink.

Once you determine the boundary, annotating your program is easy:  put
\code{@Tainted} outside the boundary, \code{@Untainted} inside, and
\code{@SuppressWarnings("tainting")} at the validation or
sanitization routines that are used at the boundary.
% (Or, the Tainting Checker may indicate to you that the boundary
% does not exist or has holes through which tainted values can pass.)

The Tainting Checker's standard default qualifier is \code{@Tainted} (see
Section~\ref{defaults} for overriding this default).  This is the safest
default, and the one that should be used for all code outside the boundary
(for example, code that reads user input).  You can set the default
qualifier to \code{@Untainted} in code that may contain sensitive sinks.

The Tainting Checker does not know the intended semantics of your program,
so it cannot warn you if you mis-annotate a sensitive sink as taking
\code{@Tainted} data, or if you mis-annotate external data as
\code{@Untainted}.  So long as you correctly annotate the sensitive sinks
and the places that untrusted data is read, the Tainting Checker will
ensure that all your other annotations are correct and that no undesired
information flows exist.

As an example, suppose that you wish to prevent SQL injection attacks.  You
would start by annotating the
\sunjavadoc{java/sql/Statement.html}{Statement} class to indicate that the
\code{execute} operations may only operate on untainted queries
(Chapter~\ref{annotating-libraries} describes how to annotate external
libraries):

\begin{Verbatim}
  public boolean execute(@Untainted String sql) throws SQLException;
  public boolean executeUpdate(@Untainted String sql) throws SQLException; 
\end{Verbatim}


\section{\code{@Tainted} and \code{@Untainted} can be used for many purposes\label{tainting-many-uses}}

The \code{@Tainted} and \code{@Untainted} annotations have only minimal
built-in semantics.  In fact, the Tainting Checker provides only a small
amount of functionality beyond the Subtyping Checker
(Section~\ref{subtyping-checker}).  This lack of hard-coded behavior means that
the annotations can serve many different purposes.  Here are just a few
examples:

\begin{itemize}
\item
  Prevent SQL injection attacks:  \code{@Tainted} is external input,
  \code{@Untainted} has been checked for SQL syntax.
\item
  Prevent cross-site scripting attacks:  \code{@Tainted} is external input,
  \code{@Untainted} has been checked for JavaScript syntax.
\item
  Prevent information leakage:  \code{@Tainted} is secret data, 
  \code{@Untainted} may be displayed to a user.
\end{itemize}

In each case, you need to annotate the appropriate untainting/sanitization
routines.  This is similar to the \code{@Encrypted} annotation
(Section~\ref{encrypted-example}), where the cryptographic functions are
beyond the reasoning abilities of the type system.  In each case, the type
system verifies most of your code, and the \code{@SuppressWarnings}
annotations indicate the few places where human attention is needed.


If you want more specialized semantics, or you want to annotate multiple
types of tainting in a single program, then you can copy the definition of
the Tainting Checker to create a new annotation and checker with a more
specific name and semantics.  See Chapter~\ref{writing-a-checker} for more
details.

% TODO: one could add a String[] value to @Untainted to distinguish different
% values, eg @Untainted{``SQL''} versus @Untainted{``HTML''}.

% LocalWords:  quals untaint PolyTainted mis untainting sanitization java

\htmlhr
\chapter{Basic checker\label{basic-checker}}

The Basic checker enforces only subtyping rules.  It operates over
annotations specified by a user on the command line.  Thus, users can
create a simple type checker without writing any code beyond definitions of
the type qualifier annotations.

The Basic checker can accommodate all of the type system enhancements that
can be declaratively specified (see Chapter~\ref{writing-a-checker}).
This includes type introduction rules (implicit
annotations, e.g., literals are implicitly considered \code{@\refclass{nullness/quals}{NonNull}}) via
the \code{@\refclass{quals}{ImplicitFor}} meta-annotation, and other features such as
flow-sensitive type qualifier inference (Section~\ref{type-refinement}) and
qualifier polymorphism (Section~\ref{qualifier-polymorphism}).

The Basic checker is also useful to type system designers who wish to
experiment with a checker before writing code; the Basic checker
demonstrates the functionality that a checker inherits from the Checker
Framework.

If you need typestate analysis, then you can extend a typestate checker,
much as you would extend the Basic Checker if you do not need typestate
analysis.  For more details (including a definition of ``typestate''), see
Chapter~\ref{typestate-checker}.

For type systems that require special checks (e.g., warning about
dereferences of possibly-null values), you will need to write code and
extend the framework as discussed in Chapter~\ref{writing-a-checker}.


\section{Using the Basic checker\label{basic-using}}

The Basic checker is used in the same way as other checkers (using the
\code{-processor checkers.basic.BasicChecker} option; see Chapter~\ref{using-a-checker}), except that it
requires an additional annotation processor argument via the standard
``\code{-A}'' switch:

\begin{itemize}

\item
\code{-Aquals}: this option specifies a comma-no-space-separated list of
the fully-qualified class
names of the annotations used as qualifiers in the custom type system.  For
example,

\begin{alltt}
  javac -processor checkers.fenum.BasicChecker
        \textit{-Aquals=myproject.quals.MyQual,myproject.quals.OtherQual} MyFile.java ...
\end{alltt}

It serves the same purpose as the \code{@\refclass{quals}{TypeQualifiers}}
annotation used by other checkers (see section
\ref{writing-compiler-interface}).

The annotations listed in \code{-Aquals} must be accessible to
the compiler during compilation in the classpath.  In other words, they must
already be compiled (and, typically, be on the javac bootclasspath)
before you run the Basic checker with \code{javac}.  It
is not sufficient to supply their source files on the command line.

\end{itemize}

To suppress a warning issued by the basic checker, use a 
\code{@\sunjavadoc{java/lang/SuppressWarnings.html}{SuppressWarnings}}
annotation, with the argument being the unqualified, uncapitalized name of
any of the annotations passed to \code{-Aquals}.  This will suppress all
warnings, regardless of which of the annotations is involved in the
warning.  (As a matter of style, you should choose one of the annotations
as your \code{@SuppressWarnings} key and stick with it for that entire type
hierarchy.)


\section{Basic checker example\label{basic-example}\label{encrypted-example}}

Consider a hypothetical \code{Encrypted} type qualifier, which denotes that the
representation of an object (such as a \code{String}, \code{CharSequence}, or
\code{byte[]}) is encrypted. To use the Basic checker for the \code{Encrypted}
type system, follow three steps.

\begin{enumerate}
\item
 Define an annotation for the \code{Encrypted} qualifier:

\begin{Verbatim}
package myquals;

import java.lang.annotation.Target;
import java.lang.annotation.ElementType;

import checkers.quals.*;

/**
 * Denotes that the representation of an object is encrypted.
 * ...
 */
@TypeQualifier
@SubtypeOf(Unqualified.class)
@Target({ElementType.TYPE_USE, ElementType.TYPE_PARAMETER})
public @interface Encrypted {}
\end{Verbatim}

Don't forget to compile this class:

\begin{Verbatim}
$ javac myquals/Encrypted.java
\end{Verbatim}

The resulting \<.class> file should either be on your classpath, or on the
processor path (set via the \<-processorpath> command-line option to javac).

\item 
  Write \code{@Encrypted} annotations in your program (\code{YourProgram.java}):

\begin{Verbatim}
import myquals.Encrypted;

...

public @Encrypted String encrypt(String text) {
    // ...
}

// Only send encrypted data!
public void sendOverInternet(@Encrypted String msg) {
    // ...
}

void sendText() {
    // ...
    @Encrypted String ciphertext = encrypt(plaintext);
    sendOverInternet(ciphertext);
    // ...
}

void sendPassword() {
    String password = getUserPassword();
    sendOverInternet(password);
}
\end{Verbatim}

You may also need to add \code{@SuppressWarnings} annotations to the
\code{encrypt} and \code{decrypt} methods.  Analyzing them is beyond the
capability of any realistic type system.

\item
  Invoke the compiler with the Basic checker, specifying the
  \code{@Encrypted} annotation using the \code{-Aquals} option.
  You should add the \code{Encrypted} classfile to the processor classpath:

\begin{Verbatim}
$ javac -processorpath myqualspath -processor checkers.basic.BasicChecker \
        -Aquals=myquals.Encrypted YourProgram.java

YourProgram.java:42: incompatible types.
found   : java.lang.String
required: @myquals.Encrypted java.lang.String
    sendOverInternet(password);
                     ^
\end{Verbatim}

\end{enumerate}


\htmlhr
\chapter{Typestate checker\label{typestate-checker}}

In a regular type system, a variable has the same type throughout its
scope.
In a typestate system, a variable's type can change as operations
are performed on it.

The most common example of typestate is for a \<File> object.  Assume a file
can be in two states, \<@Open> and \<@Closed>.  Calling the \<close()> method
changes the file's state.  Any subsequent attempt to read, write, or close
the file will lead to a run-time error.  It would be better for the type
system to warn about such problems, or guarantee their absence, at compile
time.

Just as you can extend the Basic Checker to create a type checker, you can
extend a typestate checker to create a type checker that supports typestate
analysis.  Two extensible typestate analyses that build on the Checker
Framework are available.  One is by Adam Warski:
\myurl{http://www.warski.org/typestate.html}.
The other is by Daniel Wand:
\myurl{http://typestate.ewand.de/}.


\section{Comparison to flow-sensitive type refinement\label{typestate-vs-type-refinement}}

The Checker Framework's flow-sensitive type refinement
(Section~\ref{type-refinement}) implements a form of typestate analysis.
For example, after code that tests a variable against null, the Nullness
Checker (Chapter~\ref{nullness-checker}) treats the variable's type as
\<@NonNull \emph{T}>, for some \<\emph{T}>\@.

For many type systems, flow-sensitive type refinement is sufficient.  But
sometimes, you need full typestate analysis.  This section compares the
two.  (Dependent types and unused variables
(Section~\ref{unused-fields-and-dependent-types}) also have similarities
with typestate analysis and can occasionally substitute for it.  For
brevity, this discussion omits them.)

A typestate analysis is easier for a user to create or extend.
Flow-sensitive type refinement is built into the Checker Framework and is
optionally extended by each checker.  Modifying the rules requires writing
Java code in your checker.  By contrast, it is possible to write a simple
typestate checker declaratively, by writing annotations on the methods
(such as \<close()>) that change a reference's typestate.

A typestate analysis can change a reference's type to something that is not
consistent with its original definition.  For example, suppose that a
programmer decides that the \<@Open> and \<@Closed> qualifiers are
incomparable --- neither is a subtype of the other.  A typestate analysis
can specify that the \<close()> operation converts an \<@Open File> into a
\<@Closed File>.  By contrast, flow-sensitive type refinement can only give
a new type that is a subtype of the declared type --- for flow-sensitive
type refinement to be effective, \<@Closed> would need to be a child of
\<@Open> in the qualifier hierarchy (and \<close()> would need to be
treated specially by the checker).


% LocalWords:  TODO ImplicitFor Aquals TypeQualifiers sourcepath java NonNull
% LocalWords:  CharSequence classpath nullness quals SuppressWarnings classfile
% LocalWords:  uncapitalized processorpath Warski MyFile YourProgram

\htmlhr
\chapter{Advanced type system features\label{advanced-type-system-features}}

This section describes features that are automatically supported by every
checker written with the Checker Framework.
You may wish to skim or skip this section on first reading.  After you have
used a checker for a little while and want to be able to express more
sophisticated and useful types, or to understand more about how the Checker
Framework works, you can return to it.


\section{Polymorphism and generics\label{polymorphism}}

\subsection{Generics (parametric polymorphism or type polymorphism)\label{generics}}

The Checker Framework fully supports
type-qualified Java generic types (also known in the research literature as ``parametric
polymorphism'').  Before running any checker, we recommend that you eliminate
raw types from your code (e.g., your code should use \code{List<...>} as
opposed to \code{List}).
Using generics helps prevent type errors just as using a pluggable
type-checker does.
% Should say why, or what are the consequences of violating this.

When instantiating a generic type,
clients supply the qualifier along with the type argument, as in
\code{List<@NonNull String>}.


\paragraph{Restricting instantiation of a generic class}

When you define a generic class, the generic type parameters can restrict
how the class may be instantiated.  For example, given the definition
\verb|class G<T extends Number> {...}|,
a client can instantiate it as \code{G<Integer>} but not \code{G<Date>}.
Similarly, type qualifiers on the generic type parameters can restrict on
how the class may be instantiated.  For example, a generic list class may
indicate that it can hold only non-null values.  Similarly, a generic map
class might indicate it requires an immutable key type, but that it
supports both nullable and non-null value types.


There are two ways to restrict the type qualifiers that may be used on
the actual type argument when instantiating a generic class.

The first technique is the standard Java approach of using the
\code{extends} or \code{super} clause to supply an upper or lower bound.
For example:

\begin{Verbatim}
  MyClass<T extends @NonNull Object> { ... }

  MyClass<@NonNull String> m1;       // OK
  MyClass<@Nullable String> m2;      // error
\end{Verbatim}

The second technique is to write a type annotation on the declaration of a
generic type parameter, which specifies the exact annotation that is
required on the actual type argument, rather than just a bound.  For example:

\begin{Verbatim}
  class MyClassNN<@NonNull T> { ... }
  class MyClassNble<@Nullable T> { ... }

  MyClassNN<@NonNull Number> v1;     // OK
  MyClassNN<@Nullable Number> v2;    // error
  MyClassNble<@NonNull Number> v4;   // error
  MyClassNble<@Nullable Number> v3;  // OK
\end{Verbatim}

A way to view a type annotation on a generic type parameter declaration is
as syntactic sugar for the annotation on both the \<extends> and the
\<super> clauses of the declaration.  For example, these two declarations
have the same effect:

\begin{Verbatim}
  class MyClassNN<@NonNull T> { ... }
  class MyClassNN<T extends @NonNull Object super @NonNull void> { ... }
\end{Verbatim}

\noindent
except that the latter is not legal Java syntax.  The syntactic sugar is
necessary because of two limitations in Java syntax:  it is illegal to
specify both the upper and the
lower bound, and it is impossible to specify a type annotation for a lower
bound without also specifying a type (use of \<void> is illegal).

If a type parameter declaration is annotated with \code{@A}, and a bound is
also given, then the annotation applies everywhere that there is no
explicit annotation.  For example, the following pairs of declarations are
identical.

\begin{Verbatim}
  class MyClass<@A T> { ... }
  class MyClass<T extends @A Object super @A void> { ... }

  class MyClass<@A T extends Number> { ... }
  class MyClass<T extends @A Number super @A void> { ... }

  class MyClass<@A T extends @B Number> { ... }
  class MyClass<T extends @B Number super @A void> { ... }

  class MyClass<@A T super Number> { ... }
  class MyClass<T extends @A Object super @A Number> { ... }

  class MyClass<@A T super @B Number> { ... }
  class MyClass<T extends @A Object super @B Number> { ... }
\end{Verbatim}

We can see from the above that almost all of the following types mean
different things:

\begin{Verbatim}
  class MyList1<@Nullable T> { ... }
  class MyList2<@NonNull T> { ... }
  class MyList3<T extends @Nullable Object> { ... }
  class MyList4<T extends @NonNull Object> { ... } // same as MyList2
\end{Verbatim}

\<MyList1> must be instantiated with a nullable type. 
The implementation must be able to consume (store) a null
value and produce (retrieve) a null value.

\<MyList2> and \<MyList4> must be instantiated with non-null type.
The implementation has to account for only non-null values --- it
does not have to account for consuming or producing null.

\<MyList3> may be instantiated either way:
with a nullable type or a non-null type.  The implementation must consider
that it may be instantiated either way --- flexible enough to support either
instantiation, yet rigorous enough to impose the correct constraints of the
specific instantiation.  It must also itself comply with the constraints of
the potential instantiations.

One way to express the difference is by comparing what expressions are
legal in the implementation of the list --- that is, what expressions may
appear in the ellipsis, such as inside a method's body.  Suppose each class
has, in the ellipsis, these declarations:

\begin{Verbatim}
  T t;
  @Nullable T nble;      // Section "Type annotations on a use of a generic type variable", below,
  @NonNull T nn;         // further explains the meaning of "@Nullable T" and "@NonNull T".
  void add(T arg) { ... }
  T get(int i) { ... }
\end{Verbatim}

\noindent
Then the following expressions would be legal, inside a given implementation.
(Compilable source code appears as file
\<checker-framework/checkers/tests/nullness/GenericsExample.java>.)

\begin{tabular}{|l|c|c|c|c|} \hline
                        & MyList1 & MyList2 & MyList3 & MyList4 \\ \hline
  t = null;             & OK      & error   & error   & error   \\ \hline
  t = nble;             & OK      & error   & error   & error   \\ \hline
  nble = null;          & OK      & OK      & OK      & OK      \\ \hline
  nn = null;            & error   & error   & error   & error   \\ \hline
  t = this.get(0);      & OK      & OK      & OK      & OK      \\ \hline
  nble = this.get(0);   & OK      & OK      & OK      & OK      \\ \hline
  nn = this.get(0);     & error   & OK      & error   & OK      \\ \hline
  this.add(t);          & OK      & OK      & OK      & OK      \\ \hline
  this.add(nble);       & OK      & error   & error   & error   \\ \hline
  this.add(nn);         & OK      & OK      & OK      & OK      \\ \hline
\end{tabular}


\medskip

%% This text is not very helpful.
% The
% implementation of \code{MyList2} may only place non-null objects in the
% list and may assume that retrieved elements are non-null.  The
% implementation of \code{MyList3} is similar in that it may only place
% non-null objects in the list, because it might be instantiated as, say,
% \code{MyList3<@NonNull Date>}.  When retrieving elements from the list,
% the implementation of \code{MyList3} must account for the fact that
% elements of \code{MyList3} may be null, because it might be instantiated
% as, say, \code{MyList3<@Nullable Date>}.
The differences are more
significant when the qualifier hierarchy is more complicated than just
\<@Nullable> and \<@NonNull>.


\paragraph{Defaults for bounds}
Ordinarily, a type parameter declaration with no extends clause means the
type parameter can be instantiated with any type argument at all.  For
example:

\begin{Verbatim}
  class C<T> { ... }
  class C<T extends Object> { ... }  // identical to previous line
\end{Verbatim}

\noindent
However, instantiation may be restricted if a default qualifier is in
effect (see Section~\ref{defaults}).  For example, the Nullness checker
(Chapter~\ref{nullness-checker}) uses a (configurable) default of
\<@NonNull> (see Section~\ref{null-defaults}).  That means that either
declaration above is interpreted as

\begin{Verbatim}
  class C<T extends @NonNull Object> { ... }
\end{Verbatim}

\noindent
and an instantiation such as \code{C<@Nullable Number>} is illegal.
In such a case, to permit all type arguments, the programmer would write

\begin{Verbatim}
  class C<T extends @Nullable Object> { ... }
\end{Verbatim}


It is possible to set the default qualifier for upper bounds separately
from other default qualifiers, by writing an annotation such as
\<@DefaultQualifier(value="Nullable", locations={DefaultLocation.UPPER\_BOUNDS})>.


\paragraph{Type annotations on a use of a generic type variable}

A type annotation on a generic type variable overrides/ignores any type
qualifier (in the same type hierarchy) on the corresponding actual type
argument.  For example, suppose that \code{T} is a formal type parameter.
Then using \code{@Nullable T} within the scope of \code{T} applies the type
qualifier \code{@Nullable} to the (unqualified) Java type of \code{T}.
This feature is only rarely used.

Here is an example of applying a type annotation to a generic type
variable:

\begin{Verbatim}
  class MyClass2<T> {
    ...
    @Nullable T myField = null;
    ...
  }
\end{Verbatim}

\noindent
The type annotation does not restrict how \code{MyClass2} may be
instantiated.  In other words, both
\code{MyClass2<@NonNull String>} and \code{MyClass2<@Nullable String>} are
legal, and in both cases \code{@Nullable T} means \code{@Nullable String}.
In \code{MyClass2<@Interned String>},
\code{@Nullable T} means \code{@Nullable @Interned String}.

% Note that a type annotation on a generic type variable does not act like
% other type qualifiers.  In both cases the type annotation acts as a type
% constructor, but as noted above they act slightly differently.


% %% This isn't quite right because a type qualifier is itself a type
% %% constructor.
% More formally, a type annotation on a generic type variable acts as a type
% constructor rather than a type qualifier.  Another example of a type
% constructor is \code{[]}.  Just as \code{T[]} is not the same type as
% \code{T}, \code{@Nullable T} is not (necessarily) the same type as
% \code{T}.


\subsection{Qualifier polymorphism\label{qualifier-polymorphism}}

The Checker Framework also supports type \emph{qualifier} polymorphism for
methods, which permits a single method to have multiple different qualified
type signatures.

To \emph{define} a polymorphic qualifier, mark the definition with
\<@\refclass{quals}{PolymorphicQualifier}>.  For example,
\<@\refclass{nullness/quals}{PolyNull}> is a polymorphic type
qualifier for the Nullness type system:

\begin{Verbatim}
  @PolymorphicQualifier
  @Target(ElementType.TYPE_USE)
  public @interface PolyNull { }
\end{Verbatim}

To \emph{use} a polymorphic qualifier, just write it on a type.
For example, you can write \<@PolyNull> anywhere that you would write
\<@NonNull> or \<@Nullable>.

A method written using a polymorphic qualifier conceptually has multiple
versions, somewhat like a template in C++ or the generics feature of Java.
In each version, each instance of the polymorphic qualifier has been
replaced by the same other qualifier from the hierarchy.  See the examples
below in Section~\ref{qualifier-polymorphism-examples}.

The method body must type-check with all signatures.  A method call is
type-correct if it type-checks under any one of the signatures.  If a call
matches multiple signatures, then the compiler uses the most specific
matching signature for the purpose of type-checking.  This is just like
Java's rule for resolving overriding methods, though there is no effect on
run-time dispatch or behavior.

Polymorphic qualifiers can be used on a method signature or body.
They may not be used on classes or fields.

%% I don't see why this is necessarily true; one could define @PolyNull1
%% and @PolyNull2.  It's not so relevant to the manual anyway, and raising
%% the point just makes type system bigots criticize the Checker Framework.
% Qualifier polymorphism is limited to a single qualifier variable per method.


\paragraph{Examples of using polymorphic qualifiers\label{qualifier-polymorphism-examples}}

As an example of the use of \<@PolyNull>, method \sunjavadoc{java/lang/Class.html#cast(java.lang.Object)}{Class.cast}
returns null if and only if its argument is \<null>:

\begin{Verbatim}
  @PolyNull T cast(@PolyNull Object obj) { ... }
\end{Verbatim}

\noindent
This is like writing:

\begin{Verbatim}
   @NonNull T cast( @NonNull Object obj) { ... }
  @Nullable T cast(@Nullable Object obj) { ... }
\end{Verbatim}

\noindent
except that the latter is not legal Java, since it defines two
methods with the same Java signature.


As another example, consider

\begin{Verbatim}
  @PolyNull T max(@PolyNull T x, @PolyNull T y);
\end{Verbatim}

\noindent
which is like writing

\begin{Verbatim}
   @NonNull T max( @NonNull T x,  @NonNull T y);
  @Nullable T max(@Nullable T x, @Nullable T y);
\end{Verbatim}

\noindent
Another way of thinking about which one of the two \code{max} variants is
selected is that the nullness annotations of (the declared types of) both
arguments are \emph{unified} to a type that is a supertype of both, also
known as the \emph{least upper bound} or lub.  If both
arguments are \code{@NonNull}, their unification (lub) is \<@NonNull>, and the
method return type is \<@NonNull>.  But if even one of the arguments is \<@Nullable>,
then the unification (lub) is \<@Nullable>, and so is the return type.


\paragraph{Use multiple polymorphic qualifiers in a method signature\label{qualifier-polymorphism-multiple-qualifiers}}

%% I can't think of a non-clumsy way to say this.
% Each method containing a polymorphic qualifier is (conceptually) expanded
% into multiple versions completely independently.

Usually, it does not make sense to write only a single instance of a polymorphic
qualifier in a method definition:  if you write one instance of (say)
\<@PolyNull>, then you should use at least two.  (An exception is a
polymorphic qualifier on an array element type; this section ignores that
case, but see below for further details.)

For example, there is no point to writing

\begin{Verbatim}
  void m(@PolyNull Object obj)
\end{Verbatim}

\noindent
which expands to

\begin{Verbatim}
  void m(@NonNull Object obj)
  void m(@Nullable Object obj)
\end{Verbatim}

This is no different (in terms of which calls to the method will
type-check) than writing just

\begin{Verbatim}
  void m(@Nullable Object obj)
\end{Verbatim}

The benefit of polymorphic qualifiers comes when one is used multiple times
in a method, since then each instance turns into the same type qualifier.
Most frequently, the polymorphic qualifier appears on at least one formal
parameter and also on the return type.  It can also be useful to have
polymorphic qualifiers on (only) multiple formal parameters, especially if
the method side-effects one of its arguments.
For example, consider

\begin{Verbatim}
void moveBetweenStacks(Stack<@PolyNull Object> s1, Stack<@PolyNull Object> s2) {
  s1.push(s2.pop());
}
\end{Verbatim}

\noindent
In this example, if it is acceptable to rewrite your code to use Java
generics, the code can be even cleaner:

\begin{Verbatim}
<T> void moveBetweenStacks(Stack<T> s1, Stack<T> s2) {
  s1.push(s2.pop());
}
\end{Verbatim}


%% It would be nice to give an example that isn't too contrived.


\paragraph{Using a single polymorphic qualifier on an element type\label{qualifier-polymorphism-element-types}}

There is an exception to the general rule that a polymorphic qualifier
should be used multiple times in a signature.  It can make sense to use a
polymorphic qualifier just once, if it is on an array or generic element
type.

For example, consider a routine that returns the first index, in an array
or collection, of a given element:

\begin{Verbatim}
  public static int indexOf(@PolyNull Object[] a, Object elt) { ... }

  public static int indexOf(Collection<@PolyNull Object> a, Object elt) { ... }
\end{Verbatim}

If \<@PolyNull> were replaced with either \<@Nullable> or \<@NonNull>, then
some safe client calls would be rejected.

Of course, it would be better style to use a generic method, as in either
of these signatures (and likewise for the \<Collection> version):

\begin{Verbatim}
 public static <T> int indexOf(T[] a, /*@Nullable*/ Object elt) { ... }
 public static <T> int indexOf(T[] a, T elt) { ... }
\end{Verbatim}

In conclusion, use of a single polymorphic qualifier may be necessary in
legacy code, but can be avoided by use of better code style.


\section{Unused fields and dependent types\label{unused-fields-and-dependent-types}}

In an inheritance hierarchy, subclasses often introduce new methods and
fields.  For example, a \<Marsupial> (and its subclasses such as
\<Kangaroo>) might have a variable indicating the size of the animal's
pouch.  Because such variables would not exist in superclasses such as
\<Mammal> and \<Animal>, any attempt to use them would be a compile-time
error.

If you cannot use subtypes but wish to enforce similar requirements using
type qualifiers, you can do so.  To restrict which methods may be invoked,
you can write an annotation on a method receiver; then the method may only
be invoked on an expression whose type has the given annotation (or one of
its subtypes).  Section~\ref{unused-fields} describes how to restrict which
fields may be accessed:  in other words, a given field may only be accessed
from an expression whose type has a given qualifier.  Then,
Section~\ref{dependent-types} describes an even more powerful mechanism, by
which the qualifier of a field depends on the qualifier of the expression
from which the field was accessed.
(Also see the discussion of typestate checkers, in
Chapter~\ref{typestate-checker}.)


\subsection{Unused fields\label{unused-fields}}

A Java subtype can have more fields than its supertype.  For example:

\begin{Verbatim}
class Mammal extends Animal { ... }
class Marsupial extends Mammal {
  ...
  int pouchSize;  // pouch capacity, in cubic centimeters
  ...
}
\end{Verbatim}

You can simulate
the same effect for type qualifiers:  a given field may not be accessed via
a reference with a supertype qualifier, but can be accessed via a reference
with a subtype qualifier.
For example:

\begin{Verbatim}
@interface Mammal { }
@interface Marsupial { }
class Animal {
  @Unused(when=Mammal.class)
  int pouchSize;  // pouch capacity, in cubic centimeters
  ...
}

@Marsupial Animal joey = ...;
... joey.pouchSize ...    // OK
@Mammal Animal mae = ...;
... mae.pouchSize ...    // compile-time error
\end{Verbatim}

The \code{@\refclass{quals}{Unused}} annotation
on a field declares that the field may \emph{not} be accessed via a receiver of
the given qualified type (or any \emph{super}type).

(It would probably be clearer to replace \<@Unused> by an annotation that
indicates when the field \emph{may} be used.)


\subsection{Dependent types\label{dependent-types}}

A variable has a \emph{dependent type} if its type depends on some other
value or type.
%  --- the type is dynamically, not statically, determined.
% (Type-safety can still be statically determined, though.)

The Checker Framework supports a form of dependent types, via the
\code{@\refclass{quals}{Dependent}}\code{@\refclass{quals}{Dependent}} annotation.
This annotation changes the type of a field or variable, based on the
qualified type of the receiver (\code{this}).  This can be viewed as a more
expressive form of polymorphism (see Section~\ref{polymorphism}).  It can
also be seen as a way of linking the meanings of two type qualifier
hierarchies.

Here is a restatement of the example of Section~\ref{unused-fields}, using
\code{@\refclass{quals}{Dependent}}:

\begin{Verbatim}
@interface Mammal { }
@interface Marsupial { }
class Animal { ...
  // pouch capacity, in cubic centimeters
  // (non-null if this animal is a marsupial)
  @Nullable @Dependent(result=NonNull.class, when=Marsupial.class) Integer pouchSize;
  ...
}

@Marsupial Animal joey = ...;
... joey.pouchSize.intValue() ...    // OK
@Mammal Animal mae = ...;
... mae.pouchSize.intValue() ...    // compile-time error:
                                    //   dereference of possibly-null mae.pouchSize
\end{Verbatim}

However, when the \code{@\refclass{quals}{Unused}} annotation is sufficient, you
should use it instead of \code{@Dependent}.

% TO DO:  give an example where @Dependent is actually needed


\section{The effective qualifier on a type (defaults and inference)\label{effective-qualifier}}

A checker sometimes treats a type as having a slightly different qualifier
than what is written on the type --- especially if the programmer wrote no
qualifier at all.
Most readers can skip this section on first reading, because you will
probably find the system simply ``does what you mean'', without forcing
you to write too many qualifiers in your program.
In particular, qualifiers in method bodies are extremely rare.

  The following steps determine the effective
qualifier on a type --- the qualifier that the checkers treat as being present.

\begin{enumerate}
\item
  The type system adds implicit qualifiers.  Implicit qualifiers can be
  built into a type system (Section~\ref{writing-type-introduction}), in
  which case the type system's documentation should explain all of the type
  system's implicit qualifiers.  Or, a programmer may introduce an implicit
  annotation on each use of class $C$ by writing a qualifier on the
  declaration of class $C$.

\begin{itemize}
\item
  Example 1 (built-in):  In the Nullness type system,
  \<enum> values are never null, nor is a method receiver.
\item
  Example 2 (built-in):  In the Interning type system, string literals
  and \<enum> values are always interned.
\end{itemize}

\item
  If a type qualifier is present in the source code, that qualifier is used.

  If the type has an implicit qualifier, then it is an error to write an
  explicit qualifier that is equal to (redundant with) or a supertype of
  (weaker than) the implicit qualifier.  A programmer may strengthen
  (write a subtype of) an implicit qualifier, however.

\item
  If there is no implicit or explicit qualifier on a type, then a default
  qualifier may be applied; see Section~\ref{defaults}.

  \smallskip

  At this point, every type has a qualifier.

\item
  The type system may refine a qualified type on a local variable --- that
  is, treat it as a subtype of how it was declared or defaulted.  This
  refinement is always sound and has the effect of eliminating false
  positive error messages.  See Section~\ref{type-refinement}.

  % Type
  % qualifier refinement is implemented by the \refclass{flow}{Flow} class.

\end{enumerate}



\subsection{Default qualifier for unannotated types\label{defaults}}

A type system designer, or an end-user programmer, can cause unannotated
references to be treated as if they had a default annotation.

There are several defaulting mechanisms, for convenience and flexibility.
When determining the default qualifier for a use of a type, the following
rules are used in order, until one applies.
\begin{itemize}
\item
  Use the innermost user-written \code{@DefaultQualifier}, as explained in
  this section.
\item
  Use the default specified by the type system designer
  (Section~\ref{typesystem-defaults}).
\item
  Use \code{@\refclass{quals}{Unqualified}}, which the framework
  inserts to avoid ambiguity and simplify the programming interface for
  type system designers.  Users do not have to worry about this detail,
  but type system implementers can rely on the fact that some
  qualifier is present.
\end{itemize}

% (Implementation detail:  setting defaults is implemented by the
% \refclass{util}{QualifierDefaults} class.)


The end-user programmer specifies a default qualifier by writing the \code{@\refclass{quals}{DefaultQualifier}}
annotation on a package, class, method, or variable declaration.  The
argument to \<@\refclass{quals}{DefaultQualifier}> is the \code{String}
name of an annotation.  It may be a short name like \code{"NonNull"}, if an
appropriate import statement exists.  Otherwise, it should be
fully-qualified, like \code{"checkers.nullness.quals.NonNull"}.
The optional second argument indicates where the default
applies.  If the second argument is omitted, the specified annotation is
the default in all locations.  See the Javadoc of \refclass{quals}{DefaultQualifier} for details.

For example, using the Nullness type system (Chapter~\ref{nullness-checker}):

\begin{Verbatim}
import checkers.quals.*;        // for DefaultQualifier[s]
import checkers.nullness.quals.NonNull;

@DefaultQualifier("NonNull"),
class MyClass {

  public boolean compile(File myFile) { // myFile has type "@NonNull File"
    if (!myFile.exists())          // no warning: myFile is non-null
      return false;
    @Nullable File srcPath = ...;  // must annotate to specify "@Nullable File"
    ...
    if (srcPath.exists())          // warning: srcPath might be null
      ...
  }

  @DefaultQualifier("Mutable")
  public boolean isJavaFile(File myfile) {  // myFile has type "@Mutable File"
    ...
  }
}
\end{Verbatim}

If you wish to write multiple
\<@\refclass{quals}{DefaultQualifier}> annotations at a single location,
use
\<@\refclass{quals}{DefaultQualifiers}> instead.  For example:

\begin{Verbatim}
@DefaultQualifiers({
  @DefaultQualifier("NonNull"),
  @DefaultQualifier("Mutable")
})
\end{Verbatim}


If \code{@DefaultQualifier}[\code{s}] is placed on a package (via the
\<package-info.java> file), then it applies to the given package \emph{and}
all subpackages.
% This is slightly at odds with Java's treatment of packages of different
% names as essentially unrelated, but is more intuitive and useful.

Recall that an annotation on a class definition indicates an implicit
qualifier (Section~\ref{effective-qualifier}) that can only be
strengthened, not weakened.  This can lead to unexpected results if
the default qualifier applies to a class definition.  Thus, you may want to
put explicit qualifiers on class declarations (which prevents the default
from taking effect), or exclude class declarations from defaulting.


%% Don't even bother to bring this up; it will just sow confusion without
%% being helpful.
% For some type systems, a user may not specify a default qualifier, or doing
% so prevents giving any other qualifier to any reference.  This is a
% consequence of the design of the type system; see
% Section~\ref{bottom-qualifier}.


When a programmer omits an \<extends> clause at a declaration of a type
parameter, the default still applies to the implicit upper bound.  For
example, consider these two declarations:

\begin{Verbatim}
  class C<T> { ... }
  class C<T extends Object> { ... }  // identical to previous line
\end{Verbatim}

\noindent
The two declarations are treated identically by Java, and the default
qualifier applies to the \<Object> upper bound whether it is implicit or
explicit.  (The @NonNull default annotation applies only to the upper bound
in the \<extends> clause, not to the lower bound in the inexpressible
implicit \<super void> clause.)


\subsection{Automatic type refinement (flow-sensitive type qualifier inference)\label{type-refinement}}

In order to reduce the burden of annotating types in your program, the
checkers soundly treat certain variables and expressions as having a
subtype of their declared or defaulted (Section~\ref{defaults})
type.  This functionality
never introduces unsoundness or causes an error to be missed:  it merely
suppresses false positive warnings.

By default, all checkers, including new checkers that you write, can take
advantage of this functionality.  Most of the time, users don't have to
think about, and may not even notice, this feature of the framework.  The
checkers simply do the right thing even when a programmer forgets an
annotation on a local variable, or when a programmers writes an
unnecessarily general type in a declaration.

If you are curious or want more details about this feature, then read on.

As an example, the Nullness checker (Chapter~\ref{nullness-checker}) can automatically
determine that certain variables are non-null, even if they were explicitly
or by default annotated as nullable.
The checker treats a variable or expression as \code{@\refclass{nullness/quals}{NonNull}}
\begin{itemize}
\item
starting at the time that it is either
assigned a non-null value or checked against null (e.g., via an assertion,
\code{if} statement, or being dereferenced)
\item
until it might be re-assigned (e.g.,
via an assignment that might affect this variable, or via a method call
that might affect this variable).
\end{itemize}

As with explicit annotations, the implicitly non-null types permit
dereferences and assignments to non-null types, without
compiler warnings.

Consider this code, along with comments indicating whether the
Nullness checker (Chapter~\ref{nullness-checker}) issues a warning.  Note that the same expression may yield a
warning or not depending on its context.

\begin{Verbatim}
  // Requires an argument of type @NonNull String
  void parse(@NonNull String toParse) { ... }

  // Argument does NOT have a @NonNull type
  void lex(@Nullable String toLex) {
    parse(toLex);        // warning:  toLex might be null
    if (toLex != null) {
      parse(toLex);      // no warning:  toLex is known to be non-null
    }
    parse(toLex);        // warning:  toLex might be null
    toLex = new String(...);
    parse(toLex);        // no warning:  toLex is known to be non-null
  }
\end{Verbatim}

If you find examples where you think a value should be inferred to have
(or not have) a
given annotation, but the checker does not do so, please submit a bug
report (see Section~\ref{reporting-bugs}) that includes a small piece of
Java code that reproduces the problem.

% Flow-sensitive non-null inference has been implemented for the following
% varieties of expressions:
%
% \begin{itemize}
% \item null checks in if/else statements
% \item null checks in assert statements
% \item null checks that result in a return or thrown exception, or call System.exit
% \item assignments from new class/array expressions
% \end{itemize}
%
% \emph{Note:} The items in the above list exclude complex null checks, i.e., not
% of the form \code{x != null}. Support for these types of checks will be available in a
% future release.


% TODO:  Is NonNull inferred for any parameters or fields, or just for locals?

Type inference is never performed for method parameters of non-private
methods and for non-private fields, because unknown client code could use
them in arbitrary ways.  The inferred information is never written to the
\code{.class} file as user-written annotations are.

The inference indicates when a variable can be treated as having a subtype
of its declared type --- for instance, when an otherwise nullable type can be
treated as a \code{@\refclass{nullness/quals}{NonNull}} one.  The inference never treats a variable as
a supertype of its declared type (e.g., an expression of \code{@\refclass{nullness/quals}{NonNull}}
type is never inferred to be treated as possibly-null).

\subsection{Fields and flow sensitivity analysis}

Flow sensitivity analysis infers the type of fields in some restricted cases:

\begin{itemize}

\item
A final initialized field:
Type inference is performed for final fields that are initialized to a
compile-time constant at the declaration site; so the type of \code{protocol}
is \code{@NonNull String} in the following declaration:

\begin{Verbatim}
    public final String protocol = "https";
\end{Verbatim}

Please note that such inferred type may leak to the public interface of the
class.  To override such behavior, you can explicitly insert the desired
annotation, e.g.

\begin{Verbatim}
    public final @Nullable String protocol = "https";
\end{Verbatim}

\item
Within method bodies:
Type inference is performed for fields in the context of method bodies,
like local variables, but method invocations invalidate any inferred
information.  Consider the following example, where \code{name} is a nullable
field:

\begin{Verbatim}
class DBObject {
  @Nullable Date updatedAt;

  void update() {
    if (updatedAt == null)
        updatedAt = new Date();
    // updatedAt is nonnull
    log("Updating object at " + updatedAt.getTime());

    persistData();
    // updatedAt is nullable again
    log.debug("Saved object updated at " + updatedAt.getTime()); // invalid!
  }
}
\end{Verbatim}

Here the call to \code{persistData()} invalidates the inferred non-null type
of \code{updatedAt}.

When methods do not modify any object state or have any identity side-effects
(e.g. \code{log()} method here), you can annotate these methods as
\code{Pure}.  Annotating them as \code{Pure}, would cause the flow analyzer to
carry the inferred types across the method invocation boundary.

\end{itemize}


\subsection{Inherited defaults}

In certain situations, it would be convenient for an annotation on a
superclass member to be automatically inherited by subclasses that override
it.  This feature would reduce both annotation effort and program
comprehensibility.  In general, a program is read more often than it is
edited/annotated, so the Checker Framework does not currently support this
feature.  Here are more detailed justifications:

\begin{itemize}

\item
  Currently, a user can determine the annotation on a parameter or return
  value by looking at a single file.  If annotations could be inherited
  from supertypes, then a user would have to examine all supertypes to
  understand the meaning of an unannotated type in a given file.

\item
  Different annotations might be inherited from a supertype and an
  interface, or from two interfaces.  Presumably, the subtype's annotations
  would be stronger than either (the greatest lower bound in the type
  system), or an error would be thrown if no such annotations existed.

\end{itemize}

If these issues can be resolved, then the feature may be added in the
future.  Or, it may be added optionally, and each type-checker
implementation can enable it if desired.


% If you add a Javadoc link to this location, also add the qualifier to the
% list below.
\section{Writing Java expressions as annotation arguments\label{java-expressions-as-arguments}}

Sometimes, it is necessary to write a Java expression as the argument to an
annotation.  As of this writing, tho annotations that take a Java
expression as an argument are:

\begin{itemize}
\item \code{@\refclass{nullness/quals}{KeyFor}}
\item \code{@\refclass{nullness/quals}{NonNullOnEntry}}
\item \code{@\refclass{nullness/quals}{AssertNonNullAfter}}
\item \code{@\refclass{nullness/quals}{AssertNonNullIfTrue}}
\item \code{@\refclass{nullness/quals}{AssertNonNullIfFalse}}
\end{itemize}

The expression is a subset of legal Java expressions:

\begin{itemize}
\item
  the receiver object, \<this>.
\item
  a formal parameter.  Write \<\#> followed by the zero-based parameter
  index.  For example: \<\#0>, \<\#2>.
\item
  a local variable.  This is not applicable for method annotations, but is
  applicable to type annotations such as
  \code{@\refclass{nullness/quals}{KeyFor}}.  Write the variable name.  For
  example: \<myLocalVar>.
\item
  a field of any expression.  For example:  \<next>,
  \<this.next>, \<\#0.next>, \<myLocalVar.next>.
\item a method invocation on any expression.
  The method must be pure and have no formal parameters.  For example:
  \<myClass.getPackage()>, \<myClass.getSuperclass()>,
  \<myClass.getComponentType()>.

  \textbf{Warning:}  currently, annotations that use method calls are
  \emph{not} checked.  The annotation is trusted, and other code will rely
  on it, but it is not verified that other code establishes or maintains
  the validity of the annotation.  Such expressions are still useful if a
  human verifies their correctness.  They are used in the JDK annotations,
  for example.
\end{itemize}

You may optionally omit a leading ``\<this.>'', just as in Java.  Thus, 
\<this.next> and \<next> are equivalent, assuming that there is no
shadowing definition of \<next>.

(A side note:  The formal parameter syntax \<\#0> may seem less convenient
than writing the formal parameter name.  This syntax is necessary because
in the \<.class> file, no parameter name information is available.  Running
the compiler without a checker should create legal annotations in the
\<.class> file, so we cannot rely on the checker to translate names to
indices.)


\section{Inexpressible types\label{inexpressible-types}}

The Type Annotations syntax~\cite{jsr308} is designed to be easy to read.  As a result,
there are types that it cannot express.  An example is the type of
\<Collection.toArray()>, which returns an array of objects, where the
objects have the same annotation as the elements of the receiver.

A possible annotation would be

\begin{Verbatim}
public @Polynull Object [] toArray() ArrayList<@PolyNull E> { ... }
\end{Verbatim}

\noindent
except that this is illegal syntax:  ``\code{ArrayList<@PolyNull E>}'' is
not legal in the receiver position.  (This is a motivation for
\ahref{http://types.cs.washington.edu/jsr308/specification/java-annotation-design.html#receiver-type-parameter-annotations}{extending}
the Type Annotations syntax.)

The annotated libraries (Section~\ref{annotating-libraries}) contain a less-precise annotation for
\code{toArray}.  The Nullness Checker special-cases \code{toArray} to
act as if it had the above annotation.  The cases that
are currently being handled are described in
\refclass{nullness}{CollectionToArrayHeuristics}.
This approach would be possible for other checkers and other methods as the
need arises.


% LocalWords:  MyClass quals PolymorphicQualifier DefaultQualifier subpackages
% LocalWords:  DefaultQualifiers actuals toArray CollectionToArrayHeuristics nn
% LocalWords:  MyList Nullness DefaultLocation nullness PolyNull util java TODO
% LocalWords:  QualifierDefaults nullable lub persistData updatedAt nble
% LocalWords:  subtype's

\htmlhr
\chapter{Suppressing warnings\label{suppressing-warnings}}

%% This feels redundant.
% The Checker Framework is sound:  whenever your code contains an error, the
% Checker Framework will warn you about the error.  The Checker Framework is
% conservative:  it may issue warnings when your code is safe and never
% misbehaves at run time.

When the Checker Framework reports a warning, it's best to change the code
or its annotations, to eliminate the warning.  Alternately, you can
suppress the warning, which does not change the code but prevents the
Checker Framework from reporting this particular warning to you.

You may wish to suppress checker warnings because of unannotated libraries
or un-annotated portions of your own code, because of application
invariants that are beyond the capabilities of the type system, because of
checker limitations, because you are interested in only some of the
guarantees provided by a checker, or for other reasons.
Suppressing a warning is similar to writing a cast in a Java
program:  the programmer knows more about the type than the type system does
and uses the warning suppression or cast to convey that information to the
type system.

You can suppress a single warning message (or those in a single method or
class) by using the following mechanisms:

\newcounter{lastsinglesuppression}
\begin{itemize}
\item
  the \code{@SuppressWarnings} annotation
  (Section~\ref{suppresswarnings-annotation}), or
\item
  the \code{@AssumeAssertion} string in an \<assert> message (Section~\ref{assumeassertion}).
\end{itemize}

You can suppress warnings throughout the codebase by using the following mechanisms:

\begin{itemize}
\item
  the \code{-AsuppressWarnings} command-line option (Section~\ref{suppresswarnings-command-line}),
\item
  the \code{-AskipUses} and \code{-AonlyUses} command-line options (Section~\ref{askipuses}),
\item
  the \code{-AskipDefs} and \code{-AonlyDefs} command-line options (Section~\ref{askipdefs}),
\item
  the \code{-AuseDefaultsForUncheckedCode=source} command-line
  option (Section~\ref{compiling-libraries}),
\item
  the \code{-Alint} command-line option (Section~\ref{alint}), or
\item
  not using the \code{-processor} command-line option
  (Section~\ref{no-processor}).
\end{itemize}

Some type checkers can suppress warnings via
\begin{itemize}
\item
  checker-specific mechanisms (Section~\ref{checker-specific-suppression}).
\end{itemize}

\noindent
We now explain these mechanisms in turn.

% See the @SuppressWarningsKey annotation and the getSuppressWarningsKey method.

\section{\code{@SuppressWarnings} annotation\label{suppresswarnings-annotation}}

\begin{sloppypar}
You can suppress specific errors and warnings by use of the
\code{@SuppressWarnings} annotation, for example
\code{@SuppressWarnings("interning")} or \code{@SuppressWarnings("nullness")}.
Section~\ref{suppresswarnings-annotation-syntax} explains the syntax of the
argument string.
\end{sloppypar}

A \sunjavadocanno{java/lang/SuppressWarnings.html}{SuppressWarnings}
annotation may be placed on program declarations such as a local
variable declaration, a method, or a class.  It suppresses all warnings
related to the given checker, for that program element.
Section~\ref{suppresswarnings-annotation-locations} discusses where the
annotation may be written in source code.

Section~\ref{suppresswarnings-best-practices} gives best practices for
writing \<@SuppressWarnings> annotations.


\subsection{\code{@SuppressWarnings} syntax\label{suppresswarnings-annotation-syntax}}

The \code{@SuppressWarnings} annotation takes a string argument.

The most common usage is \code{@SuppressWarnings("\emph{checkername}")}, as
in \code{@SuppressWarnings("interning")} or
\code{@SuppressWarnings("nullness")}.  The argument \emph{checkername} is
in lower case and is derived from the way you invoke the checker.  For
example, if you invoke a checker as
\code{javac -processor MyNiftyChecker ...},
then you would suppress its error messages with
\code{@SuppressWarnings("mynifty")}.  (An exception is the Subtyping
Checker, for which you use the annotation name; see
Section~\ref{subtyping-using}).  While not recommended, using
\code{@SuppressWarnings("all")} will suppress all warnings for all
checkers.

The \<@SuppressWarnings> argument string can also  be of the form
\emph{checkername:messagekey}, in which case only
errors/warnings relating to the given message key are suppressed.  For example,
\code{cast.unsafe} is the messagekey for warnings about an unsafe cast, and
\code{cast.redundant} is the messagekey for warnings about a redundant cast.

Each warning from the compiler gives the most specific
suppression key that can be used to suppress that warning.
An example is \<dereference.of.nullable> in

%BEGIN LATEX
\begin{smaller}
%END LATEX
\begin{Verbatim}
MyFile.java:107: error: [dereference.of.nullable] dereference of possibly-null reference myList
          myList.add(elt);
          ^
\end{Verbatim}
%BEGIN LATEX
\end{smaller}
%END LATEX

\noindent
With the \code{-AshowSuppressWarningKeys} command-line option,
the compiler lists every key that would suppress the warning,
not just the most specific one.

%% This is true, but relevant mostly to developers, not users.
% For a list of all message keys for a given checker, see two files:
% \begin{enumerate}
% \item \code{checker-framework/checker/src/org/checkerframework/checker/\emph{checkername}/messages.properties}
% \item \code{checker-framework/framework/src/org/checkerframework/common/basetype/messages.properties}
% \end{enumerate}
%
% \noindent
% You need to check the latter file because
% each checker is built on the \code{basetype} checker and inherits its
% properties.


\subsection{Where \code{@SuppressWarnings} can be written\label{suppresswarnings-annotation-locations}}

\<@SuppressWarnings> is a declaration annotation, so it may be placed on
program declarations such as a local variable declaration, a method, or a
class.  It cannot be used on statements, expressions, or types.  To reduce
the scope of a \<@SuppressWarnings> annotation, it is sometimes desirable
to extract part of an expression into a local variable, so that warnings
can be suppressed just for that local variable's initializer expression.

As an example, consider suppressing a warnings at a cast that you know is safe.  Here is an example
that uses the Tainting Checker (Section~\ref{tainting-checker}); assume
that \<expr> has compile-time (declared) type \<@Tainted String>, but you
know that the run-time value of \<expr> is untainted.

%BEGIN LATEX
\begin{smaller}
%END LATEX
\begin{Verbatim}
  @SuppressWarnings("tainting:cast.unsafe") // expr is untainted because ... [explanation goes here]
  @Untainted String myvar = expr;
\end{Verbatim}
%BEGIN LATEX
\end{smaller}
%END LATEX

\noindent
It would have been \emph{illegal} to write

%BEGIN LATEX
\begin{smaller}
%END LATEX
\begin{Verbatim}
  @Untainted String myvar;
  ...
  @SuppressWarnings("tainting:cast.unsafe") // expr is untainted because ...
  myvar = expr;
\end{Verbatim}
%BEGIN LATEX
\end{smaller}
%END LATEX

\noindent
This does not work because
Java does not permit annotations (such as \<@SuppressWarnings>) on
assignments or other statements or expressions.


\subsection{Good practices when suppressing warnings\label{suppresswarnings-best-practices}}

\subsubsection{Suppress warnings in the smallest possible scope\label{suppresswarnings-best-practices-smallest-scope}}

If a particular expression causes a
false positive warning, you should extract that expression into a local variable
and place a \code{@SuppressWarnings} annotation on the variable
declaration, rather than suppressing warnings for a larger expression or an
entire method body.  See Section~\ref{suppresswarnings-annotation-locations}.

%% I'm not sure how this is related to the smallest possible scope.
% As another example, if you have annotated the signatures but not the bodies
% of the methods in a class or package, put a \code{@SuppressWarnings}
% annotation on the class declaration or on the package's
% \code{package-info.java} file.


\subsubsection{Use a specific argument to \code{@SuppressWarnings}\label{suppresswarnings-best-practices-specific-argument}}


\label{compiler-message-keys}

It is best to use the most specific possible message key to suppress just a
specific error that you know to be a false positive.  The checker outputs
this message key when it issues an error.  If you use a broader
\<@SuppressWarnings> annotation, then it may mask other errors that you
needed to know about.

The example of Section~\ref{suppresswarnings-annotation-locations} could
have been written as any one of the following, with the last one being the
best style:

\begin{Verbatim}
  @SuppressWarnings("tainting")              // suppresses all tainting-related warnings
  @SuppressWarnings("tainting:cast")         // suppresses tainting warnings about casts
  @SuppressWarnings("tainting:cast.unsafe")  // suppresses tainting warnings about unsafe casts
\end{Verbatim}





\subsubsection{Justify why the warning is a false positive\label{suppresswarnings-best-practices-justification}}

A \<@SuppressWarnings> annotation asserts that the code is actually
correct or safe (that is, no undesired behavior will occur), even though
the type system is unable to prove that the code is correct or safe.

Whenever you write a \<@SuppressWarnings> annotation, you should also
write, typically on the same line, a code comment
explaining why the code is actually correct.  In some cases you might also
justify why the code cannot be rewritten in a simpler way that would be
amenable to type-checking.

This documentation will help you and others to understand the reason for
the \<@SuppressWarnings> annotation.  It will also help if you decide to
audit your code to verify all the warning suppressions.


\section{\code{@AssumeAssertion} string in an \<assert> message\label{assumeassertion}}

\begin{sloppypar}
You can suppress a warning by \<assert>ing that some property is true, and
placing the string \<@AssumeAssertion(\emph{warningkey})> in the assertion
message.
\end{sloppypar}

For example, in this code:

\begin{Verbatim}
  assert x != null : "@AssumeAssertion(nullness)";
  ... x.f ...
\end{Verbatim}

\noindent
the Nullness Checker assumes that \<x> is non-null from the \<assert>
statement forward, and so the expression \<x.f> cannot throw a null pointer
exception.

The \<assert> expression must be an expression that would affect flow-sensitive
type qualifier refinement (Section~\ref{type-refinement}), if the
expression appeared in a conditional test.  Each type system has its own
rules about what type refinement it performs.

The warning key is exactly as in the \<@SuppressWarnings> annotation
(Section~\ref{suppresswarnings-annotation}).  The same good practices apply
as for \<@SuppressWarnings> annotations, such as writing a comment
justifying why the assumption is safe
(Section~\ref{suppresswarnings-best-practices}).

The \<-AassumeAssertionsAreEnabled> and \<-AassumeAssertionsAreDisabled>
command-line options (Section~\ref{type-refinement-assertions}) do not
affect processing of \<assert> statements that have \<@AssumeAssertion> in
their message.  Writing \<@AssumeAssertion> means that the assertion would
succeed if it were executed, and the Checker Framework makes use of that
information regardless of the \<-AassumeAssertionsAreEnabled> and
\<-AassumeAssertionsAreDisabled> command-line options.


%% Redundant.
% If the string \<@AssumeAssertion(\emph{warningkey})> does not appear in the
% assertion message, then the Checker Framework treats the assertion as
% being used for defensive programming.  That is, the programmer believes
% that the assertion might fail at run time, so the Checker Framework should
% not make any inference, which would not be justified.

%% Users should never see assertions anyway -- they are for programmers.
% A downside of putting the string in the assertion message is that if the
% assertion ever fails, then a user might see the string and be confused.
% This should never be a problem, since
% the programmer should write the string should only if the programmer has
% reasoned that the
% assertion can never fail.

% (Another way of stating the Nullness Checker's use of assertions is as an
% additional caveat to the guarantees provided by a checker
% (Section~\ref{checker-guarantees}).  The Nullness Checker prevents null
% pointer errors in your code under the assumption that assertions are
% enabled, and it does not guarantee that all of your assertions succeed.)


\subsection{Suppressing warnings and defensive programming\label{defensive-programming}}

This section explains the distinction between two different uses for
assertions (and for related methods like JUnit's \<Assert.assertNotNull>).

Assertions are commonly used for two distinct purposes:  documenting how
the program works and debugging the program when it does not work
correctly.  By default, the Checker Framework assumes that each assertion
is used for debugging:  the assertion might fail at run time, and the programmer
wishes to be informed at compile time about such run-time errors.  On the
other hand, if you write the \<@AssumeAssertion> string in the \<assert>
message, then the Checker Framework assumes that you have used some other
technique to verify that the assertion can never fail at run time, so the
checker assumes the assertion passes and does not issue a warning.

Distinguishing the purpose of each assertion is important for precise
type-checking.
% In particular, the Checker Framework would miss many errors
% if it assumed that every assertion succeeds at run time.
Suppose that a
programmer encounters a failing test, adds an assertion to aid debugging, and fixes the
test.  The programmer leaves the assertion in the program if the programmer
is worried that the program might fail in a similar way in the future.
% The assertion indicates the potential for failure at this point in the code.
The Checker Framework should not assume that the assertion succeeds ---
doing so would defeat the very purpose of the Checker Framework, which is
to detect errors at compile time and prevent them from occurring at run
time.

On the other hand, assertions sometimes document facts that a programmer
has independently verified to be true, and the Checker Framework can
leverage these assertions in order to avoid issuing false positive
warnings.  The programmer marks such assertions with the \<@AssumeAssertion>
string in the \<assert> message.  Only do so if you are sure
that the assertion always succeeds at run time.


% In each case, the assertion or method indicates an application invariant --- a
% fact that should always be true.  There are two distinct reasons a
% programmer may have written the invariant, depending on whether the
% programmer is 100\% sure that the application invariant holds.
%
% \begin{enumerate}
% \item
%   A programmer might write it as \textbf{defensive programming}.  This causes
%   the program to throw an exception, which is useful for debugging because
%   it gives an earlier run-time indication of the error.
%   A programmer would use an assertion in this way if the programmer is not
%   100\% sure that the application invariant holds.
%
%   % , or even to document what the program
%   % is intended to do.
%
% \item
%   A programmer might write it to \textbf{suppress} false positive
%   \textbf{warning messages} from a checker.  A programmer would use an
%   assertion this way if the programmer is 100\% sure that the application
%   invariant holds, and the reference can never be null at run time.
%
% \end{enumerate}


Sometimes methods such as
\refmethod{checker/nullness}{NullnessUtils}{castNonNull}{-T-} are used
instead of assertions.  Just as for assertions, you can treat them as
debugging aids or as documentation.
If you know that a particular codebase uses
a nullness-checking method not for defensive programming but to indicate
facts that are guaranteed to be true (that is, these assertions will never
fail at run time), then you can suppress
warnings related to it.
Annotate its definition just as
\refmethod{checker/nullness}{NullnessUtils}{castNonNull}{-T-} is annotated (see the
source code for the Checker Framework).
% TODO:
% For an assert statement, XXXXX.
Also, be sure to document the intention in the method's Javadoc, so that
programmers do not
accidentally misuse it for defensive programming.


If you are annotating a codebase that already contains precondition checks,
such as:

\begin{Verbatim}
  public String get(String key, String def) {
    checkNotNull(key, "key"); //NOI18N
    ...
  }
\end{Verbatim}

\noindent
then you should mark the appropriate parameter as \<@NonNull> (which is the
default).  This will prevent the checker from issuing a warning about the
\<checkNotNull> call.



\section{\code{-AsuppressWarnings} command-line option\label{suppresswarnings-command-line}}

Supplying the \<-AsuppressWarnings> command-line option is equivalent to
writing a \<@SuppressWarnings> annotation on every class that the compiler
type-checks.  The argument to \<-AsuppressWarnings> is a comma-separated
list of warning suppression keys, as in
\<-AsuppressWarnings=purity,uninitialized>.

When possible, it is better to write a \<@SuppressWarnings> annotation with a
smaller scope, rather than using the \<-AsuppressWarnings> command-line option.


\section{\code{-AskipUses} and \code{-AonlyUses} command-line options\label{askipuses}}

You can suppress all errors and warnings at all \emph{uses} of a given
class, or suppress all errors and warnings except those at uses of a given
class.  (The class itself is still type-checked, unless you also use
the \code{-AskipDefs} or \code{-AonlyDefs} command-line option, see~\ref{askipdefs}).

Set the \code{-AskipUses} command-line option to a
regular expression that matches class names (not file names) for which warnings and errors
should be suppressed.
Or, set the \code{-AonlyUses} command-line option to a
regular expression that matches class names (not file names) for which warnings and errors
should be emitted; warnings about uses of all other classes will be suppressed.

For example, suppose that you use
``{\codesize\verb|-AskipUses=^java\.|}'' on the command line
(with appropriate quoting) when invoking
\code{javac}.  Then the checkers will suppress all warnings related to
classes whose fully-qualified name starts with \codesize\verb|java.|, such
as all warnings relating to invalid arguments and all warnings relating to
incorrect use of the return value.

To suppress all errors and warnings related to multiple classes, you can use
the regular expression alternative operator ``\code{|}'', as in
``{\codesize\verb+-AskipUses="java\.lang\.|java\.util\."+}'' to suppress
all warnings related to uses of classes belong to the \code{java.lang} or
\code{java.util} packages.

You can supply both \code{-AskipUses} and \code{-AonlyUses}, in which case
the \code{-AskipUses} argument takes precedence, and \code{-AonlyUses} does
further filtering but does not add anything that \code{-AskipUses} removed.

Warning:  Use the \code{-AonlyUses} command-line option with care,
because it can have unexpected results.  For example, if the
given regular expression does not match classes in the JDK, then the
Checker Framework will suppress every warning that involves a JDK class
such as \<Object> or \<String>.  The meaning of \code{-AonlyUses} may be
refined in the future.  Oftentimes \code{-AskipUses} is more useful.

% The desired meaning of -AonlyUses is tricky, because what is a "use"?
% Maybe only check calls of methods on the class (though don't check
% argument expressions) and field accesses, but nothing else (such as
% extends clauses that happen to use the class).  But then we would also
% want to suppress warnings related to assignments where a method use or
% field access is the right-hand side.  I'm going to punt on this for now.


\section{\code{-AskipDefs} and \code{-AonlyDefs} command-line options\label{askipdefs}}

You can suppress all errors and warnings in the \emph{definition} of a given
class, or suppress all errors and warnings except those in the definition
of a given class.  (Uses of the class are still type-checked, unless you also use
the \code{-AskipUses} or \code{-AonlyUses} command-line option,
see~\ref{askipuses}).

Set the \code{-AskipDefs} command-line option to a
regular expression that matches class names (not file names) in whose definition warnings and errors
should be suppressed.
Or, set the \code{-AonlyDefs} command-line option to a
regular expression that matches class names (not file names) whose
definitions should be type-checked.

For example, if you use
``{\codesize\verb|-AskipDefs=^mypackage\.|}'' on the command line
(with appropriate quoting) when invoking
\code{javac}, then the definitions of
classes whose fully-qualified name starts with \codesize\verb|mypackage.|
will not be checked.

If you supply both \code{-AskipDefs} and \code{-AonlyDefs}, then
\code{-AskipDefs} takes precedence.

Another way not to type-check a file is not to pass it on the compiler
command-line:  the Checker Framework type-checks only files that are passed
to the compiler on the command line, and does not type-check any file that
is not passed to the compiler.  The \code{-AskipDefs} and \code{-AonlyDefs}
command-line options
are intended for situations in which the build system is hard to understand
or change.  In such a situation, a programmer may find it easier to supply
an extra command-line argument, than to change the set of files that is
compiled.

A common scenario for using the arguments is when you are starting out by
type-checking only part of a legacy codebase.  After you have verified the
most important parts, you can incrementally check more classes until you
are type-checking the whole thing.


\section{\code{-Alint} command-line option\label{alint}}

\label{lint-options}

The \code{-Alint} option enables or disables optional checks, analogously to
javac's \code{-Xlint} option.
Each of the distributed checkers supports at least the following lint options:

% For the current list of lint options supported by all checkers, see
% method BaseTypeChecker.getSupportedLintOptions().

% For the per-checker list, search for "@SupportedLintOptions" in the
% checker implementations.


\begin{itemize}

\item
  \code{cast:unsafe} (default: on) warn about unsafe casts that are not
  checked at run time, as in \code{((@NonNull String) myref)}.  Such casts
  are generally not necessary when flow-sensitive local type refinement is
  enabled.

\item
  \code{cast:redundant} (default: on) warn about redundant
  casts that are guaranteed to succeed at run time,
  as in \code{((@NonNull String) "m")}.  Such casts are not necessary,
  because the target expression of the cast already has the given type
  qualifier.

\item
  \code{cast} Enable or disable all cast-related warnings.

\item
\begin{sloppypar}
  \code{all} Enable or disable all lint warnings, including
  checker-specific ones if any.  Examples include \code{redundantNullComparison} for the
  Nullness Checker (see Section~\ref{lint-nulltest-section}) and \<dotequals> for
  the Interning Checker (see Section~\ref{lint-dotequals}).  This option
  does not enable/disable the checker's standard checks, just its optional
  ones.
\end{sloppypar}

\item
  \code{none} The inverse of \<all>:  disable or enable all lint warnings,
  including checker-specific ones if any.

\end{itemize}

% This syntax is different from -Xlint that uses a colon instead of an
% equals sign, because javac forces the use of the equals sign.

\noindent
To activate a lint option, write \code{-Alint=} followed by a
comma-delimited list of check names.  If the option is preceded by a
hyphen (\code{-}), the warning is disabled.  For example, to disable all
lint options except redundant casts, you can pass
\code{-Alint=-all,cast:redundant} on the command line.

Only the last \code{-Alint} option is used; all previous \code{-Alint}
options are silently ignored.  In particular, this means that \<-Alint=all
-Alint=cast:redundant> is \emph{not} equivalent to
\code{-Alint=-all,cast:redundant}.


\section{No \code{-processor} command-line option\label{no-processor}}

You can also compile parts of your code without use of the
\code{-processor} switch to \code{javac}.  No checking is done during
such compilations, so no warnings are issued related to pluggable
type-checking.


\section{Checker-specific mechanisms\label{checker-specific-suppression}}

Finally, some checkers have special rules.  For example, the Nullness
checker (Chapter~\ref{nullness-checker}) uses
the special \<castNonNull> method to suppress warnings
(Section~\ref{suppressing-warnings-with-assertions}).
This manual also explains special mechanisms for
suppressing warnings issued by the Fenum Checker
(Section~\ref{fenum-suppressing}) and the Units Checker
(Section~\ref{units-suppressing}).



\htmlhr
\chapter{Handling legacy code\label{legacy-code}}

Section~\ref{get-started-with-legacy-code} describes a methodology for
applying annotations to legacy code.  This chapter tells you what to do if,
for some reason, you cannot change your code in such a way as to eliminate
a checker warning.

Also recall that you can convert checker errors into warnings via the
\code{-Awarns} command-line option; see Section~\ref{checker-options}.


\section{Checking partially-annotated programs:  handling unannotated code\label{unannotated-code}}

Sometimes, you wish to type-check only part of your program.
You might focus on the most mission-critical or error-prone part of your
code.  When you start to use a checker, you may not wish to annotate
your entire program right away.
% Not having source code is *not* a reason.
You may not have
enough knowledge to annotate poorly-documented libraries that your program uses.

If annotated code uses unannotated code, then the checker may issue
warnings.  For example, the Nullness Checker (Chapter~\ref{nullness-checker}) will
warn whenever an unannotated method result is used in a non-null context:

\begin{Verbatim}
  @NonNull myvar = unannotated_method();   // WARNING: unannotated_method may return null
\end{Verbatim}

If the call \emph{can} return null, you should fix the bug in your program by
removing the \refqualclass{checker/nullness/qual}{NonNull} annotation in your own program.

If the library call \emph{never} returns null,
there are several ways to eliminate the compiler warnings.
\begin{enumerate}
\item Annotate \code{unannotated\_method} in full.  This approach provides
  the strongest guarantees, but may require you to annotate additional
  methods that \code{unannotated\_method} calls.  See
  Chapter~\ref{annotating-libraries} for a discussion of how to annotate
  libraries for which you have no source code.
\item Annotate only the signature of \code{unannotated\_method}, and
  suppress warnings in its body.  Two ways to suppress the warnings are via a
  \code{@SuppressWarnings} annotation or by not running the checker on that
  file (see Section~\ref{suppressing-warnings}).
\item Suppress all warnings related to uses of \code{unannotated\_method}
  via the \code{skipUses} processor option
  (see Section~\ref{askipuses}).
  Since this can suppress more warnings than you may expect,
  it is usually better to annotate at least the method's signature.  If you
  choose the boundary between the annotated and unannotated code wisely,
  then you only have to annotate the signatures of a limited number of
  classes/methods
  (e.g., the public interface to a library or package).

\end{enumerate}

Chapter~\ref{annotating-libraries} discusses adding annotations to
signatures when you do not have source code available.
Section~\ref{suppressing-warnings} discusses suppressing warnings.


\section{Backward compatibility with earlier versions of Java\label{backward-compatibility}}

Sometimes, your code needs to be \emph{compiled} by people who are using a
Java 5/6/7 compiler, which does not support type annotations.
You can handle this situation by writing annotations in comments (Sections~\ref{annotations-in-comments}--\ref{uncommenting-annotations}).

If your code just needs to be \emph{run} by people who are not using a Java
8 JVM, supply an appropriate \<-target> command-line option to javac.  As
discussed in Section~\ref{no-modular-type-checking-java7-jvm}, the
disadvantage is that this makes it more difficult for clients of your
library to use pluggable type-checking to verify their own code against the
\<.class> or \<.jar> files that you supply;
Section~\ref{declaration-annotations-for-java7} gives a partial solution.


\subsection{Annotations in comments\label{annotations-in-comments}}

A Java 4 compiler does not permit use of
annotations.
A Java 5/6/7 compiler only permits annotations on
declarations --- it does not permit annotations on generic arguments,
casts, \<extends> clauses, method receivers, etc.

So that your code can be compiled by any Java compiler (for any version of
the Java language), you may write any single annotation inside a
\code{/*}\ldots\code{*/} Java comment, as in \code{List</*@NonNull*/ String>}.
The Checker Framework compiler treats the code exactly as if you had not written the
\code{/*} and \code{*/}.
In other words, the Checker Framework compiler will recognize the
annotation (when it is targeting a Java 8 or later JVM),
but your code will still compile with any Java compiler.

%% This is true, but obvious and not important enough to take up space in
%% the manual.
% In a single program, you may write some annotations in comments, and others
% outside of comments.  There is not much point in doing so, however.

\begin{sloppypar}
By default, the Checker Framework compiler ignores any comment that contains spaces at the
beginning or end, or between the \code{@} and the annotation name.
In other words, it reads \code{/*@NonNull*/} as an annotation but ignores
\code{/* @NonNull*/} and \code{/*@ NonNull*/} and \code{/*@NonNull */}.
This
feature enables backward compatibility with code that contains comments
that start with \code{@} but are not annotations.  (The
ESC/Java~\cite{FlanaganLLNSS02}, JML~\cite{LeavensBR2006:JML}, and
Splint~\cite{Evans96} tools all use ``\code{/*@}'' or ``\code{/*~@}'' as a
comment marker.)
Compiler flag
\code{-XDTA:spacesincomments} causes the compiler to parse annotation comments
even when they contain spaces.  You may need to use
\code{-XDTA:spacesincomments} if you use Eclipse's ``Source $>$ Correct
Indentation'' command, since it inserts space in comments.  But the
annotation comments are less readable with spaces, so it's even better to disable
inserting spaces:  in the Formatter preferences, in the Comments tab,
unselect the ``enable block comment formatting'' checkbox.
\end{sloppypar}

Compiler flag \code{-XDTA:noannotationsincomments} causes the compiler
to ignore annotation comments.  With this compiler flag, the
Checker Framework compiler behaves like a standard Java 8 compiler that does
not support annotations in comments.  If your code already contains
comments of the form \</*@...*/> that look like type annotations, and
you want the Checker Framework compiler not to try to interpret them,
then you can either selectively add spaces to the comments or use
\code{-XDTA:noannotationsincomments} to turn off all annotation
comments.

\textbf{Note:} Annotations in comments is a feature of the javac compiler
that is
distributed along with the Checker Framework.  It is \emph{not}
supported by the mainline OpenJDK javac.  This is the key
difference between the Checker Framework compiler and the OpenJDK compiler.


\subsubsection{Annotations in comments do not appear in Java 5/6/7 \code{.class} files\label{annotations-in-java7-class-files}}

The Checker Framework compiler ignores annotations in comments when
targeting a Java 5/6/7 JVM, for example when the \<-target 7> command-line
option is supplied.

It would be possible for the Checker Framework compiler to read the
annotations in comments and place them in the Java 5/6/7 \<.class> file so
that they are available when type-checking client code.  However, this
would have two problems.  First, it would only be use useful to the Checker
Framework compiler, because a standard Java 8 compiler will not look for
type annotations in Java 5/6/7 bytecode.  Second, the type annotations
make reference to parts of the Java 8 JDK, such as
\sunjavadoc{java/lang/annotation/ElementType.html\#TYPE\_USE}{ElementType.TYPE\_USE}.
% What are the exact consequences?  A warning or a crash?
Therefore, trying to run the \<.class> file on a Java 5/6/7 JVM
would cause warnings or crashes.


\subsection{Import statements and receiver parameters in comments\label{receivers-and-imports-in-comments}}

There is a more powerful mechanism that permits arbitrary code to be
written in a comment.  Format the comment as ``\code{/*>>>}\ldots\code{*/}'',
with the first three characters of the comment being greater-than signs. As
with annotations in comments, the commented code is ignored by ordinary
compilers but is treated like code by the
Checker Framework compiler.

This mechanism is intended for two purposes.
First, it supports the receiver (\<this> parameter) syntax.  For example,
to specify a method that does not modify its receiver:

\begin{Verbatim}
public boolean method1(/*>>> @ReadOnly MyClass this*/) { ... }
public boolean method2(/*>>> @ReadOnly MyClass this, */ String argument) { ... }
\end{Verbatim}

Second, it can be used for import statements:

\begin{Verbatim}
/*>>>
import org.checkerframework.checker.nullness.qual.*;
import org.checkerframework.checker.regex.qual.*;
*/
\end{Verbatim}

\noindent
If the import statements are \emph{not} commented out, then every time you
compile the code (even when not doing pluggable type-checking),
the annotation definitions (e.g., the \code{checker.jar}
or \code{checker-qual.jar} file) must be on the classpath.
(This is done automatically if you use the Checker Framework compiler.)
Commenting out the import statements also eliminates Eclipse
warnings about unused import statements, if all uses of the imported
qualifier are themselves in comments and thus invisible to Eclipse.

A third use is for writing multiple annotations inside one
comment, as in \code{/*>>> @NonNull @Interned */ String s;}.
However, it is better style to write multiple annotations each
inside its own comment, as in \</*@NonNull*/ /*@Interned*/ String s;>.

It would be possible to abuse the \code{/*>>>...*/} mechanism to inject
code only when using
the Checker Framework compiler.  Doing so is not a sanctioned use of the
mechanism.


\subsection{Migrating away from annotations in comments\label{uncommenting-annotations}}

Suppose that your codebase currently uses annotations in comments, but you
wish to remove the comment characters around your annotations, because in
the future you will use only compilers that support type annotations and
your code will only run on Java 8 or later JVMs.
This Unix command removes
the comment characters, for all Java files in the current
working directory or any subdirectory.

\begin{Verbatim}
   find . -type f -name '*.java' -print \
     | xargs grep -l -P '/\*\s*@([^ */]+)\s*\*/' \
     | xargs perl -pi.bak -e 's|/\*\s*@([^ */]+)\s*\*/|@\1|g'
\end{Verbatim}

You can customize this command:
\begin{itemize}
\item
To process comments with embedded spaces and asterisks, change
two instances of ``\verb|[^ */]|'' to ``\verb|[^/]|''.
\item
To ignore comments with leading or trailing spaces, remove the four
instances of ``\verb|\s*|''.
\item
  To not make backups, remove ``\verb|.bak|''.
\end{itemize}

% TODO: adapt it to do so.
The command does not handle the \code{>>>} comments; you will need to
adapt the above command to do so, or remove them in another way.


\subsection{No modular type-checking when targeting Java 5/6/7\label{no-modular-type-checking-java7-jvm}}

The Checker Framework's type annotations utilize a Java 8 feature that
allows them to be placed on any type use, including generic type parameters
as in \code{List<@NonNull String>}.  A downside is that use of these type
annotations creates a dependency on Java 8, which means that the compiled
program requires a Java 8 or later JDK at run time.

To ensure that your program can run on a Java 5/6/7 JVM, use a command-line
option such as \<-target 7> when doing normal compilation to produce
classfiles.  Before doing so, you will do pluggable type-checking, using the
\<-target 8> command-line option (or no \<-target> command-line option) to
javac; you may wish to supply the \<-proc:only> command-line argument so
that the type-checking step does not overwrite existing classfiles.

Here are the disadvantages of this approach:

\begin{itemize}
\item
It produces classfiles that contain no trace of your type annotations.
This means that modular type-checking (also known as separate compilation)
is not possible.

You need to compile your entire application every time you
do pluggable type-checking, rather than just compiling a subset of the
files.  Furthermore, clients of your code cannot do pluggable
type-checking to verify that they are using your code correctly, unless
they re-compile your code (or at least all the interfaces that they use)
every time that they compile their own.

\item
It makes pluggable type-checking a
different step than ``real'' compilation, rather than both happening at the
same time.  You will do pluggable type-checking first, and when it works or
when you want to create a binary to distribute to others, you will compile
with an ordinary Java compiler.
\end{itemize}

One way to enable clients to do pluggable type-checking is to provide a
version of your library compiled for Java 8 or later, with the type
annotations.  Clients will do type-checking against this version of the
library, but will do normal compilation and execution using the Java 5, 6,
or 7 version of your library.

Section~\ref{declaration-annotations-for-java7} gives an alternative
approach with its own advantages and disadvantages.


\subsection{Distributing declaration annotations instead of type annotations\label{declaration-annotations-for-java7}}

If it is important to you to distribute Java 5/6/7 classfiles against which
clients can do some type-checking, this section gives a way to do so.

The idea is to
use annotations that are Java 5/6/7 declaration annotations.
This approach requires you to use annotations that are declared in
different packages than usual and that have slightly different names.

\begin{itemize}
\item
At code locations that are legal for both declaration and type
annotations (such as for fields, method returns, and method parameters),
write annotations normally (not in comments).
\item
At locations where a declaration annotation is not permitted
(such as generic type parameters and \<extends> clauses), write
annotations in comments.
\end{itemize}

Here are some disadvantages of this approach:

\begin{itemize}
\item
  You need to use nonstandard names for
  some annotations, and to remember which annotations to write in comments
  and which to write normally.
\item
  It produces classfiles that contain only some of your type annotations
  --- the ones that were not written in comments.
  If your code uses type annotations at
  locations such as generic type parameters and \<extends> clauses, then
  modular type-checking will not observe them;
  the implications of that were described above.
\end{itemize}


Here are more details about the approach.
Suppose you wish to run the Nullness Checker using Java 6 or 7
declaration annotations rather than type annotations.  You have two options.

\begin{enumerate}
\item
At locations where declaration annotations are possible,
use aliased annotations from other projects.  For example, the aliased
annotations for the Nullness Checker are listed in
Section~\ref{nullness-related-work}.

\begin{sloppypar}
At locations where only type annotations are possible, use the
``\<*Type>'' compatibility annotations from package
\<org.checkerframework.checker.nullness.compatqual>
in comments.  For example, the Nullness Checker
declares these declaration annotations:
\refqualclass{checker/nullness/compatqual}{NullableType},
\refqualclass{checker/nullness/compatqual}{NonNullType},
\refqualclass{checker/nullness/compatqual}{PolyNullType},
\refqualclass{checker/nullness/compatqual}{MonotonicNonNullType}, and
\refqualclass{checker/nullness/compatqual}{KeyForType}.
\end{sloppypar}

\item
At locations where declaration annotations are possible,
use ``\<*Decl>'' compatibility annotations from package
\<org.checkerframework.checker.nullness.compatqual>.
For example, the Nullness Checker
declares these declaration annotations:
\refqualclass{checker/nullness/compatqual}{NullableDecl},
\refqualclass{checker/nullness/compatqual}{NonNullDecl},
\refqualclass{checker/nullness/compatqual}{PolyNullDecl},
\refqualclass{checker/nullness/compatqual}{MonotonicNonNullDecl}, and
\refqualclass{checker/nullness/compatqual}{KeyForDecl}.

At locations where only type annotations are possible, use the regular
Checker Framework type annotations in comments.
\end{enumerate}

Notice that in each case, the declaration annotations and type annotations
have distinct names.  This enables a programmer to import both sets of
annotations without a name conflict.  But, you must remember to use the
correct name, depending on where the annotations are written.

Eventually, when backward compatibility with Java 7 and earlier is not important,
you should refactor your codebase to use only the regular Checker Framework
annotations, and not to write them in comments.




% LocalWords:  quals skipUses un AskipUses Alint annotationname javac's Awarns
% LocalWords:  Xlint dotequals castNonNull XDTA spacesincomments Formatter jsr
% LocalWords:  unselect checkbox classpath Djsr bak Nullness nullness java lang
% LocalWords:  checkername util myref nulltest html ESC buildfile mynifty Fenum
% LocalWords:  MyNiftyChecker messagekey basetype uncommenting Anomsgtext
% LocalWords:  AskipDefs mypackage jsr308 Djsr308 Makefile PLXCOMP expr
%  LocalWords:  TODO AsuppressWarnings AssumeAssertion AonlyUses AonlyDefs
%  LocalWords:  ing warningkey redundantNullComparison qual proc Decl
%  LocalWords:  noannotationsincomments lastsinglesuppression classfiles
%%  LocalWords:  AshowSuppressWarningKeys NullableType NonNullType JUnit's
%%  LocalWords:  PolyNullType MonotonicNonNullType KeyForType NullableDecl
%%  LocalWords:  NonNullDecl PolyNullDecl MonotonicNonNullDecl KeyForDecl
%%  LocalWords:  refactored AassumeAssertionsAreEnabled assertNotNull
%%  LocalWords:  AassumeAssertionsAreDisabled NullnessUtils checkNotNull
%%  LocalWords:  ElementType AuseSafeDefaultsForUnannotatedSourceCode

\htmlhr
\chapter{Annotating libraries\label{annotating-libraries}}

When annotated code uses an unannotated library, a checker may issue warnings.
As described in Section~\ref{unannotated-code}, the best way to correct
this problem is to add annotations to the library.  (Alternately, you can instead
suppress all warnings related to an unannotated library by use of the
\code{-AskipUses} command-line option; see
Section~\ref{suppressing-warnings}.)  If you have source code for the
library, you can easily add the annotations.
This section tells you
how to add annotations to a library for which you have no source code,
because the library is distributed only in binary form
(as \code{.class} files, possibly packaged in a \code{.jar} file).
This section is also useful if you do not wish to edit the
library's source code.

You can make the annotations known to the checkers in two ways.

\begin{itemize}

\item
  You can write annotations in a ``stub
  file'' containing classes with no method bodies.
  Section~\ref{stub} describes how to create and use stub files.

\item
  You can insert annotations in the compiled
  \code{.class} files of the library.
  You would express the annotations textually, typically as an annotation index file, and
  then insert them in the library by using the Annotation File Utilities
  (\myurl{http://types.cs.washington.edu/annotation-file-utilities/}).
  See the Annotation File Utilities documentation for full details.

\end{itemize}

The Checker Framework distribution contains annotations for popular
libraries, such as the JDK\@.  It uses both of the above mechanisms.  The
Nullness, Javari, IGJ, and Interning Checkers use an annotated JDK
(Section~\ref{skeleton}), and all other checkers use stub files
(Section~\ref{stub}).

If you annotate additional libraries, please share them with us so that we
can distribute the annotations with the Checker Framework; see
Section~\ref{reporting-bugs}.
You can determine the correct annotations for a library either
automatically by running an inference tool, or manually by reading the
documentation.  Presently, type inference tools are available for the
Nullness (Section~\ref{nullness-inference}) and Javari
(Section~\ref{javari-inference}) type systems.


\section{Choosing between stub files and annotated \<.class> files\label{stub-vs-class-files}}

A checker can read annotations either from a stub file or from a library's
\<.class> files.  This section helps you choose between the two alternatives.

Once created, a stub file can be used directly; this makes it an easy way
to get started with library annotations.
When provided by the author of the checker, a stub file is used
automatically, with no need for the user to supply a command-line option.

Inserting annotations in a library's \<.class> files takes an extra step
using an external tool, the Annotation File Utilities
(\myurl{http://types.cs.washington.edu/annotation-file-utilities/}).
However, this process does not suffer the limitations of stub files, such
as its inability to handle declaration annotations
(Section~\ref{stub-limitations}).


\section{Using stub classes\label{stub}\label{stub-creating-and-using}}

A stub file contains ``stub classes'' that contain annotated signatures,
but no method bodies.  A
checker uses the annotated signatures at compile time, instead of or in
addition to annotations that appear in the library.

Section~\ref{stub-creating} describes how to create stub classes.
Section~\ref{stub-using} describes how to use stub classes.
These sections illustrate stub classes via the example of creating a \code{@\refclass{interning/quals}{Interned}}-annotated
version of \code{java.lang.String}.  You don't need to repeat these steps
to handle \code{java.lang.String} for the Interning Checker,
but you might do something similar for a different class and/or checker.

% First, you must install the skeleton class generator
% (Section~\ref{skeleton-installing}).

\subsection{Creating a stub file\label{stub-creating}}

Every Java file is a stub file.  If you have access to the Java file, then
it is usually best to use the Java file as the stub file, without removing
any of the parts that the stub file format permits you to.  Just add
annotations to the signatures, leaving the method bodies unchanged.
This approach retains the original
documentation and source code, making it easier for a programmer to
double-check the annotations.  It also enables creation of diffs, easing
the process of upgrading when a library adds new methods.  And, the
annotations are in a format that the library maintainers can even
incorporate.

The downside of this approach is that the stub files are larger.  This can
slow down parsing.  Furthermore, a programmer must search the stub file
for a given method rather than just skimming one or two pages of signatures.

If you do not have access to the library source code, then you can create a
stub file from the class file (Section~\ref{stub-creating}),
and then annotate it.  The rest of this section describes this approach.


\begin{enumerate}

\item
  Create a stub file by running the stub class generator.  (\<checkers.jar>
  must be on your classpath.)

\begin{Verbatim}
  cd nullness-stub
  java checkers.util.stub.StubGenerator java.lang.String > String.astub
\end{Verbatim}

  Supply it with the fully-qualified name of the class for which you wish to
  generate a stub class.  The stub class generator prints the
  stub class to standard out, so you may wish to redirect its output to a
  file.

\item
  Add import statements for the annotations.  So you would need to
add the following import statement at the beginning of the file:

\begin{Verbatim}
  import checkers.interning.quals.Interned;
\end{Verbatim}

\item
  Add annotations to the stub class.  For example, you might annotate
  the \sunjavadoc{java/lang/String.html#intern()}{String.intern()} method as follows:

\begin{Verbatim}
  @Interned String intern();
\end{Verbatim}

  You may also remove irrelevant parts of the stub file; see
  Section~\ref{stub-format}.

\end{enumerate}

\subsection{Using a stub file\label{stub-using}}

The \code{-Astubs} argument causes the Checker Framework to read
annotations from annotated stub classes in preference to the unannotated
original library classes.  For example:

%BEGIN LATEX
\begin{smaller}
%END LATEX
\begin{Verbatim}
  javac -processor checkers.interning.InterningChecker -Astubs=String.astub:stubs MyFile.java MyOtherFile.java ...
\end{Verbatim}
%BEGIN LATEX
\end{smaller}
%END LATEX

Each stub path entry is a file or a directory; specifying a directory is
equivalent to specifying every file in it whose name ends with
\code{.astub}.  The stub path entries are delimited by
\<File.pathSeparator> (`\<:>' for Linux and Mac, `\<;>' for Windows).

A checker automatically reads the stub file \code{jdk.astub}.  (The checker
author should place it in the same directory as the Checker class, i.e.,
the subclass of \code{BaseTypeVisitor}.)  Programmers should only use the
\<-Astubs> argument for additional stub files they create themselves.

% \textbf{The following is not implemented yet}
% A library writers should create a file \code{library.astub} on the
% classpath (in the resources directory or the binary jars).
% The Checker Framework automatically imports all the stub files named
% \code{library.astub} found in the classpath.  





\subsection{Stub file format\label{stub-format}}

Every Java file is a valid stub file.  However, you can omit information
that is not relevant to pluggable type-checking; this makes the stub file
smaller and easier for people to read and write.

As an illustration, a stub file for the Interning type system
(Chapter~\ref{interning-checker}) could be:

\begin{Verbatim}
  import checkers.interning.quals.Interned;
  package java.lang;
  @Interned class Class<T> { }
  class String {
    @Interned String intern();
  }
\end{Verbatim}





The stub file format is allowed to differ from Java source code in the
following ways:
\begin{description}

\item{\textbf{Method bodies:}}
  The stub class does not require method bodies for classes; any method
  body may be replaced by a semicolon (\code{;}), as in an interface or
  abstract method declaration.

\item{\textbf{Method declarations:}}
  You only have to specify the methods that you need to annotate.
  Any method declaration may be omitted, in which case the checker reads
  its annotations from library's \<.class> files.  (If you are using a stub class, then
  typically the library is unannotated.)

\item{\textbf{Declaration specifiers:}}
  Declaration specifiers (e.g., \<public>, \<final>, \<volatile>)
  may be omitted.

\item{\textbf{Import statements:}}
  The only required import statements are the ones to import type
  annotations.  Such imports must be at the beginning of the
  file.  Other import statements are optional.

\item{\textbf{Multiple classes and packages:}}
  The stub file format permits having multiple classes and packages.
  The packages are separated by a package statement:
  \<package my.package;>.  Each package declaration may occur only once; in
  other words, all classes from a package must appear together.

\end{description}


\subsection{Limitations\label{stub-limitations}}

The stub file reader has several limitations:

\begin{itemize}
\item
  % Still a problem as of 9/2/2009.
  It does not handle \code{enum}s.
\item
  % Still a problem as of 9/2/2009.
  It only handles type annotations, not declaration annotations (e.g.,
  IGJ's \<@Assignable> or Interning's \<@UsesObjectEquals>).
\item
  It does not handle nested classes.  To work around this, it permits a
  top-level class to be written with a \<\$> in its name, and applies the
  annotations to the appropriate nested class.
\end{itemize}

If these limitations are a problem, then you should insert annotations
in the library's \<.class> files instead.


% Label "skeleton" is for old links from the Javarifier manual, to prevent
% them from being broken links.

\section{Using distributed annotated JDKs\label{skeleton-using}\label{skeleton}}

The Checker Framework distribution contains
annotated JDKs at the path \<checkers/jdk/jdk.jar>.
The \<javac> that is distributed with the Checker Framework uses the
annotated JDKs by default.

If you use a different \<javac>, then you must add a
\code{-Xbootclasspath/p:} argument, which causes the compiler to read
annotations from annotated JDK classes in preference to the unannotated
original library classes.  Supply \code{-Xbootclasspath/p:} in addition to
whatever other arguments you usually use, including \code{-classpath}.  For example:

%BEGIN LATEX
\begin{smaller}
%END LATEX
\begin{Verbatim}
  javac -processor checkers.nullness.NullnessChecker -Xbootclasspath/p:${CHECKERS}/jdk/jdk.jar my_source_files
\end{Verbatim}
% Unconfuse Emacs font lock mode: $
%BEGIN LATEX
\end{smaller}
%END LATEX

If you do not supply the \code{-Xbootclasspath/p:} option, the checker will
print a message warning you to do so.  In the unlikely event that you want
to suppress this warning, use \code{-Anocheckjdk}.


The annotated JDK should \emph{not} be in your classpath at run time, only
at compile time.

The supplied annotated JDK is a version of JDK 6.  If you wish to have an
annotated version of JDK 7, you will need to create it yourself.  Running
\<ant jdk.jar> from the \<checkers/> directory will perform this process.


% Skeleton classes are inferior to stub classes for two reasons.  First,
% skeleton files must be on the classpath during compilation but must
% \emph{not} be on the classpath during execution; this is inconvenient and
% error-prone.  Second, the skeleton files contain incorrect values for
% certain static final fields.  These incorrect values can lead to
% run-time problems unless the Java code is re-compiled without the skeleton
% classes after type-checking is complete.



% \section{Installing the skeleton class generator\label{skeleton-installing}}
%
% Source code for the skeleton class generator tool is included in the
% Checker Framework
% distribution, but because the tool has additional dependencies, the provided
% build script does not build the tool by default.
%
% Follow these steps to install the skeleton class generator:
%
% \begin{enumerate}
%
% \item
%   Install the annotation file utilities, using the instructions at
%   \myurl{http://types.cs.washington.edu/annotation-file-utilities/}.
%   Per those instructions, the \code{annotation-file-utilities.jar} file
%   should be on your classpath.
%
% % TODO This item should become optional; tell people to install the AFU in
% % the right place.
% \item
%   Update the \code{build.properties} file in the Checker Framework distribution so
%   that the \code{annotation-utils.lib} property specifies the location of
%   the \code{annotation-file-utilities.jar} library.
%
% \item
%   Build the skeleton class generator tool by running \code{ant
%     skeleton-util dist} in the \code{checkers} directory.  This updates the
%   \code{checkers.jar} file to contain the skeleton class generator.
%   \code{checkers.jar} should already be on your classpath (see
%   Section~\ref{installation}).
%
% \end{enumerate}


% LocalWords:  plugin utils util dist RuntimeException NonNull TODO AFU enum
% LocalWords:  sourcepath Nullness javac classpath src quals pathSeparator JDKs
% LocalWords:  IGJ's jdk Astubs skipUses astub AskipUses toArray IGJ
% LocalWords:  CollectionToArrayHeuristics BaseTypeVisitor Xbootclasspath
% LocalWords:  Interning's UsesObjectEquals Anocheckjdk

\htmlhr
\section{How to create a new checker\label{writing-a-checker}}

\newcommand{\TreeAPIBase}{http://java.sun.com/javase/6/docs/jdk/api/javac/tree/com/sun/source}
\newcommand{\refTreeclass}[2]{\ahref{\TreeAPIBase{}/#1/#2.html?is-external=true}{\<#2>}}
\newcommand{\ModelAPIBase}{http://java.sun.com/javase/6/docs/api/javax/lang/model}
\newcommand{\refModelclass}[2]{\ahref{\ModelAPIBase{}/#1/#2.html?is-external=true}{\<#2>}}

This section describes how to extend the Checker Framework to create a checker
--- a type-checking compiler plugin that detects bugs or verifies their
absence.  After a programmer annotates a program,
the checker plugin verifies that the code is consistent
with the annotations.
If you only want to \emph{use} a checker, you do not need to read this
section.

Writing a simple checker is easy!  For example, here is a complete, useful
type checker:

\begin{Verbatim}
@TypeQualifier
@SubtypeOf(Unqualified.class)
public @interface Encrypted {}
\end{Verbatim}

This checker is so short because it builds on the Basic Checker
(Section~\ref{basic-checker}).
See Section~\ref{basic-example} for more details about this particular checker.

You can also customize a typestate checker.
Two of these are available.  One is by Adam Warski:  
\myurl{http://www.warski.org/typestate.html}.
The other is by Daniel Wand:
\myurl{http://typestate.ewand.de/}.


The rest of this section contains many details for people who want to more write powerful
checkers.
You won't need all of the details, at least at first.
In addition to reading this section of the manual, you may find it helpful
to examine the implementations of the checkers that are distributed with
the Checker Framework, or to create your checker by modifying another one.
The Javadoc documentation of the framework and the checkers is in the
distribution and is also available online at
\myurl{http://types.cs.washington.edu/checker-framework/current/doc/}.

If you write a new checker, let us know so we can mention it here, link to
it from our webpages, or include it in the Checker Framework distribution.


\subsection{The parts of a checker\label{parts-of-a-checker}}

The Checker Framework provides abstract base classes (default
implementations), and a specific checker overrides as little or as much of
the default implementations as necessary.
%
Sections~\ref{writing-typequals}--\ref{writing-compiler-interface} describe
the components of a type system as written using the Checker Framework:

\begin{description}

\item{\ref{writing-typequals}}
  \textbf{Type qualifiers and hierarchy.}  You define the annotations for
  the type system and the subtyping relationships among qualified types
  (for instance, that \<@NonNull Object> is a subtype of \<@Nullable
  Object>).

\item{\ref{writing-type-introduction}}
  \textbf{Type introduction rules.}  For some types and
  expressions, a qualifier should be treated as present even if a
  programmer did not explicitly write it.  For example, in the Nullness
  type system every literal
  other than \<null> has a \<@\refclass{nullness/quals}{NonNull}> type; examples of literals include \<"some
  string"> and \<java.util.Date.class>.

\item{\ref{extending-visitor}}
  \textbf{Type rules.}  You specify the the type system semantics (type
  rules), violation of which yields a type error.  There are two types of
  rules.  Your checker automatically inherits rules related to the type
  hierarchy, such as that every assignment and
  pseudo-assignment satisfies a subtyping relationship.  You write any
  additional rules.  For example, in the
  Nullness type system, only references with a \<@\refclass{nullness/quals}{NonNull}> type may be
  dereferenced.

\item{\ref{writing-compiler-interface}}
  \textbf{Interface to the compiler.}  The compiler interface indicates
  which annotations are part of the type system, which command-line options
  and \<@SuppressWarnings> annotations the checker recognizes, etc.
\end{description}


\subsection{Annotations: Type qualifiers and hierarchy\label{writing-typequals}}

A type system designer specifies the qualifiers in the type system and
the type hierarchy that relates them.

%% True, but seems irrelevant here, so it detracts from the message.
% Each qualifier restricts the values that
% a type can represent.  For example \<@NonNull String> type can only
% represent non-null values, indicating that the variable may not hold
% \<null> values.

Type qualifiers are defined as Java annotations~\cite{JSR269}.  In Java, an
annotation is defined using the Java \code{@interface} keyword.
Write the \<@\refclass{quals}{TypeQualifier}> annotation on the annotation definition
to indicate that the annotation represents a type qualifier (e.g.,
\<@\refclass{nullness/quals}{NonNull}> or \<@\refclass{interning/quals}{Interned}>) and should be processed by the checker.  For example:

\begin{Verbatim}
  // Define an annotation for the @NonNull type qualifier.
  @TypeQualifier
  public @interface NonNull { }
\end{Verbatim}

\noindent
(An annotation that is written on an annotation
definition, such as \<@\refclass{quals}{TypeQualifier}>, is called a \emph{meta-annotation}.)

% \noindent
% The \<@TypeQualifier> meta-annotation
% distinguishes it from an ordinary
% annotation that applies to a declaration (e.g., \<@Deprecated> or
% \<@Override>).
% The framework ignores any annotation whose
% declaration does not bear the \<@TypeQualifier> meta-annotation (with minor
% exceptions, such as \<@SuppressWarnings>).

The type hierarchy induced by the qualifiers can be defined either
declaratively via meta-annotations (Section~\ref{declarative-hierarchy}), or procedurally through
subclassing \refclass{types}{QualifierHierarchy} or
\refclass{types}{TypeHierarchy} (Section~\ref{procedural-hierarchy}).


\subsubsection{Declaratively defining the qualifier and type hierarchy\label{declarative-hierarchy}}

Declaratively, the type system designer uses two meta-annotations (written
on the declaration of qualifier annotations) to specify the qualifier
hierarchy.

\begin{itemize}

\item \code{@\refclass{quals}{SubtypeOf}} denotes that a qualifier is the subtype of
  another qualifier or qualifiers, specified as an array of class
  literals.  For example, for any type $T$,
  \code{@\refclass{nullness/quals}{NonNull}} $T$ is a subtype of \code{@\refclass{nullness/quals}{Nullable}} $T$:

  \begin{Verbatim}
    @TypeQualifier
    @SubtypeOf( { Nullable.class } )
    public @interface NonNull { }
  \end{Verbatim}

  (The actual definition of \refclass{nullness/quals}{NonNull} is slightly more complex.)


  %% True, but a distraction.  Move to Javadoc?
  % (It would be more natural to use Java subtyping among the qualifier
  % annotations, but Java forbids annotations from subtyping one another.)
  %
  \<@\refclass{quals}{SubtypeOf}> accepts multiple annotation classes as an argument,
  permitting the type hierarchy to be an arbitrary DAG\@.  For example,
  in the IGJ type system (Section~\ref{igj-annotations}), \<@\refclass{igj/quals}{Mutable}>
  and \<@\refclass{igj/quals}{Immutable}> induce two mutually exclusive subtypes of the
  \<@\refclass{igj/quals}{ReadOnly}> qualifier.

  As a special case, the root qualifier needs to be annotated with
  \<@Subtype( \{ \} )>.  The root qualifier is the qualifier that is
  a supertype of all other qualifiers.  \refclass{nullness/quals}{Nullable}
  is the root of the Nullness type system, hence is defined as:

  \begin{Verbatim}
    @TypeQualifier
    @SubtypeOf( { } )
    public @interface Nullable { }
  \end{Verbatim}

  All type qualifiers, except for polymorphic qualifiers, need to be
  properly annotated with \refclass{quals}{SubtypeOf}.

  If the root of the hierarchy is the unqualified type, then its children
  will use \code{@SubtypeOf(Unqualified.class)}, but no \code{@SubtypeOf(
    \{ \} )} annotation on the root is necessary.  For an example, see the
  \<Encrypted> type system of Section~\ref{writing-a-checker}.

\item \<@\refclass{quals}{PolymorphicQualifier}> denotes that a qualifier is a
  polymorphic qualifier.  For example:

  \begin{Verbatim}
    @TypeQualifier
    @PolymorphicQualifier
    public @interface PolyNull { }
  \end{Verbatim}

  For a description of polymorphic qualifiers, see
  Section~\ref{qualifier-polymorphism}.  A polymorphic qualifier needs
  no \<@\refclass{quals}{SubtypeOf}> meta-annotation and need not be
  mentioned in any other \<@\refclass{quals}{SubtypeOf}>
  meta-annotation.

\end{itemize}

\urldef{\isSubtypeURL}\url{doc/checkers/basetype/BaseTypeChecker.html#isSubtype(checkers.types.AnnotatedTypeMirror,%20checkers.types.AnnotatedTypeMirror)}

The declarative and procedural mechanisms for specifying the hierarchy can
be used together.  In particular, when using the \<@\refclass{quals}{SubtypeOf}>
meta-annotation, further customizations may be
performed procedurally (Section~\ref{procedural-hierarchy})
by overriding the \ahref{\isSubtypeURL}{\code{isSubtype}} method in the checker class
(Section~\ref{writing-compiler-interface}).
However, the declarative mechanism is sufficient for most type systems.


\subsubsection{Procedurally defining the qualifier and type hierarchy\label{procedural-hierarchy}}

\urldef{\createQualifierHierarchyURL}\url{doc/checkers/basetype/BaseTypeChecker.html#createQualifierHierarchy()}
\urldef{\createTypeHierarchyURL}\url{doc/checkers/basetype/BaseTypeChecker.html#createTypeHierarchy()}

While the declarative syntax suffices for many cases, more complex
type hierarchies can be expressed by overriding, in \refclass{basetype}{BaseTypeChecker},
either \ahref{\createQualifierHierarchyURL}{\<createQualifierHierarchy>} or \ahref{\createTypeHierarchyURL}{\<createTypeHierarchy>} (typically
only one of these needs to be overridden).
For more details, see the Javadoc of those methods and of the classes
\refclass{types}{QualifierHierarchy} and \refclass{types}{TypeHierarchy}.

The \refclass{types}{QualifierHierarchy} class represents the qualifier hierarchy (not the
type hierarchy), e.g., \refclass{igj/quals}{Mutable}
is a subtype of \refclass{igj/quals}{ReadOnly}.  A type-system designer may subclass
\refclass{types}{QualifierHierarchy} to express customized qualifier
relationships (e.g., relationships based on annotation
arguments).

The \refclass{types}{TypeHierarchy} class represents relationships between
annotated types, rather than merely type qualifiers, e.g., \<@Mutable
Date> is a subtype of \<@ReadOnly Date>.  The default \refclass{types}{TypeHierarchy} uses
\refclass{types}{QualifierHierarchy} to determine all subtyping relationships.
The default \refclass{types}{TypeHierarchy} handles
generic type arguments, array components, type variables, and
wild-cards in a similar manner to the Java standard subtype
relationship but with taking qualifiers into consideration.  Some type
systems may need to override that behavior.  For instance, the Java
Language Specification specifies that two generic types are subtypes only
if their type arguments are identical:  for example,
\code{List<Date>} is not a subtype of \code{List<Object>}, or of any other
generic \code{List}.
(In the technical jargon, the generic arguments are ``invariant''.)
The Javari type system overrides this
behavior to allow some type arguments to change covariantly in a type-safe
manner (e.g.,
\code{List<@Mutable Date>} is a subtype of \code{List<@QReadOnly Date>}).


\subsubsection{Defining a default annotation\label{typesystem-defaults}}

% This paragraph is out of place.

\urldef{\setAbsoluteDefaultsURL}\url{doc/checkers/util/QualifierDefaults.html#setAbsoluteDefaults(javax.lang.model.element.AnnotationMirror,%20java.util.Set)}

A type system designer may set a default annotation.  A user may override
the default; see Section~\ref{defaults}.

The type system designer may specify a default annotation declaratively,
using the \code{@\refclass{quals}{DefaultQualifierInHierarchy}}
meta-annotation.
Note that the default will apply to any source code that the checker reads,
including stub libraries, but will not apply to compiled \code{.class}
files that the checker reads.

Alternately, the type system designer may specify a default procedurally,
by calling the
\ahref{\setAbsoluteDefaultsURL}{\<QualifierDefaults.setAbsoluteDefaults>}
method.  You may do this even if you have declaratively defined the
qualifier hierarchy; see the Nullness checker's implementation for an
example.


Recall that defaults are distinct
from implicit annotations; see Sections~\ref{effective-qualifier}
and~\ref{writing-type-introduction}.


\subsubsection{Bottom qualifier\label{bottom-qualifier}}

It is usually a good idea to have a bottom qualifier in your type hierarchy
--- a qualifier that is a (direct or indirect) subtype of every other
qualifier.  The reason is that this is the natural type for the \code{null}
value, which is can be viewed as having any type at all.

Users should never write the bottom qualifier explicitly; it is merely used
for the \code{null} value.

You might write the bottom qualifier like this:

\begin{Verbatim}
  package myTypeQuals;

  import checkers.quals.*;
  import com.sun.source.tree.Tree;

  @TypeQualifier
  @SubtypeOf({Prototype.class, NonPrototype.class})
  @ImplicitFor(trees={Tree.Kind.NULL_LITERAL})
  public @interface PrototypeBottom {}
\end{Verbatim}



\subsection{Type Factory: Implicit annotations\label{writing-type-introduction}}

For some types and expressions, a qualifier should be treated as present
even if a programmer did not explicitly write it.  For example, every
literal (other than \<null>) has a \<@\refclass{nullness/quals}{NonNull}> type.

The implicit annotations may be specified declaratively and/or procedurally.


\subsubsection{Declaratively specifying implicit annotations}

The \<@\refclass{quals}{ImplicitFor}> meta-annotation indicates implicit annotations.
When written on a qualifier, \refclass{quals}{ImplicitFor}
specifies the trees (AST nodes) and types for which the framework should
automatically add that qualifier.

In short, the types and trees can be
specified via any combination of five fields:

  \begin{itemize}
  \item \code{trees}: an array of
    \ahref{\TreeAPIBase{}/tree/Tree.Kind.html?is-external=true}{\code{com.sun.source.tree.Tree.Kind}}, e.g.,
    \code{NEW\_ARRAY} or \code{METHOD\_INVOCATION}
  \item \code{types}: an array of
    \refModelclass{type}{TypeKind}, e.g., \code{ARRAY}
    or \code{BOOLEAN}
  \item \code{treeClasses}: an array of class literals for classes
    implementing \refTreeclass{tree}{Tree}, e.g.,
    \code{LiteralTree.class} or \code{ExpressionTree.class}
  \item \code{typeClasses}: an array of class literals for classes
    implementing \code{javax.lang.model.type.TypeMirror}, e.g.,
    \code{javax.lang.model.type.PrimitiveType}.  Often you should use
    a subclass of \refclass{types}{AnnotatedTypeMirror}
  \item \code{stringPatterns}: an array of regular expressions that will
    be matched against
    string literals, e.g., \code{"[01]+"} for a binary number.  Useful
    for annotations that indicate the format of a string.
  \end{itemize}

For example, consider the definitions of the \<@\refclass{nullness/quals}{NonNull}> and \<@\refclass{nullness/quals}{Nullable}>
type qualifiers:

%BEGIN LATEX
\begin{smaller}
%END LATEX
\begin{Verbatim}
  @TypeQualifier
  @SubtypeOf( { Nullable.class } )
  @ImplicitFor(
    types={TypeKind.PACKAGE},
    typeClasses={AnnotatedPrimitiveType.class},
    trees={
      Tree.Kind.NEW_CLASS,
      Tree.Kind.NEW_ARRAY,
      Tree.Kind.PLUS,
      // All literals except NULL_LITERAL:
      Tree.Kind.BOOLEAN_LITERAL, Tree.Kind.CHAR_LITERAL, Tree.Kind.DOUBLE_LITERAL, Tree.Kind.FLOAT_LITERAL,
      Tree.Kind.INT_LITERAL, Tree.Kind.LONG_LITERAL, Tree.Kind.STRING_LITERAL
    })
  public @interface NonNull {  }


  @TypeQualifier
  @SubtypeOf({})
  @ImplicitFor(trees={Tree.Kind.NULL_LITERAL})
  public @interface Nullable { }
\end{Verbatim}
%BEGIN LATEX
\end{smaller}
%END LATEX

For more details, see the Javadoc for the \refclass{quals}{ImplicitFor}
  annotation, and the Javadoc for the javac classes that are linked from
it.  (You only need to understand a small amount about the javac AST, such
as the
\ahref{\TreeAPIBase{}/tree/Tree.Kind.html?is-external=true}{\code{Tree.Kind}}
and
\refModelclass{type}{TypeKind}
enums.  All the information you need is in the Javadoc, and
Section~\ref{javac-tips} can help you get started.)


\subsubsection{Procedurally specifying implicit annotations}


The Checker Framework provides a representation of annotated types,
\refclass{types}{AnnotatedTypeMirror}, that extends the standard \<TypeMirror>
interface but integrates a representation of the annotations into a
type representation.  A checker's \emph{type factory} class, given an AST
node, returns the annotated type of that expression.  The Checker
Framework's abstract
\emph{base type factory} class, \refclass{types}{AnnotatedTypeFactory},
supplies a uniform, Tree-API-based interface
for querying the annotations on a program element, regardless of
whether that element is declared in a source file or in a class file.
It also handles default annotations, and it optionally performs
flow-sensitive local type inference.

\refclass{types}{AnnotatedTypeFactory} inserts the qualifiers that the programmer
explicitly inserted in the code.  Yet, certain constructs should be
treated as having a type qualifier even when the programmer has not
written one.  The type system designer may subclass
\refclass{types}{AnnotatedTypeFactory} and override
\code{annotateImplicit(Tree,AnnotatedTypeMirror)} and
\code{annotateImplicit(Element,AnnotatedTypeMirror)} to account for
such constructs.


\subsection{Visitor: Type rules\label{extending-visitor}}

A type system's rules define which operations on values of a
particular type are forbidden.

The framework provides a \textit{base visitor class},
\refclass{basetype}{BaseTypeVisitor}, that performs type-checking at each node of a
source file's AST\@.  It uses the visitor design pattern to traverse
Java syntax trees as provided by Sun's
\ahref{http://java.sun.com/javase/6/docs/jdk/api/javac/tree/index.html}{Tree
API},
and issues a warning whenever the type system induced by the type
qualifier is violated.

A checker's visitor overrides one method in the base visitor for each special
rule in the type qualifier system.  Most type-checkers
override only a few methods in \refclass{basetype}{BaseTypeVisitor}.  For example, the
visitor for the Nullness type system of Section~\ref{nullness-checker} consists
of a single 4-line method that warns if an expression of nullable type
is dereferenced, as in:
\begin{Verbatim}
  myObject.hashCode();  // invalid dereference
\end{Verbatim}



By default, \refclass{basetype}{BaseTypeVisitor} performs subtyping checks that are
similar to Java subtype rules, but taking the type qualifiers into account.
\refclass{basetype}{BaseTypeVisitor} issues these errors:

\begin{itemize}

\item invalid assignment (type.incompatible) when an assignment from
  an expression type to an incompatible type.  The assignment may be a
  simple assignment, or pseudo-assignment like return expressions or
  argument passing in a method invocation

  In particular, in every assignment and pseudo-assignment, the
  left-hand side of the assignment is a supertype of (or the same type
  as) the right-hand side.  For example, this assignment is not
  permitted:

  \begin{Verbatim}
    @Nullable Object myObject;
    @NonNull Object myNonNullObject;
    ...
    myNonNullObject = myObject;  // invalid assignment
  \end{Verbatim}

\item invalid generic argument (generic.argument.invalid) when a type
  is bound to an incompatible generic type variable

\item invalid method invocation (method.invocation.invalid) when a
  method is invoked on an object whose type is incompatible with the
  method receiver type

\item invalid overriding parameter type (override.parameter.invalid)
  when a parameter in a method declaration is incompatible with that
  parameter in the overridden method's declaration

\item invalid overriding return type (override.return.invalid) when a
  parameter in a method declaration is incompatible with that
  parameter in the overridden method's declaration

\item invalid overriding receiver type (override.receiver.invalid)
  when a receiver in a method declaration is incompatible with that
  receiver in the overridden method's declaration

\end{itemize}


\subsection{The checker class:  Compiler interface\label{writing-compiler-interface}}

A checker's entry point is a subclass of \refclass{basetype}{BaseTypeChecker}.  This entry
point, which we call the checker class, serves two
roles:  an interface to the compiler and a factory for constructing
type-system classes.

Because the Checker Framework provides reasonable defaults, oftentimes the
checker class has no work to do.  Here are the complete definitions of the
checker classes for the Interning and Nullness checkers:

\begin{Verbatim}
  @TypeQualifiers({ Interned.class, PolyInterned.class })
  @SupportedLintOptions({"dotequals"})
  public final class InterningChecker extends BaseTypeChecker { }

  @TypeQualifiers({ Nullable.class, Raw.class, NonNull.class, PolyNull.class })
  @SupportedLintOptions({"flow", "cast", "cast:redundant"})
  public class NullnessChecker extends BaseTypeChecker { }
\end{Verbatim}


\urldef{\getSupportedTypeQualifiersURL}\url{http://types.cs.washington.edu/checker-framework/current/doc/checkers/basetype/BaseTypeChecker.html#getSupportedTypeQualifiers()}

The checker class must be annotated by
\code{@\refclass{quals}{TypeQualifiers}}, which lists the annotations
that make up the type hierarchy for this checker (including
polymorphic qualifiers), provided as an array of class literals.  Each
one is a type qualifier whose definition bears the
\code{@\refclass{quals}{TypeQualifier}} meta-annotation (or is
returned by the
\ahref{\getSupportedTypeQualifiersURL}{\<BaseTypeChecker\-.getSupportedTypeQualifiers>}
method).

\urldef{\reportURL}\url{http://types.cs.washington.edu/checker-framework/current/doc/checkers/source/SourceChecker.html#report(checkers.source.Result,%20java.lang.Object)}

The checker class bridges between the compiler and the checker plugin.  It
invokes the type-rule check visitor on every Java source file being
compiler, and provides a simple API, \ahref{\reportURL}{\<report>}, to issue
errors using the compiler error reporting mechanism.

Also, the checker class follows the factory method pattern to
construct the concrete classes (e.g., visitor, factory) and annotation
hierarchy representation.  It is a convention that, for
a type system named Foo, the compiler
interface (checker), the visitor, and the annotated type factory are
named as \<FooChecker>, \<FooVisitor>, and \<FooAnnotatedTypeFactory>.
\refclass{basetype}{BaseTypeChecker} uses the convention to
reflectively construct the components.  Otherwise, the checker writer
must specify the component classes for construction.

A checker can customize the default error messages through a
\sunjavadoc{java/util/Properties.html}{Properties}-loadable text file named
\<messages.properties> that appears in the same directory as the checker class.
The property file keys are the strings passed to \ahref{\reportURL}{\<report>}
(like \code{type.incompatible}) and the values are the strings to be
printed (\code{cannot assign ...}).
The \<messages.properties> file only need to mention the new messages that
the checker defines.
It is also allowed to override messages defined in superclasses, but this
is rarely needed.

\subsubsection{Bundling multiple checkers}

% TODO: Have better explanation on why one would want to bundle checkers
% and why passing checkers in commandline is error-prone

Users need to specify the checker class name in command line
\<-processor> flag to invoke each checker.  When multiple related
checkers need to be run together as a unit, users will have to pass
each checker class name, like:

\begin{Verbatim}
  javac -processor DistanceUnitChecker -processor SpeedUnitChecker ... files ...
\end{Verbatim}

Alternatively, an aggregate checker class is declared to combine these
multiple checkers.  \refclass{util}{AggregateChecker} forms a
convenient base class for such situation, where the checkers can be
declared in one method, like the following:

\begin{Verbatim}
  public class UnitCheckers extends AggregateChecker {
    protected abstract Collection<Class<? extends SourceChecker>>
    getSupportedCheckers() {
      return Arrays.asList(DistanceUnitChecker.class, SpeedUnitChecker);
    }
  }
\end{Verbatim}

Now, users can simply pass \<UnitCheckers> a single argument to the commandline:

\begin{Verbatim}
  javac -processor UnitCheckers ... files ...
\end{Verbatim}

\subsection{Testing framework\label{testing-framework}}

[This section should discuss the testing framework that is used for
testing the distributed checkers.]


\subsection{Debugging options\label{debugging-options}}

The Checker Framework provides debugging options that can be helpful when
writing checker. These are provided via the standard \code{javac} ``\code{-A}''
switch, which is used to pass options to an annotation processor.

\begin{itemize}

\item \code{-Anomsgtext}: use message keys (such as ``\code{type.invalid}'')
rather than full message text when reporting errors or warnings

\item \code{-Ashowchecks}: print debugging information for each
pseudo-assignment check (as performed by \refclass{basetype}{BaseTypeVisitor}; see Section
\ref{extending-visitor} above)

\item \code{-Afilenames}: prints the name of each file before type-checking it

\end{itemize}

The following example demonstrates how these options are used:

%BEGIN LATEX
\begin{smaller}
%END LATEX
\begin{Verbatim}
$ javac -processor checkers.interning.InterningChecker \
    examples/InternedExampleWithWarnings.java -Ashowchecks -Anomsgtext -Afilenames

[InterningChecker] InterningExampleWithWarnings.java
 success (line  18): STRING_LITERAL "foo"
     actual: DECLARED @checkers.interning.quals.Interned java.lang.String
   expected: DECLARED @checkers.interning.quals.Interned java.lang.String
 success (line  19): NEW_CLASS new String("bar")
     actual: DECLARED java.lang.String
   expected: DECLARED java.lang.String
examples/InterningExampleWithWarnings.java:21: (not.interned)
    if (foo == bar)
            ^
 success (line  22): STRING_LITERAL "foo == bar"
     actual: DECLARED @checkers.interning.quals.Interned java.lang.String
   expected: DECLARED java.lang.String
1 error
\end{Verbatim}
%BEGIN LATEX
\end{smaller}
%END LATEX

You can use any standard debugger to observe the execution of your checker.
Set the main class to \code{com.sun.tools.javac.Main} and the bootclasspath
to include the JSR308 langtools.


%% Not relevant to most readers.  Can go in a README file in our repository.
% \subsection{Putting your checker in the repository\label{writing-repository}}
%
% This section is relevant only if you wish to add your checker to the source code
% repository for the Checker Framework --- for example, to include your
% checker in the Checker Framework distribution.
%
% The checkers appear in directory \code{annotations/checkers/} of
% the \code{annotations} repository.  It contains the following relevant
% subdirectories:
% \begin{itemize}
% \item
%   \code{manual/}: Documentation for your checker goes here.
% \item
%   \code{src/checkers/\emph{annotation\_name}/}: Code for the checker,
%   in a directory that is a sibling of of \code{quals/}, \code{nonnull/},
%   etc.
% \item
%   \code{jdk/\emph{annotation\_name}/}: Annotated ``skeleton class''
%   versions of the JDK and other libraries (see Section~\ref{skeleton}).
% \item
%   \code{tests/\emph{annotation\_name}/}: Inputs and outputs for the test
%   suite for the checker.  A single top-level test suite class goes in
%   \code{tests/src/tests/}.
% \end{itemize}


\subsection{javac implementation survival guide\label{javac-tips}}

The implementation of Sun's javac compiler can be a bit daunting to a
newcomer, and its documentation does not particularly help a newcomer to
get oriented.  But do not lose heart!
This section helps you to understand the small part of javac
that you need in order to write a checker.
Other useful resources include the Java Infrastructure Developer's guide at
\url{http://wiki.netbeans.org/Java_DevelopersGuide} and the
compiler mailing list archives at
\url{http://news.gmane.org/gmane.comp.java.openjdk.compiler.devel}
(subscribe at \url{http://mail.openjdk.java.net/mailman/listinfo/compiler-dev}).


The Checker Framework uses Sun's Tree API to access a program's AST\@.
This is specific to the Sun JDK\@.  In the future, the Checker Framework
can be migrated to use the Java Model AST of JSR 198 (Extension API for
Integrated Development Environments)~\cite{JSR198}, which gives access to
the entire source code of a method in an implementation-neutral way.


A \refTreeclass{tree}{Tree} is an AST node; it represents an arbitrary code snippet such
as a method definition, a block, a statement, etc.

The \<Tree> interface has many subinterfaces, that specify what
kind of node is being handled. Trees are usually processed by a class
implementing the
\refTreeclass{tree}{TreeVisitor}
interface, through the \<accept> method on \<Tree>. Common
implementations of \<TreeVisitor> that you may want to extend are
\refTreeclass{util}{SimpleTreeVisitor},
that visits a single node based on its type,
\refTreeclass{util}{TreeScanner},
that visits all subnodes recursively, and
\refTreeclass{util}{TreePathScanner},
that visits all subnodes recursively and stores the
\refTreeclass{util}{TreePath}
corresponding the the currently visited \<Tree>. (Also note that the
iterator given by \<TreePath> used to have an implementation
\ahref{http://bugs.sun.com/view_bug.do?bug_id=6473148}{bug}.)

In order to determine the kind of an object that extends \<Tree>, use
the
\ahref{\TreeAPIBase{}/tree/Tree.html#getKind()}{getKind}
method, as opposed to the \<instanceof> operator, since a \<Tree>
implementation might opt to implement more than one interface from
this API\@.  There is a utility class to perform operations on trees,
\refTreeclass{util}{Trees},
but the framework is intended to do all the low-level tree processing,
so you probably should not need to use this class.

An
\refModelclass{element}{Element}
represents a program element such as packages, classes or methods.
\<Element> has 5 subinterfaces:
\refModelclass{element}{ExecutableElement}
represents methods, constructors or initializers (anything invocable);
\refModelclass{element}{PackageElement}
represents package elements, and contain package information;
\refModelclass{element}{TypeElement}
represents the element of a class or an interface (note that
\<TypeElement> is an \<Element>, not a \<Type>; the corresponding
\<Type> is represented by
\refModelclass{type}{DeclaredType};
\refModelclass{element}{TypeParameterElement}
represents an element of a formal type parameter of a something with
generics, and
\refModelclass{element}{VariableElement}
represents the element associated with a variable. There is an
\refModelclass{element}{ElementVisitor}
interface for visiting objects that \<Element>, in a similar manner to the \<Tree> visitors, with similar
provided implementations. Use the \<asType> method from \<Element> to obtain a \<TypeMirror> for the element.

Again, \<Element> is an interface, so use getKind() to obtain the kind
of an \<Element>, as opposed to the instanceof operator, since an
implementation of \<Element> might also implement other element
interfaces. There is an utility class for handling elements,
\refModelclass{util}{Elements};
the appropriate instance can be obtained by using the
\sunjavadoc{javax/annotation/processing/ProcessingEnvironment.html#getElementUtils()}{getElementUtils{}}
method on the \<ProcessingEnvironment> object visible on factories and
checkers. The framework should do most of the element processing that
requires \<Elements>, unless you are doing something non-trivial.

A
\refModelclass{type}{TypeMirror}
represents a Java type. It is yet another interface you should be
familiar with, with various subinterfaces, notable ones being
\refModelclass{type}{DeclaredType}
for class and interface types, and
\refModelclass{type}{ExecutableType}
for method, constructor and initializer types.

Note that a
\refTreeclass{tree}{MethodTree}
resolves into a \<ExecutableType>, while a
\refTreeclass{tree}{MethodInvocationTree}
resolves into a \<DeclaredType> if the return type is a class or an
interface, an
\refModelclass{type}{ArrayType}
if the return type is an array, a
\refModelclass{type}{NoType}
if the return type is void, or a
\refModelclass{type}{PrimitiveType}
if the return type is primitive.

Not every \<Tree> corresponds to an \<Element> (such as a
\<BlockTree>), not every \<Tree> corresponds to a \<TypeMirror>
(again, such as a \<BlockTree>), and not every \<TypeMirror> has a
corresponding \<Element> (such as primitive types or arrays).

As one could expect by this point, \<TypeMirror> is an interface, so
use the appropriate getKind() method to distinguish the types, as
opposed to the instanceof operator, since those are interfaces, and
more than one can be implemented by a same object.

Note that the \<TypeMirror> API makes no guarantees that the same type
will always be represented by the same object; use the method
recommended on the API if you need to compare two types.

\refModelclass{type}{TypeVisitor}
and implementations of visitors for \<TypeMirror> are provided, but
those classes should not be used or extended directly on the
framework, since all checker plugin classes are meant to visit
\<AnnotatedTypeMirror> instead, modifying the annotations as needed.
A
\refModelclass{util}{Types}
utility class is provided by the \<ProcessingEnvironment> as well, if
you need to do more complex operations with types. In general, you
should use \<AnnotatedTypeMirror> and its subclasses as opposed to
using \<TypeMirror> and its subinterfaces.

An \refclass{types}{AnnotatedTypeMirror}
(defined in the Checker Framework, not in javac) represents an
annotated type --- a type along with all its annotations.  It is
modeled after Sun's
\refModelclass{type}{TypeMirror}.
Similarly modeled visitors are presented: a
\refclass{types/visitors}{AnnotatedTypeVisitor} interface, implemented by
\refclass{types/visitors}{SimpleAnnotatedTypeVisitor}
for visiting just one node,
\refclass{types/visitors}{AnnotatedTypeScanner}
for visiting every node recursively.

In short: a
\refTreeclass{tree}{Tree}
represents some snippet of code, an
\refModelclass{element}{Element}
represents some program element, and a
\refModelclass{type}{TypeMirror}
represents a Java type, but you usually should use
\refclass{types}{AnnotatedTypeMirror},
provided by the Checker Framework, instead of \<TypeMirror>, as our
implementation carries along with the types the annotation information
at every node level.  The
\refclass{types}{AnnotatedTypeFactory}
(or its extension on your framework plugin) is responsible for
producing \<AnnotatedTypeMirror> objects for \<Tree> and \<Element>
parameters it receives; those \<AnnotatedTypeMirror> objects are then
processed by the visitor class and checked by the checker class on
your checker plugin.


% LocalWords:  plugin javac's SourceChecker AbstractProcessor getMessages quals
% LocalWords:  getSourceVisitor SourceVisitor getFactory AnnotatedTypeFactory
% LocalWords:  SupportedAnnotationTypes SupportedSourceVersion TreePathScanner
% LocalWords:  TreeScanner visitAssignment AssignmentTree AnnotatedClassTypes
% LocalWords:  SubtypeChecker SubtypeVisitor NonNull isSubtype getClass nonnull
% LocalWords:  AnnotatedClassType isAnnotatedWith hasAnnotationAt TODO src jdk
% LocalWords:  processor NullnessChecker InterningChecker Nullness Nullable igj
% LocalWords:  AnnotatedTypeMirrors BaseTypeChecker BaseTypeVisitor basetype
% LocalWords:  Aqual Anqual java CharSequence getAnnotatedType UseLovely IGJ
% LocalWords:  AnnotatedTypeMirror LovelyChecker Anomsgtext Ashowchecks enums
% LocalWords:  Afilenames dereferenced SuppressWarnings declaratively SubtypeOf
% LocalWords:  TypeQualifier TypeHierarchy GraphQualifierHierarchy ReadOnly Foo
% LocalWords:  QualifierHierarchy QualifierRoot createQualifierHierarchy util
% LocalWords:  createTypeHierarchy QReadOnly ImplicitFor treeClasses TypeMirror
% LocalWords:  LiteralTree ExpressionTree typeClasses annotateImplicit nullable
% LocalWords:  TypeQualifiers getSupportedTypeQualifiers FooChecker nullness
% LocalWords:  FooVisitor FooAnnotatedTypeFactory basicstyle InterningVisitor
% LocalWords:  InterningAnnotatedTypeFactory QualifierDefaults TypeKind getKind
% LocalWords:  setAbsoluteDefaults PolymorphicQualifier TreeVisitor subnodes
% LocalWords:  SimpleTreeVisitor TreePath instanceof subinterfaces TypeElement
% LocalWords:  ExecutableElement PackageElement DeclaredType VariableElement
% LocalWords:  TypeParameterElement ElementVisitor javax getElementUtils NoType
% LocalWords:  ProcessingEnvironment ExecutableType MethodTree ArrayType Warski
% LocalWords:  MethodInvocationTree PrimitiveType BlockTree TypeVisitor blog
% LocalWords:  AnnotatedTypeVisitor SimpleAnnotatedTypeVisitor html langtools
% LocalWords:  AnnotatedTypeScanner bootclasspath asType stringPatterns
% LocalWords:  DefaultQualifierInHierarchy invocable

\htmlhr
\chapter{Integration with external tools\label{external-tools}}

This chapter discusses how to run a checker from your favorite IDE\@.

Or, if your favorite isn't here, you should customize how it runs the
javac command on your behalf.  See the IDE documentation to learn how to
customize it, adapting the instructions for javac in Section~\ref{running}.
If you make another tool support running a checker, please
inform us via the
\ahref{http://groups.google.com/group/checker-framework-discuss}{mailing
  list} or
\ahref{http://code.google.com/p/checker-framework/issues/list}{issue tracker} so
we can add it to this manual.

This chapter also discusses type inference tools (see
Section~\ref{type-inference-tools}).


\section{Javac Compiler\label{javac-installation}}

If you use \code{javac} compiler from the command line, then you can use
the Type annotations compiler (a variant of the OpenJDK \code{javac}) that
is bundled with the Checker Framework.  The bundled \code{javac} recognizes
type annotations, and annotations in comments (see
Section~\ref{annotations-in-comments}).  (Eventually, you will be able to
use any Java compiler, such as the OpenJDK compiler, but Oracle has been
slow to incorporate all the patches, so the bundled \code{javac} is
superior, for the purpose of pluggable type-checking, and is equivalent in
all other respects.)

This section describes how you can install and use the bundled
\code{javac}, using either Unix/Linux/MacOS (see
Section~\ref{unix-installation}) or Windows (see
Section~\ref{windows-installation}).
The instructions are identical to those of Section~\ref{installation},
but are given as commands that you can cut and paste into your command shell.


%%% *****
%%% UPDATE
%%% *****

%%% Note that much of this section is duplicated with the "Windows
%%% installation" section.  That is better for users, even though it is
%%% longer and makes the maintainers keep two versions in sync.
\subsection{Unix/Linux/MacOS installation\label{unix-installation}}

These instructions assume that you use the bash or sh shell.  If you use a
different shell, you may need to slightly adjust the commands.

\begin{enumerate}

\item
  Download the latest Checker Framework distribution
  % (\ahrefurl{http://types.cs.washington.edu/checker-framework/current/checkers.zip})
  and unzip it.  You can put it anywhere you like by changing the
  definition of environment variable \code{JSR308} below; a standard place
  is in a
  new directory named \code{jsr308}.

\begin{Verbatim}
  export JSR308=$HOME/jsr308
  mkdir -p ${JSR308}
  cd ${JSR308}
  # or:  wget http://types.cs.washington.edu/checker-framework/current/checkers.zip
  curl -O http://types.cs.washington.edu/checker-framework/current/checkers.zip
  unzip checkers.zip
  chmod +x checker-framework/checkers/binary/javac
  checker-framework/checkers/binary/javac -version
\end{Verbatim}
% unconfuse Emacs LaTeX mode: $

The output of the last command should be:

\begin{Verbatim}
  javac 1.7.0-jsr308-1.2.1
\end{Verbatim}


\item
  Place the following commands in your \code{.bashrc} file:
\begin{Verbatim}
  export JSR308=$HOME/jsr308
  export CHECKERS=$JSR308/checker-framework/checkers
  export PATH=$CHECKERS/binary:${PATH}
\end{Verbatim}
% unconfuse Emacs LaTeX mode: $

% It is not necessary to add checkers.jar to your classpath, because the
% shipped compiler already does so.
%   export CLASSPATH=$JSR308/checker-framework/checkers/checkers.jar:${CLASSPATH}

Also execute them on the command line, or log out and back in.  Then,
verify that the installation works.  From the command line, run:

\begin{Verbatim}
  javac -version
\end{Verbatim}

The output should be:

\begin{Verbatim}
  javac 1.7.0-jsr308-1.2.1
\end{Verbatim}

\end{enumerate}

That's all there is to it!  Now you are ready to start using the checkers with
the new \code{javac} compiler.

\subsection{Windows installation\label{windows-installation}}

\begin{enumerate}

\item
  Download the latest Checker Framework distribution
  % (\ahrefurl{http://types.cs.washington.edu/checker-framework/current/checkers.zip})
  and unzip it to create a \<checkers> directory.  You can put it anywhere
  you like; a standard place is in a new directory under \<C:\ttbs{}Program
  Files>.

\begin{enumerate}
\item
  Save the file
  \ahrefurl{http://types.cs.washington.edu/checker-framework/current/checkers.zip}
  to your Desktop.
\item
  Double-click the \<checkers.zip> file on your computer.  Click on
  the \<checkers> directory, then Select \<Extract all files>, and use
  \<C:\ttbs{}Program Files> as the destination.  You will obtain a new
  \<C:\ttbs{}Program Files\ttbs{}checker-framework> folder.
\item
  Verify that the installation works.  From a Windows command prompt, run
  (all on one line, and don't forget to replace the \verb|\Path\To\...|!):

% Do I need to quote the space in "Program Files"?
\begin{Verbatim}
  set CHECKERS = C:\Program Files\checker-framework\checkers
  java -Xbootclasspath/p:%CHECKERS%\binary\jsr308-all.jar -jar C:%CHECKERS%\binary\jsr308-all.jar -version
\end{Verbatim}

The output should be:

\begin{Verbatim}
  javac 1.7.0-jsr308-1.2.1
\end{Verbatim}

\end{enumerate}


\item
  In order to use the updated compiler when you type \code{javac}, add the
  directory \<C:\ttbs{}Program Files\ttbs{}checker-framework\ttbs{}checkers\ttbs{}binary> to the
  beginning of your path variable.  Also set a CHECKERS variable.

% Instructions stolen from http://www.webreference.com/js/tips/020429.html

To set an environment variable, you have two options:  make the change
temporarily or permanently.
\begin{itemize}
\item
To make the change \textbf{temporarily}, type at the command shell prompt:

\begin{alltt}
path = \emph{newdir};%PATH%
\end{alltt}

For example:

\begin{Verbatim}
set CHECKERS = C:\Program Files\checker-framework\checkers
path = %CHECKERS%\binary;%PATH%
\end{Verbatim}

This is a temporary change that endures until the window is closed, and you
must re-do it every time you start a new command shell.

\item
To make the change \textbf{permanently},
Right-click the \<My Computer> icon and
select \<Properties>. Select the \<Advanced> tab and click the
\<Environment Variables> button. You can set the variable as a ``System
Variable'' (visible to all users) or as a ``User Variable'' (visible to
just this user).  Both work; the instructions below show how to set as a
``System Variable''.
In the \<System Variables> pane, select
\<Path> from the list and click \<Edit>. In the \<Edit System Variable>
dialog box, move the cursor to the beginning of the string in the
\<Variable Value> field and type the full directory name (not using the
\verb|%CHECKERS%| environment variable) followed by a
semicolon (\<;>).

% This is for the benefit of the Ant task.
Similarly, set the CHECKERS variable.

This is a permanent change that only needs to be done once ever.
\end{itemize}


% It is not necessary to add checkers.jar to your classpath, because the
% shipped compiler already does so.
%   export CLASSPATH=$JSR308/checker-framework/checkers/checkers.jar:${CLASSPATH}

Now, verify that the installation works.  From the command line, run:

\begin{Verbatim}
  javac -version
\end{Verbatim}

The output should be:

\begin{Verbatim}
  javac 1.7.0-jsr308-1.2.1
\end{Verbatim}

\end{enumerate}


\section{Ant task\label{ant-task}}

If you use the \ahref{http://ant.apache.org/}{Ant} build tool to compile
your software, then you can add an Ant task that runs a checker.  We assume
that your Ant file already contains a compilation target that uses the
\code{javac} task.

\begin{enumerate}
\item
Set the \code{jsr308javac} property:

%BEGIN LATEX
\begin{smaller}
%END LATEX
\begin{Verbatim}
  <property environment="env"/>

  <dirname property="checkers" file="${env.CHECKERS}" />

  <presetdef name="jsr308.javac">
    <javac fork="yes">
      <!-- JSR308 related compiler arguments -->
      <compilerarg value="-version"/>
      <!-- optional, so .class files work with older JVMs: <compilerarg line="-target 5"/> -->
      <compilerarg value="-implicit:class"/>
      <compilerarg line="-Awarns -Xmaxwarns 10000"/>
      <compilerarg value="-J-Xbootclasspath/p:${checkers}/binary/jsr308-all.jar"/>

      <classpath>
        <pathelement location="${checkers}/checkers.jar"/>
      </classpath>
    </javac>
  </presetdef>
\end{Verbatim}
% Unconfuse Emacs font lock mode: $
%BEGIN LATEX
\end{smaller}
%END LATEX

\item Duplicate the compilation target, then modify it slightly as
indicated in this example:

%BEGIN LATEX
\begin{smaller}
%END LATEX
\begin{Verbatim}
  <target name="check-nullness"
          description="Check for null pointer dereferences"
          depends="clean,...">
    <!-- use jsr308.javac instead of javac -->
    <jsr308.javac ... >
      <compilerarg line="-processor checkers.nullness.NullnessChecker"/>
      <compilerarg value="-Xbootclasspath/p:${checkers}/jdk/jdk.jar"/>
      <!-- optional, for implicit imports: <compilerarg value="-J-Djsr308_imports=checkers.nullness.quals.*"/> -->
      <!-- optional, to not check uses of library methods: <compilerarg value="-AskipUses=^(java\.awt\.|javax\.swing\.)"/> -->
      ...
    </jsr308.javac>
  </target>
\end{Verbatim}
%BEGIN LATEX
\end{smaller}
%END LATEX
% Unconfuse Emacs font lock mode: $

Fill in each ellipsis (\ldots) from the original compilation target.  But,
don't include any \code{-source} argument with value other than \code{1.8}
or \code{8}.  Doing so will disable the annotations in
comments feature (see Section~\refwithpage{annotations-in-comments}).

In the example, the target is named \code{check-nullness}, but you can
name it whatever you like.
\end{enumerate}

\subsection{Explanation\label{ant-task-explanation}}

This section explains each part of the Ant task.

\begin{enumerate}
\item Definition of \code{jsr308.javac}:

The \code{fork} field of the \code{javac} task
ensures that an external javac program is called.  Otherwise, Ant will run
javac via a Java method call, and there is no guarantee that it will get
the JSR 308 version that is distributed with the Checker Framework.

The \code{-version} compiler argument is just for debugging; you may omit
it.

The \code{-target 5} compiler argument is optional, if you use Java 5 in
ordinary compilation when not performing pluggable type-checking (see
Section~\refwithpage{java5-class-files}).

The \code{-implicit:class} compiler argument causes annotation processing
to be performed on implicitly compiled files.  (An implicitly compiled file
is one that was not specified on the command line, but for which the source
code is newer than the \code{.class} file.)  This is the default, but
supplying the argument explicitly suppresses a compiler warning.

The \code{-Awarns ...} compiler argument is optional, and causes the checker to
treat errors as warnings so that you can see all errors in all files rather
than only the errors in the first file; see Section~\ref{running}.

\item The \code{check-nullness} target:

The target assumes the existence of a \code{clean} target that removes all
\code{.class} files.  That is necessary because Ant's \code{javac} target
doesn't re-compile \code{.java} files for which a \code{.class} file
already exists.

The \code{-processor ...} compiler argument indicates which checker to
run.  You can supply additional arguments to the checker as well.

\end{enumerate}


\section{Maven plugin\label{maven-plugin}}

If you use the \ahref{http://maven.apache.org/}{Maven} project tool,
then you can specify the distributed checkers as part of your build
process.

\begin{enumerate}

\item First, you need to add the repositories in your \code{pom.xml} file:

\begin{Verbatim}
    <repositories>
        <repository>
            <id>checker-framework-repo</id>
            <url>http://types.cs.washington.edu/m2-repo</url>
        </repository>
    </repositories>
    <pluginRepositories>
        <pluginRepository>
            <id>checker-framework-repo</id>
            <url>http://types.cs.washington.edu/m2-repo</url>
        </pluginRepository>
    </pluginRepositories>
\end{Verbatim}

\item Then, to use the annotations used by the distributed checkers, you'll
have to declared as a dependency:

\begin{Verbatim}
    <dependencies>
        <!-- annotations for the standard checkers: nullness, interning, mutability -->
        <dependency>
            <groupId>types.checkers</groupId>
            <artifactId>checkers-quals</artifactId>
            <version>1.1.1</version>
        </dependency>

        <!-- other dependencies -->
    </dependencies>
\end{Verbatim}

\item And finally, you need to attach the plugin to your build lifecycle:

\begin{Verbatim}
    <build>
        <plugins>
            <plugin>
                <groupId>types.checkers</groupId>
                <artifactId>checkersplugin</artifactId>
                <version>0.1</version>
                <executions>
                    <execution>
                        <!-- run the checkers after compilation; this can also be any later phase -->
                        <phase>process-classes</phase>
                        <goals>
                            <goal>check</goal>
                        </goals>
                    </execution>
                </executions>
                <configuration>
                    <!-- required configuration options -->
                    <!-- a list of processors to run -->
                    <processors>
                        <processor>checkers.nullness.NullnessChecker</processor>
                        <processor>checkers.interning.InterningChecker</processor>
                    </processors>


                    <!-- other optional configuration -->
                    <!-- full path to a java executable that should be used to create the forked JVM -->
                    <executable>/opt/java1.6/bin/java</executable>
                    <!-- should an error reported by a checker cause a build failure, or only be logged as a warning; defaults to true -->
                    <failOnError>true|false</failOnError>
                    <!-- a list of patterns to include, in the standard maven syntax; defaults to **/*.java -->
                    <includes>
                        <include>org/company/important/**/*.java</include>
                    </includes>
                    <!-- a list of patterns to exclude, in the standard maven syntax; defaults to an empty list -->
                    <excludes>
                        <exclude>org/company/notimportant/**/*.java</exclude>
                    </excludes>
                    <!-- additional parameters passed to the JSR308 java compiler -->
                    <javacParams>-Alint</javacParams>
                    <!-- additional parameters to pass to the forked JVM -->
                    <javaParams>-Xdebug</javaParams>
                    <!-- versions of checkers to use; defaults to the current newest version: 1.0.6 -->
                    <checkersVersion>0.8.8</checkersVersion>
                </configuration>
            </plugin>
        </plugins>
    </build>
\end{Verbatim}

\end{enumerate}

The plugin was contributed by Adam Warski.
% [Thanks Adam!]


\section{Gradle\label{gradle}}

% This information came from:
% http://jira.codehaus.org/browse/GRADLE-342
% http://docs.codehaus.org/display/GRADLE/Gradle+0.8+Breaking+Changes

\ahref{http://gradle.org/}{Gradle} lets you add command-line arguments to a
\code{javac} invocation by setting the compilerArgs property of the
compiler options.  This is adequate for running the Checker Framework,
because it is run by specifying command-line arguments.  See the
instructions elsewhere in this manual for a list of command-line arguments.

To specify command-line arguments, set
\code{compile.options.compilerArgs}.  Here is a possible example:

\begin{Verbatim}
allprojects {
  tasks.withType(Compile).allTasks { Compile compile ->
    compile.options.debug = true
    compile.options.compilerArgs = [
      '-version',
      '-implicit:class',
      '-Awarns', '-Xmaxwarns', '10000',
      '-J-Xbootclasspath/p:${env.CHECKERS}/binary/jsr308-all.jar',
      '-processor', 'checkers.nullness.NullnessChecker',
      '-Xbootclasspath/p:${env.CHECKERS}/jdk/jdk.jar']
  }
}
\end{Verbatim}

You don't need to use a special version of \code{javac}; you only need to
give the \code{-J-Xbootclasspath/p:...} argument.  If you choose to use a
special version of \code{javac} instead of supplying the command-line
argument, then you can do so in the following way:

\begin{Verbatim}
tasks.withType(Compile).allObjects { compile ->
  compile.options.fork.executable = "$CHECKERS/binary/javac"
}
\end{Verbatim}
% extra $ to unconfuse Emacs's LaTeX mode


\section{IntelliJ IDEA\label{intellij}}

IntelliJ IDEA (Maia release)
\ahref{http://blogs.jetbrains.com/idea/2009/07/type-annotations-jsr-308-support/}{supports}
the Type Annotations (JSR-308) syntax.
See \url{http://blogs.jetbrains.com/idea/2009/07/type-annotations-jsr-308-support/}.

\section{Eclipse\label{eclipse}}

There are two ways to run a checker from within the Eclipse IDE:  via Ant
or using an Eclipse plug-in.


\paragraph{Using an Ant task}

Add an Ant target as described in Section~\ref{ant-task}.  You can
run the Ant target by executing the following steps
(instructions copied from
\myurl{http://www.eclipse.org/documentation/?topic=/org.eclipse.platform.doc.user/gettingStarted/qs-84_run_ant.htm}):

\begin{enumerate}

\item
  Select \code{build.xml} in one of the navigation views and choose
  {\bf Run As $>$ Ant Build...} from its context menu.

\item
  A launch configuration dialog is opened on a launch configuration
  for this Ant buildfile.

\item
  In the {\bf Targets} tab, select the new ant task (e.g., check-interning).

\item
  Click {\bf Run}.

\item
  The Ant buildfile is run, and the output is sent to the Console view.

\end{enumerate}

\paragraph{Eclipse plug-in for the Checker Framework}

The Checker Plugin is an Eclipse plugin that enables the use of the Checker
Framework.
Its website (\myurl{http://types.cs.washington.edu/checker-framework/eclipse/}).
The website contains instructions for installing and using the plugin.
% The plugin has been substantially improved through a Google Summer of Code 2010 project
% and supports all checkers that are distributed with the Checker Framework.

\paragraph{Eclipse plug-in for Type Annotations}

A prototype version of Type Annotations support for Eclipse is
available from the Eclipse project.  The goal is to enable full support for
writing
type annotations outside of comments.  You do not need this to run the
Checker Framework, whether or not you write your type annotations in comments.

% Username/password is necessary according to Jon McCord, 5/23/2010
(Update:  this apparently needs a username and password, so it may not be
publicly available.)
Use the following information to check
out the CVS repository:
\begin{description}
\item[Host:]                 dev.eclipse.org
\item[Repository path:] /cvsroot/eclipse
\item[Module name:]    org.eclipse.jdt.core
\item[Branch:]             JSR\_308
\end{description}


\section{tIDE\label{tide}}

tIDE, an open-source Java IDE, supports the Checker Framework.  See its
documentation at \myurl{http://tide.olympe-network.com/}.


\section{Type inference tools\label{type-inference-tools}}

\subsection{Varieties of type inference}

There are two different tasks that are commonly called ``type inference''.

\begin{enumerate}
\item
  Type inference during type checking (Section~\ref{type-refinement}):
  During type checking, if certain variables have no type qualifier, the
  type-checker determines whether there is some type qualifier that would
  permit the program to type check.  If so, the type checker uses that type
  qualifier, but never tells the programmer what it was.  Each time the
  type checker runs, it re-infers the type qualifier for that variable.  If
  no type qualifier exists that permits the program to type-check, the
  type-checker issues a type warning.

  This variety of type inference is built into the Checker Framework.  Every
  checker can take advantage of it at no extra effort.  However, it only
  works within a method, not across method boundaries.

  Advantages of this variety of type inference include:
  \begin{itemize}
  \item
    If the type qualifier is obvious to the programmer, then omitting it
    can reduce annotation clutter in the program.
  \item
    The type inference can take advantage of only the code currently being
    compiled, rather than having to be correct for all possible calls.
    Additionally, if the code changes, then there is no old annotation to
    update.
  \end{itemize}


\item
  Type inference to annotate a program (Section~\ref{type-inference-to-annotate}):
  As a separate step before type checking, a type inference tool takes the
  program as input, and outputs a set of type qualifiers that would
  type-check.  These qualifiers are inserted into the source code or the
  class file.  They can be viewed and adjusted by the programmer, and can
  be used by tools such as the type checker.

  This variety of type inference must be provided by a separate tool.  It
  is not built into the Checker Framework.

  Advantages of this variety of type inference include:
  \begin{itemize}
  \item
    The program contains documentation in the form of type qualifiers,
    which can aid programmer understanding.
  \item
    Error messages may be more comprehensible.  With type inference
    during type checking, error messages can be obscure, because the
    compiler has already inferred (possibly incorrect) types for a number
    of variables.
  \item
    A minor advantage is speed:  type-checking can be modular, which can be
    faster than re-doing type inference every time the
    program is type-checked.
  \end{itemize}

\end{enumerate}

Advantages of both varieties of inference include:
\begin{itemize}
\item
  Less work for the programmer.
\item
  The tool chooses the most general type, whereas a programmer might
  accidentally write a more specific, less generally-useful annotation.
\end{itemize}


Each variety of type inference has its place.  When using the Checker
Framework, type inference during type checking is performed only
\emph{within} a method (Section~\ref{type-refinement}).  Every method
signature (arguments and return values) and field must be explicitly annotated,
either by the programmer or by a separate type checking tool
(Section~\ref{type-inference-to-annotate}).  This choice reduces programmer
effort (typically, a programmer does not have to write any qualifiers
inside the body of a method) while still retaining modular checking and
documentation benefits.


\subsection{Type inference to annotate a program\label{type-inference-to-annotate}}

This section lists tools that take a program and output a set of
annotations for it.

Section~\ref{nullness-inference} lists several tools that infer
annotations for the Nullness Checker.

Section~\ref{javari-inference} lists a tool that infers
annotations for the Javari Checker, which detects mutation errors.


% LocalWords:  jsr plugin Warski xml buildfile tIDE java Awarns pom lifecycle
% LocalWords:  IntelliJ Maia newdir classpath Unconfuse nullness Gradle
% LocalWords:  compilerArgs Xbootclasspath

\htmlhr
\chapter{Frequently Asked Questions (FAQs)\label{faq}}

These are some common questions about the Checker Framework and about
pluggable type-checking in general.  Feel free to suggest improvements to
the answers, or other questions to include here.

% Not supported by Hevea, so don't bother; instead do by hand:
% \minitoc

%BEGIN LATEX
~
%END LATEX

%BEGIN LATEX
\newcommand{\faqtocpara}[1]{\paragraph{#1} ~}
%END LATEX
%HEVEA \newcommand{\faqtocpara}[1]{\textbf{#1}}


\noindent
\textbf{Contents:}

\faqtocpara{\ref{faq-motivation-section}: Motivation for pluggable type-checking}
\\ \ref{never-make-type-errors}: I don't make type errors, so would pluggable type-checking help me?
\\ \ref{faq-qualifiers-vs-subclasses}: When should I use type qualifiers, and when should I use subclasses?

\faqtocpara{\ref{faq-getting-started-section}: Getting started}
\\ \ref{faq-annotate-existing-program}: How do I get started annotating an existing program?
\\ \ref{faq-first-checker}: Which checker should I start with?
\\ \ref{faq-typequals-vs-subtypes}: Should I use pluggable types or Java subtypes?
\\ \ref{faq-checker-framework-dev}: How can I join the checker-framework-dev mailing list?

\faqtocpara{\ref{faq-usability-section}: Usability of pluggable type-checking}
\\ \ref{faq-ease-of-use}: Are type annotations easy to read and write?
\\ \ref{faq-code-clutter}: Will my code become cluttered with type annotations?
\\ \ref{faq-slowdown}: Will using the Checker Framework slow down my program?  Will it slow down the compiler?
\\ \ref{faq-shorten-command-line}: How do I shorten the command line when invoking a checker?

\faqtocpara{\ref{faq-warnings-section}: How to handle warnings}
\\ \ref{faq-handling-warnings}: What should I do if a checker issues a warning about my code?
\\ \ref{faq-interpreting-warnings}: What does a certain Checker Framework warning message mean?
\\ \ref{faq-no-absolute-guarantee}: Can a pluggable type-checker guarantee that my code is correct?
\\ \ref{faq-concurrency}: What guarantee does the Checker Framework give for concurrent code?
\\ \ref{faq-awarns}: How do I make compilation succeed even if a checker issues errors?
\\ \ref{faq-100-warnings}: Why does the checker always say there are 100 errors or warnings?
\\ \ref{faq-type-i-did-not-write}: Why does the Checker Framework report an error regarding a type I have not written in my program?
\\ \ref{faq-run-time-checking}: How can I do run-time monitoring of properties that were not statically checked?

\faqtocpara{\ref{faq-syntax-section}: Syntax of type annotations}
\\ \ref{faq-receiver}: What is a ``receiver''?
\\ \ref{faq-annotation-after-type}: What is the meaning of an annotation after a type, such as \<@NonNull Object @Nullable>?
\\ \ref{faq-array-syntax-meaning}: What is the meaning of array annotations such as \<@NonNull Object @Nullable []>?
\\ \ref{faq-varargs-syntax-meaning}: What is the meaning of varargs annotations such as \<@English String @NonEmpty~...>?
\\ \ref{faq-type-qualifier-on-class-declaration}: What is the meaning of a type qualifier at a class declaration?
\\ \ref{faq-no-annotation-on-types-and-declarations}: Why shouldn't a qualifier apply to both types and declarations?
\\ \ref{faq-annotate-fully-qualified-name}: How do I annotate a
fully-qualified type name?

\faqtocpara{\ref{faq-semantics-section}: Semantics of type annotations}
\\ \ref{faq-list-map-nonnull-typeargs}: Why are the type parameters to \<List> and \<Map> annotated as \<@NonNull>?
\\ \ref{faq-typestate}: How can I handle typestate, or phases of my program with different data properties?
\\ \ref{faq-implicit-bounds}: Why are explicit and implicit bounds defaulted differently?

\faqtocpara{\ref{faq-create-a-checker-section}: Creating a new checker}
\\ \ref{faq-create-a-checker}: How do I create a new checker?
\\ \ref{faq-declarative-syntax-for-type-rules}: Why is there no declarative syntax for writing type rules?

\faqtocpara{\ref{faq-other-tools-section}: Relationship to other tools}
\\ \ref{faq-type-checking-vs-bug-detectors}: Why not just use a bug detector (like FindBugs)?
\\ \ref{faq-eclipse}: How does the Checker Framework compare with Eclipse's Null Analysis?
\\ \ref{faq-jml}: How does pluggable type-checking compare with JML?
\\ \ref{faq-checker-framework-part-of-java}: Is the Checker Framework an official part of Java?
\\ \ref{faq-jsr-305}: What is the relationship between the Checker Framework and JSR 305?
\\ \ref{faq-jsr-308}: What is the relationship between the Checker Framework and JSR 308?


\section{Motivation for pluggable type-checking\label{faq-motivation-section}}

\subsection{I don't make type errors, so would pluggable type-checking help me?\label{never-make-type-errors}}

Occasionally, a developer says that he makes no errors that type-checking
could catch, or that any such errors are unimportant because they have low
impact and are easy to fix.  When I investigate the claim, I invariably
find that the developer is mistaken.

Very frequently, the developer has underestimated what type-checking can
discover.  Not every type error leads to an exception being thrown; and
even if an exception is thrown, it may not seem related to classical types.
Remember that a type system can discover
null pointer dereferences,
incorrect side effects,
security errors such as information leakage or SQL injection,
partially-initialized data,
wrong units of measurement,
and many other errors.
Every programmer makes errors sometimes and works with other people
who do.
Even where type-checking does not discover a
problem directly, it can indicate code with bad smells, thus revealing
problems, improving documentation, and making future maintenance easier.

There are other ways to discover errors, including extensive testing and
debugging.  You should continue to use these.
But type-checking is a good complement to these.  Type-checking is more
effective for some problems, and less effective for other problems.  It can
reduce (but not eliminate) the time and effort that you spend on other
approaches.  There are many important errors that type-checking and other
automated approaches cannot find; pluggable type-checking gives you more
time to focus on those.


\subsection{When should I use type qualifiers, and when should I use subclasses?\label{faq-qualifiers-vs-subclasses}}

In brief, use subtypes when you can, and use type qualifiers when you cannot
use subtypes.
For more details, see Section~\ref{when-to-use-type-qualifiers}.



\section{Getting started\label{faq-getting-started-section}}

\subsection{How do I get started annotating an existing program?\label{faq-annotate-existing-program}}

See Section~\ref{get-started-with-legacy-code}.


\subsection{Which checker should I start with?\label{faq-first-checker}}

You should start with a property that matters to you.  Think about what
aspects of your code cause the most errors, or cost the most time during
maintenance, or are the most common to be incorrectly-documented.  Focusing
on what you care about will give you the best benefits.

When you first start out with the Checker Framework, it's usually best to
get experience with an existing type-checker before you write your own new
checker.

Many users are tempted to start with the
\ahrefloc{nullness-checker}{Nullness Checker} (see
\chapterpageref{nullness-checker}), since null pointer errors are common
and familiar.  The Nullness Checker works very well, but be warned of three
facts that make the absence of null pointer exceptions challenging to
verify.

\begin{enumerate}
\item
  Dereferences happen throughout your codebase, so there are a lot of
  potential problems.  By contrast, fewer lines of code are related to
  locking, regular expressions, etc., so those properties are easier to
  check.
\item
  Programmers use \<null> for many different purposes.  More seriously,
  programmers write run-time tests against \<null>, and those are difficult
  for any static analysis to capture.
\item
  The Nullness Checker interacts with initialization and map keys.
\end{enumerate}

If null pointer exceptions are most important to you, then by all means use
the Nullness Checker.  But if you just want to try \emph{some}
type-checker, there are others that are easier to use.

we do not recommend indiscriminately running all the checkers on your code.
The reason is that each one has a cost --- not just at compile time, but
also in terms of code clutter and human time to maintain the annotations.
If the property is important to you, is difficult for people to reason
about, or has caused problems in the past, then you should run that
checker.  For other properties, the benefits may not repay the effort to
use it.  You will be the best judge of this for your own code, of course.

The \ahrefloc{linear-checker}{Linear Checker} (see
\chapterpageref{linear-checker}) has not been extensively tested.
The
\ahrefloc{igj-checker}{IGJ Checker} (see \chapterpageref{igj-checker}),
\ahrefloc{javari-checker}{Javari Checker} (see
\chapterpageref{javari-checker}), and some of the third-party checkers (see
\chapterpageref{third-party-checkers})
have known bugs that limit their
usability.  (Report the ones that affect you, and the Checker Framework
developers will prioritize fixing them.)


\subsection{Should I use pluggable types or Java subtypes?\label{faq-typequals-vs-subtypes}}

% Old label, from when this discussion was in the introduction
\label{when-to-use-type-qualifiers}

For some programming tasks, you can use either a Java subclass or a type
qualifier.  As an example that your code currently uses \code{String} to
represent an address.  You could use Java subclasses by creating a new
\code{Address} class and refactor your code to use it, or you could use
type qualifiers by creating an \code{@Address} annotation and applying it
to some uses of \code{String} in your code.  As another example, suppose
that your code currently uses \code{MyClass} in two different ways that
should not interact with one another.  You could use Java subclasses by
changing MyClass into an interface or abstract class, defining two
subclasses, and ensuring that neither subclass ever refers to the other
subclass nor to the parent class.

If  Java subclasses solve your problem, then that is probably better.
We do not encourage you to use type qualifiers as a poor substitute for
classes.  An advantage of using classes is that the Java type-checker
always runs; by contrast, it is possible to forget to run the pluggable
type-checker.  However, here are some reasons type qualifiers may be a
better choice.

\begin{description}

\item[Backward compatibility]
Using a new class may make your code incompatible with existing libraries or
clients.  Brian Goetz expands on this issue in an article on the
pseudo-typedef antipattern~\cite{Goetz2006:typedef}.  Even if compatibility
is not a concern, a code change may introduce bugs, whereas adding
annotations does not change the run-time behavior.  It is possible to add
annotations to existing code, including code you do not maintain or cannot
change.  For code that strictly cannot be changed, you can add
annotations in comments (see Section~\ref{annotations-in-comments}), or you
can write library annotations (see Chapter~\ref{annotating-libraries}).

\item[Broader applicability]
Type annotations can be applied to primitives and to final classes such as
\code{String}, which cannot be subclassed.

\item[Richer semantics and new supertypes]
Type qualifiers permit you to remove operations, with a compile-time
guarantee.  An example is that an immutable version of a type prohibits
calling mutator methods
(see Chapters~\ref{igj-checker} and~\ref{javari-checker}).  More
generally, type qualifiers permit creating a new supertype, not just a
subtype, of an existing Java type.

\item[More precise type-checking]
The Checker Framework is able to verify the correctness of code that the
Java type-checker would reject.  Here are a few examples.
\begin{itemize}
\item
  It uses a dataflow analysis to determine a more precise type for
  variables after conditional tests or assignments.
\item
  It treats certain Java constructs more precisely, such as
  reflection (see Chapter~\ref{reflection-resolution}).
\item
  It includes special-case logic for type-checking specific methods, such
  as the Nullness Checker's treatment of \code{Map.get}.
\end{itemize}


\item[Efficiency]
  Type qualifiers have no run-time representation.  Therefore, there is no
  space overhead for separate classes or for wrapper classes for
  primitives.  There is no run-time overhead for due to extra dereferences
  or dynamic dispatch for methods that could otherwise be statically
  dispatched.

\item[Less code clutter]
  The programmer does not have to convert primitive types to wrappers,
  which would make the code both uglier and slower.  Thanks to defaults and
  type inference (Section~\ref{defaults}),
  you may be able to write and think in terms of the
  original Java type, rather than having to explicitly write one of the
  subtypes in all locations.

\end{description}

\subsection{How can I join the checker-framework-dev mailing list?\label{faq-checker-framework-dev}}

The \code{checker-framework-dev@googlegroups.com} mailing list is for
Checker Framework developers.  Anyone is welcome to
\href{https://groups.google.com/forum/#!forum/checker-framework-dev}{join
  \code{checker-framework-dev}}, after they have had several pull requests
accepted.

Anyone is welcome to send mail to the
\code{checker-framework-dev@googlegroups.com} mailing list --- for
implementation details it is generally a better place for discussions than
the general \code{checker-framework-discuss@googlegroups.com} mailing list,
which is for user-focused discussions.

Anyone is welcome to
\href{https://groups.google.com/forum/#!forum/checker-framework-discuss}{join
  \code{checker-framework-discuss@googlegroups.com}} and send mail to it.


\section{Usability of pluggable type-checking\label{faq-usability-section}}

\subsection{Are type annotations easy to read and write?\label{faq-ease-of-use}}

% This FAQ also appears in the JSR 308 FAQ.
% When I update one, also update the other.

The papers
\href{http://homes.cs.washington.edu/~mernst/pubs/pluggable-checkers-issta2008-abstract.html}{``Practical
  pluggable types for Java''}~\cite{PapiACPE2008}
and
\href{http://homes.cs.washington.edu/~mernst/pubs/pluggable-checkers-icse2011-abstract.html}{``Building
  and using pluggable type-checkers''}~\cite{DietlDEMS2011}
discuss case studies in
which programmers
found type annotations to be natural to read and write.  The code
continued to feel like Java, and the type-checking errors were easy to
comprehend and often led to real bugs.

You don't have to take our word for it, though.  You can try the
Checker Framework for yourself.

The difficulty of adding and verifying annotations depends on your program.
If your program is well-designed and -documented, then skimming the
existing documentation and writing type annotations is extremely easy.
Otherwise, you may find yourself spending a lot of time trying to
understand, reverse-engineer, or fix bugs in your program, and then just a
moment writing a type annotation that describes what you discovered.  This
process inevitably improves your code.  You must decide whether it is a
good use of your time.  For code that is not causing trouble now and is
unlikely to do so in the future (the code is bug-free, and you do not
anticipate changing it or using it in new contexts), then the
effort of writing type annotations for it may not be justified.


\subsection{Will my code become cluttered with type annotations?\label{faq-code-clutter}}

% This FAQ also appears in the JSR 308 FAQ.
% When I update one, also update the other.

In summary:  annotations do not clutter code; they are used much
less frequently than generic types, which Java programmers find acceptable;
and they reduce the overall volume of documentation that a codebase needs.

As with any language feature, it is possible to write ugly code that
over-uses annotations.  However, in normal use, very few annotations need
to be written.  Figure 1 of the paper
\href{http://homes.cs.washington.edu/~mernst/pubs/pluggable-checkers-issta2008-abstract.html}{Practical
  pluggable types for Java}~\cite{PapiACPE2008} reports data for over
350,000 lines of type-annotated code:

\begin{itemize}
\item
    1 annotation per 62 lines for nullness annotations (\<@NonNull>, \<@Nullable>, etc.)
    % (/ (+ 4640 3961 10798) (+ 107 35 167))
\item
    1 annotation per 1736 lines for interning annotations (\<@Interned>)
    % (/ 224048 129)
\item
    1 annotation per 27 lines for immutability annotations (IGJ type system)
    % (/ (+ 6236 18159 30507 8691 59221 26828) (+ 315 1125 1386 384 1815 450))
\end{itemize}

% ICSE 2011 paper says:
% Signature String Checker: less than 1 annotation per 500 lines of code

These numbers are for annotating existing code.  New code that
is written with the type annotation system in mind is cleaner and more
correct, so it requires even fewer annotations.

Each annotation that a programmer writes replaces a sentence or phrase of
English descriptive text that would otherwise have been written in the
Javadoc.  So, use of annotations actually reduces the overall size of the
documentation, at the same time as making it machine-processable
and less ambiguous.


\subsection{Will using the Checker Framework slow down my program?  Will it slow down the compiler?\label{faq-slowdown}}

Using the Checker Framework has no impact on the execution of your program:
the compiler emits the identical bytecodes as the Java 8
compiler and so there is no run-time effect.  Because there is no run-time
representation of type qualifiers, there is no way to use reflection to
query the qualifier on a given object, though you can use reflection to
examine a class/method/field declaration.

Using the Checker Framework does increase compilation time.  In theory it
should only add a few percent overhead, but our current implementation
can double the compilation time --- or more, if you run many pluggable
type-checkers at once.  This is especially true if you run pluggable
type-checking on every file (as we recommend) instead of just on the ones
that have recently changed.
Nonetheless, compilation with pluggable type-checking still feels like
compilation, and you can do it as part of your normal development process.


\subsection{How do I shorten the command line when invoking a checker?\label{faq-shorten-command-line}}

\begin{sloppypar}
The compile options to javac can be a pain to type; for example,
\code{javac -processor org.checkerframework.checker.nullness.NullnessChecker ...}.
See Section~\ref{checker-auto-discovery} for a way to avoid the need for
the \code{-processor} command-line option.
\end{sloppypar}


\section{How to handle warnings and errors\label{faq-warnings-section}}

\subsection{What should I do if a checker issues a warning about my code?\label{faq-handling-warnings}}

For a discussion of this issue, see Section~\ref{handling-warnings}.


\subsection{What does a certain Checker Framework warning message mean?\label{faq-interpreting-warnings}}

Search through this manual for the text of the warning message.
Oftentimes the manual explains it.  If not, ask on the \href{https://groups.google.com/forum/#!forum/checker-framework-discuss}{mailing list}.


\subsection{Can a pluggable type-checker guarantee that my code is correct?\label{faq-no-absolute-guarantee}}

Each checker looks for certain errors.  You can use multiple checkers to
detect more errors in your code, but you will never have a guarantee that
your code is completely bug-free.

If the type-checker issues no warning, then you have a guarantee that your
code is free of some particular error.  There are some limitations to the
guarantee.

Most importantly, if you run a pluggable checker on only part of a program, then
you only get a guarantee that those parts of the program are error-free.
For example, suppose you have type-checked a framework that clients
are intended to extend.  You should recommend that clients
run the pluggable checker.  There is no way to force users to do so, so you
may want to retain dynamic checks or use other mechanisms to detect errors.

Section~\ref{checker-guarantees} states other limitations to a checker's
guarantee, such as regarding concurrency.  Java's type system is also
unsound in certain situations, such as for arrays and casts (however, the
Checker Framework is sound for arrays and casts).  Java uses dynamic checks
is some places it is unsound, so that errors are thrown at run time.  The
pluggable type-checkers do not currently have built-in dynamic checkers to
check for the places they are unsound.
Writing dynamic checkers would be an interesting and valuable project.

Other types of dynamism in a Java application do not jeopardize the
guarantee, because the type-checker is conservative.  For example, at a
method call, dynamic dispatch chooses some implementation of the method,
but it is impossible to know at compile time which one it will be.  The
type-checker gives a guarantee no matter what implementation of the method
is invoked.

% This paragraph is weak.

Even if a pluggable checker cannot give an ironclad
guarantee of correctness, it is still useful.  It can find errors,
exclude certain types of possible problems (e.g., restricting the
possible class of problems), improve documentation, and increase confidence
in your software.


\subsection{What guarantee does the Checker Framework give for concurrent code?\label{faq-concurrency}}

The Lock Checker (see Chapter~\ref{lock-checker}) offers a way to detect
and prevent certain concurrency errors.


By default, the Checker Framework assumes that the code that it is checking
is sequential:  that is, there are no concurrent accesses from another
thread.  This means that the Checker Framework is unsound for concurrent
code, in the sense that it may fail to issue a warning about errors that
occur only when the code is running in a concurrent setting.
For example, the Nullness Checker issues no warning for this
code:

\begin{Verbatim}
  if (myobject.myfield != null) {
    myobject.myfield.toString();
  }
\end{Verbatim}

\noindent
This code is safe when run on its own.
However, in the presence of multithreading, the call to \<toString> may
fail because another thread may set \<myobject.myfield> to \<null> after
the nullness check in the \<if> condition, but before the \<if> body is
executed.

If you supply the \<-AconcurrentSemantics> command-line option, then the
Checker Framework assumes that any field can be changed at any time.  This
limits the amount of flow-sensitive type qualifier refinement
(Section~\ref{type-refinement}) that the Checker Framework can do.


% If you are concerned about concurrency, then the ``fix''
% of putting data in a local variable doesn't fix the problem,
% just masks it from one particular checker.  This is bad style and may make
% debugging harder rather than easier.
%
% For instance, suppose you you have
%
% if (x.val != null) {
%   x.val = x.val + 1;
% }
%
% which can suffer a null pointer exception if another thread nulls out
% x.val.  The underlying problem is the possible concurrency error:  the user
% should have used locks or some other mechanism to protect access to x.val.
%
% Changing this to
%
% myval = x.val;
% if (myval != null) {
%   x.val = myval + 1;
% }
%
% does not fix the concurrency error, because no locking has been introduced.
% The code still has a data race that can lose updates or corrupt data
% structures.  The code has been transformed so that the Nullness Checker
% does not issue a warning, but this is scant comfort since the code is no
% more correct than it was before.
%
% If you want to detect concurrency errors, then it is better to use a
% correct checker that is concurrency-aware, rather than an unsound one that
% encourages incorrect workarounds.  Another way to put this is that a static
% checker should encourage better overall design, not just different bad
% designs.


\subsection{How do I make compilation succeed even if a checker issues errors?\label{faq-awarns}}

Section~\ref{running} describes the \<-Awarns> command-line
option that turns checker errors into warnings, so type-checking errors
will not cause \<javac> to exit with a failure status.


\subsection{Why does the checker always say there are 100 errors or warnings?\label{faq-100-warnings}}

By default, javac only reports the first 100 errors or warnings.
Furthermore, once javac encounters an error, it doesn't try compiling any
more files (but does complete compilation of all the ones that it has
started so far).

To see more than 100 errors or warnings, use the javac options \<-Xmaxerrs>
and \<-Xmaxwarns>.  To convert Checker Framework errors into warnings so
that javac will process all your source files, use the option \<-Awarns>.
See Section~\ref{running} for more details.


\subsection{Why does the Checker Framework report an error regarding a type I have not written in my program?\label{faq-type-i-did-not-write}}


Sometimes, a Checker Framework warning message will mention a type you have
not written in your program.  This is typically because a default has been
applied where you did not write a type; see Section~\ref{defaults}.  In
other cases, this is because flow-sensitive type refinement has given an
expression a more specific type than you wrote or than was defaulted; see
Section~\ref{type-refinement}.


\subsection{How can I do run-time monitoring of properties that were not statically checked?\label{faq-run-time-checking}}

Some properties are not checked statically (see
Chapter~\ref{suppressing-warnings} for reasons that code might not be
statically checked).  In such cases, it would be desirable to check the
property dynamically, at run time.
Currently, the Checker Framework has no support for adding code to perform
run-time checking.

Adding such support would be an interesting and valuable project.
An example would be an option that causes the Checker Framework to
automatically insert a run-time check anywhere that static checking is
suppressed.
% such as casts
If you
are able to add run-time verification functionality, we would gladly
welcome it as a contribution to the Checker Framework.

Some checkers have library methods that you can explicitly insert in your
source code.
Examples include the Nullness Checker's
\refmethod{checker/nullness}{NullnessUtils}{castNonNull}{-T-} method (see
Section~\ref{suppressing-warnings-with-assertions}) and the Regex Checker's
\<RegexUtil> class (see Section~\ref{regexutil-methods}).
But, it would be better to have more general support that does not require
the user to explicitly insert method calls.


\section{Syntax of type annotations\label{faq-syntax-section}}

There is also a separate FAQ for the type annotations syntax
(\url{http://types.cs.washington.edu/jsr308/current/jsr308-faq.html}).


\subsection{What is a ``receiver''?\label{faq-receiver}}

The \emph{receiver} of a method is the \<this> formal parameter, sometimes
also called the ``current object''.  Within the method declaration, \<this>
is used to refer to the receiver formal parameter.  At a method call, the
receiver actual argument is written before the method name.

The method \<compareTo> takes \emph{two} formal parameters.  At a call site
like \<x.compareTo(y)>, the two arguments are \<x> and \<y>.  It is
desirable to be able to annotate the types of both of the formal
parameters, and doing so is supported by both Java's type annotations
syntax and by the Checker Framework.

A type annotation on the receiver is treated exactly like a type annotation
on any other formal parameter.  At each call site, the type of the argument
must be a consistent with (a subtype of or equal to) the declaration of the
corresponding formal parameter.  If not, the type-checker issues a warning.

Here is an example.  Suppose that \<@A Object> is a supertype of \<@B
Object> in the following declaration:

\begin{Verbatim}
  class MyClass {
    void requiresA(@A MyClass this) { ... }
    void requiresB(@B MyClass this) { ... }
  }
\end{Verbatim}

\noindent
Then the behavior of four different invocations is as follows:

\begin{Verbatim}
  @A MyClass myA = ...;
  @B MyClass myB = ...;

  myA.requiresA()    // OK
  myA.requiresB()    // compile-time error
  myB.requiresA()    // OK
  myB.requiresB()    // OK
\end{Verbatim}

The invocation \<myA.requiresB()> does not type-check because the actual
argument's type is not a subtype of the formal parameter's type.

A top-level constructor does not have a receiver.  An inner class
constructor does have a receiver, whose type is the same as the containing
outer class.  The receiver is distinct from the object being constructed.
In a method of a top-level class, the receiver is named \<this>.  In a
constructor of an inner class, the receiver is named \<Outer.this> and the
result is named \<this>.


\subsection{What is the meaning of an annotation after a type, such as \<@NonNull Object @Nullable>?\label{faq-annotation-after-type}}

In a type such as \<@NonNull Object @Nullable []>, it may appear that the
\<@Nullable> annotation is written \emph{after} the type \<Object>.  In
fact, \<@Nullable> modifies \<[]>.  See the next FAQ, about array
annotations (Section~\ref{faq-array-syntax-meaning}).


\subsection{What is the meaning of array annotations such as \<@NonNull Object @Nullable []>?\label{faq-array-syntax-meaning}}

You should parse this as:
(\textbf{\<@NonNull Object>}) (\textbf{\<@Nullable []>}).
Each annotation precedes the component of the type that it qualifies.

Thus,
\<@NonNull Object @Nullable []> is a possibly-null array of non-null
objects.  Note that the first token in the type,
``\<@NonNull>'', applies to the element
type \<Object>, not to the array type as a whole.  The annotation \<@Nullable> applies to the
array (\<[]>).

Similarly,
\<@Nullable Object @NonNull []> is a non-null array of possibly-null
objects.


Some older tools interpret a declaration like \<@NonEmpty String[] var> as
``non-empty array of strings''.  This is in conflict with the Java type
annotations specification, which defines it as meaning ``array of
non-empty strings''.
% (and has since October 2007)
If you use one of these
older tools, you will find this incompatibility confusing.
You will have to live with it until the older
tool is updated to conform to the Java specification, or until you
transition to a newer tool that conforms to the Java specification.


\subsection{What is the meaning of varargs annotations such as \<@English String @NonEmpty~...>?\label{faq-varargs-syntax-meaning}}

Varargs annotations are treated similarly to array annotations.
(A way to remember this is that
when you write a varargs formal parameter such as
\<void method(String... x) \ttlcb\ttrcb>, the Java compiler generates a
method that takes an array of strings; whenever your source code calls the
method with multiple arguments, the Java compiler packages them up into an
array before calling the method.)

Either of these annotations

\begin{Verbatim}
  void method(String @NonEmpty [] x) {}
  void method(String @NonEmpty ... x) {}
\end{Verbatim}

\noindent
applies to the array:  the method takes a non-empty array of strings, or
the varargs list must not be empty.

Either of these annotations

\begin{Verbatim}
  void method(@English String [] x) {}
  void method(@English String ... x) {}
\end{Verbatim}

\noindent.
applies to the element type. The annotation documents that the method takes an array of English strings.


\subsection{What is the meaning of a type qualifier at a class declaration?\label{faq-type-qualifier-on-class-declaration}}

% TODO: use a more realistic example.

Writing an annotation on a class declaration makes that annotation implicit
for all uses of the class (see Section~\ref{effective-qualifier}).  If you
write \<class @MyQual MyClass \ttlcb\ ... \ttrcb>, then every unannotated
use of \<MyClass> is \<@MyQual MyClass>.  A user is permitted to strengthen
the type by writing a more restrictive annotation on a use of MyClass, such
as \<@MyMoreRestrictiveQual MyClass>.


\subsection{Why shouldn't a qualifier apply to both types and declarations?\label{faq-no-annotation-on-types-and-declarations}}

It is bad style for an annotation to apply to both types and declarations.
In other words, every annotation should have a \<@Target> meta-annotation,
and the \<@Target> meta-annotation should list either only declaration
locations or only type annotations.  (It's OK for an annotation to target
both \<ElementType.TYPE\_PARAMETER> and \<ElementType.TYPE\_USE>, but no
other declaration location along with \<ElementType.TYPE\_USE>.)

Sometimes, it may seem tempting for an annotation to apply to both type
uses and (say) method declarations.  Here is a hypothetical example:

\begin{quote}
  ``Each \<Widget> type may have a \<@Version> annotation.
  I wish to prove that versions of widgets don't get assigned to
  incompatible variables, and that older code does not call newer code (to
  avoid problems when backporting).

  A \<@Version> annotation could be written like so:

\begin{Verbatim}
  @Version("2.0") Widget createWidget(String value) { ... }
\end{Verbatim}

\<@Version("2.0")> on the method could mean that the \<createWidget> method
only appears in the 2.0 version.  \<@Version("2.0")> on the return type
could mean that the returned \<Widget> should only be used by code that
uses the 2.0 API of \<Widget>.  It should be possible to specify these
independently, such as a 2.0 method that returns a value that allows the
1.0 API method invocations.''
\end{quote}

Both of these are type properties and should be specified with type
annotations.  No method annotation is necessary or desirable.  The best way
to require that the receiver has a certain property is to use a type
annotation on the receiver of the method.  (Slightly more formally, the
property being checked is compatibility between the annotation on the type
of the formal parameter receiver and the annotation on the type of the
actual receiver.)  If you do not know what ``receiver'' means, see the next
question.


Another example of a type-and-declaration annotation that represents poor
design is JCIP's \<@GuardedBy> annotation~\cite{Goetz2006}.  As discussed
in Section~\ref{jcip-annotations}, it means two different things when
applied to a field or a method.  To reduce confusion and increase
expressiveness, the Lock Checker (see Chapter~\ref{lock-checker}) uses the
\<@Holding> annotation for one of these meanings, rather than overloading
\<@GuardedBy> with two distinct meanings.


\subsection{How do I annotate a fully-qualified type name?\label{faq-annotate-fully-qualified-name}}

If you write a fully-qualified type name in your program, then the Java
language requires you to write a type annotation on the simple name part,
such as
\begin{Verbatim}
  entity.hibernate. @Nullable User x;
\end{Verbatim}

If you try to write the type annotation before the entire fully-qualified
name, such as
\begin{Verbatim}
  @Nullable entity.hibernate.User x;  // illegal Java syntax
\end{Verbatim}
\noindent
then you will get an error like one of the following:
\begin{Verbatim}
error: scoping construct for static nested type cannot be annotated
error: scoping construct cannot be annotated with type-use annotation
\end{Verbatim}


\section{Semantics of type annotations\label{faq-semantics-section}}


\subsection{Why are the type parameters to \<List> and \<Map> annotated as \<@NonNull>?\label{faq-list-map-nonnull-typeargs}}

The annotation on \<java.util.Collection> only allows non-null elements:

\begin{Verbatim}
  public interface Collection<E extends @NonNull Object> {
    ...
  }
\end{Verbatim}

\noindent
Thus, you will get a type error if you write code like
\code{Collection<@Nullable Object>}.
A nullable
type parameter is also forbidden for certain other collections, including
\<AbstractCollection>, \<List>, \<Map>, and \<Queue>.

% AbstractCollection has no documentation of its own regarding nullness,
% but it implements Collection.

% The JML specifications of the add() method says
%       @   signals (NullPointerException)
%       @             (* not allowed to add null *);
%       ...
%       @   signals (NullPointerException)
%       @             (* not allowed to add null *);
% In other words, the method might throw NullPointerException, but the JML
% spec does not say under what circumstances.

The \<extends @NonNull Object> bound is a direct consequence of the design
of the collections classes; it merely formalizes the Javadoc specification.
The Javadoc for \<Collection> states:

\begin{quote}
  Some list implementations have restrictions on the elements that they may
  contain. For example, some implementations prohibit null elements, \ldots
\end{quote}

Here are some consequences of the requirement to detect all nullness errors
at compile time.  If even one subclass of a given collection class may
prohibit null, then the collection class and all its subclasses must
prohibit null.  Conversely, if a collection class is specified to accept
null, then all its subclasses must honor that specification.

The Checker Framework's annotations make apparent a flaw in the JDK
design, and helps you to avoid problems that might be caused by that flaw.


\paragraph{Justification from type theory\label{faq-list-map-nonnull-typeargs-junification-from-type-theory}}
Suppose \<B> is a subtype of \<A>.
Then an overriding method in \<B> must have a stronger (or equal) signature
than the overridden method in~\<A>.  In a stronger signature, the formal
parameter types may be supertypes, and the return type may be a subtype.
Here are examples:

\begin{Verbatim}
  class A           {  @NonNull Object Number m1( @NonNull Object arg) { ... } }
  class B extends A { @Nullable Object Number m1( @NonNull Object arg) { ... } } // error!
  class C extends A {  @NonNull Object Number m1(@Nullable Object arg) { ... } } // OK
  class D           { @Nullable Object Number m2(@Nullable Object arg) { ... } }
  class E extends D {  @NonNull Object Number m2(@Nullable Object arg) { ... } } // OK
  class F extends D { @Nullable Object Number m2( @NonNull Object arg) { ... } } // error!
\end{Verbatim}

According to these rules, since some subclasses of \<Collection> do not
permit nulls, then \<Collection> cannot either:

\begin{Verbatim}
  // does not permit null elements
  class PriorityQueue<E> implements Collection<E> {
    boolean add(E);
    ...
  }
  // must not permit null elements, or PriorityQueue would not be a subtype of Collection
  interface Collection<E> {
    boolean add(E);
    ...
  }
\end{Verbatim}


\paragraph{Justification from checker behavior\label{faq-list-map-nonnull-typeargs-justification-from-behavior}}

Suppose that you changed the bound in the \<Collection> declaration to
\<extends @Nullable Object>.  Then, the checker would issue no warning for
this method:

\begin{Verbatim}
  static void addNull(Collection l) {
    l.add(null);
  }
\end{Verbatim}

\noindent
However, calling this method \emph{can} result in a null pointer exception,
for instance caused by the following code:

\begin{Verbatim}
  addNull(new PriorityQueue());
\end{Verbatim}

\noindent
Therefore, the bound must remain as \<extends @NonNull Object>.

By contrast, this code is OK because \<ArrayList> is documented to support
null elements:

\begin{Verbatim}
  static void addNull(ArrayList l) {
    l.add(null);
  }
\end{Verbatim}

\noindent
Therefore, the upper bound in \<ArrayList> is \<extends @Nullable Object>.
Any subclass of \<ArrayList> must also support null elements.




% Every implementation of List seems to permit null.
% Examples of Collection that do not permit null:
% BlockingQueue family:
%   BlockingQueue, BlockingDeque, ArrayBlockingQueue, DelayQueue, LinkedBlockingDeque, LinkedBlockingQueue, PriorityBlockingQueue, SynchronousQueue
% PriorityQueue
% probably lots of other queues.

% A similar argument applies to \<Map>.
% For example, \<ConcurrentHashMap> and \<Hashtable> implement \<Map> but do
% not permit \<null> to be used as a key or value.  Therefore, \<Map> must
% not permit \<null> to be used as a key or value


% The Checker Framework is designed to warn you whenever your code might
% throw a null pointer exception.  If you want to be safe, you will never put
% \<null> in a \<List> of unknown provenance, because that \<List> might not
% accept null.



\paragraph{Suppressing warnings\label{faq-list-map-nonnull-typeargs-suppressing-warnings}}

Suppose your program has a list variable, and you know that any list referenced
by that variable will definitely support null elements.  Then, you can suppress the
warning:

\begin{Verbatim}
  @SuppressWarnings("nullness:generic.argument") // any list passed to this
  method will support null elements
  static void addNull(List l) {
    l.add(null);
  }
\end{Verbatim}

\noindent
You need to use \<@SuppressWarnings("nullness:generic.argument")>
whenever you use a collection that may contain \<null> elements in
contradiction to its documentation.  Fortunately, such uses are relatively
rare.


For more details on suppressing nullness warnings, see
Section~\ref{suppressing-warnings-nullness}.


\subsection{How can I handle typestate, or phases of my program with different data properties?\label{faq-typestate}}

Sometimes, your program works in phases that have different behavior.  For
example, you might have a field that starts out null and becomes non-null
at some point during execution, such as after a method is called.  You can
express this property as follows:

\begin{enumerate}
\item
Annotate the field type as \refqualclass{checker/nullness/qual}{MonotonicNonNull}.
\item
Annotate the method that sets the field as \refqualclass{checker/nullness/qual}{EnsuresNonNull}\<(">\emph{\<myFieldName>}\<")>.
(If method \<m1> calls method \<m2>, which actually sets the field, then
you would probably write this annotation on both \<m1> and \<m2>.)
\item
Annotate any method that depends on the field being non-null as
\refqualclass{checker/nullness/qual}{RequiresNonNull}\<(">\emph{\<myFieldName>}\<")>.
The type-checker will verify that such a method is only called when the
field isn't null --- that is, the method is only called after the setting
method.
\end{enumerate}

You can also use a typestate checker (see
\chapterpageref{typestate-checker}), but they have not been as extensively
tested.


\subsection{Why are explicit and implicit bounds defaulted differently?\label{faq-implicit-bounds}}

The following two bits of code have the same semantics under Java, but are
treated differently by the Checker Framework's CLIMB-to-top defaulting
rules (Section~\ref{climb-to-top}):

\begin{Verbatim}
class MyClass<T> { ... }
class MyClass<T extends Object> { ... }
\end{Verbatim}

The difference is the annotation on the upper bound of the type argument
\<T>.  They are treated in the following.

\begin{Verbatim}
class MyClass<T>  ==  class MyClass<T extends @TOPTYPEANNO Object> { ... }
class MyClass<T extends Object>  ==  class MyClass<T extends @DEFAULTANNO Object>
\end{Verbatim}

\noindent
\<@TOPTYPEANNO> is the top annotation in the type qualifier hierarchy.  For
example, for the nullness type system, the top type annotation is
\<@Nullable>; as shown in Figure~\ref{fig-nullness-hierarchy}.
\<@DEFAULTANNO> is the default annotation for the type system.  For
example, for the nullness type system, the default type annotation is
\<@NonNull>.

In some type systems, the top qualifier and the default are the same.  For
such type systems, the two code snippets shown above are treated the same.
An example is the regular expression type system; see
Figure~\ref{fig-regex-hierarchy}.

The CLIMB-to-top rule reduces the code edits required to annotate an
existing program, and it treats types written in the program consistently.

When a user writes no upper bound, as in
\code{class C<T> \ttlcb\ ... \ttrcb},
then Java permits the class to be instantiated with any type parameter.
The Checker Framework behaves exactly the same, no matter what the default
is for a particular type system -- and no matter whether the user has
changed the default locally.

When a user writes an upper bound, as in
\code{class C<T extends OtherClass> \ttlcb\ ... \ttrcb},
then the Checker Framework treats this occurrence of \<OtherClass> exactly
like any other occurrence, and applies the usual defaulting rules.  Use of
\<Object> is treated consistently with all other types in this location and
all other occurrences of \<Object> in the program.

It is uncommon for a user to write \<Object> as an upper bound with no type
qualifier:
\code{class C<T extends Object> \ttlcb\ ... \ttrcb}.
It is better style to write no upper bound or to write an explicit type
annotation on \<Object>.


\section{Creating a new checker\label{faq-create-a-checker-section}}

\subsection{How do I create a new checker?\label{faq-create-a-checker}}

In addition to using the checkers that are distributed with the Checker
Framework, you can write your own checker to check specific properties that
you care about.  Thus, you can find and prevent the bugs that are most
important to you.

Chapter~\ref{writing-a-checker} gives
complete details regarding how to write a checker.  It also suggests places
to look for more help, such as the \href{api/}{Checker Framework
API documentation (Javadoc)} and the source code of the distributed
checkers.

To whet your interest and demonstrate how easy it is to get started, here
is an example of a complete, useful type-checker.

\begin{Verbatim}
  @SubtypeOf(Unqualified.class)
  @Target({ElementType.TYPE_USE, ElementType.TYPE_PARAMETER})
  public @interface Encrypted { }
\end{Verbatim}

Section~\ref{subtyping-example} explains this checker and tells
you how to run it.


\subsection{Why is there no declarative syntax for writing type rules?\label{faq-declarative-syntax-for-type-rules}}

A type system implementer can declaratively specify the type qualifier
hierarchy (Section~\ref{declarative-hierarchy}) and the type introduction rules
(Section~\ref{declarative-type-introduction}).  However, the Checker
Framework uses a procedural syntax for specifying type-checking
rules (Section~\ref{extending-visitor}).
A declarative syntax might be more concise, more readable, and more
verifiable than a procedural syntax.

We have not found the procedural syntax to be the most important impediment
to writing a checker.

Previous attempts to devise a declarative syntax
for realistic type systems have failed; see a technical
paper~\cite{PapiACPE2008} for a discussion.  When an
adequate syntax exists, then the Checker Framework can be extended to
support it.


\section{Relationship to other tools\label{faq-other-tools-section}}


\subsection{Why not just use a bug detector (like FindBugs)?\label{faq-type-checking-vs-bug-detectors}}

Pluggable type-checking finds more bugs than a bug detector does, for any
given variety of bug.

A bug detector like \href{http://findbugs.sourceforge.net/}{FindBugs}~\cite{HovemeyerP2004,HovemeyerSP2005},
\href{http://jlint.sourceforge.net/}{Jlint}~\cite{Artho2001}, or
\href{http://pmd.sourceforge.net/}{PMD}~\cite{Copeland2005} aims to find \emph{some}
of the most obvious bugs in your program.  It uses a lightweight analysis,
then uses heuristics to discard some of its warnings.  Thus, even if the tool
prints no warnings, your code might still have errors --- maybe the
analysis was too weak to find them, or the tool's heuristics classified the
warnings as likely false positives and discarded them.

A type-checker aims to find \emph{all} the bugs (of certain varieties).
It requires you to write type qualifiers in your program, or to use a tool
that infers types.  Thus, it requires more work from the programmer, and in
return it gives stronger guarantees.

Each tool is useful in different circumstances, depending on how important
your code is and your desired level of confidence in your code.  For more
details on the comparison, see Section~\ref{other-tools}.  For a case study
that compared the nullness analysis of FindBugs, Jlint, PMD, and the
Checker Framework, see section 6 of the paper
\href{http://homes.cs.washington.edu/~mernst/pubs/pluggable-checkers-issta2008.pdf}{``Practical pluggable types for Java''}~\cite{PapiACPE2008}.


\subsection{How does the Checker Framework compare with Eclipse's null analysis?\label{faq-eclipse}}

Eclipse comes with a
\href{http://help.eclipse.org/luna/index.jsp?topic=\%2Forg.eclipse.jdt.doc.user\%2Ftasks\%2Ftask-using_null_annotations.htm}{null analysis} that
can detect potential null pointer errors in your code.  Eclipse's built-in
analysis differs from the Checker Framework in several respects.

The Checker Framework's Nullness Checker
(see~\chapterpageref{nullness-checker}) is more precise:  it does a deeper
semantic analysis, so it issues fewer false positives than Eclipse.  For
example, the Nullness Checker handles initialization and map key checking,
it supports method pre- and post-conditions, and it includes a powerful
dataflow analysis.

Eclipse assumes that all code is multi-threaded, which cripples its local
type inference.  By contrast, the Checker Framework allows the user to
specify whether code will be run concurrently or not via the
\<-AconcurrentSemantics> command-line option (see
Section~\ref{faq-concurrency}).

The Checker Framework is easier to run in integration scripts or in
environments where not all developers are using Eclipse.
% It is possible to use ecj as one's compiler:
% https://wiki.eclipse.org/JDT/FAQ#Can_I_use_JDT_outside_Eclipse_to_compile_Java_code.3F

Eclipse handles only nullness properties and is not extensible, whereas the
Checker Framework comes with over 20 type-checkers (for a list,
see~\chapterpageref{introduction}) and is extensible to more properties.

There are also some benefits to Eclipse's Null Analysis.
It is faster than the Checker Framework, in part because it is less featureful.
It is built into Eclipse, so you do not have to download and install a
separate Eclipse plugin as you do for the Checker Framework (see
Section~\ref{eclipse-plugin}).
Its IDE integration is tighter and slicker.

(If you know of other differences, please let us know at
checker-framework-dev@googlegroups.com so we can update the manual.)


\subsection{How does pluggable type-checking compare with JML?\label{faq-jml}}

\href{http://www.cs.ucf.edu/~leavens/JML/}{JML}, the Java Modeling
Language~\cite{LeavensBR2006:JML}, is a language for writing formal
specifications.

\textbf{JML aims to be more expressive than pluggable type-checking.}
A programmer can write a JML specification that
describes arbitrary facts about program behavior.  Then, the programmer can
use formal reasoning or a theorem-proving tool to verify that the code
meets the specification.  Run-time checking is also possible.
By contrast, pluggable type-checking can express a more limited set of
properties about your program.  Pluggable type-checking annotations are
more concise and easier to understand.

\textbf{JML is not as practical as pluggable type-checking.}
The JML toolset is less mature.  For instance, if your code uses
generics or other features of Java 5, then you cannot use JML.
However, JML has a run-time checker, which the Checker Framework currently
lacks.


\subsection{Is the Checker Framework an official part of Java?\label{faq-checker-framework-part-of-java}}

The Checker Framework is not an official part of Java.
The Checker Framework relies on
type annotations, which are part of Java 8.  See the
\href{http://types.cs.washington.edu/jsr308/current/jsr308-faq.html#pluggable-type-checking-in-java}{Type
  Annotations (JSR 308) FAQ} for more details.


\subsection{What is the relationship between the Checker Framework and JSR 305?\label{faq-jsr-305}}

JSR 305 aimed to define official Java names for some annotations, such as
\<@NonNull> and \<@Nullable>.  However, it did not aim to precisely define
the semantics of those annotations nor to provide a reference
implementation of an annotation processor that validated their use;
as a result, JSR 305 was of limited utility as a specification.
JSR 305 has been abandoned; there has been
no activity by its expert group since
% January
2009.

By contrast, the Checker Framework precisely defines the meaning of a set
of annotations and provides powerful type-checkers that validate them.
However, the Checker Framework is not an official part of the Java
language; it chooses one set of names, but another tool might choose other
names.

In the future, the Java Community Process might revitalize JSR 305 or
create a replacement JSR to standardize the names and
meanings of specific annotations, after there is more experience with their
use in practice.

% JSR 305 didn't specify the semantics of its annotations, and where it did
% they were idiosyncratic -- essentially mimicking the FindBugs tool, but
% not useful for any other defect detection tool.  A revitalization of JSR
% 305 would have to start over from scratch in order to clearly specify a
% semantics that is general and useful for a whole range of tools.
% The Java community does not yet understand all the subtleties well enough
% to set the annotations in stone in the Java specification yet; it is
% better for the community to experiment with different approaches, such as
% those of FindBugs, IntelliJ, Eclipse, and the Checker Framework, so that
% we can come to consensus before deciding on an official set.


The Checker Framework defines annotations \<@NonNull> and \<@Nullable> that
are compatible with annotations defined by JSR 305, FindBugs, IntelliJ, and
other tools; see Section~\ref{nullness-related-work}.


\subsection{What is the relationship between the Checker Framework and JSR 308?\label{faq-jsr-308}}

JSR 308, also known as the Type Annotations specification, dictates the
syntax of type annotations in Java SE 8:  how they are expressed in the
Java language.

JSR 308 does not define any type annotations such as \<@NonNull>, and it does
not specify the semantics of any annotations.  Those tasks are left to
third-party tools.  The Checker Framework is one such tool.

The Checker Framework makes use of Java SE 8's type annotation syntax, but
the Checker Framework can
be used with previous versions of the Java language via the
annotations-in-comments feature (Section~\ref{annotations-in-comments}).


% LocalWords:  IGJ toolset AbstractCollection ConcurrentHashMap NullnessUtils
% LocalWords:  castNonNull createWidget backporting JCIP's GuardedBy Awarns PMD
% LocalWords:  ElementType nullness bytecodes Jlint Hashtable SuppressWarnings
%  LocalWords:  RegexUtil compareTo myA requiresB nullable java myobject
%  LocalWords:  multithreading myfield Regex NonEmpty Xmaxerrs Xmaxwarns
%  LocalWords:  igj javari MonotonicNonNull EnsuresNonNull myFieldName pre
%  LocalWords:  m1 m2 RequiresNonNull AconcurrentSemantics MyQual plugin
%  LocalWords:  MyClass MyMoreRestrictiveQual TOPTYPEANNO DEFAULTANNO api
%%  LocalWords:  OtherClass Goetz antipattern subclassed varargs
%%  LocalWords:  featureful

\htmlhr
\chapter{Troubleshooting and getting help\label{troubleshooting}}

Please read the entire manual, including this chapter and the FAQ
(Chapter~\ref{faq}), because the manual might already answer your question.
If not, you can use the mailing list,
\code{checker-framework-discuss@googlegroups.com}, to ask other users for
help.  For archives and to subscribe, see \url{http://groups.google.com/group/checker-framework-discuss}.
To report bugs, use the issue tracker at
\url{http://code.google.com/p/checker-framework/issues/list}.
If you want to help out, you can choose a bug and fix it, or select a
project from the ideas list at
\url{http://code.google.com/p/checker-framework/wiki/Ideas}.


\section{Common problems and solutions\label{common-problems}}

\begin{itemize}

\item
To verify that you are using the compiler you think you are, you can add
\code{-version} to the command line.  For instance, instead of running
\code{javac -g MyFile.java}, you can run \code{javac \underline{-version} -g
  MyFile.java}.  Then, javac will print out its version number in addition
to doing its normal processing.


\item
If you get the error

%BEGIN LATEX
\begin{smaller}
%END LATEX
\begin{Verbatim}
com.sun.tools.javac.code.Symbol$CompletionFailure: class file for com.sun.source.tree.Tree not found
\end{Verbatim}
% Unconfuse Emacs by matching the "$" in the above Verbatim
%BEGIN LATEX
\end{smaller}
%END LATEX

\noindent
then you are using the source installation and file \code{tools.jar} is not
on your classpath.  See the installation instructions
(Section~\ref{installation}).


\item
If you get an error such as

\begin{Verbatim}
package checkers.nullness.quals does not exist
\end{Verbatim}

  \noindent
  despite no apparent use of \code{import checkers.nullness.quals.*;} in
  the source code, then perhaps
  \code{jsr308\_imports} is set as a Java system property, a shell
  environment variable, or a command-line option (see
  Section~\ref{jsr308_imports}).  You can solve this by unsetting the
  variable/option, or by ensuring that the \code{checkers.jar} file is on
  your classpath.


\item
If a checker seems to be ignoring the annotation on a method, then it is
possible that the checker is reading the method's signature from its
\code{.class} file, but the \code{.class} file was not created by the JSR
308 compiler.  You can check whether the annotations actually appear in the
\code{.class} file by using the \code{javap} tool.

If the annotations do not appear in the \code{.class} file, here are two
ways to solve the problem:
\begin{itemize}
\item
  Re-compile the method's class with the Type Annotations compiler.  This will
  ensure that the type annotations are written to the class file, even if
  no type-checking happens during that execution.
\item
  Pass the method's file explicitly on the command line when type-checking,
  so that the compiler reads its source code instead of its \code{.class}
  file.
\end{itemize}

\item
If the compiler reports that it cannot find a method from the
JDK or another external library, then maybe the stub/skeleton file for that
class is incomplete.  You can edit it to add the missing method.  The
libraries appear, for example, at \code{checkers/jdk/nullness/src/} for the
Nullness checker.

The error might take one of these forms:

\begin{Verbatim}
method sleep in class Thread cannot be applied to given types
cannot find symbol: constructor StringBuffer(StringBuffer)
\end{Verbatim}

\item
If you get an error like the following when using the Ant task
(Section~\ref{ant-task}),

%BEGIN LATEX
\begin{smaller}
%END LATEX
\begin{Verbatim}
...\build.xml:59: Error running ${env.CHECKERS}\binary\javac.bat compiler
\end{Verbatim}
% Unconfuse Emacs by matching the "$" in the above Verbatim
%BEGIN LATEX
\end{smaller}
%END LATEX

\noindent
then the problem may be that you have not set the CHECKERS environment
variable, as described in Section~\ref{windows-installation}.  Or, maybe
you made it a user variable instead of a system variable.

\end{itemize}


\subsection{Known problems in the framework\label{known-problems}}

\begin{itemize}

\item
  The framework may not parse annotations from skeleton files if the
  skeleton files are older than the classfiles.  Running \code{ant
    touch-jdk} solves this problem, by applying the 
  \code{touch} program to each distributed skeleton file.

% Mahmood will address.  -MDE 3/19/2009
\item The framework is missing a check for type argument subtyping in
  method invocations if the type arguments are inferred.

% Mahmood will address.  -MDE 3/19/2009
\item The checks for enclosed types are not yet fully tested.

\end{itemize}

\subsection{Known problems in the Nullness checker}

\begin{itemize}
\item
  The Nullness checker is often able to determine that a call to
  \code{Map.get()} will not return null.  This enables the checker to avoid
  issuing false positive warnings, in circumstances like the following.

\begin{Verbatim}
    @NonNull String value;
    if (myMap.containsKey(key)) {
      value = myMap.get(key);
    }
    for (String keyInMap : myMap.keySet()) {
        value = myMap.get(keyInMap);
    }
\end{Verbatim}

  The Nullness checker can sometimes fail to issue a warning if the map is
  modified or re-assigned between the check of \code{containsKey} and the
  call to \code{get}.

% The solution is to merge flow with the Map.get heuristics.
% And to do forward instead of backward analysis.


\item 
  The Nullness checker issues a warning when a constructor does not
  initialize every non-null field.  However, because the checker does not
  fully implement all of Java's definite assignment rules (e.g., for
  \code{finally} blocks), the checker sometimes issues a false positive
  warning.  The checker's behavior is sound but unnecessarily restrictive.
  If you encounter this problem in practice, please submit a bug report so
  that we can improve the checker.

\end{itemize}



\section{How to report problems\label{reporting-bugs}}

If you have a problem with any checker, or with the Checker Framework,
please file a bug at 
\url{http://code.google.com/p/checker-framework/issues/list}.
(First, check whether there is an existing bug report for that issue.)

Alternately (especially if your communication is not a bug report), you can
send mail to checker-framework-dev@googlegroups.com.
We welcome suggestions, annotated libraries, bug fixes, new
features, new checker plugins, and other improvements.

Please ensure that your bug report is clear and that it is complete.
Otherwise, we may be unable to understand it or to reproduce it, either of
which would prevent us from fixing the bug.  Your bug report will be most
helpful if you:

\begin{itemize}
\item
  Add \code{-version -verbose} to the javac options.  This causes the compiler to output
  debugging information, including its version number.
\item
  Indicate exactly what you did.  Don't skip any steps, and don't merely
  describe your actions in words.  Show the exact commands by attaching a
  file or using cut-and-paste from your command shell;
\item
  Include all files that are necessary to reproduce the problem.  This
  includes every file that is used by any of the commands you reported, and
  possibly other files as well.
\item
  Indicate exactly what the result was by attaching a file or using
  cut-and-paste from your command shell (don't merely describe it in
  words).  Also indicate what you expected the result to be --- remember, a
  bug is a difference between desired and actual outcomes.
\end{itemize}


\section{Building from source\label{build-source}}

The Checker Framework release (Section~\ref{installation}) contains everything that
most users need, both to use the distributed checkers and to write your own
checkers.  This section describes how to re-build its binaries from source.
% Doing
% so permits you to examine and modify the implementation of the distributed
% checkers and of the checker framework.  It may also help you to debug
% problems more effectively.


\subsection{Obtain the source}

Obtain the latest source code from the version control repository:

\begin{Verbatim}
export JSR308=$HOME/jsr308
mkdir -p $JSR308
cd $JSR308
hg clone https://jsr308-langtools.googlecode.com/hg/ jsr308-langtools
hg clone https://checker-framework.googlecode.com/hg/ checker-framework
\end{Verbatim}

\noindent
(Alternately, you could use the version of the source code that is packaged
in the Checker Framework release.)


\subsection{Build the Type Annotations compiler}

\begin{enumerate}
\item
Set the \<JAVA\_HOME> environment variable to the location of your JDK 6 or 7 installation. Most likely it is already set for Ant to work.
\item
Compile the Type Annotations javac compiler and the javap tool:

\begin{Verbatim}
  cd $JSR308/jsr308-langtools/make
  ant clean build-javac build-javap
\end{Verbatim}

\item
 Add the \<jsr308-langtools/dist/bin> directory to the front of your PATH environment variable.
  Example command:

\begin{Verbatim}
  export PATH=$JSR308/jsr308-langtools/dist/bin:${PATH}
\end{Verbatim}

\end{enumerate}

% JSR 308 extends the Java language to permit annotations to appear on types,
% as in \code{List<@NonNull String>} (see Section~\ref{writing-annotations}).
% This change will be part of the Java 7 language.  We recommend that you
% write annotations in comments, as in \code{List</*@NonNull*/ String>} (see
% Section~\ref{annotations-in-comments}).  The JSR 308 compiler still reads
% such annotations, but this syntax permits you to use a compiler other than
% the JSR 308 compiler.  For example, you can compile your code with a Java 5
% compiler, and you can use a checker as an external tool in an IDE.


\subsection{Build the Checker Framework\label{building}}

% Building (compiling) the checkers and framework from source creates the
% \code{checkers.jar} file.  A pre-compiled \code{checkers.jar} is included
% in the distribution, so building it is optional.  It is mostly useful for
% people who are developing compiler plug-ins (type-checkers).  If you only
% want to \emph{use} the compiler and existing plug-ins, it is sufficient to
% use the pre-compiled version.

\begin{enumerate}
% \item
% Edit \code{checkers/build.properties} file so that the
% \code{compiler.lib} property specifies the location of the JSR 308
% \code{javac.jar} library.  (If you also installed the JSR 308 compiler from
% source, and you made the \code{checkers} and \code{jsr308-langtools} directories
% siblings, then you don't need to edit \code{checkers/build.properties}.)

\item
Run \code{ant} to create \<checkers.jar>:

\begin{Verbatim}
  cd $JSR308/checker-framework/checkers
  ant
\end{Verbatim}

\item Add \code{tools.jar} and \code{checkers.jar} to your classpath
  (If you do not do this, you will have to supply the \code{-cp} option
  whenever you run \code{javac} and use a checker plugin.)
  Example command:

%BEGIN LATEX
\begin{smaller}
%END LATEX
\begin{Verbatim}
  export CLASSPATH=${CLASSPATH}:$JAVA_HOME/lib/tools.jar:$JSR308/checker-framework/checkers/checkers.jar
\end{Verbatim}
%BEGIN LATEX
\end{smaller}
%END LATEX
  %% In Cygwin, are reversed slashes required?

\item Test that everything works:

  \begin{itemize}

  \item Run \code{ant all-tests} in the \code{checkers} directory:
\begin{Verbatim}
cd checkers
ant all-tests
\end{Verbatim}

  \item Run the Nullness checker examples (see
    Section~\refwithpage{nullness-example}).

  \end{itemize}

\end{enumerate}

% \subsection{Adjust classpath}

% Building the Checker Framework requires use of a Java 7 compiler.  You may
% use either the OpenJDK compiler or the JSR 308 compiler.  The latter has a
% few extra features and tends to get bug fixes more quickly.

% The following instructions give detailed steps for installing the source
% release of the Checker Framework.


% \item Download and install the JSR 308 implementation; follow the instructions at
% % alternative: \urldef{\JsrInstallingUrl}\url{http://types.cs.washington.edu/checker-framework/current/README-jsr308.html#installing}
% {\codesize\url{http://types.cs.washington.edu/checker-framework/current/README-jsr308.html#installing}}.
% This creates a \code{jsr308-langtools} directory.
% 
% \item Download the Checker Framework distribution zipfile from
% \myurl{http://types.cs.washington.edu/checker-framework/current/checkers.zip},
% and unzip it to create a \code{checkers} directory.  We recommend that the
% \code{checkers} directory and the \code{jsr308-langtools} directory be siblings.
% Example commands:
% 
% \begin{Verbatim}
%   cd $JSR308
%   wget http://types.cs.washington.edu/checker-framework/current/checkers.zip
%   unzip checkers.zip
% \end{Verbatim}
% 
% You will also need to adjust the path to \<javac> in any Ant buildfiles,
% etc.

% \item Optionally edit property \code{compiler.lib} in file
%   \code{checkers/build.properties}.  You don't have to do this if the
%   \code{checkers} directory and the \code{jsr308-langtools} directory are
%   siblings.


% (A checkers implementation builds on
% standard mechanisms such as JSR 269 annotation processing, but also
% accesses the compiler's AST. In the long run, a checker built using the
% Checker Framework should not be dependent on any compiler specifics.)
% If you do not place the annotations in 
% then you should also disable Eclipse's on-the-fly syntax checking.




% \subsection{TO DO:  The short instructions (for Linux only)}
% 
% %%% This comment does not seem to be correct any longer.
% %% This text is identically reproduced at ../../jsr308-langtools/README-jsr308.html
% %% so if you change either one, change the other also!
% 
% The following commands install
% the JSR 308 \code{javac} compiler and the Checker
% Framework, or update an existing installation.
% It currently works only on \textbf{Linux}.
% For more details, or if anything goes wrong, see the comments in the 
% \code{Makefile-jsr308-install} file.
% 
% \begin{enumerate}
% 
% \item
%   Execute the following commands:
% 
% \begin{Verbatim}
%   cd
%   wget -nv -N http://types.cs.washington.edu/jsr308/Makefile-jsr308-install
%   make -f Makefile-jsr308-install
% \end{Verbatim}
% 
% \item
% Set some environment variables according to the instructions at the top of file
% \code{Makefile-jsr308-install}.
% 
% \end{enumerate}



\section{Learning more\label{learning-more}}

The technical paper ``Practical pluggable types for Java''~\cite{PapiACPE2008}
(\myurl{http://www.cs.washington.edu/homes/mernst/pubs/pluggable-checkers-issta2008.pdf})
gives more technical detail about many
aspects of the Checker Framework and its implementation.
%
The technical
paper also describes case
studies in which each of the checkers found
previously-unknown errors in real software.


\section{Comparison to other tools\label{other-tools}}

A pluggable type-checker, such as those created by the Checker Framework,
aims to help you prevent or detect all errors of a given variety.  An
alternate approach is to use a bug detector such as
\ahref{http://findbugs.sourceforge.net/}{FindBugs},
\ahref{http://artho.com/jlint/}{JLint}, or
\ahref{http://pmd.sourceforge.net/}{PMD}.

A pluggable type-checker
differs from a bug detector in several ways:
\begin{itemize}
\item
  A type-checker aims to find \emph{all} errors.  Thus, it can verify the
  \emph{absence} of errors:  if the type checker says there are no null
  pointer errors in your code, then there are none.  (This guarantee only
  holds for the code it checks, of course; see
  Section~\ref{checker-guarantees}.)

  A bug detector aims to find \emph{some} of the most obvious errors.  Even
  if it reports no errors, then there may still be errors in your code.

  Both types of tools may issue false positive warnings; see
  Section~\ref{suppressing-warnings}.

\item
  A type-checker requires you to annotate your code with type qualifiers,
  or to run an inference tool that does so for you.  A bug detector may not
  require annotations.  This means that it may be easier to get started
  running a bug detector.

\item
  A type-checker may use a more sophisticated and complete analysis.
  A bug detector typically does a more lightweight analysis, coupled with
  heuristics to suppress false positives.

  As one example, a type-checker can take advantage of annotations on
  generic type parameters, such as \code{List<@NonNull String>}, permitting
  it to be much more precise for code that uses generics.

\end{itemize}

A case study~\cite[\S6]{PapiACPE2008} compared the Checker Framework's nullness
checker with those of FindBugs, JLint, and PMD\@.  The case study was on a
well-tested program in daily use.  The Checker Framework tool found 8
nullness errors.  None of the other tools found any errors.

Also see the
\ahref{http://types.cs.washington.edu/jsr308/}{JSR 308}~\cite{jsr308}
documentation for a detailed discussion of related work.



\section{Credits and changelog\label{credits}}

The key developers of the Checker Framework are Mahmood Ali, Telmo Correa,
Michael D. Ernst, and Matthew M. Papi.
Many users have provided valuable feedback, for which we are grateful.

%% Not so accurate, since Mahmood is really an author of the nullness and
%% interned checkers too.
% The Checker Framework was implemented by 
% The nullness checker was implemented by Matthew M. Papi.
% The interning checker was implemented by Matthew M. Papi.
% The Javari checker was implemented by Telmo Correa.
% The IGJ checker was implemented by Mahmood Ali.
% The basic checker was implemented by Matthew M. Papi.

Differences from previous versions of the checkers and framework can be found
in the \code{changelog-checkers.txt} file.  This file is included in the
Checker Framework distribution and is also available on the web at
\myurl{http://types.cs.washington.edu/checker-framework/current/changelog-checkers.txt}.






% LocalWords:  jsr unsetting plugins langtools zipfile cp plugin Nullness txt
% LocalWords:  nullness classpath NonNull MyObject javac uref changelog MyEnum
% LocalWords:  subtyping containsKey proc classfiles SourceChecker javap jdk
% LocalWords:  MyFile buildfiles


\htmlhr
\bibliographystyle{alpha}
\bibliography{bibstring-unabbrev,types,ernst,invariants,generals,alias,concurrency}

\end{document}

% LocalWords:  pt TODO JavaDocs Arg api HEVEA html ernst IGJ igj javari
